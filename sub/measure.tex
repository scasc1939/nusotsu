\documentclass[../sotsu.tex]{subfiles}

\begin{document}


\section{測度}
\label{sec:measure}

実ユークリッド空間$ℝ^n$の適当な部分集合においては,``体積''が定義できる.
この体積という概念を一般化したものが\ruby{測度}{そく|ど}である.

この節の議論は\citeauthor{ito-lebesgue-1963}『\citetitle{ito-lebesgue-1963}』による.
同書において集合の和$A + B$とかかれているのは,直和$A \sqcup B$の意味である.
測度について$m(A \overset{直和}{+} B) = m(A) + m(B)$であることを考えれば合理的な表記ともいえるが,
本稿ではベクトル空間の\refdfn[和空間]{dfn:sum-of-vector-space}とまぎらわしいので用いない.


\subsection{ユークリッド空間における体積}

測度の定義をする前に,
通常のユークリッド空間における「体積」とは何かについて考える.

$X = ℝ$を実数の空間とする.
$ℝ$の部分集合であって,
\begin{equation}
    \label{eq:interval-example}
    (a, b] \coloneq \Set{  x \in ℝ  \given  a < x \leq b  },
    \qquad -\infty \leq a < b < \infty
\end{equation}
とあらわされるものを\word{区間}[く|かん][くかん]という\cite{ito-lebesgue-1963}.

$X = ℝ^n$を実ユークリッド空間とする.
ユークリッド空間の``領域''を,直積
\begin{equation}
    I \coloneq (a_1, b_1] \times \dots \times (a_n, b_n],
    \qquad -\infty \leq a_i < b_i < \infty
\end{equation}
で定める.
$I$は,$n$-次元の直方体(長方形)領域である.

さらに,有限個の区間$I_i$($i = 1, 2, \dots, N$)の\refdfn[直和(非交和)]{dfn:disjoint-union}
\begin{equation}
    I_1 \sqcup I_2 \sqcup \dots \sqcup I_N
\end{equation}
を\word{区間塊}という.
いくつかの直方体が組み合わさっている領域が区間塊である.
直和であるから,直方体同士はくっついていても離れていてもよいが,重なってはいけない.

$N$-次元ユークリッド空間における区間全体の集合を$ℑ_N$,
区間塊全体の集合を$𝔉_N$とかく\cite{ito-lebesgue-1963}.


\subsection{有限加法族}
\label{sec:finitely-additive-class}

\begin{definition}
    \label{dfn:finitely-additive-class}
    空間$X$の部分集合の\refdfn[族]{dfn:family-of-sets}(つまり$X$の部分集合からなる集合)$𝔉$が以下の3つを満たすとき,
    $𝔉$を\word{有限加法族}という.
    \begin{enumerate}
        \item \label{field-of-sets:empty-or-X}
            $X, \emptyset \in 𝔉$である.
        \item \label{field-of-sets:complementation}
            $A \in 𝔉$なら$A^{\complement} \in 𝔉$
            (ただし補集合$A^{\complement} = X \setminus A$).
        \item \label{field-of-sets:operation} 
            $A, B \in 𝔉$ならば,
            $A \cup B \in 𝔉$,
            $A \cap B \in 𝔉$,
            $A \setminus B \in 𝔉$が成り立つ.
    \end{enumerate}
\end{definition}

実は,\cref{field-of-sets:empty-or-X}は$X$か空集合$\emptyset$どちらか片方だけで十分であるし,
\cref{field-of-sets:operation}の3条件はどれかひとつで十分である.

任意の空間$X$について,その\refdfn[部分集合全体の集合]{dfn:power-set}$𝔉_{\mathrm{max}} \coloneq \pset(X)$は最大の有限加法族である.
また,$𝔉_{\mathrm{min}} \coloneq \{ X, \emptyset \}$は最小の有限加法族である.


\begin{example}
    $n$-次元ユークリッド空間における区間塊全体の集合$𝔉_n$は有限加法族である\cite[\S 4]{ito-lebesgue-1963}.
\end{example}

\begin{definition}
    \label{dfn:content-measure}
    $X$を空間,$𝔉$を有限加法族とする.
    $𝔉$の元(つまり$X$の部分集合)$A$を実数にうつす関数$m$が以下の条件を満たすとき,
    $m$を\word{有限加法的測度}という.
    \begin{enumerate}
        \item \label{content-measure:positivity}
            任意の$A \in 𝔉$に対し,$0 \leq m(A) \leq +\infty$である.
            特に,$m(\emptyset) = 0$である.
        \item \label{content-measure:additivity}(\word{有限加法性})
            $A, B \in 𝔉$かつ$A \cap B = \emptyset$(交わらない)なら,
            $m(A \sqcup B) = m(A) + m(B)$である.

            より一般に,$A_1, \dots, A_n \in 𝔉$が$A_i \cup A_j = \emptyset$($i \neq j$),
            つまり互いに交わらないなら,
            \begin{equation*}
                m \ab( \bigsqcup_{i=1}^{n} A_i ) = \sum_{i=1}^{n} m(A_i)
            \end{equation*}
    \end{enumerate}
    以下の性質は定義からただちに導かれるが,
    重要なので定義として加えておく.
    \begin{enumerate}[resume]
        \item \label{content-measure:monotonicity}(\word{単調性})
            $A \subset B$($\in 𝔉$)なら$m(A) \leq m(B)$である.
        \item $A \subset B$($\in 𝔉$)なら$m(A \setminus B) = m(A) - m(B)$である.
        \item \label{content-measure:subadditivity}(\word{有限劣加法性})
            $A_1, \dots, A_n \in 𝔉$なら,
            \begin{equation*}
                m \ab( \bigcup_{i=1}^{n} A_i ) \leq \sum_{i=1}^{n} m(A_i)
            \end{equation*}
    \end{enumerate}
    これらに加えて,
    さらに以下が成り立つとき(必ず成り立つとは限らない),
    $m$は\word{完全加法的}\index{かんせんかほうてき@完全加法的}であるという.
    \begin{enumerate}[resume]
        \item \refdfn[高々可算個]{dfn:at-most-countable}の集合$A_1, A_2, \dotsc$が$A_i \cup A_j = \emptyset$($i \neq j$)を満たし,
            $A \coloneq \bigsqcup_{n=1}^{\infty} A_n \in 𝔉$であるなら,
            \begin{equation*}
                m(A) = \sum_{n=1}^{\infty} m(A_n)
            \end{equation*}
    \end{enumerate}
\end{definition}

\cref{content-measure:additivity}より,
2つの領域$A$,$B$が全く重なっていなければ,
2つを合わせた領域$A \cup B$の体積は,$A$の体積と$B$の体積の和になるはずである.
また,$A$と$B$が一部重なっている場合,
$A \cup B$の体積は単純な和よりもすこし小さくなるはずであるということを
\cref{content-measure:subadditivity}は主張している.

\cref{content-measure:monotonicity}より,
領域$A$が完全に$B$に含まれているのなら,$A$の体積が$B$の体積を超えることはない.



\begin{proposition}
    \label{thm:Lebesgue-finitely-additive-measure}
    $N$-次元ユークリッド空間$ℝ^N$における区間塊$E = I_1 + \dots + I_n$
    ($E \in 𝔉_N$,$I_i \in ℑ$)に対し,
    \begin{equation}
        \begin{split}
            m^* (E)  &\coloneq  \prod_\nu (b_\nu - a_\nu)  \\
                     &=  (b_1 - a_1) \times \dots \times (b_N - a_N)
                     \quad \text{ただし} \quad I_\nu = (a_\nu, b_\nu]
        \end{split}
    \end{equation}
    は$𝔉_N$上の有限加法的測度である(\cref{eq:interval-example}を参照).
    さらに$m^*$は完全加法的である.
\end{proposition}



\subsection{外測度と可測集合}

\begin{definition}[外測度]
    \label{dfn:outer-measure}
    $X$を集合,$Γ$を$X$の部分集合から実数への関数とする.
    任意の部分集合$A, B$について,
    \begin{enumerate}
        \item (非負性)$0 \leq Γ(A) \leq +\infty$であり,さらに$Γ(\emptyset) = 0$である.
        \item (単調性)$A \subset B$なら$Γ(A) \leq Γ(B)$である.
        \item (劣加法性)$Γ \ab( \bigcup_{n=1}^{\infty} A_n ) \leq \sum_{n=1}^{\infty} Γ(A_n)$
    \end{enumerate}
    の3つが満たされるとき,
    $Γ$を\word{カラテオドリの外測度}(Carathéodory outer measure)%
    \index{かいそくと@外測度!からておとりの@カラテオドリの\indexdash}%
    \index{からておとりのかいそくと@カラテオドリの外測度|see{外測度}}%
    \cite{rikagaku-eiwa},
    あるいは単に\word{外測度}[がい|そく|ど][かいそくと](outer measure)という\cite{ito-lebesgue-1963}%
    \footnote{\person{Constantin Carathéodory}{1873}{1950}.ドイツの数学者.熱力学第二法則の定式化でも知られる.\cite{nipponica}}.
\end{definition}


\cref{sec:finitely-additive-class}で議論したように,
ある空間$X$と有限加法族$𝔉$が与えられたとき,
集合$A \in 𝔉$に対して\refdfn-[有限加法的測度]{dfn:finitely-additive-class}$m(A)$が定義できたのであった.
次の\namecref{thm:finite-additive-to-outer-measure}を用いると,
$X$の部分集合すべてに対して$m$が定義できるように拡張することができる.


\begin{theorem}
    \label{thm:finite-additive-to-outer-measure}
    空間$X$を考え,$𝔉$を有限加法族,$m$を$𝔉$上の有限加法的測度とする.
    $𝔉$に属する集合$E$の測度は$m(E)$と定義できる.
    これを$X$の任意の集合$A$($A \in 𝔉$とは限らない)に対して拡張しよう.

    $A \subset X$を集合とする.
    高々可算個の$𝔉$の集合の組$\{ E_n \}$で$A$を覆う%
    \footnote{
        つまり,
        $\displaystyle A \subset \bigcup_{n=1}^{\infty} E_n$となるような集合の組$ \{ E_n \} $
        (ただし$E_k \in 𝔉$)であり,
        $E_1, E_2, \dotsc$と数えると可算無限個(ないしは有限個)である.
        $E_k$は可算集合でなくてもよい.
    }.
    このような覆いかたは必ず存在し\footnote{たとえば1個の集合$X$で$A$を覆うことができる.},
    またそれぞれの覆いかた$\{ E_n \}$について
    $\sum_{n=1}^{\infty} m(E_n)$を計算することができる.

    このような$A$の覆いかたすべてを考え,
    そのうえで
    \begin{equation*}
        Γ(A)  \coloneq  \inf_{ \{ E_n \} } \sum_{n=1}^{\infty} m(E_n)
    \end{equation*}
    と定めると,$Γ$は\refdfn-[外測度]{dfn:outer-measure}である.
\end{theorem}

\begin{proof}
    \cref{dfn:outer-measure}の条件を確認すればよい.
\end{proof}


応用上重要なのは,次のルベーグ外測度である.

\begin{definition}
    \label{dfn:Lebesgue-outer-measure}
    \cref{thm:Lebesgue-finitely-additive-measure}%
    で定義した$m^*$を%
    \cref{thm:finite-additive-to-outer-measure}%
    の方法によって拡張した$\mu^*$を,
    \word{ルベーグ外測度}(Lebesgue outer measure)%
    \index{るへえく@ルベーグ!かいそくと@\indexdash 外測度|see{外測度}}%
    \index{かいそくと@外測度!るへえく@ルベーグ\indexdash}%
    という.
\end{definition}


定義より,\refdfn-[外測度]{dfn:outer-measure}$Γ$は空間$X$に含まれるすべての集合に対して定義できるが,
すべての集合に対して$Γ$が良い``体積''になるとは限らない.
例えば$𝔉 \coloneq \{ X, \emptyset \}$として,
$m(X) = 1$,$m(\emptyset) = 0$となるように有限加法的測度$m$を定める%
\footnote{もちろんこの定義がwell-definedであるのは$X \neq \emptyset$のときのみである.}.
そうすれば外測度$Γ$は
\begin{equation*}
    Γ (A)  =  
    \begin{cases}
        1  &  \text{when} \;  A \neq \emptyset  \\
        0  &  \text{when} \;  A   =  \emptyset
    \end{cases}
    , 
    \quad 
    A \subset X
\end{equation*}
である.
したがって,任意の$A, B \subset X$($A, B \neq \emptyset$)に対して,$Γ (A) = Γ (B) = Γ (A \cup B) = 1$であるが,
これは直感的な体積の性質に反する.
では,$Γ (E)$が``良い''体積になる,具体的には
\begin{enumerate}
    \item 任意の集合$E \subset X$に対し,$0 \leq Γ (E) \leq \infty$であり,
    \item 高々可算個の集合$E_1, E_2, \dotsc$に対し,
        $Γ (E_1 \sqcup E_2 \sqcup \dotsb) = Γ(E_1) + Γ(E_2) + \dotsb$が成り立つ
\end{enumerate}
ような集合$E$の条件とは何だろうか.

\begin{definition}
    \label{dfn:measurable-set}
    $Γ$を空間$X$で定義された外測度とする.
    集合$E \subset X$が\word{$Γ$-可測},あるいは単に\word{可測}[か|そく](measurable)\index{かそく@可測}であるとは,
    任意の集合$A \subset X$に対し,
    \begin{equation}
        \label{eq:measurable-set-1}
        Γ (A) = Γ (A \cap E) + Γ (A \cap E^{\complement})
    \end{equation}
    が成り立つことをいう.
    あるいは,任意の$B \subset E$,$C \subset E^{\complement}$に対し,
    \begin{equation}
        \label{eq:measurable-set-2}
        Γ (B \sqcup C) = Γ (B) + Γ (C)
    \end{equation}
    が成り立つことをいう.
\end{definition}

\cref{eq:measurable-set-1}と\cref{eq:measurable-set-2}は,
$B = A \cap E$,$C = A \cap E^{\complement}$と置きなおすことで移り変わる,
同値な定義である.

次に,どのような集合が可測になるかを見る.
まず定理を列挙し,証明はあとでまとめて行う.

\begin{proposition}
    \label{thm:complement-of-measurable-set-is-measurable}
    $Γ$-可測な集合$E$の補集合$E^{\complement}$は$Γ$-可測である.
\end{proposition}


\begin{proposition}
    \label{thm:zero-set-is-measurable}
    $Γ (E) = 0$なら$E$は可測である.
\end{proposition}


\begin{proposition}
    \label{thm:disjoint-union-of-measurable-sets-is-measurable}
    $Γ$-可測な集合系$E_n$($n = 1, 2, \dots$)が互いに交わらない(つまり$i \neq j$なら$E_i \cap E_j = \emptyset$である)とき,
    $S = \bigsqcup_{n=1}^{\infty} E_n$は$Γ$-可測であり,さらに$Γ(S) = \sum_{n=1}^{\infty} Γ(E_n)$である\cite[定理5.3]{ito-lebesgue-1963}.
\end{proposition}


\begin{proposition}
    \label{thm:finite-union-and-intersection-of-measurable-sets-are-measurable}
    $E_i$($i = 1, 2, \dots$)が$Γ$-可測なら,
    $\bigcap_{i=1}^{n} E_i$,$\bigcup_{i=1}^{n} E_i$は可測である.
\end{proposition}


\begin{corollary}
    \label{thm:countable-set-minus-countable-set-is-countable}
    $E, F$が$Γ$-可測なら,$E \setminus F$は$Γ$-可測である.
\end{corollary}



% %%%%%%%%%%%%%%%%%%%%%%%%%%%%%%%%%%%%%%%%
% ここから証明
% %%%%%%%%%%%%%%%%%%%%%%%%%%%%%%%%%%%%%%%%

\begin{proof}[\cref{thm:complement-of-measurable-set-is-measurable}の証明]
    $Γ$-可測の定義(\cref{dfn:measurable-set})より明らかである.
\end{proof}

\begin{proof}[\cref{thm:zero-set-is-measurable}の証明]
    任意の$A \subset X$に対して,
    $Γ$の単調性より$Γ (A \cap E) \leq Γ (E) = 0$であることに注意すると,
    \[
        Γ (A \cap E) + Γ (A \cap E^{\complement})
        \leq Γ (A \cap E^{\complement})
        \leq Γ (A)
    \]
    である.
    $\geq$は$Γ$の劣加法性から明らかであるので,
    $Γ (A \cap E) + Γ (A \cap E^{\complement}) = Γ (A)$.
    よって$E$は可測である.
\end{proof}

\cref{thm:disjoint-union-of-measurable-sets-is-measurable}の証明は技術的に面倒なので省略する.

\cref{thm:finite-union-and-intersection-of-measurable-sets-are-measurable}を示す前に,
以下の\namecref{thm:intersection-of-measurable-sets-is-measurable}を示す.

\begin{lemma}
    \label{thm:intersection-of-measurable-sets-is-measurable}
    集合$E, F \subset X$が$\Gamma$-可測なら,
    $E \cap F$も$\Gamma$-可測である.
\end{lemma}

\begin{proof}[\cref{thm:intersection-of-measurable-sets-is-measurable}の証明]
    $A \subset (E \cap F)$,
    $B \subset (E \cap F)^{\complement} = E^{\complement} \cup F^{\complement}$を任意にとる.
    $B = B_1 + B_2$,$B_1 = B \cap F$,$B_2 = B \cap F^{\complement}$と分割すると,
    $\Gamma$の劣加法性より
    \begin{equation*}
        \begin{split}
            \mkern100mu
            \Gamma (A) + \Gamma (B)
                &\leq \Gamma (A) + \Gamma (B_1) + \Gamma (B_2)
            \shortintertext{$E$は可測で,$A \subset E$,$B_1 \subset E^{\complement}$であるから,}
                &= \Gamma (A \sqcup B_1) + \Gamma (B_2)
            \shortintertext{$F$は可測で,$A, B_1 \subset F$,$B_2 \subset F^{\complement}$だから,}
                &= \Gamma (A \sqcup B_1 \sqcup B_2) \\
                &= \Gamma (A \sqcup B)
        \end{split}
    \end{equation*}
    逆向きの不等号$\geq$は$\Gamma$の劣加法性より明らかであるから,
    $\Gamma (A) + \Gamma (B) = \Gamma (A \sqcup B)$.
\end{proof}


\begin{proof}[\cref{thm:finite-union-and-intersection-of-measurable-sets-are-measurable}の証明]
    $\bigcap$については,
    \cref{thm:intersection-of-measurable-sets-is-measurable}を繰り返し用いればよい.
    $\bigcup$は,
    \begin{equation*}
        \bigcup_{i=1}^{n} E_i  =  \ab( \bigcap_{i=1}^{n} E_i^{\complement} )^{\complement}
    \end{equation*}
    および可測集合の補集合が可測である(\cref{thm:complement-of-measurable-set-is-measurable})ことから従う.
\end{proof}


\begin{proof}[\cref{thm:countable-set-minus-countable-set-is-countable}の証明]
    $E \setminus F = E \cap F^{\complement}$と書けるので,
    \cref{thm:complement-of-measurable-set-is-measurable}と\cref{thm:intersection-of-measurable-sets-is-measurable}より示される.
\end{proof}



\subsection{測度}

\begin{definition}
    \label{dfn:sigma-additive-class}
    $X$を空間とする.
    集合族$𝚺$が以下の条件を満たすとき,
    \word{$\sigma$-加法族}%
    \index{sかほうそく@$\sigma$-加法族}%
    という\footnote{
        \word{完全加法族}[][かんせんかほうそく]という場合や,
        単に\word{加法族}[][かほうそく]ということもある.
    }.
    \begin{enumerate}
        \item $X, \emptyset \in 𝚺$である.
        \item $A \in 𝚺$なら$A^{\complement} \in 𝚺$である.
        \item $𝚺$に属する可算個の集合$A_1, A_2, \dotsc \in 𝚺$に対し,
            \(  \bigcup_{n=1}^{\infty} A_n  \in  𝚺  \).
    \end{enumerate}
    より一般に,
    $𝚺$に属する集合の\refdfn-[和]{dfn:union-of-set}・差・\refdfn[重なり]{dfn:intersection-of-set}を可算無限回とってできた集合は$𝚺$に属する.
\end{definition}

$\sigma$-加法族は\refdfn-[有限加法族]{dfn:finitely-additive-class}である.



\begin{definition}
    \label{dfn:measure}
    $𝚺$を$\sigma$-加法族とする.
    以下の条件を満たす$\mu$を\word{測度}[そく|ど](measure)\index{そくと@測度}という.
    \begin{enumerate}
        \item 任意の$A \in 𝚺$に対し,$0 \leq \mu(A) \leq \infty$である.
        \item \refdfn-[高々可算個]{dfn:at-most-countable}の集合\footnote{$A_i$が可算集合といっているのではない.}%
            $A_1, A_2, \dotsc \in 𝚺$が互いに交わらないとき($i \neq j$なら$A_i \cap A_j = \emptyset$),
            \begin{equation}
                \mu \ab( \bigsqcup_{n=1}^{\infty} A_n ) 
                    = \sum_{n=1}^{\infty} \mu (A_n)
            \end{equation}
    \end{enumerate}
    これらより,
    \begin{enumerate}[resume]
        \item \label{measure:monotonicity}(\word{単調性})
            $A, B \in 𝚺$が$A \subset B$であるなら$\mu(A) \leq \mu(B)$である.
        \item $m(A \setminus B) = \mu (A) - \mu (B)$である.
        \item \label{measure:subadditivity}(\word{劣加法性})
            $A_1, A_2, \dotsc \in 𝔉$(高々可算個)なら,
            \begin{equation*}
                \mu \ab( \bigcup_{i=1}^{n} A_i ) \leq \sum_{i=1}^{n} \mu(A_i)
            \end{equation*}
    \end{enumerate}
    $(X, 𝚺, \mu)$を\word{測度空間}という.
\end{definition}


\begin{proposition}
    空間$X$上の外測度$Γ$に対して,
    \begin{enumerate}
        \item $Γ$-可測である集合全体$𝔐_Γ$は$\sigma$-加法族をなす.
        \item $Γ$は$𝔐_Γ$上の\refdfn-[測度]{dfn:measure}である.
    \end{enumerate}
\end{proposition}

\begin{proof}
    \begin{enumerate}
    \item まず,任意の$A \subset X$に対して$Γ(A) = Γ(\emptyset) + Γ(A \cap X) = Γ(A \cap \emptyset) + Γ(A \cap \emptyset^{\complement})$であるから,
        $\emptyset$は可測,つまり$\emptyset \in 𝔐_Γ$.
        次に,$E \subset X$が可測なら$E^{\complement}$が可測であるのは\cref{eq:measurable-set-1}より明らかである.
        最後に,$E_1, E_2, \dotsc \subset X$が可測であるとすると,
        $E \coloneq \bigcup_{n} E_n \in \symfrak{M}_{\Gamma}$というのは,
        \cref{thm:disjoint-union-of-measurable-sets-is-measurable}の主張そのもの%
        \footnote{
            より正確に言えば,
            $F_1 \coloneq E_1$,
            $F_2 \coloneq E_2 \setminus F_1$,
            $F_3 \coloneq E_3 \setminus (F_1 \sqcup F_2)$,
            ……と$F_n$を定義すると,
            $F_n$は可測であり(なぜなら$A, B$が可測なら$A \setminus B$),
            $E = \bigcup_{n=1}^{\infty} E_n = \bigsqcup_{n=1}^{\infty} F_n$であるから,
            \cref{thm:disjoint-union-of-measurable-sets-is-measurable}より$E$は可測.
        }.
    \item \refdfn-[外測度]{dfn:outer-measure}の公理および\cref{thm:disjoint-union-of-measurable-sets-is-measurable}よりいえる.
    \end{enumerate}
\end{proof}



\begin{definition}
    \label{dfn:Lebesgue-measure}
    \refdfn-[ルベーグ外測度]{dfn:Lebesgue-outer-measure}%
    $\mu^*$の定義域を$𝔐_{\mu^*}$に制限したもの$\mu$を,
    \word{ルベーグ測度}(Lebesgue measure)
    \index{るへえく@ルベーグ!\indexdash そくと@測度|see{測度}}%
    \index{そくと@測度!るへえく@ルベーグ\indexdash}%
    という.
\end{definition}


ルベーグ測度がゼロである領域は考慮しなくてよい.

\begin{definition}
    \label{dfn:almost-everywhere}
    $E \subset X$で
    点$\symbf{x} \in E$を実数(または複素数)にうつす関数$f$が\word{ほとんどいたるところ}(almost everywhere)ゼロであるとは,
    ルベーグ測度$0$の集合$E_0$を除いた各点でゼロになることをいう.
    すなわち
    \begin{equation*}
        f(\symbf{x}) = 0  ,
        \quad  \text{for}   \; \symbf{x} \in E \setminus E_0  
        \quad  \text{where} \; \mu(E_0) = 0
    \end{equation*}
    このとき,
    \begin{align*}
        f(\symbf{x}) &= 0  \quad  \text{a.e.} \  \symbf{x} \in E
        &
        f(\symbf{x}) &= 0  \quad  \text{a.e.}
    \end{align*}
    などと書く.
\end{definition}



位相的性質を入れる.

\begin{proposition}
    \label{thm:minimal-sigma-additive-class}
    $(X, 𝚺, \mu)$を\refdfn-[測度空間]{dfn:measure}とする.
    $X$の任意の部分集合$E$に対し,
    $E$を含む最小の\refdfn[$\sigma$-加法族]{dfn:sigma-additive-class}$𝚺$が存在する.
    % この$B$のことを$𝚺[E]$とかく.
\end{proposition}

\begin{proof}
    $E$を含む$\sigma$-加法族は少なくともひとつ存在する(たとえば$\pset(X)$).
    そこで,$E$を含む$\sigma$-加法族の族$\sequ{𝚺_\lambda}[\lambda \in \Lambda]$をとって
    \begin{equation*}
        𝚺 \coloneq \bigcap_{\lambda \in \Lambda} 𝚺_\lambda
    \end{equation*}
    とすれば,これは$E$を含む最小の$\sigma$-加法族である.
\end{proof}


\begin{definition}
    \label{dfn:Borel-set}
    $N$-次元ユークリッド空間$\symbb{R}^N$に\cref{thm:minimal-sigma-additive-class}を適用すると,
    $\symbb{R}^N$の\refdfn-[開集合]{dfn:open-set}$U \in \topology$すべてを含む\refdfn[$\sigma$-加法族]{dfn:sigma-additive-class}$𝔅$をとることができる.
    この$𝔅$に属する集合を\word{ボレル集合}(Borel set)という.
\end{definition}

$\sigma$-加法族の定義から,
\cref{dfn:Borel-set}の$𝚺$は$\symbb{R}^N$の\refdfn-[閉集合]{dfn:closed-set}(開集合の補集合)すべてを含む.




\end{document}