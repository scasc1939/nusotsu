\documentclass[../sotsu.tex]{subfiles}

\begin{document}


\section{測度}

実ユークリッド空間$\symbb{R}^n$の適当な部分集合においては、``体積''が定義できる。
この体積という概念を一般化したものが\ruby{測度}{そく|ど}である。

この節の議論は\citeauthor{ito-lebesgue-1963}『\citetitle{ito-lebesgue-1963}』による。
同書において集合の和$A + B$とかかれているのは、直和$A \sqcup B$の意味である。
測度について$m(A \overset{直和}{+} B) = m(A) + m(B)$であることを考えれば合理的な表記ともいえるが、
本稿ではベクトル空間の\refdfn[和空間]{dfn:sum-of-vector-space}とまぎらわしいので用いない。


\subsection{ユークリッド空間における体積}

$X = \symbb{R}$を実数の空間とする。
$\symbb{R}$の部分集合であって、
\begin{equation}
    (a, b] \coloneq \Set{  x \in \symbb{R}  \given  a < x \leq b  },
    \qquad -\infty \leq a < b < \infty
\end{equation}
とあらわされるものを\word{区間}[く|かん][くかん]という。

$X = \symbb{R}^n$を実ユークリッド空間とする。
ユークリッド空間の``領域''を




\end{document}