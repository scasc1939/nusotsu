\documentclass[../sotsu.tex]{subfiles}

\begin{document}


\section{商空間とテンソル積}

\subsection{商空間}

ベクトル空間の\refdfn[商集合]{dfn:quotient-set}に対応する.

\begin{definition}
    \label{dfn:quotient-space}
    $V$を$\symbb{K}$上のベクトル空間,$W$を$V$の\refdfn[部分ベクトル空間]{dfn:vector-subspace}とする.
    $V$上の\refdfn-[同値関係]{dfn:equivalence-relation}$\sim$を
    \begin{equation}
        𝒙 \sim 𝒚
            \iff  𝒙 - 𝒚 \in W
    \end{equation}
    と定義し,
    $\sim$による同値類を$\eqclass{𝒙}$とかくことにする.
    商集合を$V/W$とかき,\word{商空間}とよぶ.
\end{definition}

\begin{proposition}
    商空間$V/W$は次のような和とスカラー倍のもとでベクトル空間をなす.
    \begin{subequations}
    \begin{align}
        \eqclass{𝒙} \mathbin{\bar{+}} \eqclass{𝒚} &\coloneq \eqclass{𝒙 + 𝒚}  \\
        c \mathbin{\bar{\scaprod}} \eqclass{𝒙} &\coloneq \eqclass{c𝒙}
    \end{align}
    \end{subequations}
\end{proposition}

\begin{proof}
    $\mathbin{\bar{+}}$と$\mathbin{\bar{\scaprod}}$がwell-definedであることを示せば十分だろう.
    
    まず$𝒙 \sim 𝒙'$,$𝒚 \sim 𝒚'$とする.
    このとき$\symbf{w} \coloneq 𝒙 + 𝒚 \in W$,
    $\symbf{w}' \coloneq 𝒙 + 𝒚 \in W$であるから,
    $\symbf{w} - \symbf{w}' \in W$である.
    つまり$(𝒙 + 𝒚) - (𝒙' + 𝒚') \in W$,
    だから$𝒙 + 𝒚 \sim 𝒙' + 𝒚'$である.

    次に,$𝒙 - 𝒙' \in W$だから,
    $c𝒙 - c𝒙' = c(𝒙 - 𝒚) \in W$.
    よって,$c𝒙 \sim c𝒙'$である.
\end{proof}


\subsection{テンソル積}
\label{sec:tensor-product-of-vector}

$V, W$を$\symbb{K}$上のベクトル空間とする.
$𝒙, 𝒙' \in V$,$𝒚, 𝒚' \in W$,$c \in \symbb{K}$について,
\begin{subequations}
\label{eq:bilinear-map}
\begin{align}
    (𝒙 + 𝒙', \  𝒚) &= (𝒙, 𝒚) + (𝒙', 𝒚)
    &
    (c𝒙, 𝒚) &= c(𝒙, 𝒙)
    \\
    (𝒙, \  𝒚 + \symbf{y'}) &= (𝒙, 𝒚) + (𝒙, 𝒚')
    &
    (𝒙, c𝒚) &= c(𝒙, 𝒙)
\end{align}
\end{subequations}
が成り立つような$(𝒙, 𝒚)$を,\word{テンソル積}\index{てんそるせき@テンソル積}という.
$(𝒙, 𝒚)$の全体もまた,ベクトル空間をなすので,
この空間を$V \otimes W$とかく.
また$(𝒙, 𝒚) \eqcolon 𝒙 \otimes 𝒚$とかく.

なお,\cref{eq:bilinear-map}のような条件を満たす写像$(\bigdot, \bigdot)$を\word{双線形写像}という.

定義を見てもわかりづらいので,
具体例でみてみよう.
$V = V' = W = W' = \symbb{R}^2$とする.
テンソル積は
\begin{equation*}
    \begin{pmatrix}
        a \\ b
    \end{pmatrix}
    \otimes
    \begin{pmatrix}
        c \\ d
    \end{pmatrix}
    \coloneq
    \begin{pmatrix}
        a 
        \begin{pmatrix}
            c \\ d
        \end{pmatrix}
        \\
        b
        \begin{pmatrix}
            c \\ d
        \end{pmatrix}
    \end{pmatrix}
    =
    \begin{pmatrix}
        ac \\ ad \\ bc \\ bd
    \end{pmatrix}
\end{equation*}
とかける.

一方,\( V = V' = W = W' = \text{$2 \times 2$-行列全体の集合} \)とする.
このとき,テンソル積は
\begin{equation*}
    \begin{pmatrix}
        a  & b  \\ c  & d 
    \end{pmatrix}
    \otimes 
    \begin{pmatrix}
        \alpha & \beta \\ \gamma & \delta
    \end{pmatrix}
    =
    \begin{pmatrix}
        a
        \begin{pmatrix}
            \alpha & \beta \\ \gamma & \delta
        \end{pmatrix}
        &
        b
        \begin{pmatrix}
            \alpha & \beta \\ \gamma & \delta
        \end{pmatrix}
        \\
        c
        \begin{pmatrix}
            \alpha & \beta \\ \gamma & \delta
        \end{pmatrix}
        &
        d
        \begin{pmatrix}
            \alpha & \beta \\ \gamma & \delta
        \end{pmatrix}
    \end{pmatrix}
    =
    \begin{pmatrix}
        a \alpha  &  a \beta   &  b \alpha  &  b \beta   \\
        a \gamma  &  a \delta  &  b \gamma  &  b \delta  \\
        c \alpha  &  c \beta   &  d \alpha  &  d \beta   \\
        c \gamma  &  c \delta  &  d \gamma  &  d \delta  
    \end{pmatrix}
\end{equation*}
とかける.



\subsection{線形写像のテンソル積}
\label{sec:tensor-product-of-linear-map}

$V, V', W, W'$をベクトル空間,
$f \colon V \to V'$,$g \colon W \to W'$を線形写像とする.
線形写像$f \otimes g \colon V \otimes W \to V' \otimes W'$を
\begin{equation}
    \label{eq:tensor-product-of-linear-map}
    (f \otimes g)(𝒙 \otimes 𝒚)
        = f(𝒙) \otimes g(𝒚)
\end{equation}
で定めたものを,$f$と$g$の\word{テンソル積}\index{てんそるせき@テンソル積}という%
\footnote{
    \cref{eq:tensor-product-of-linear-map}にある3つの$\otimes$はそれぞれ別の演算である.
    ひとつめは$\otimes \colon \hom(V, V') \times \hom(W, W') \to \hom ( V \otimes W, \  V' \otimes W' )$,
    ふたつめは$\otimes \colon V \times W \to V \otimes W$,
    みっつめは$\otimes \colon V' \times W' \to V' \otimes W'$である.
}.

\end{document}