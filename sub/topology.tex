\documentclass[../sotsu.tex]{subfiles}

\begin{document}

\section{距離空間と位相空間}
\label{sec:topological-space}

\subsection{距離空間}
\label{sec:metric-space}

\begin{definition}[距離]
    \label{dfn:distance}
    $X$を集合とする.写像$d \colon X \times X \to ℝ$が以下を満たすとき,
    これを\word{距離}(distance)あるいは\word{計量}(metric)という.
    \begin{itemize}
        \item 任意の$x, y, z \in X$に対し,
        \begin{enumerate}
            \item \label{dist:positivity} $d(x, y) \geq 0$である.ただし,$d(x, y) = 0$となるのは$x = y$のときに限る.
            \item \label{dist:symmetry} $d(x, y) = d(y, x)$である.
            \item \label{dist:triangle-inequality}$d(x, z) \leq d(x, y) + d(y, z)$である.
        \end{enumerate}
    \end{itemize}
    このとき,$(X, d)$を\word{距離空間}(metric space)\index{きより@距離!くうかん@\indexdash 空間}という.
\end{definition}


\begin{example}
    \label{eg:distance-in-Euclidean-space}
    $X = ℝ^n$(実ユークリッド空間)とする.
    $𝒙, 𝒚 \in ℝ^n$に対して,
    \begin{align*}
        d_1      (𝒙, 𝒚) =        \sum_{i = 1}^{n} \abs{x_i - y_i},
        \quad
        d_2      (𝒙, 𝒚) = \sqrt{ \sum_{i = 1}^{n} \abs{x_i - y_i}^2 },
        \quad
        d_\infty (𝒙, 𝒚) = \max_{1 \leq i \leq n}  \abs{x_i - y_i}^2
    \end{align*}
    と定めると,$d_1, d_2, d_\infty$はいずれも距離である.
    $d_1$を\ltword{タクシー距離}(taxicab distance)または\ltword{マンハッタン距離}(Manhattan distance)\footnote{
        マンハッタンとはアメリカ・ニューヨークの中心地区.
        道路が整然と格子状に並んだマンハッタンを歩くときの移動距離ということで,この名がついたという.
        だとしたら名古屋距離でもよさそうだが,
        あまり用語を気にしてはいけない.
    },
    $d_2$を\ltword{ユークリッド距離}(Euclidean distance)\index{ゆうくりつと@ユークリッド!きより@\indexdash 距離},
    $d_\infty$を\ltword{チェビシェフ距離}(Chebyshev distance)という.
    ふつう距離といったらユークリッド距離$d_2$を指す.
\end{example}


\begin{example}
    $X = C([a, b])$(閉区間$[a, b]$上の実数値関数.ただし$-\infty < a < b < \infty$)とする.
    関数$f, g$の距離を
    \begin{equation*}
        d(f, g) \coloneq \max_{a < x < b} \abs{ f(x) - g(x) }
    \end{equation*}
    で定めれば,これは距離になっている.
\end{example}


\cref{sec:norm-space}で扱うノルムは,
ベクトル空間上の距離である.

\begin{proposition}
    \label{thm:norm-is-distance}
    $V$をノルム空間\refdfn{dfn:norm},$\norm{\bigdot}$を$V$上のノルムとする.
    $𝒙, 𝒚 \in V$に対し,$\norm{𝒙 \tminus 𝒚}$は,$V$上の距離である.
    すなわち,ノルム空間は距離空間である.
\end{proposition}

\begin{proof}
    ノルムの公理(\cref{dfn:norm})から距離の公理が導かれることを示せばよい.
\end{proof}




\subsection{内部と閉包,開集合と閉集合}
\label{sec:interior-and-closure}

この節では,
ヒルベルト空間を扱ううえで必要となる\ltjruby{開}{かい}集合・\ltjruby{閉}{へい}集合という概念を定義する.
なお,ヒルベルト空間において有用であるのは\cref{sec:complete-metric-space}での閉集合の定義であるが,
ややわかりづらいので,まずは$\varepsilon$-近傍を用いて直感にあった定義をする.


\begin{definition}[$\varepsilon$-近傍]
    \label{dfn:epsilon-neighborhood}
    $(X, d)$を距離空間,$\varepsilon > 0$をある実数とする.
    ある点$a \in X$に対し,\ltword{$\varepsilon$-|近|傍}[|きん|ぼう]($\varepsilon$-neighborhood)%
    \index{eきんほう@$\varepsilon$-近傍}%
    を
    \begin{equation}
        \neigh(a; \varepsilon) \coloneq \Set{  x \in X  \given  d(x, a) < \varepsilon  }
    \end{equation}
    で定める\footnote{
        なお,$\varepsilon$の代わりに$\delta$などほかの文字を使ったものを,
        $\delta$-近傍などという.
    }.
\end{definition}

$\varepsilon$-近傍とはいわば,
ある点を中心とする半径$\varepsilon$の球(円)のことである.

距離空間における様々な概念は,
$\varepsilon$-近傍を使って定義される.
まずは内部と閉包を定義しよう.
$(X, d)$を距離空間,$A \subset X$を部分集合とする.

\begin{definition}
    \label{dfn:interior}
    $a \in X$が$A$の\ltjruby{内|点}{ない|てん}\index{ないてん@内点}であるとは,
    ある$\varepsilon > 0$が存在して,
    $\neigh(a; \varepsilon) \subset A$となることをいう.

    $A$の内点全体の集合を,$A$の\ltword{内|部}(interior)\index{ないふ@内部}といい,
    $A^\circ$,$\interr{A}$,$A^i$などとあらわす.
\end{definition}

\begin{definition}
    \label{dfn:closure}
    $a \in X$が$A$の\ltjruby{触|点}{しょく|てん}であるとは,
    任意の$\varepsilon > 0$に対し,
    $\neigh(a; \varepsilon) \cap A \neq \emptyset$となることをいう.

    $A$の触点全体の集合を,$A$の\ltword{閉|包}[へい|ほう](closure)\index{へいほう@閉包}といい,
    $\bar{A}$,$\operatorname{cl}(A)$,$A^a$などとあらわす.
\end{definition}

平たく言えば,
$A$から境界をのぞいたものが内部$\interr{A}$,
$A$に境界を入れたものが閉包$\clsr{A}$である.

例えば,$X = \symbb{R}$,
$A = (a, b] = \Set{ x \in \symbb{R} \given a < x \leq b }$
(ただし$-\infty < a < b < \infty$)とする.
このとき,$\interr{A} = (a, b)$,
$\clsr{A} = [a, b]$である.


\begin{example}
    \label{thm:interior-and-closure-of-emptyset}
    $(X, d)$を距離空間とすると,
    $\interr{X} = X$,$\interr{\emptyset} = \emptyset$,
    $\clsr{X} = X$,$\clsr{\emptyset} = \emptyset$である.
\end{example}



\begin{proposition}
    \label{thm:interior-closure-inclusion}
    距離空間$(X, d)$の部分集合$A \subset X$に対し,
    つねに$\interr{A} \subset A \subset \clsr{A}$である.
\end{proposition}

\begin{proof}
    \begin{enumerate}
        \item まず$\interr{A} \subset A$を示す.
            $a \in \interr{A}$をとると,
            内部の定義より,ある$\varepsilon$-近傍$\neigh(a; \varepsilon)$が存在して,
            \[  \neigh(a; \varepsilon) \subset A  \]
            である.ここで$a \in \neigh(a; \varepsilon)$であるから,
            $a \in A$が示される.
        \item 次に$A \subset \clsr{A}$を示す.
            $a \in A$を任意にとる.
            このとき,任意の$\varepsilon > 0$に対して$a \in \neigh(a; \varepsilon)$だから,
            \[ \neigh(a; \varepsilon) \cap A \neq \emptyset \]
            である.
            よって$a \in \clsr{A}$.\qedhere
    \end{enumerate}
\end{proof}


\begin{proposition}
    \label{thm:inclusion-for-int-and-cl}
    距離空間の部分集合$A, B$について,
    $A \subset B$であるとき,
    以下が成り立つ.
    \begin{enumerate}
        \item $\interr{A} \subset \interr{B}$
        \item $\clsr{A} \subset \clsr{B}$
    \end{enumerate}
\end{proposition}

\begin{proof}
    \begin{enumerate}
        \item $a \in \interr{A}$とする.
            定義より,ある$\varepsilon > 0$が存在して,
            \[  \neigh(a; \varepsilon) \subset A \subset B  \]
            すなわち$a \in \interr{B}$である.
        \item $a \in \clsr{A}$とする.
            定義より,任意の$\varepsilon > 0$に対し,
            \[  \emptyset
                \neq \neigh(a; \varepsilon) \cap A 
                \quad \subset \quad 
                \neigh(a; \varepsilon) \cap B
                  \]
            だから$\neigh(a; \varepsilon) \cap B  \neq  \emptyset$.
            よって$a \in \clsr{B}$である.\qedhere
    \end{enumerate}
\end{proof}


内部と閉包は,次のような関係で結ばれる.

\begin{lemma}
    \label{thm:int-comp-equal-comp-cl}
    距離空間の部分集合$A$について,
    $(\interr{A})^\complement = \clsr{(A^\complement)}$である.
\end{lemma}

\begin{proof}
    次のような同値変形によって示される.
    \begin{equation*}
        \begin{split}
            &\qquad 
            \text{$x \in (\interr{A})^\complement$}
            \\ & \iff
            \text{$x \notin \interr{A}$}
            \\ & \iff
            \text{任意の$\varepsilon > 0$に対し,$\neigh(x; \varepsilon) \not\subset A$}
            \\ & \iff
            \text{任意の$\varepsilon > 0$に対し,$\neigh(x; \varepsilon) \cap A^\complement \neq \emptyset$}
            \\ & \iff
            \text{$x \in \clsr{(A^\complement)}$}
            \qedhere
        \end{split}
    \end{equation*}
\end{proof}

\cref{thm:int-comp-equal-comp-cl}より,
内部に関して成立する性質は,
その補集合をとることで閉包に対する性質として書き換えられる.


\begin{lemma}
    \label{thm:interior-of-epsilon-neighborhood}
    距離空間の$\varepsilon$-近傍$\neigh(a; \varepsilon)$は,
    内部をとっても変わらない.
    つまり,$\interr{\neigh(a; \varepsilon)} = \neigh(a; \varepsilon)$である.
\end{lemma}

\begin{proof}
    \cref{thm:interior-closure-inclusion}より,
    $\neigh(a; \varepsilon) \subset \interr{\neigh(a; \varepsilon)}$を示せば十分である.
    距離を$d$と書く.
    $x \in \neigh(a; \varepsilon)$を任意にとったとき,
    $\delta \coloneq \varepsilon - d(x, a)$とおくと,
    $\neigh(x; \delta) \subset \neigh(a; \varepsilon)$である%
    \footnote{
        実際,任意の$y \in \neigh(x; \delta)$に対し,
        三角不等式より$d(a, y) \leq d(a, x) + d(x, y) < d(a, x) + \delta = \varepsilon$なので,
        $y \in \neigh(a; \varepsilon)$.
    }.
    したがって,
    内部の定義より$x \in \interr{\neigh(a; \varepsilon)}$がいえる.
\end{proof}

上の\namecref{thm:interior-of-epsilon-neighborhood}より,
以下の重要な性質が示される.

\begin{theorem}
    \label{thm:int-of-int-and-cl-of-cl}
    内部と閉包について,次の\ltjruby{冪|等|性}{べき|とう|せい}\index{へきとう@冪等}が成り立つ.
    \begin{enumerate}
        \item $\interr{(\interr{A})} = \interr{A}$
        \item $\clsr{(\clsr{A})} = \clsr{A}$
    \end{enumerate}
\end{theorem}

\begin{proof}
    \cref{thm:interior-closure-inclusion}より,
    $\interr{(\interr{A})} \supset \interr{A}$および$\clsr{(\clsr{A})} \subset \clsr{A}$を示せば十分である.
    \begin{enumerate}
        \item すべての$a \in \interr{A}$について,
            $x \in \interr{(\interr{A})}$となること,
            つまり,ある$\varepsilon > 0$が存在して,
            $\neigh(a; \varepsilon) \subset \interr{A}$となることを示せばよい.
            そこで$a \in \interr{A}$とすれば,
            内部の定義より,ある$\varepsilon > 0$が存在して,$\neigh(a; \varepsilon) \subset A$である.
            内部をとる操作は包含を保ち\refthm{thm:inclusion-for-int-and-cl},
            また$\interr{\neigh(a; \varepsilon)} = \neigh(a; \varepsilon)$\refthm{thm:interior-of-epsilon-neighborhood}なので,
            \[  A \supset \interr{\neigh(a; \varepsilon)} = \neigh(a; \varepsilon)  \]
        \item \cref{thm:int-of-int-and-cl-of-cl}を使うと,
            \[
                \clsr{(\clsr{A})} 
                = (\interr{((\clsr{A})^\complement)})^\complement  
                = (
                    \interr{(
                        (
                            (\interr{A^\complement})^\complement
                        )^\complement
                    )}
                  )^\complement  
                = (\interr{(\interr{A^\complement})})^\complement  
            \]
            である.
            ここで$\interr{(\interr{(A^\complement)})} = \interr{(A^\complement)}$だったから,
            \[
                \clsr{(\clsr{A})} 
                = (\interr{(\interr{A^\complement})})^\complement  
                = (\interr{(A^\complement)})^\complement
                = \clsr{A}
                \qedhere
            \]
    \end{enumerate}
\end{proof}


\begin{proposition}
    \label{thm:int-of-intersection-and-cl-of-union}
    $A, B$を距離空間の部分集合とする.
    \begin{enumerate}
        \item $\interr{(A \cap B)} = \interr{A} \cap \interr{B}$
        \item $\clsr{(A \cup B)} = \clsr{A} \cup \clsr{B}$
    \end{enumerate}
\end{proposition}

なお,$\interr{(A \cup B)} = \interr{A} \cup \interr{B}$や,
$\clsr{(A \cap B)} = \clsr{A} \cap \clsr{B}$は,
一般には成立しない.
例えば距離空間$\symbb{R}$の部分集合$A = [a, b]$,$B = [b, c]$とすれば,
$A \cup B = [a, c]$である.
ところが,
\[  \interr{(A \cup B)} = (a, c)
    \neq
    (a, b) \cup (b, c) 
    = \interr{A} \cup \interr{B}  \]
である.
左辺は$b \in \symbb{R}$を含むが,右辺は含まない.

\begin{proof}
    \begin{enumerate}
        \item 
            \bluehead{$\interr{(A \cap B)} \subset \interr{A} \cap \interr{B}$について}\quad
            $A \cap B \subset A$なので,$\interr{(A \cap B)} \subset \interr{A}$.
            $A \cap B \subset B$なので,$\interr{(A \cap B)} \subset \interr{B}$.
            よって,$\interr{(A \cap B)} \subset \interr{A} \cap \interr{B}$である.

            \bluehead{$\interr{(A \cap B)} \supset \interr{A} \cap \interr{B}$について}\quad
            $x \in \interr{A} \cap \interr{B}$とする.
            $x \in \interr{A}$だから,
            ある$\varepsilon > 0$が存在し,$\neigh(x; \varepsilon) \subset A$.
            また,$x \in \interr{B}$だから,
            ある$\varepsilon' > 0$が存在し,$\neigh(x; \varepsilon') \subset B$.
            そこで,$\varepsilon$と$\varepsilon'$のうち小さいほうを$\bar{\varepsilon}$とすれば,
            $\neigh(x; \bar{\varepsilon}) \subset A \cap B$である.
            よって$x \in \interr{(A \cap B)}$が示された.
        \item 
            \cref{thm:int-comp-equal-comp-cl}を使えば,
            $\interr$に帰着される.
            \begin{equation*}
                \begin{split}
                    \clsr{(A \cup B)}
                    = ( \interr{(A \cup B)^\complement} )^\complement
                    &= ( \interr{( A^\complement \cap B^\complement )} )^\complement
                    \\
                    &= \clsr{ ( ( A^\complement \cap B^\complement )^\complement ) }
                    = \clsr{ ( A \cup B ) }
                \qedhere
                \end{split}
            \end{equation*}
    \end{enumerate}
\end{proof}






\begin{definition}[開集合と閉集合]
    \label{dfn:open-set-and-closed-set}
    \label{dfn:open-set}
    \label{dfn:closed-set}
    $(X, d)$を距離空間とする.
    \begin{enumerate}
        \item 
            $A \subset X$の内部$\interr{A}$が$A$自身と一致するとき,
            $A$を\ltword{開|集|合}[かい|しゅう|ごう](open set)\index{かいしゆうこう@開集合}
            (あるいは\ltjruby{開}{かい}である)という.
        \item 
            $A \subset X$の閉包$\clsr{A}$が$A$自身と一致するとき,
            $A$を\ltword{閉|集|合}[へい|しゅう|ごう](closed set)\index{へいしゆうこう@閉集合}
            (あるいは\ltjruby{閉}{へい}である) という.
    \end{enumerate}
\end{definition}


\begin{example}[自明な開集合・閉集合]
    \label{thm:emptyset-is-open-and-closed}
    $(X, d)$を距離空間とすると.
    \cref{thm:interior-and-closure-of-emptyset}より,
    $X$と$\emptyset$はいずれも開かつ閉である.
\end{example}


\begin{example}
    \cref{thm:int-of-int-and-cl-of-cl}(冪等性)より,
    距離空間の任意の部分集合$A$について,
    その内部$\interr{A}$は開集合であり,
    閉包$\clsr{A}$は閉集合であることがわかる.
\end{example}


\begin{example}
    \label{thm:epsilon-neighborhood-is-open}
    \cref{thm:interior-of-epsilon-neighborhood}より,
    $\varepsilon$-近傍は開集合である.
\end{example}


次の\namecref{thm:complement-of-open-set-is-closed}で示される開集合と閉集合の関係は,
距離空間を扱ううえで極めて重要である.
距離空間を一般化した位相空間では,
これを開集合・閉集合の公理とする.

\begin{theorem}
    \label{thm:complement-of-open-set-is-closed}
    開集合の補集合は閉集合である.
    逆に,閉集合の補集合は開集合である.
\end{theorem}

\begin{proof}
    \cref{thm:int-comp-equal-comp-cl}を使うと,
    \begin{equation*}
        \begin{split}
            \quad \text{$A$が開集合}
            & \iff
            \interr{A} = A
            \\ & \iff
            (\interr{A})^\complement = A^\complement
            \\ & \iff
            \clsr{(A^\complement)} = A^\complement
            \iff
            \text{$A^\complement$は閉集合}
            \qedhere
        \end{split}
    \end{equation*}
\end{proof}

\begin{theorem}
    \label{thm:union-and-intersection-of-open-sets}
    開集合について,以下が成り立つ.
    \begin{enumerate}
        \item $U_1, \dots, U_n$が開集合であれば,
            $U_1 \cap \dots \cap U_n$も開集合である.
        \item $U_\lambda$($\lambda \in \Lambda$:添字)が開集合であれば,
            $\bigcup_{\lambda \in \Lambda} U_\lambda$は開集合である.
    \end{enumerate}
\end{theorem}


\begin{corollary}
    \label{thm:union-and-intersection-of-closed-sets}
    閉集合について,以下が成り立つ.
    \begin{enumerate}
    \item $A_1, \dots, A_n$が閉集合であれば,
        $A_1 \cup \dots \cup A_n$も閉集合である.
    \item $A_\lambda$($\lambda \in \Lambda$:添字)が開集合であれば,
        $\bigcap_{\lambda \in \Lambda} A_\lambda$は開集合である.
    \end{enumerate}
\end{corollary}


\begin{proof}[\cref{thm:union-and-intersection-of-open-sets}の証明]
    \begin{enumerate}
        \item $n = 2$の場合に示せば十分である.
            $U_1, U_2$を開集合,
            $U \coloneq U_1 \cap U_2$とすれば,
            \cref{thm:int-of-intersection-and-cl-of-union}より,
            \begin{equation*}
                \interr{U} = \interr{U_1} \cap \interr{U_2}
                    = U_1 \cap U_2
                    = U
            \end{equation*}
            だから,$U$は開集合である.
        \item $U \coloneq \bigcup_{\lambda \in \Lambda} U_\lambda$とする.
            $U \subset \interr{U}$を示せばよい.
            そこで$x \in U$とすると,定義より,
            ある$\lambda_0 \in \Lambda$が存在して,
            $x \in U_{\lambda_0}$となる.
            $U_{\lambda_0}$は開集合なので,$x \in \interr{U_{\lambda_0}}$.
            よってある$\varepsilon > 0$が存在し,
            $\neigh(x; \varepsilon) \subset U_{\lambda_0}$となる.
            したがって$\neigh(x; \varepsilon) \subset U$であるから,
            $x \in \interr{U}$である.
            \qedhere
    \end{enumerate}
\end{proof}


\begin{proof}[\cref{thm:union-and-intersection-of-closed-sets}の証明]
    $U$が開集合であれば$U^\complement$が閉集合であることに加えて
    \begin{enumerate}
        \item $(U_1 \cap \dots \cap U_n)^\complement = U_1^\complement \cup \dots \cup U_n^\complement$
        \item $\ab( \bigcup_{\lambda \in \Lambda} U_\lambda )^\complement = \bigcap_{\lambda \in \Lambda} U_\lambda^\complement $
    \end{enumerate}
    (\cref{thm:de-Morgan-of-n-sets,thm:de-Morgan-of-family-of-sets})を使えば\cref{thm:union-and-intersection-of-open-sets}に帰着される.
\end{proof}


\begin{proposition}
    \label{thm:interior-and-closure-by-all-open-and-closed-sets}
    距離空間の部分集合$A$について,
    \begin{enumerate}
        \item $A$の\refdfn-[内部]{dfn:interior}$\interr{A}$は,
            $A$に含まれる\emph{開}集合すべての和集合に一致する.
        \item $A$の\refdfn-[閉包]{dfn:closure}$\clsr{A}$は,
            $A$を含む\emph{閉}集合すべての共通部分に一致する.
    \end{enumerate}
\end{proposition}

\begin{proof}
    \begin{enumerate}
        \item $A$に含まれる開集合の族を$\sequ{U_\lambda}[\lambda \in \Lambda]$とおく.
            示したいのは
                \[  \interr{A} = \bigcup_{\lambda \in \Lambda} U_\lambda  \]
        
            \bluehead{$\subset$であること}\quad
            $x \in \interr{A}$をとる.
            $x$は$A$の内点であるから,ある$\varepsilon > 0$が存在して,
            $\neigh(x; \varepsilon) \subset A$となる.
            また\cref{thm:epsilon-neighborhood-is-open}より$\varepsilon$-近傍は開であるから,
            $\neigh(x; \varepsilon)$は$A$に含まれる開集合である.
            したがって,ある$\lambda_0 \in \Lambda$に対して$U_{\lambda_0} = \neigh(x; \varepsilon)$となる.
            よって,$x \in U_{\lambda_0} \subset \bigcup_{\lambda \in \Lambda} U_\lambda$が成り立つ.

            \bluehead{$\supset$であること}\quad
            $U_\lambda$の定義より$A \supset \bigcup_{\lambda \in \Lambda} U_\lambda$である.
            ここで内部が包含を保つこと\refthm{thm:inclusion-for-int-and-cl},
            開集合系の任意の和集合は開であること\refthm{thm:union-and-intersection-of-open-sets}より,
            \[  \interr{A} 
                    \supset \interr{\ab(\bigcup_{\lambda \in \Lambda} U_\lambda)}
                    = \bigcup_{\lambda \in \Lambda} U_\lambda  \]

        \item 略.\qedhere
    \end{enumerate}
\end{proof}


\begin{corollary}
    \label{thm:interior-or-closure-is-max-or-min}
    距離空間の部分集合$A$について,
    \begin{enumerate}
        \item $A$の\refdfn-[内部]{dfn:interior}$\interr{A}$は,
            $A$に含まれる\emph{開}集合のうち最大である.
        \item $A$の\refdfn-[閉包]{dfn:closure}$\clsr{A}$は,
            $A$を含む\emph{閉}集合のうち最小である.
    \end{enumerate}
\end{corollary}

\begin{proof}
    \cref{thm:interior-and-closure-by-all-open-and-closed-sets}より明らかである.
\end{proof}


\begin{definition}[近傍]
    \label{dfn:neighborhood}
    距離空間の点$x$を内点として含む集合,
    $x$の\ltword{近|傍}[きん|ぼう](neighborhood)という.
    また,近傍のうち\refdfn-[開]{dfn:open-set}であるものを,
    \ltword{開|近|傍}[かい|きん|ぼう]という\footnote{
        開近傍のことを近傍とよぶこともある.
    }.

    $x$の近傍全体の集合(集合族)を\ltword{近傍系}といい,
    $\Neigh(x)$と書くことにする.
\end{definition}

\refdfn-[$\varepsilon$-近傍]{dfn:epsilon-neighborhood}は近傍のひとつである.





\subsection{収束列とコーシー列}

ヒルベルト空間を扱ううえで避けて通れないのが,「\ltjruby{完|備}{かん|び}」という概念である.
完備性を定義するための準備として,ある値に収束する数列について議論する.

これから$X$の点列といった場合,
自然数で順序付けられた$X$の可算部分集合,つまり
$\sequ{x_i}[i \in \symbb{N}] = (x_1, x_2, \dots, x_i, \dotsc)$であり,$i \in \symbb{N}$,$x_i \in X$であるものをいうことにする.

集合$X$上の点列を,自然数全体から$X$への写像$f \colon \symbb{N} \to X$ととらえることもできる.
この立場からは,$X$上の点列全体の集合を$X^{\symbb{N}}$とかく.

\begin{definition}[収束列]
    \label{dfn:convergent-sequence}
    $(X, d)$を距離空間とする.
    $X$の点列$\sequ{x_i}[i \in \symbb{N}]$が$x \in X$に\ltword{収束する}(converge)とは,
    任意の$\varepsilon > 0$に対し,ある$N \in \symbb{N}$が存在して,
    任意の$n > N$に対し,$d(x_n, x) < \varepsilon$となることをいう.
    
    このときの$x$のことを\ltword{極|限}[きょく|げん](limit)\index{きよくけん@極限}という\cite{uchida-set-2020}.
    \ltjruby{収|束|先}{しゅう|そく|さき}ということもある.

    また,$X$の点列$\sequ{x_i}[i \in \symbb{N}]$が%
    \ltword{収束列}(convergent sequence)%
    \index{しゆうそくれつ@収束列}%
    であるとは,
    この点列がある$x \in X$に収束することをいう.
\end{definition}

記号$\lim$を用いると,点列$\sequ{x_i}$が$x$に収束することを
\[  \lim_{n \to \infty} d(x_n, x) = 0  \]
とかける.

収束列の極限はただ一つに定まる.
実際,$X$の点列$(x_i)$の極限が$x$と$x'$の2つあったとすると,
\begin{equation*}
    \begin{split}
        d(x, x') &\leq d(x, x_n) + d(x_n, x)  \qquad \text{(三角不等式)}  \\
            &= d(x_n, x) + d(x_n, x)  
            \xrightarrow{n \to \infty} 0
    \end{split}
\end{equation*}
であるので,\refdfn-[距離の公理]{dfn:distance}より$x = x'$である.


\begin{definition}[コーシー列]
    \label{dfn:Cauchy-sequence}
    $(X, d)$を距離空間とする.
    $X$の点列$(x_1, x_2, \dotsc)$が
    \ltword{コーシー列}(Cauchy sequence)%
    \index{こおしいれつ@コーシー列}%
    であるとは,
    任意の$\varepsilon > 0$に対し,ある$N \in \symbb{N}$が存在して,
    任意の$n, m > N$に対し,$ d(x_n, x_m) < \varepsilon $となることをいう\cite{uchida-set-2020}.
\end{definition}

記号$\lim$を用いると,コーシー列の定義は
$\lim_{n, m \to \infty} d(x_n, x_m) = 0$とかける.

収束列の定義とコーシー列の定義はよく似ているが,前者は収束先$x \in X$の存在を要請しているのに対し,後者はそうでない.
収束列とコーシー列には,次のような関係がある.
\begin{theorem}
    \label{thm:convergent-is-Cauchy}
    距離空間の収束列は常にコーシー列である.
\end{theorem}

\begin{proof}
    $(x_i)$を収束列,その極限を$x \in X$とする.
    定義より,任意の$\delta > 0$に対し,ある$N > \symbb{N}$が存在して,
    任意の$n > N$に対し,$d(x_n, x) < \delta$である.
    すると,任意の$m, n > N$に対し,
    \begin{equation*}
        \begin{split}
            d(x_m, x_n) &\leq d(x_m, x) + d(x, x_n)  \qquad \text{(三角不等式)}  \\
                &= d(x_m, x) + d(x_n, x)  \\
                &< 2\delta
        \end{split}
    \end{equation*}
    である.
    $\delta \coloneq \varepsilon/2$とおけば,コーシー列の条件\refdfn{dfn:Cauchy-sequence}が成立する.
\end{proof}

距離空間において,すべての収束列はコーシー列であるが,その逆は必ずしも成立しない.

たとえば,$\symbb{Q}$の点列$(1, 1.4, 1.41, 1.414, \dotsc)$は明らかにコーシー列である.
しかし,この点列の収束先は$\sqrt{2} \notin \symbb{Q}$であり,$\symbb{Q}$の収束列でない.

また,開区間$(0, 1) \subset ℝ$の点列$(0.1, 0.01, 0.001, 0.0001, \dotsc)$は明らかにコーシー列である.
しかし,この点列の収束先は$0 \notin (0, 1)$であり,$(0, 1)$の収束列でない.


\subsection{距離空間の完備性}
\label{sec:complete-metric-space}


\begin{definition}[稠密]
    \label{dfn:dense}
    $(X, d)$を距離空間とする.
    $A \subset B$をみたす$X$の2つの部分集合について,
    $B \subset \clsr{A}$であるとき,
    $A$は$B$において\ltword{稠|密}[ちゅう|みつ](dense)\index{ちゆうみつ@稠密}であるという\footnote{
        稠密を「ち\emph{ょ}うみつ」と読む人もいる.
    }\cite{iwanami-functional}.

    特に$A$の閉包が距離空間に一致するとき($\clsr{A} = X$),
    「$X$において」を省略して「$A$は稠密である」ということがある.
\end{definition}


\begin{proposition}
    距離空間の部分集合$A$が$B$において稠密である必要十分条件は,
    任意の$x \in B$の任意の近傍が$A$と共通部分を持つことである\cite{iwanami-functional}.
\end{proposition}


\begin{proof}
    \bluehead{必要性}\quad 
    $x \in B$の近傍を$N$とおく.
    $x$は$N$の内点であるから,
    $\neigh(x; \varepsilon) \subset N$となる$\varepsilon > 0$が存在する.
    一方,$A$が$B$において稠密という仮定から,
    任意の$x \in B$は$A$の閉包に含まれる.
    すなわち任意の$\varepsilon > 0$に対して,
    $\neigh(x; \varepsilon) \cap A \neq \emptyset$である.
    したがって,$N \cap A \neq \emptyset$がいえる.

    \bluehead{十分性}\quad 
    任意の$x \in B$に対して,その$\varepsilon$-近傍$\neigh(x; \varepsilon)$を考えると,
    仮定より任意の$\varepsilon > 0$に対して$\neigh(x; \varepsilon) \cap A \neq 0$だから,
    $x \in \clsr{A}$.
\end{proof}


\begin{definition}
    \label{dfn:separable}
    距離空間が\refdfn-[稠密]{dfn:dense}な\refdfn-[高々可算集合]{dfn:at-most-countable}をもつとき,
    \ltword{可|分}[か|ぶん](separable)\index{かふん@可分}であるという.
\end{definition}


\begin{definition}
    \label{dfn:complete-metric-space}
    距離空間$(X, d)$において,任意のコーシー列が収束列であるとき,\ltword{完|備}[かん|び](complete)であるという.
    このとき,$(X, d)$のことを\ltword{完備距離空間}(complete metric space)という.
\end{definition}


\begin{proposition}[実数の完備性]
    \label{thm:completeness-of-real-numbers}
    実数全体の集合$ℝ$は完備である.
\end{proposition}

\begin{proof}
    実数の連続性から従う.
    % あるいは,実数を有理数列のうちコーシー列であるものの\refdfn-[同値類]{dfn:equivalence-class}として公理的に構成することでわかる\cite[付録]{uchida-set-2020}.
\end{proof}


\begin{corollary}
    \label{thm:completeness-of-complex-numbers}
    複素数全体の集合$\symbb{C}$は完備である.
\end{corollary}



\subsubsection*{点列による開集合と閉集合の定義}

\cref{sec:interior-and-closure}で扱った開集合・閉集合は,
点列を使って定義することもできる.
むしろ,ヒルベルト空間においては点列による定義のほうが有用である.
そこで,まず点列の収束を用いて\refdfn-[閉包]{dfn:closure}を定義しよう.

\begin{definition}[点列による閉包の定義]
    \label{dfn:closure-by-sequence}
    $(X, d)$を距離空間,$A \subset X$を部分集合とする.
    $A$に属するすべての収束列の極限からなる集合を
    $A$の\ltword{閉|包}[へい|ほう](closure)\index{へいほう@閉包}といい,
    $\clsr{A}$とかく.
    つまり
    \begin{equation}
        \clsr{A} \coloneq \Set{ x \in X  \given  \exists (x_i) \subset A \text{ s.t. } \lim_{n \to \infty} d(x_i, x) = 0 }
    \end{equation}
\end{definition}

\cref{dfn:closure}で与えた閉包の定義と一致する.
$A$の閉包というのは,$A$に適当な元を加えて\refdfn-[閉集合]{dfn:closed-set}にしたもの,
あるいは$A$のコーシー列が収束するようにしたものと考えることができる.

すでに\cref{thm:inclusion-for-int-and-cl}で示したことだが,
$A$の閉包が$A$を含むことを,
点列のことばで証明しよう.

\begin{corollary}
    任意の$A \subset X$の閉包$\clsr{A}$について,
    $A \subset \clsr{A}$である.
\end{corollary}

\begin{proof}
    $x \in A$とすると,
    $A$の点列$(x, x, x, \dotsc)$は明らかに$x$に収束するので$x \in \clsr{A}$である.
\end{proof}



\subsection{位相空間}

この節では,距離空間を一般化した概念である位相空間(topological space)について述べる.
集合に「位相を入れる」ことで,
距離空間(特に$\varepsilon$-近傍)で定義した開集合・閉集合といったものを,
距離を使わずに定義できるようになる.

なお,ここで扱う位相空間(topological space)は,
物理学でよく扱う位相空間(phase space)とは関係のない概念である.

まずは距離空間における位相について定義しておこう.
位相の定義には同値なものがいくつかある.
例えば,はじめに``\refdfn-[開集合]{dfn:open-set}''を定義するか,
あるいはその代わりに``\refdfn-[閉集合]{dfn:closed-set}''を定義することがよくある.
ほかに``\refdfn-[内部]{dfn:interior}'',
``\refdfn-[閉包]{dfn:closure}'',
もしくは``近傍系''のいずれかを定義することによっても,
まったく同じ位相が得られる.
ここでは開集合を使って位相を定義することにする.

\begin{definition}[距離位相]
    \label{dfn:topology-by-distance}
    距離空間$(X, d)$において,すべての開集合$A \in \pset(X)$
    (つまり$A \subset_{\text{open}} X$)
    を集めた集合族を,
    (距離$d$から導かれる)\ltword{位|相}[い|そう](topology)\index{いそう@位相}といい,
    $\topology_d$であらわす.
\end{definition}

ここでの定義では,「位相=開集合系」である.
一般の位相は,
距離空間の場合に調べた開集合の性質(\cref{thm:emptyset-is-open-and-closed},
\cref{thm:union-and-intersection-of-open-sets})を抽象化することで定義される.

\begin{definition}[位相]
    \label{dfn:topology}
    \label{dfn:topological-space}
    $X$を集合とする.
    冪集合$\pset(X)$の部分集合であって,
    以下の条件を満たす$\topology \subset \pset(X)$を\ltword{位相}(topology)という.
    \begin{enumerate}
        \item $X, \emptyset \in \topology$
        \item $U_1, \dots, U_n \in \topology$なら,$U_1 \cap \dots \cap U_n \in \topology$
        \item すべての添字$\lambda$について$U_\lambda \in \topology$なら,$\bigcup_{\lambda} U_\lambda \in \topology$
    \end{enumerate}
    このとき,$(X, \topology)$を%
    \ltword{位|相|空|間}[い|そう|くう|かん](topological space)%
    \index{いそう@位相!くうかん@\indexdash 空間}%
    という.
\end{definition}


\begin{itemize}
    \item \cref{dfn:topology}より,
    \begin{enumerate}
        \item $A$が開集合であるとは,$A \in \topology$であることと定める.
    \end{enumerate}

    \item \cref{thm:complement-of-open-set-is-closed}より,
    \begin{enumerate}[resume]
        \item $A$が閉集合であるとは,$A^\complement \in \topology$であることと定める.
    \end{enumerate}

    \item \cref{thm:interior-and-closure-by-all-open-and-closed-sets}より,
    \begin{enumerate}[resume]
        \item $A$の内部$\interr{A}$を「$A$に含まれる開集合すべての和集合」,
        \item $A$の閉包$\clsr{A}$を「$A$を含む閉集合すべての共通部分」と定める.
    \end{enumerate}
\end{itemize}




\subsection{位相空間の連続写像}




\subsection{コンパクト集合}

% % 一般の位相空間における話
% %%%%%%%%%%%%%%%%%%%%

% \begin{definition}[被覆]
%     $X$を集合,$A \subset X$をその部分集合とする.
%     $\symfrak{G} \subset \pset(X)$が$A \subset \bigcup \symfrak{G}$をみたすとき,
%     $\symfrak{G}$は$A$の\ltword{被|覆}[ひ|ふく]\index{ひふく@被覆}である(あるいは$\symfrak{G}$は$A$を被覆する)という\cite[\S 22]{uchida-set-2020}.
% \end{definition}

% 特に,位相空間$(X, \topology)$の部分集合$A \subset X$において,
% 開集合の族$\symfrak{G} \in \topology$が$A$を被覆するとき,
% $\symfrak{G}$は$A$の\ltword{開被覆}\index{かいひふく@開被覆}であるという.

% \begin{definition}[コンパクト集合]
%     \label{dfn:compact-set}
%     位相空間$(X, \topology)$の部分集合$A \subset X$について,
%     $A$の任意の開被覆$\symfrak{G} \subset \topology$に対し,
%     有限個の開集合$O_1, \dots, O_n \in \symfrak{G}$が存在して
%     \begin{equation*}
%         A \subset O_1 \cup \dots \cup O_n
%     \end{equation*}
%     とかけるとき,$A$は\ltword{コンパクト}(compact)\index{こんはくと@コンパクト}であるという.

%     $X$自身がコンパクトであるとき,$(X, \topology)$を\ltword{コンパクト空間}\index{こんはくと@コンパクト!くうかん@\indexdash 空間}という\cite[\S 22]{uchida-set-2020}.
% \end{definition}

% 一般の位相空間におけるコンパクト集合の定義はわかりづらいが,距離空間においては簡明な言いかえが存在する.

% \begin{proposition}
%     \label{thm:compact-set-is-bounded-closed-set}
%     距離空間$(X, d)$におけるコンパクト集合は,有界な閉集合である\cite[\S 22]{uchida-set-2020}.
% \end{proposition}

% \begin{theorem}
%     \label{thm:Heine-Borel-theorem}
%     通常の位相に関して(つまり通常の開集合・閉集合を考えた場合),
%     $ℝ$の任意の閉区間$[a, b]$はコンパクトである(\ltword{ハイネ--ボレルの被覆定理}(Heine--Borel theorem)\index{はいねほれるのひふくていり@ハイネ--ボレルの被覆定理})\cite[\S 22]{uchida-set-2020}.
% \end{theorem}


% 距離空間の話
% %%%%%%%%%%%%%%%%%%%%

\begin{definition}
    \label{dfn:compact-set}
    距離空間$(X, d)$において,有界な閉集合$A \subset X$を\word{コンパクト集合}(compact set)という.
\end{definition}





\end{document}