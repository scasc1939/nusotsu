\documentclass[../sotsu.tex]{subfiles}

\begin{document}

\section{距離空間と位相空間}

\subsection{距離空間}

\begin{definition}[距離]
    \label{dfn:distance}
    $X$を集合とする.写像$d \colon X \times X \to \symbb{R}$が以下を満たすとき,
    \word{距離}(distance)あるいは\word{計量}(metric)という.
    \begin{itemize}
        \item 任意の$x, y, z \in X$に対し,
        \begin{enumerate}
            \item $d(x, y) \geq 0$である.ただし,$d(x, y) = 0$となるのは$x = y$のときに限る.
            \item $d(x, y) = d(y, x)$である.
            \item $d(x, z) \leq d(x, y) + d(y, z)$である.
        \end{enumerate}
    \end{itemize}
    $(X, d)$を\word{距離空間}[][きよりくうかん](metric space)という.
\end{definition}


\begin{example}
    $X = \symbb{R}^n$(実ユークリッド空間)とする.
    $\symbf{x}, \symbf{y} \in \symbb{R}^n$に対して,
    $d_2 (\symbf{x}, \symbf{y}) = \sqrt{ \sum_{i = 1}^{n} \abs{x_i - y_i}^2 }$と定めると,これは距離である.

    特に,実数全体の集合$\symbb{R}$は,距離$d(x, y) \coloneq \abs{x - y}$に対して距離空間になる.
\end{example}

\begin{proposition}
    \label{thm:norm-is-distance}
    $V$をノルム空間\refdfn{dfn:norm},$\norm{\bigdot}$を$V$上のノルムとする.
    $\symbf{x}, \symbf{y} \in V$に対し,$\norm{\symbf{x} \tminus \symbf{y}}$は,$V$上の距離である.
    すなわち,ノルム空間は距離空間である.
\end{proposition}

\begin{proof}
    ノルムの定義より従う.
\end{proof}


\subsection{内部と閉包}


\begin{definition}
    $(X, d)$を距離空間、$\varepsilon > 0$をある実数とする。
    ある点$a \in X$に対し、\word{$\varepsilon$-近傍}%
    \index{eきんほう@$\varepsilon$-近傍}%
    を
    \begin{equation}
        \neigh(a; \varepsilon) \coloneq \Set{  x \in X  \given  d(x, a) < \varepsilon  }
    \end{equation}
    で定める。
\end{definition}

$(X, d)$を距離空間、$A \subset X$を部分集合とする。

\begin{definition}
    $a \in X$が$A$の内点であるとは、
    ある$\varepsilon > 0$が存在して、
    $\neigh(a; \varepsilon) \subset A$となることをいう。

    $A$の内点全体の集合を、$A$の\word{内部}[][ないぶ](interior)といい、
    $\operatorname{int}(A)$、$A^i$などとあらわす。
\end{definition}

\begin{definition}
    $a \in X$が$A$の触点であるとは、
    任意の$\varepsilon > 0$に対し、
    $\neigh(a; \varepsilon) \cap A \neq \emptyset$となることをいう。

    $A$の触点全体の集合を、$A$の\word{閉包}[へい|ほう][へいほう](closure)といい、
    $\bar{A}$、$\operatorname{cl}(A)$、$A^a$などとあらわす。
\end{definition}


\begin{definition}[開集合と閉集合]
    \label{dfn:open-set-and-closed-set}
    $(X, d)$を距離空間とする。
    \begin{enumerate}
        \item $A \subset X$の内部が$A$自身と一致するとき、
            $A$は\word{開集合}[][かいしゆうこう](open set)であるとは、
        \item $A \subset X$の$A$の閉包$\cl{A}$が$A$と一致するとき、
            $A$は\word{閉集合}[][へいしゆうこう](closed set)である
    \end{enumerate}
\end{definition}



\subsection{収束列とコーシー列}

ヒルベルト空間を扱ううえで避けて通れないのが,「\ruby{完備}{かん|び}」という概念である.
完備性を定義するための準備として,ある値に収束する数列について議論する.

これから$X$の点列といった場合,
$\symbb{N}$で順序付けられた$X$の加算部分集合,つまり
$(x_1, x_2, \dots, x_i, \dotsc)$であり,$i \in \symbb{N}$,$x_i \in X$であるものをいうことにする.

\begin{definition}[収束列]
    \label{dfn:convergent-sequence}
    $(X, d)$を距離空間とする.
    $X$の点列$(x_1, x_2, \dotsc)$が$x \in X$に\word{収束する}(converge)とは,
    任意の$\varepsilon > 0$に対し,ある$N \in \symbb{N}$が存在して,
    任意の$n > N$に対し,$d(x_n, x) < \varepsilon$となることをいう.
    
    このときの$x$のことを\word{極限}[きょく|げん](limit)という.
    \ruby{収束先}{しゅう|そく|さき}ということもある.

    また,$X$の点列$(x_1, x_2, \dotsc)$が\word{収束列}(convergent sequence)であるとは,
    点列がある$x \in X$に収束することをいう.
\end{definition}

記号$\lim$を用いると,点列$(x_i)$が$x$に収束することを
\[  \lim_{n \to \infty} d(x_n, x) = 0  \]
とかける.

収束列の極限はただ一つに定まる.
実際,$X$の点列$(x_i)$の極限が$x$と$x'$の2つあったとすると,
\begin{equation*}
    \begin{split}
        d(x, x') &\leq d(x, x_n) + d(x_n, x)  \qquad \text{(三角不等式)}  \\
            &= d(x_n, x) + d(x_n, x)  
            \xrightarrow{n \to \infty} 0
    \end{split}
\end{equation*}
であるので,$x = x'$である\refdfn{dfn:distance}.


\begin{definition}[コーシー列]
    \label{dfn:Cauchy-sequence}
    $(X, d)$を距離空間とする.
    $X$の点列$(x_1, x_2, \dotsc)$が\word{コーシー列}(Cauchy sequence)であるとは,
    任意の$\varepsilon > 0$に対し,ある$N \in \symbb{N}$が存在して,
    任意の$n, m > N$に対し,$ d(x_n, x_m) < \varepsilon $となることをいう.
\end{definition}

記号$\lim$を用いると,コーシー列の定義は
$\lim_{n, m \to \infty} d(x_n, x_m) = 0$とかける.

収束列の定義とコーシー列の定義はよく似ているが,前者は収束先$x \in X$の存在を要請しているのに対し,後者はそうでない.
収束列とコーシー列には,次のような関係がある.
\begin{theorem}
    \label{thm:convergent-is-Cauchy}
    距離空間の収束列は常にコーシー列である.
\end{theorem}

\begin{proof}
    $(x_i)$を収束列,その極限を$x \in X$とする.
    定義より,任意の$\delta > 0$に対し,ある$N > \symbb{N}$が存在して,
    任意の$n > N$に対し,$d(x_n, x) < \delta$である.
    すると,任意の$m, n > N$に対し,
    \begin{equation*}
        \begin{split}
            d(x_m, x_n) &\leq d(x_m, x) + d(x, x_n)  \qquad \text{(三角不等式)}  \\
                &= d(x_m, x) + d(x_n, x)  \\
                &< 2\delta
        \end{split}
    \end{equation*}
    である.
    $\delta \coloneq \varepsilon/2$とおけば,コーシー列の条件\refdfn{dfn:Cauchy-sequence}が成立する.
\end{proof}

距離空間において,すべての収束列はコーシー列であるが,その逆は必ずしも成立しない.
コーシー列が収束列でない例をいくつか挙げる.

\begin{example}
    $\symbb{Q}$の点列$(1, 1.4, 1.41, 1.414, \dotsc)$は明らかにコーシー列である.
    しかし,この点列の収束先は$\sqrt{2} \notin \symbb{Q}$であり,$\symbb{Q}$の収束列でない.
\end{example}

\begin{example}
    開区間$(0, 1) \subset \symbb{R}$の点列$(0.1, 0.01, 0.001, 0.0001, \dotsc)$は明らかにコーシー列である.
    しかし,この点列の収束先は$0 \notin (0, 1)$であり,$(0, 1)$の収束列でない.
\end{example}


\subsection{距離空間の完備性}

\begin{definition}
    距離空間$(X, d)$において,任意のコーシー列が収束列であるとき,\word{完備}[かん|び](complete)であるという.
    このとき,$(X, d)$のことを\word{完備距離空間}(complete metric space)という.
\end{definition}

\begin{proposition}[実数の完備性]
    実数全体の集合$\symbb{R}$は完備である.
\end{proposition}

\begin{proof}
    実数を,有理数列のうちコーシー列であるものの同値類\refdfn{dfn:equivalence-class}として公理的に構成することでわかる.
\end{proof}

\begin{definition}[閉包]
    \label{dfn:closure-by-sequence}
    $(X, d)$を距離空間,$\symcal{D} \subset X$を部分集合とする.
    $\symcal{D}$に属するすべての収束列の極限からなる集合を
    $\symcal{D}$の\word{閉包}[へい|ほう](closure)といい,$\cl{\symcal{D}}$とかく.
    つまり
    \begin{equation}
        \cl{\symcal{D}} \coloneq \Set{ x \in X  \given  \exists (x_i) \subset \symcal{D} \text{ s.t. } \lim_{n \to \infty} d(x_i, x) = 0 }
    \end{equation}
\end{definition}

$x \in \symcal{D}$であることと$x \in \cl{\symcal{D}}$であることは,定義上は全く関係ない.

\begin{corollary}
    任意の$\symcal{D} \subset X$の閉包$\cl{\symcal{D}}$について,
    $\symcal{D} \subset \cl{\symcal{D}}$である.
\end{corollary}

\begin{proof}
    $x \in \symcal{D}$とする.
    $\symcal{D}$の点列$(x, x, x, \dotsc)$は明らかに$x$に収束するので$x \in \cl{\symcal{D}}$である.
\end{proof}

次に稠密を定義する.

\begin{definition}[稠密]
    $(X, d)$を距離空間とする.部分集合$A \subset X$の閉包$\cl{A}$が$X$に一致するとき,
    $A$は$X$の\word{稠密}[ちゅう|みつ][ちゆうみつ](dense)な部分集合であるという.
\end{definition}

\begin{proposition}
    $A$が$X$の稠密な部分集合である必要十分条件は,
    $x \in X$の任意の近傍が$A$と共通部分を持つことである.
\end{proposition}

\begin{proof}
    
\end{proof}


\subsection{位相空間}

\begin{definition}
    \label{dfn:topology}
    距離空間$(X, d)$において、すべての開集合$A \in \pset(X)$
    (つまり$A \subset_{\text{open}} X$)
    を集めた集合族を\word{位相}[い|そう][いそう](topology)といい、
    $\topology_d$であらわす。
\end{definition}

定義より$\topology_d \subset \pset(X)$である。

\begin{definition}
    \label{dfn:topological-space}
    距離空間$(X, d)$において位相$\topology_d$が定められているとき、
    $(X, \topology_d)$を\word{位相空間}[い|そう|くう|かん][いそうくうかん]という。
\end{definition}

実は位相空間というのは、距離空間よりも広い概念である。
しかし、ここでは距離空間の別の見方が位相空間であるとしておく。



\subsection{コンパクト集合}

\begin{definition}[被覆]
    $X$を集合、$A \subset X$をその部分集合とする。
    $\symfrak{G} \subset \pset(X)$が$A \subset \bigcup \symfrak{G}$をみたすとき、
    $\symfrak{G}$は$A$の\word{被覆}[ひ|ふく][ひふく]である(あるいは$\symfrak{G}$は$A$を被覆する)という\cite[\S 22]{uchida-set-2020}。
\end{definition}

特に、位相空間$(X, \topology)$の部分集合$A \subset X$において、
開集合の族$\symfrak{G} \in \topology$が$A$を被覆するとき、
$\symfrak{G}$は$A$の\word{開被覆}[][かいひふく]であるという。

\begin{definition}[コンパクト集合]
    位相空間$(X, \topology)$の部分集合$A \subset X$について、
    $A$の任意の開被覆$\symfrak{G} \subset \topology$に対し、
    有限個の開集合$O_1, \dots, O_n \in \symfrak{G}$が存在して
    \begin{equation*}
        A \subset O_1 \cup \dots \cup O_n
    \end{equation*}
    とかけるとき、$A$は\word{コンパクト}[][こんはくと](compact)であるという。

    $X$自身がコンパクトであるとき、$(X, \topology)$を\word{コンパクト空間}[][こんはくとくうかん]という\cite[\S 22]{uchida-set-2020}。
\end{definition}

\begin{proposition}
    \label{thm:compact-set-is-bounded-closed-set}
    距離空間$(X, d)$におけるコンパクト集合は、有界な閉集合である\cite[\S 22]{uchida-set-2020}。
\end{proposition}

\begin{theorem}
    \label{thm:Heine-Borel-theorem}
    通常の位相に関して(つまり通常の開集合・閉集合を考えた場合)、
    $\symbb{R}$の任意の閉区間$[a, b]$はコンパクトである(\word{ハイネ--ボレルの被覆定理}[][はいねほれるのひふくていり](Heine--Borel theorem))\cite[\S 22]{uchida-set-2020}。
\end{theorem}


\end{document}