\documentclass[../sotsu.tex]{subfiles}

\begin{document}


\section{行列}
\label{sec:matrix}

この章では行列の性質を述べる.

\subsection{行列の定義}

用語の定義にとどめる.

\begin{definition}[行列]
    $𝕂$を\refdfn[体]{dfn:field}とする.
    $𝕂$の元(すなわち数字)を縦横に$m \times n$個ならべたものを\word{行列}[ぎょう|れつ][きようれつ](matrix)という.
    \begin{equation}
        \label{eq:matrix-example}
        A \coloneq 
        \begin{pmatrix}
            a_{11}  &  a_{12}  &  \cdots  &  a_{1n}  \\
            a_{21}  &  a_{22}  &  \cdots  &  a_{2n}  \\
            \vdots  &  \vdots  &  \ddots  &  \vdots  \\
            a_{m1}  &  a_{m2}  &  \cdots  &  a_{mn}
        \end{pmatrix}
        , \qquad 
        a_{ij} \in 𝕂
    \end{equation}
    縦が\word{列}[れつ],横が\word{行}[ぎょう]である.

    \cref{eq:matrix-example}において,
    上から$i$番目,左から$j$番目にある数字$a_{ij}$のことを,
    $A$の$ij$-成分という.
    これを$(A)_{ij}$とかくこともある.
\end{definition}

\cref{eq:matrix-example}の行列$A$を,$(a_{ij})_{1 \leq i \leq m, \  1 \leq j \leq n}$とあらわすこともある.

\begin{definition}
    \label{dfn:set-of-matrix}
    $𝕂$-係数の$m \times n$-行列全体の集合を$M_{mn}(𝕂)$とあらわす.
\end{definition}

名前がついている特別な行列がいくつかある.

\begin{definition}[正方行列]
    行と列の数が同じ行列,すなわち$n \times n$-行列を\word{正方行列}[せい|ほう|ぎょう|れつ]という.
    特に大きさを明示する場合,$n$-\ruby{次}{じ}正方行列という.
\end{definition}

\begin{definition}
    正方行列のなかで,$ii$-成分が$1$,それ以外が$0$である行列
    \begin{equation}
        I \coloneq 
        \begin{pmatrix}
            1  &  0  &  \cdots  &  0  \\
            0  &  1  &  \cdots  &  0  \\
            \vdots & \vdots & \ddots & \vdots \\
            0  &  0  &  \cdots  &  1  
        \end{pmatrix}
    \end{equation}
    を\word{単位行列}[たん|い|ぎょう|れつ]という.
\end{definition}

\begin{definition}
    \label{dfn:inverse-of-matrix}
    $A$を$n$-次正方行列とする.
    \begin{equation}
        A A^{-1} = A^{-1} A = I
    \end{equation}
    となるような$n$-次正方行列$A^{-1}$が存在するとき,
    $A^{-1}$を$A$の\word{逆行列}という.
\end{definition}


\begin{definition}
    行列の和・倍・積を以下のように定義する.
    \begin{enumerate}
        \item 2つの$m \times n$行列$A = (a_{ij})$,$B = (b_{ij})$の和は,それぞれの成分の和
            \begin{equation}
                A + B \coloneq (a_{ij} + b_{ij})_{1 \leq i \leq m, \  1 \leq j \leq n}
            \end{equation}
        \item 行列$A = (a_{ij})$の$c$($\in 𝕂$)倍は,それぞれの成分の$c$倍
            \begin{equation}
                c \cdotp A \coloneq (c \cdotp a_{ij})_{1 \leq i \leq m, \  1 \leq j \leq n}
            \end{equation}
        \item $m \times n$行列$A = (a_{ij})$,$n \times l$行列$B = (b_{jk})$の積は
            \begin{equation}
                A \cdotp B \coloneq \ab( \sum_{1 \leq j \leq n} a_{ij} \cdotp b_{jk} )_{1 \leq i \leq m, \  1 \leq k \leq l}
            \end{equation}
    \end{enumerate}
\end{definition}


\begin{proposition}
    行列の演算についての性質を挙げる.
    $p \times q$行列$A$,$q \times r$行列$B$,$r \times s$行列$C$とする.
    \begin{itemize}
        \item $M_{mn}(𝕂)$は和について\refdfn[アーベル群]{dfn:group}をなす.すなわち
        \begin{enumerate}
            \item (結合律)$(A + B) + C = A + (B + C)$であり,
            \item (単位元)$A + O = O + A = A$であり,
            \item (逆元)$A + (-A) = (-A) + A$となる$-A$が存在し,
            \item (交換律)$A + B = B + A$である.
        \end{enumerate}
        \item $M_{mn}(𝕂)$は積について結合的であり,また単位元をもつ.
            すなわち,
        \begin{enumerate}
            \item (結合律)$(AB)C = A(BC)$であり,
            \item (単位元)$AI = IA = A$である.
        \end{enumerate}
        \item 
    \end{itemize}
\end{proposition}


\subsection{数ベクトル}

$m \times 1$-行列のことを数ベクトルという.

\begin{definition}
    \label{dfn:canonical-basis-of-coordinate-space}
    $\symbb{K}^m$の元のなかで,
    第$i$列の成分のみが$1$,
    ほかの成分が$0$であるもの
    \begin{equation*}
        \symbf{e}_i
            \coloneq
            \begin{pmatrix}
                0  \\  \vdots  \\  1  \\  \vdots  \\  0
            \end{pmatrix}
    \end{equation*}
    をあつめたもの
    \begin{equation*}
        \{  \symbf{e}_1, \dots, \symbf{e}_m  \}
    \end{equation*}
    を,$\symbb{K}^m$の\word{標準基底}(canonical basis)という.
\end{definition}



\subsection{置換}

\begin{definition}
    \label{dfn:permutation}
    集合$\{ 1, 2, \dots, n \}$から自身への\refdfn[写像]{dfn:map}のうち\refdfn-[全単射]{dfn:bijection}であるものを,
    $n$-文字の\word{置換}[ち|かん][ちかん]という.
    それぞれの元を$1 \mapsto \sigma(1), \  2 \mapsto \sigma(2), \  \dotsc$とうつすものを
    \begin{equation*}
        \sigma = 
        \begin{pmatrix}
                   1  &        2  & \cdots &        n  \\
            \sigma(1) & \sigma(2) & \cdots & \sigma(n)
        \end{pmatrix}
    \end{equation*}
    とかく.
    $n$文字の置換$\sigma$全体がなす集合を$\symfrak{S}_n$とかく.
\end{definition}

\begin{definition}
    \label{dfn:identity-permutation}
    置換$\sigma$のうち\refdfn-[恒等写像]{dfn:identity-map}であるものを\word{恒等置換}という.   
\end{definition}

\begin{definition}
    \label{dfn:inverse-of-permutation}
    置換$\sigma$の\refdfn-[逆写像]{dfn:inverse-map}$\sigma^{-1}$を\word{逆置換}という.
\end{definition}

任意の置換$\sigma \in \symfrak{S}_n$は全単射であるから,
逆置換$\sigma^{-1} \in \symfrak{S}_n$が必ず存在する.

\begin{definition}[置換の積]
    \label{dfn:product-of-permutation}
    $\sigma, \tau \in \symfrak{S}_n$とする.
    2つの置換の積$\sigma\tau$を,写像の合成$\sigma \circ \tau$として定める.
\end{definition}

なお,一般に$\sigma \tau \neq \tau \sigma$である.

この記法を用いると,
任意の置換$\sigma \in \symfrak{S}_n$とその逆置換$\sigma^{-1}$について,
$\sigma \sigma^{-1} = \sigma^{-1} \sigma = \text{恒等置換}$である.

\begin{proposition}
    $n$-文字の置換全体の集合$\symfrak{S}_n$は,
    \cref{dfn:product-of-permutation}で定義される積に対して群をなす.
\end{proposition}

\begin{proof}
    結合律は\refdfn[写像の合成が結合的である]{thm:map-is-associative}ことよりいえる.
    単位元は\refdfn-[恒等置換]{dfn:identity-permutation},
    逆元は\refdfn-[逆置換]{dfn:inverse-of-permutation}である.
\end{proof}

\begin{definition}
    \label{dfn:transposition}
    置換$\sigma \in \symfrak{S}_n$のうち,
    $\sigma(i) \neq i$となる$i$が2つだけのもの,
    つまり2つの数字を入れ替えるだけの置換を\word{互換}[ご|かん][こかん](transposition)という.
\end{definition}

\begin{proposition}
    任意の置換は互換の積としてあらわされる.
\end{proposition}

置換に対し,互換の積によるあらわしかたは一意ではない.

\begin{proposition}
    \label{thm:parity-of-permutation}
    任意の置換を互換に分解したとき,含まれる互換の数の偶奇は積のあらわしかたによらない.
\end{proposition}

証明は複雑なため省略するが,
これによって次に示す$\sgn$の定義を正当化される.

\begin{definition}
    \label{dfn:sign-of-permutation}
    置換$\sigma$を互換による積であらわしたとき,
    互換が奇数個なら$\sgn(\sigma) = -1$,偶数個なら$\sgn(\sigma) = +1$となるように
    $\sgn \colon \symfrak{S}_n \to \{ +1, -1 \}$を定義する.
    これを置換の\word{符号}[ふ|ごう][ふこう](sign)という.
\end{definition}

$\sgn(\sigma)$のことを$(-1)^\sigma$や$(-)^\sigma$とかくこともある.

\begin{corollary}
    \label{thm:sign-of-product-of-permutation}
    \label{thm:sign-of-inverse-of-permutation}
    任意の置換$\sigma, \tau \in \symfrak{S}_n$について,
    $\sgn(\sigma \tau) = \sgn \sigma \cdotp \sgn \tau$である.
    特に,$\sgn(\sigma^{-1}) = \sgn(\sigma)$である.
\end{corollary}

\begin{proof}
    前半は$\sigma$,$\tau$をそれぞれ互換の積であらわせば明らか.
    後半は$\sgn(\sigma) \cdotp \sgn(\sigma^{-1}) = \sgn(\sigma \cdotp \sigma^{-1}) = \sgn(\text{恒等置換}) = 1$であることからわかる.
\end{proof}



\subsection{行列式}

\begin{definition}
    $n$-次の正方行列$A = (a_{ij})$に対して,
    \begin{equation}
        \det A  \coloneq  \sum_{\sigma \in \symfrak{S}_n} \sgn(\sigma) \, a_{1, \sigma(1)} a_{2, \sigma(2)} \dotsm a_{n, \sigma(n)} 
    \end{equation}
    で定義される写像$\det \colon M_{nn}(𝕂) \to 𝕂$を\word{行列式}[ぎょう|れつ|しき][きようれつしき](determinant)という.
\end{definition}

\begin{proposition}
    $A$を$n$-次正方行列とする.
    行列式について,以下が成り立つ.
    \begin{enumerate}
        \item $\det A = \det \tp{A}$
        \item $\det (cA) = c^n \det A$
    \end{enumerate}
\end{proposition}

\begin{proof}
    \begin{enumerate}
        \item $\tp{A}$の行列式は
            \[  \det \tp{A} = \sum_{\sigma \in \symfrak{S}_n} \sgn(\sigma) \, a_{\sigma(1), 1} a_{\sigma(2), 2} \dotsm a_{\sigma(n), n}  \]
            であるが,$\sigma(i)$が連番になるように$a_{\sigma(i), i}$の順番を入れ替えることで,
            \[  \det \tp{A} = \sum_{\sigma^{-1} \in \symfrak{S}_n} \sgn(\sigma^{-1}) \, a_{1, \sigma^{-1}(1)} a_{2, \sigma^{-1}(2)} \dotsm a_{n, \sigma^{-1}(n)}  \]
            と書き換えられ,
            \cref{thm:sign-of-inverse-of-permutation}より$\sgn(\sigma) = \sgn(\sigma^{-1})$であることから従う.
        \item 行列式の定義より明らかである.
    \end{enumerate}
\end{proof}

\begin{definition}
    \label{dfn:invertible-matrix}
    正方行列$A$に対し,\refdfn-[逆行列]{dfn:inverse-of-matrix}$A^{-1}$が存在するとき,
    $A$を\word{正則行列}[せい|そく|ぎょう|れつ][せいそくきようれつ](invertible matrix)という.
\end{definition}

\begin{proposition}
    以下は同値.
    \begin{enumerate}
        \item $A$は正則行列.
        \item $A$の行列式$\det A = 0$.
        \item $A 𝒙 = \symbf{0}$の解は自明な解$𝒙 = \symbf{0}$のみである.
    \end{enumerate}
\end{proposition}



\end{document}