\documentclass[../sotsu.tex]{subfiles}

\begin{document}


\section{ヒルベルト空間}

\subsection{ヒルベルト空間}

ベクトル空間における完備性について考えてみよう.

$V$をノルム空間\refdfn{dfn:norm}とする.
$𝒙, 𝒚 \in V$の距離\refdfn{dfn:distance}は,
$d(𝒙, 𝒚) \coloneq \norm{𝒙 - 𝒚}$で定義できる(\cref{thm:norm-is-distance}).
すると,$V$の収束列\refdfn{dfn:convergent-sequence}とコーシー列\refdfn{dfn:Cauchy-sequence}は,以下のように定義できる.
\begin{itemize}
    \item $V$の点列$\sequ{𝒗_i}$が収束列であるとは,$\sequ{𝒗_i}$がある$𝒗 \in V$に収束する,
        つまり$\lim_{n \to \infty} x_n = x \in V$となることをいう.
    \item $V$の点列$\sequ{𝒗_i}$がコーシー列であるとは,
        $\lim_{m, n \to \infty} \norm{𝒗_m - 𝒗_n} = 0$であることをいう.
\end{itemize}
これらを用いて,ノルム空間の完備性を定義することができる.

\begin{definition}
    ノルム空間$V$が完備であるとは,$V$のコーシー列が収束列であることをいう.
\end{definition}

\begin{definition}[バナッハ空間]
    \label{dfn:Banach-space}
    完備なノルム空間を\word{バナッハ空間}(Banach space)という.
\end{definition}

内積空間\refdfn{dfn:inner-product}では,内積から導かれるノルムが存在するのであった.
このノルムを用いれば,完備な内積空間というものを定義することができる.

\begin{definition}[ヒルベルト空間]
    \label{dfn:Hilbert-space}
    完備な内積空間を\word{ヒルベルト空間}(Hilbert space)という.
\end{definition}

\begin{proposition}
    実ユークリッド空間$ℝ^n$および複素ユークリッド空間$ℂ^n$はヒルベルト空間である.
\end{proposition}



\subsection{射影定理}

\refdfn[距離]{dfn:distance}%
$d(𝒙, 𝒚) = \norm{𝒙 - 𝒚} = \sqrt{\iparen{𝒙 - 𝒚, \  𝒙 - 𝒚}}$である.

\begin{lemma}
    \label{thm:lemma-of-projection-theorem}
    ヒルベルト空間$\symcal{H}$とその\refdfn-[閉]{dfn:closed-set}部分ベクトル空間$\symcal{D}$について,
    ベクトル$𝒗 \in \symcal{H}$と$\symcal{D}$の距離を
    \begin{equation*}
        \symcal{d}(𝒗, \symcal{D}) 
            \coloneq \inf_{𝒘 \in \symcal{D}} d(𝒗, 𝒘)
            = \inf_{𝒘 \in \symcal{D}} \norm{𝒗 - 𝒘}
    \end{equation*}
    で定める.
    このとき,
    $\symcal{d}(𝒗, \symcal{D}) = d(𝒗, 𝒘)$となるような$𝒘 \in \symcal{D}$がただひとつ存在する.
\end{lemma}

\begin{proof}
    $\symcal{d}(𝒗, \symcal{D})$の定義より,
    $d(𝒗 - 𝒘_n) \xrightarrow{n \to \infty} \symcal{d}(𝒗, \symcal{D})$となるような$\symcal{D}$の点列$\sequ{𝒘_n}[n \in \symbb{N}]$がとれる.
    すると,
    \begin{equation*}
        \begin{split}
            d^2 (𝒘_m, 𝒘_n)
                &= \norm{(𝒘_m + 𝒗) + (𝒗 - 𝒘_n)}^2  \\
                &= 2 \ab( \norm{𝒘_m + 𝒗}^2 + \norm{𝒗 + 𝒘_n}^2 ) 
                    - \norm{(𝒘_m - 𝒗) - (𝒗 - 𝒘_n)}^2
                    \qquad \text{(\refthm[中線定理]{thm:parallelogram-law})}  \\
                &= 2 \ab( \norm{𝒗 - 𝒘_m}^2 + \norm{𝒗 - 𝒘_n}^2 )
                    - 4 \norm*{ \frac{𝒘_m + 𝒘_n}{2} - 𝒗 }^2  \\
                &\leq 2 \ab( \norm{𝒗 - 𝒘_m}^2 + \norm{𝒗 - 𝒘_n}^2 )
                    - 4 \symcal{d}(𝒗, \symcal{D})  \\
                &\xrightarrow{m, n \to \infty}
                    2 \ab\big( \symcal{d}(𝒗, \symcal{D}) + \symcal{d}(𝒗, \symcal{D}) )
                    - 4 \symcal{d}(𝒗, \symcal{D})
                = 0
        \end{split}
    \end{equation*}
    最後の不等号では,$(𝒘_m + 𝒘_n)/2 \in \symcal{D}$であること,
    および$\symcal{d}(𝒗, \symcal{D})$の最小性を使った.

    これにより,$\sequ{𝒘_n}[n \in \symbb{N}]$は\refdfn[コーシー列]{dfn:Cauchy-sequence}であり,
    ヒルベルト空間の完備性からこれはある$𝒘 \in \symcal{D}$に収束する.

    一意性を示す.$𝒘, 𝒘' \in W$がともに$\symcal{d}(𝒗, \symcal{D}) = d(𝒗, 𝒘^{(\prime)})$を満たすとして,
    $d^2 (𝒘, 𝒘')$を上の方法で計算すると,これは$0$になるので,$𝒘 = 𝒘'$である.
\end{proof}


\begin{theorem}[射影定理]
    \label{thm:projection-theorem}
    ヒルベルト空間$\symcal{H}$に対し,閉部分ベクトル空間$\symcal{D} \subset \symcal{H}$とその\refdfn[直交補空間]{dfn:orthogonal-compliment}$\symcal{D}^\perp$をとる.
    このとき,任意のベクトル$\symbf{z}$に対し,
    $\symbf{x} \in \symcal{D}$と$\symbf{y} \in \symcal{D}^\perp$がそれぞれただひとつ存在し,
    $\symcal{z} = \symcal{x} + \symcal{y}$とかける.
    すなわち$\symcal{H} = \symcal{D} \oplus \symcal{D}^\perp$と\refdfn[直和分解]{dfn:direct-sum-of-vector-space}できる.
\end{theorem}

\begin{proof}
    $\symbf{z} \in \symcal{H}$とする.
    \cref{thm:lemma-of-projection-theorem}の方法で$\symbf{x} \in \symcal{D}$をとる.
    このとき$\symbf{y} \coloneq \symbf{z} - \symbf{x} \in \symcal{D}^\perp$を示せばよい.
    任意の$\lambda \in \symbb{R}$および$\symbf{a} \in \symcal{D}$に対し,
    \begin{equation*}
        \begin{split}
            0 \leq \norm{\symbf{y}}^2
                &= \norm{\symbf{z} - \symbf{x}}^2  \\
                &\leq \norm{\symbf{z} - (\symbf{x} - \lambda \symbf{a})}^2
                    \quad \text{($\symbf{x} - \lambda \symbf{a} \in \symcal{D}$,
                                $\norm{\symbf{z} - \symbf{x}}$の最小性)} \\
                &= \norm{\symbf{y} + \lambda \symbf{a}}^2  \\
                &= \norm{\symbf{y}}^2 + 2 \lambda \Real \iparen{\symbf{y}, \symbf{a}} + \lambda^2 \norm{\symbf{a}}^2  \\
        \end{split}
    \end{equation*}
    したがって,
    \(  2 \lambda \Real \iparen{\symbf{y}, \symbf{a}} + \lambda^2 \norm{\symbf{a}}^2  \geq  0  \)
    であるが,$\lambda \in \symbb{R}$は任意なので,$\Real \iparen{\symbf{y}, \symbf{a}} = 0$.
    また,$\lambda \to \iu \lambda$として同様に考えると,$\Imaginary \iparen{\symbf{y}, \symbf{a}} = 0$.
    よって$\iparen{\symbf{y}, \symbf{a}} = 0$である.
    $\symbf{a} \in \symcal{D}$は任意だったので,$\symbf{y} \in \symcal{D}$が示された.
\end{proof}



\subsection{ヒルベルト空間の基底}

一般のベクトル空間には\cref{thm:basis-exist}より基底が存在する.
また,有限次元の内積空間では,\cref{thm:Gram-Schmidt-process}によって正規直交基底を構成できる.

それでは,無限次元の内積空間に正規直交基底が存在するのか?

それを考えるためには,
まずベクトルの無限級数というものを考えないといけない.

\begin{definition}
    ヒルベルト空間$\symcal{H}$に属する可算無限個のベクトルの列$\sequ{\symbf{u}_n}[n \in \symbb{N}]$と
    スカラー列$\sequ{c_n}[n \in \symbb{N}]$をとる.
    \begin{equation*}
        \lim_{N \to \infty} \norm*{𝒗 - \sum_{n=1}^{N} c_n \symbf{u}_n} = 0
    \end{equation*}
    となるような$𝒗 \in \symcal{H}$が存在するとき,
    級数$\sum_{n=1}^{\infty} c_n \symbf{u}_n$は$𝒗$に収束するといい,
    \begin{equation*}
        \sum_{n=1}^{\infty} c_n \symbf{u}_n = 𝒗
    \end{equation*}
    とかく\cite[\S 1.5]{arai-1997}.
\end{definition}

\begin{definition}
    $\symcal{H}$をヒルベルト空間とする.
    $\symcal{H}$に属する\refdfn[可算無限個]{dfn:countably-infinite}のベクトルの組$(\symbf{u}_n)_{n=1}^{\infty}$が,
    任意のベクトル$𝒗 \in \symcal{H}$を
    \begin{equation}
        𝒗 = \sum_{n=1}^{\infty} c_n \symbf{u}_n,
        \quad c_n \coloneq \iparen{𝒗, \symbf{u}_n}
    \end{equation}
    とあらわせるとき,
    $\sequ{\symbf{u}_n}[n \in \symbb{N}]$を%
    \word{完全正規直交系}(complete orthonormal system)%
    \index{かんせんせいきちよつこうけい@完全正規直交系}%
    という.
\end{definition}

$\sequ{\symbf{u}_\lambda}[\lambda \in \Lambda]$がベクトル空間$V$の\refdfn[基底]{dfn:basis}であるとは,
任意の$𝒗 \in V$が\emph{有限個を除きゼロである}スカラーの組$\sequ{c_\lambda}[\lambda \in \Lambda]$($c_\lambda \in \symbb{K}$)を用いて
\[  𝒗 = \sum_{\lambda \in \Lambda} c_\lambda \symbf{u}_\lambda  \]
と有限和でかけることだった.
しかし,完全正規直交系の場合は無限和(高々可算個のベクトルの線形結合)も許される.

なお,$\symcal{H}$が有限次元であるときは,
正規直交基底を指して完全正規直交系という.


\begin{theorem}
    \label{thm:basis-of-Hilbert-space}
    ヒルベルト空間$\symcal{H}$が\refdfn[可分]{dfn:separable}であれば,
    $\symcal{H}$は正規直交基底をもつ.
\end{theorem}



\end{document}