\documentclass[../sotsu.tex]{subfiles}

\begin{document}


\section{ヒルベルト空間}

ユークリッド空間$\symbb{R}^N, \symbb{C}^N$では,
任意のベクトル列$\sequ{\symbf{v}_n}$が$\norm{\symbf{v}_m - \symbf{v}_n} \to 0$($n, m \to \infty$)を満たすとき,
この列はあるベクトル$\symbf{v}_* \in \symbb{R}^N \text{ or } \symbb{C}^N$に\refdfn[収束]{dfn:convergence-of-vector-sequence}する.
自明のように思われるかもしれないが,
一般のベクトル空間(ノルム空間,内積空間)ではこの事実が成り立たない.



\subsection{ヒルベルト空間とは何か}

この節では,ベクトル空間における完備性について考えてみよう.

$V$を\refdfn-[ノルム空間]{dfn:norm}とする.
$𝒙, 𝒚 \in V$の\refdfn[距離]{dfn:distance}は,
$d(𝒙, 𝒚) \coloneq \norm{𝒙 - 𝒚}$で定義できるのであった(\cref{thm:norm-is-distance}).
距離空間で定義したように,$V$の\refdfn-[収束列]{dfn:convergent-sequence}と\refdfn-[コーシー列]{dfn:Cauchy-sequence}は,以下のように定義できる.
\begin{itemize}
    \item $V$の点列$\sequ{𝒗_n}$が収束列であるとは,$\sequ{𝒗_n}$がある$𝒗_* \in V$に収束する,
        つまり$\lim_{n \to \infty} 𝒗_n = 𝒗_* \in V$となることをいう.
    \item $V$の点列$\sequ{𝒗_n}$がコーシー列であるとは,
        $\lim_{m, n \to \infty} \norm{𝒗_m - 𝒗_n} = 0$であることをいう.
\end{itemize}
これらを用いて,ノルム空間の完備性を定義することができる.

\begin{definition}
    ノルム空間$V$が完備であるとは,$V$のコーシー列が収束列であることをいう.
\end{definition}

\begin{definition}[バナッハ空間]
    \label{dfn:Banach-space}
    完備なノルム空間を%
    \word{バナッハ空間}(Banach space)%
    \index{はなつはくうかん@バナッハ空間}%
    という\footnote{
        \person{Stefan Banach}{1892}{1945}.
        ポーランドの数学者.
        関数解析学の創始者のひとり.
        \cite{nipponica}
    }.
\end{definition}

\refdfn[内積空間]{dfn:inner-product}では,内積から導かれるノルムが存在するのであった.
このノルムを用いれば,完備な内積空間というものを定義することができる.

\begin{definition}[ヒルベルト空間]
    \label{dfn:Hilbert-space}
    完備な内積空間を\word{ヒルベルト空間}(Hilbert space)という\footnote{
        \person{David Hilbert}{1862}{1943}.
        ドイツの数学者.
        数学基礎論,代数学,幾何学,解析学,さらには理論物理学まで広い分野で功績を残し,
        20世紀前半でもっとも偉大な数学者のひとりとされる.
        ヒルベルト23の問題でも有名.
        \cite{nipponica}\cite{iwanami-sugakujiten}
    }.
\end{definition}

\begin{proposition}
    実ユークリッド空間$ℝ^n$および複素ユークリッド空間$ℂ^n$はヒルベルト空間である.
\end{proposition}





\begin{theorem}
    \label{thm:subspace-of-norm-space-is-complete}
    ノルム空間の有限次元部分空間は完備である\cite[定理1.2の系]{iwanami-functional}.
    特に,有限次元ノルム空間は完備である.
\end{theorem}

\begin{proof}
    概略だけ述べる.
    $\symbb{C}$上のノルム空間$V$の部分空間を$W$とし,
    次元$n \coloneq \dim W$とおく.
    
    $W$のコーシー列を$\sequ{\symbf{v}^{(k)}}[k \in \symbb{N}]$とおくと,
    任意の$k$に対して$\symbf{v}^{(k)} \in W$であるから,
    $W$の基底$\symbf{u}_1, \dots, \symbf{u}_n$を用いて
    \[  \symbf{v}^{(k)} = c^{(k)}_1 \symbf{u}^{(k)}_1 + \dots + c^{(k)}_n \symbf{u}^{(k)}_n  \]
    とかける.
    $\sequ{\symbf{v}^{(k)}}[k]$がコーシー列であることを使うと,
    各$i$に対して$\sequ{c^{(k)}_i}[k]$がコーシー列であることを示すことができる%
    \footnote{
        ここでは,$W$がノルム空間の有限次元部分空間であるときに,
        $W$上のノルムがすべて同値であることを使う\cite[定理1.2]{iwanami-functional}.
    }.
    すると,$\symbb{C}$の完備性\refthm{thm:completeness-of-copmlex-numbers}からこれは,
    $c^{(k)}_i \to c^*_i \in \symbb{C}$に収束する.
    そこで$\symbf{v}_* \coloneq c^*_1 \symbf{u}_1 + \dots + c^*_1 \symbf{u}_n \in W$とおくと,
    \begin{equation*}
        \begin{split}
            \norm{ \symbf{v}^{(k)} - \symbf{v}^* }
                = \norm*{ \sum_{i = 1}^{n} \ab\big(c^{(k)}_i - c^*_i) \symbf{u}_i }
                \leq \sum_{i=1}^{n} \, \abs[\big]{ c^{(k)}_i - c^*_i } \, \norm{ \symbf{u}_i }
                \xrightarrow{k \to \infty} 0
        \end{split}
    \end{equation*}
    より,$\symbf{v}^{(k)} \to \symbf{v}^* \in W$である.
\end{proof}

したがって,
有限次元のノルム空間では,完備性について検討する必要はない.



\subsection{バナッハ空間論}

\begin{proposition}
    \label{thm:Banach-subspace}
    バナッハ空間の閉部分ベクトル空間は,またバナッハ空間である.
\end{proposition}

\begin{proof}
    ノルム空間の部分ベクトル空間がノルム空間であることは\cref{thm:norm-subspace}からいえる.
    完備性は,部分ベクトル空間が閉であることの定義(\cref{dfn:closure-by-sequence})から明らかである.
\end{proof}



\subsection{ヒルベルト空間論}

バナッハ空間のときの\cref{thm:Banach-subspace}と同様に,
以下が示される.

\begin{corollary}
    \label{thm:Hilbert-subspace}
    ヒルベルト空間の閉部分ベクトル空間は,またヒルベルト空間である.
\end{corollary}




\subsection{射影定理}




\refdfn[距離]{dfn:distance}%
$d(𝒙, 𝒚) = \norm{𝒙 - 𝒚} = \sqrt{\iparen{𝒙 - 𝒚, \  𝒙 - 𝒚}}$である.

\begin{lemma}
    \label{thm:lemma-of-projection-theorem}
    ヒルベルト空間$\symcal{H}$とその\refdfn-[閉]{dfn:closed-set}部分ベクトル空間$\symcal{L}$について,
    ベクトル$𝒗 \in \symcal{H}$と$\symcal{L}$の距離を
    \begin{equation*}
        \symcal{d}(𝒗, \symcal{L}) 
            \coloneq \inf_{𝒘 \in \symcal{L}} d(𝒗, 𝒘)
            = \inf_{𝒘 \in \symcal{L}} \norm{𝒗 - 𝒘}
    \end{equation*}
    で定める.
    このとき,
    $\symcal{d}(𝒗, \symcal{L}) = d(𝒗, 𝒘)$となるような$𝒘 \in \symcal{L}$がただひとつ存在する.
\end{lemma}

\begin{proof}
    $\symcal{d}(𝒗, \symcal{L})$の定義より,
    $d(𝒗 - 𝒘_n) \xrightarrow{n \to \infty} \symcal{d}(𝒗, \symcal{L})$となるような$\symcal{L}$の点列$\sequ{𝒘_n}[n \in \symbb{N}]$がとれる.
    すると,
    \begin{equation*}
        \begin{split}
            d^2 (𝒘_m, 𝒘_n)
                &= \norm{(𝒘_m - 𝒗) + (𝒗 - 𝒘_n)}^2  \\
                &= 2 \ab( \norm{𝒘_m - 𝒗}^2 + \norm{𝒗 + 𝒘_n}^2 ) 
                    - \norm{(𝒘_m - 𝒗) - (𝒗 - 𝒘_n)}^2
                    \qquad \text{(\refthm[中線定理]{thm:parallelogram-law})}  \\
                &= 2 \ab( \norm{𝒗 - 𝒘_m}^2 + \norm{𝒗 - 𝒘_n}^2 )
                    - 4 \norm*{ \frac{𝒘_m + 𝒘_n}{2} - 𝒗 }^2  \\
                &\leq 2 \ab( \norm{𝒗 - 𝒘_m}^2 + \norm{𝒗 - 𝒘_n}^2 )
                    - 4 \symcal{d}(𝒗, \symcal{L})  \\
                &\xrightarrow{m, n \to \infty}
                    2 \ab\big( \symcal{d}(𝒗, \symcal{L}) + \symcal{d}(𝒗, \symcal{L}) )
                    - 4 \symcal{d}(𝒗, \symcal{L})
                = 0
        \end{split}
    \end{equation*}
    最後の不等号では,$(𝒘_m + 𝒘_n)/2 \in \symcal{L}$であること,
    および$\symcal{d}(𝒗, \symcal{L})$の最小性を使った.

    これにより,$\sequ{𝒘_n}[n \in \symbb{N}]$は\refdfn[コーシー列]{dfn:Cauchy-sequence}であり,
    ヒルベルト空間の完備性からこれはある$𝒘 \in \symcal{L}$に収束する.

    一意性を示す.$𝒘, 𝒘' \in W$がともに$\symcal{d}(𝒗, \symcal{L}) = d(𝒗, 𝒘^{(\prime)})$を満たすとして,
    $d^2 (𝒘, 𝒘')$を上の方法で計算すると,これは$0$になるので,$𝒘 = 𝒘'$である.
\end{proof}


\begin{theorem}[射影定理]
    \label{thm:Hilbert-projection-theorem}
    ヒルベルト空間$\symcal{H}$に対し,閉部分ベクトル空間$\symcal{L} \subset \symcal{H}$とその\refdfn[直交補空間]{dfn:orthogonal-compliment}$\symcal{L}^\perp$をとる.
    このとき,任意のベクトル$\symbf{z}$に対し,
    $\symbf{x} \in \symcal{L}$と$\symbf{y} \in \symcal{L}^\perp$がそれぞれただひとつ存在し,
    $\symbf{z} = \symbf{x} + \symbf{y}$とかける.
    すなわち$\symcal{H} = \symcal{L} \oplus \symcal{L}^\perp$と\refdfn[直和分解]{dfn:direct-sum-of-vector-space}できる.
\end{theorem}

\begin{proof}
    $\symbf{z} \in \symcal{H}$とする.
    \cref{thm:lemma-of-projection-theorem}の方法で$\symbf{x} \in \symcal{L}$をとる.
    このとき$\symbf{y} \coloneq \symbf{z} - \symbf{x} \in \symcal{L}^\perp$を示せばよい.
    任意の$\lambda \in \symbb{R}$および$\symbf{a} \in \symcal{L}$に対し,
    \begin{equation*}
        \begin{split}
            0 \leq \norm{\symbf{y}}^2
                &= \norm{\symbf{z} - \symbf{x}}^2  \\
                &\leq \norm{\symbf{z} - (\symbf{x} - \lambda \symbf{a})}^2
                    \quad \text{($\symbf{x} - \lambda \symbf{a} \in \symcal{L}$,
                                $\norm{\symbf{z} - \symbf{x}}$の最小性)} \\
                &= \norm{\symbf{y} + \lambda \symbf{a}}^2  \\
                &= \norm{\symbf{y}}^2 + 2 \lambda \Real \iparen{\symbf{y}, \symbf{a}} + \lambda^2 \norm{\symbf{a}}^2  \\
        \end{split}
    \end{equation*}
    したがって,
    \(  2 \lambda \Real \iparen{\symbf{y}, \symbf{a}} + \lambda^2 \norm{\symbf{a}}^2  \geq  0  \)
    であるが,$\lambda \in \symbb{R}$は任意なので,$\Real \iparen{\symbf{y}, \symbf{a}} = 0$.
    また,$\lambda \to \iu \lambda$として同様に考えると,$\Imaginary \iparen{\symbf{y}, \symbf{a}} = 0$.
    よって$\iparen{\symbf{y}, \symbf{a}} = 0$である.
    $\symbf{a} \in \symcal{L}$は任意だったので,$\symbf{y} \in \symcal{L}$が示された.
\end{proof}



\subsection{ヒルベルト空間の基底}

一般のベクトル空間には\cref{thm:basis-exist}より基底が存在する.
また,有限次元の内積空間では,\cref{thm:Gram-Schmidt-process}によって正規直交基底を構成できる.

それでは,無限次元の内積空間に正規直交基底が存在するのか?

それを考えるためには,
まずベクトルの無限級数というものを考えないといけない.

\begin{definition}
    ヒルベルト空間$\symcal{H}$に属する可算無限個のベクトルの列$\sequ{\symbf{u}_n}[n \in \symbb{N}]$と
    スカラー列$\sequ{c_n}[n \in \symbb{N}]$をとる.
    \begin{equation*}
        \lim_{N \to \infty} \norm*{𝒗 - \sum_{n=1}^{N} c_n \symbf{u}_n} = 0
    \end{equation*}
    となるような$𝒗 \in \symcal{H}$が存在するとき,
    級数$\sum_{n=1}^{\infty} c_n \symbf{u}_n$は$𝒗$に収束するといい,
    \begin{equation*}
        \sum_{n=1}^{\infty} c_n \symbf{u}_n = 𝒗
    \end{equation*}
    とかく\cite[\S 1.5]{arai-1997}.
\end{definition}

\begin{definition}
    $\symcal{H}$をヒルベルト空間とする.
    $\symcal{H}$に属する\refdfn[可算無限個]{dfn:countably-infinite}のベクトルの組$(\symbf{u}_n)_{n=1}^{\infty}$が,
    任意のベクトル$𝒗 \in \symcal{H}$を
    \begin{equation}
        𝒗 = \sum_{n=1}^{\infty} c_n \symbf{u}_n,
        \quad c_n \coloneq \iparen{𝒗, \symbf{u}_n}
    \end{equation}
    とあらわせるとき,
    $\sequ{\symbf{u}_n}[n \in \symbb{N}]$を%
    \word{完全正規直交系}(complete orthonormal system)%
    \index{かんせんせいきちよつこうけい@完全正規直交系}%
    という.
\end{definition}

$\sequ{\symbf{u}_\lambda}[\lambda \in \Lambda]$がベクトル空間$V$の\refdfn[基底]{dfn:basis}であるとは,
任意の$𝒗 \in V$が\emph{有限個を除きゼロである}スカラーの組$\sequ{c_\lambda}[\lambda \in \Lambda]$($c_\lambda \in \symbb{K}$)を用いて
\[  𝒗 = \sum_{\lambda \in \Lambda} c_\lambda \symbf{u}_\lambda  \]
と有限和でかけることだった.
しかし,完全正規直交系の場合は無限和(高々可算個のベクトルの線形結合)も許される.

なお,$\symcal{H}$が有限次元であるときは,
正規直交基底を指して完全正規直交系という.


\begin{theorem}
    \label{thm:basis-of-Hilbert-space}
    ヒルベルト空間$\symcal{H}$が\refdfn[可分]{dfn:separable}であれば,
    $\symcal{H}$は正規直交基底をもつ.
\end{theorem}



\end{document}