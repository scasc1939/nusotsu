\documentclass[../sotsu.tex]{subfiles}

\begin{document}


\section{ヒルベルト空間}

ユークリッド空間$\symbb{R}^N, \symbb{C}^N$では,
任意のベクトル列$\sequ{𝒗_n}$が$\norm{𝒗_m - 𝒗_n} \to 0$($n, m \to \infty$)を満たすとき,
この列はあるベクトル$𝒗_* \in \symbb{R}^N \text{ or } \symbb{C}^N$に\refdfn[収束]{dfn:convergence-of-vector-sequence}する.
自明のように思われるかもしれないが,
一般のベクトル空間(ノルム空間,内積空間)ではこの事実が成り立たない.



\subsection{ヒルベルト空間とは何か}

この節では,ベクトル空間における完備性について考えてみよう.

$V$を\refdfn-[ノルム空間]{dfn:norm}とする.
$𝒙, 𝒚 \in V$の\refdfn[距離]{dfn:distance}は,
$d(𝒙, 𝒚) \coloneq \norm{𝒙 - 𝒚}$で定義できるのであった(\cref{thm:norm-is-distance}).
距離空間で定義したように,$V$の\refdfn-[収束列]{dfn:convergent-sequence}と\refdfn-[コーシー列]{dfn:Cauchy-sequence}は,以下のように定義できる.
\begin{itemize}
    \item $V$の点列$\sequ{𝒗_n}$が収束列であるとは,$\sequ{𝒗_n}$がある$𝒗_* \in V$に収束する,
        つまり$\lim_{n \to \infty} 𝒗_n = 𝒗_* \in V$となることをいう.
    \item $V$の点列$\sequ{𝒗_n}$がコーシー列であるとは,
        $\lim_{m, n \to \infty} \norm{𝒗_m - 𝒗_n} = 0$であることをいう.
\end{itemize}
これらを用いて,ノルム空間の完備性を定義することができる.

\begin{definition}
    ノルム空間$V$が完備であるとは,$V$のコーシー列が収束列であることをいう.
\end{definition}

\begin{definition}[バナッハ空間]
    \label{dfn:Banach-space}
    完備なノルム空間を%
    \word{バナッハ空間}(Banach space)%
    \index{はなつはくうかん@バナッハ空間}%
    という\footnote{
        \person{Stefan Banach}{1892}{1945}.
        ポーランドの数学者.
        関数解析学の創始者のひとり.
        \cite{nipponica}
    }.
\end{definition}

\refdfn[内積空間]{dfn:inner-product}では,内積から導かれるノルムが存在するのであった.
このノルムを用いれば,完備な内積空間というものを定義することができる.

\begin{definition}[ヒルベルト空間]
    \label{dfn:Hilbert-space}
    完備な内積空間を\word{ヒルベルト空間}(Hilbert space)という\footnote{
        \person{David Hilbert}{1862}{1943}.
        ドイツの数学者.
        数学基礎論,代数学,幾何学,解析学,さらには理論物理学まで広い分野で功績を残し,
        20世紀前半でもっとも偉大な数学者のひとりとされる.
        ヒルベルト23の問題でも有名.
        \cite{nipponica}\cite{iwanami-sugakujiten}
    }.
\end{definition}

\begin{proposition}
    実ユークリッド空間$ℝ^n$および複素ユークリッド空間$ℂ^n$はヒルベルト空間である.
\end{proposition}





\begin{theorem}
    \label{thm:subspace-of-norm-space-is-complete}
    ノルム空間の有限次元部分空間は完備である\cite[定理1.2の系]{iwanami-functional}.
    特に,有限次元ノルム空間は完備である.
\end{theorem}

\begin{proof}
    概略だけ述べる.
    $\symbb{C}$上のノルム空間$V$の部分空間を$W$とし,
    次元$n \coloneq \dim W$とおく.
    
    $W$のコーシー列を$\sequ{𝒗^{(k)}}[k \in \symbb{N}]$とおくと,
    任意の$k$に対して$𝒗^{(k)} \in W$であるから,
    $W$の基底$\symbf{u}_1, \dots, \symbf{u}_n$を用いて
    \[  𝒗^{(k)} = c^{(k)}_1 \symbf{u}^{(k)}_1 + \dots + c^{(k)}_n \symbf{u}^{(k)}_n  \]
    とかける.
    $\sequ{𝒗^{(k)}}[k]$がコーシー列であることを使うと,
    各$i$に対して$\sequ{c^{(k)}_i}[k]$がコーシー列であることを示すことができる%
    \footnote{
        ここでは,$W$がノルム空間の有限次元部分空間であるときに,
        $W$上のノルムがすべて同値であることを使う\cite[定理1.2]{iwanami-functional}.
    }.
    すると,$\symbb{C}$の完備性\refthm{thm:completeness-of-copmlex-numbers}からこれは,
    $c^{(k)}_i \xrightarrow{k \to \infty} c^*_i \in \symbb{C}$に収束する.
    そこで$𝒗_* \coloneq c^*_1 \symbf{u}_1 + \dots + c^*_1 \symbf{u}_n \in W$とおくと,
    \begin{equation*}
        \begin{split}
            \norm{ 𝒗^{(k)} - 𝒗^* }
                = \norm*{ \sum_{i = 1}^{n} \ab\big(c^{(k)}_i - c^*_i) \symbf{u}_i }
                \leq \sum_{i=1}^{n} \, \abs[\big]{ c^{(k)}_i - c^*_i } \, \norm{ \symbf{u}_i }
                \xrightarrow{k \to \infty} 0
        \end{split}
    \end{equation*}
    より,$𝒗^{(k)} \to 𝒗^* \in W$である.
\end{proof}

したがって,
有限次元のノルム空間では,完備性について検討する必要はない.



\subsection{バナッハ空間論}

\begin{proposition}
    \label{thm:Banach-subspace}
    バナッハ空間の閉部分ベクトル空間は,またバナッハ空間である.
\end{proposition}

\begin{proof}
    ノルム空間の部分ベクトル空間がノルム空間であることは\cref{thm:norm-subspace}からいえる.
    完備性は,部分ベクトル空間が閉であることの定義(\cref{dfn:closure-by-sequence})から明らかである.
\end{proof}



\subsection{ヒルベルト空間論}

バナッハ空間のときの\cref{thm:Banach-subspace}と同様に,
以下が示される.

\begin{corollary}
    \label{thm:Hilbert-subspace}
    ヒルベルト空間の閉部分ベクトル空間は,またヒルベルト空間である.
\end{corollary}


内積の連続性から,以下がいえる.

\begin{lemma}
    \label{thm:lemma-of-inner-product-continuity}
    $\symcal{F}$をヒルベルト空間$\symcal{H}$の部分集合とする.
    $\symcal{H}$のベクトル$𝒗$が,
    任意のベクトル$𝒙 \in \symcal{F}$に対して$\iparen{𝒗, 𝒙} = 0$を満たすなら,
    任意のベクトル$𝒙_* \in \clsr{\symcal{F}}$に対しても$\iparen{𝒗, 𝒙_*} = 0$が成り立つ.
\end{lemma}

\begin{proof}
    $𝒙_*$は$\symcal{F}$の閉包に属するベクトルだから,
    $\symcal{F}$の点列$\sequ{𝒙_n}[n \in \symbb{N}]$が存在して,
    $𝒙_n \xrightarrow{n \to \infty} 𝒙_*$となる.
    このことと\refdfn[内積の連続性]{thm:inner-product-continuity}から,
    \[
        \iparen{𝒗, 𝒙_*}
        = \iparen{𝒗, \   \lim_{n \to \infty} 𝒙_n}
        = \lim_{n \to \infty} \iparen{𝒗, 𝒙_n}
        = 0
    \]
    である.
\end{proof}


\begin{proposition}
    \label{thm:Hilbert-subspace-orthogonal-compliment}
    ヒルベルト空間$\symcal{H}$の部分集合\footnote{
        $\symcal{F}$が部分ベクトル空間である必要はない.
    }$\symcal{F}$について,
    その\refdfn[直交補空間]{dfn:orthogonal-compliment}$\symcal{F}^\perp$は閉部分ベクトル空間である.
\end{proposition}

\begin{proof}
    \bluehead{$\symcal{F}$が部分ベクトル空間になること} %
    \cref{thm:vector-subspace-iff}の条件を満たすことは,
    内積の線形性から示される.
    実際,$𝒗 \in \symcal{F}$を任意のベクトルとし,
    これは,$𝒙, 𝒚 \in \symcal{F}^\perp$,
    $c \in \symbb{C}$をとると,
    \begin{gather*}
        \iparen{𝒗, \symbf{0}} = 0
        \\
        \iparen{𝒗, \  𝒙 + 𝒚}
        = \iparen{𝒗, 𝒙} + \iparen{𝒗, 𝒚}
        = 0
        \\
        \iparen{𝒗, c 𝒙}
        = c \iparen{𝒗, 𝒙}
        = 0
    \end{gather*}
    であるから,
    $\symbf{0} \in \symcal{F}^\perp$,
    $𝒙 + 𝒚 \in \symcal{F}^\perp$,
    かつ$c 𝒙 \in \symcal{F}^\perp$である.

    \bluehead{$\symcal{F}$が閉集合であること} %
    $\symcal{F}$の点列$\sequ{𝒙_n}[n \in \symbb{N}]$が$𝒙_* \in \symcal{F}$に収束することは,
    \refdfn[内積の連続性]{thm:inner-product-continuity}からわかる.
    つまり,任意の$𝒗 \in \symcal{F}$に対して
    \begin{equation*}
        \iparen{𝒗, 𝒙_*}
        = \iparen{𝒗, \lim_{n \to \infty} 𝒙_n}
        = \lim_{n \to \infty} \iparen{𝒗, 𝒙_n}
        = 0
    \end{equation*}
    だから,$𝒙_* = \lim_{n \to \infty} 𝒙_n \in \symcal{F}^\perp$である.
\end{proof}


\begin{lemma}
    \label{thm:Hilbert-subset-inclusion-orthogonal-compliment}
    ヒルベルト空間の部分集合$\symcal{F}, \symcal{G}$について,
    $\symcal{F} \subset \symcal{G}$ならば$\symcal{G}^\perp \subset \symcal{F}^\perp$である.
\end{lemma}

\begin{proof}
    \refdfn-[直交補空間]{dfn:orthogonal-compliment}の定義から,
    $𝒙 \in \symcal{G}^\perp$ならば$𝒙 \in \symcal{F}^\perp$が示される.
\end{proof}


\begin{corollary}
    \label{thm:Hilbert-subset-orthogonal-complement-of-closure}
    ヒルベルト空間の部分集合$\symcal{F}$について,
    $(\clsr{\symcal{F}})^\perp = \symcal{F}^\perp$.
\end{corollary}

\begin{proof}
    閉包の性質から$\symcal{F} \subset \clsr{\symcal{F}}$なので,
    \cref{thm:Hilbert-subset-inclusion-orthogonal-compliment}より$(\clsr{\symcal{F}})^\perp \subset \symcal{F}^\perp$がいえる.
    逆の包含を示す.
    $𝒙 \in \symcal{F}^\perp$とすると,
    任意の$𝒗 \in \symcal{F}$に対し,$\iparen{𝒗, 𝒙} = 0$である.
    よって,\cref{thm:lemma-of-inner-product-continuity}より,
    $𝒗_* \in \clsr{\symcal{F}}$に対しても$\iparen{𝒗_*, 𝒙} = 0$であるから,
    $𝒙 \in (\clsr{\symcal{F}})^\perp$がいえる.
\end{proof}


\begin{proposition}
    \label{thm:Hilbert-subspace-double-orthogonal-compliment}
    ヒルベルト空間$\symcal{H}$の部分ベクトル空間$\symcal{F}$について,
    $(\symcal{F}^\perp)^\perp = \clsr{\symcal{F}}$である.
\end{proposition}

\begin{proof}
    
    $\clsr{\symcal{F}}$は閉部分ベクトル空間であるから,
    任意の$𝒛 \in \symcal{H}$は\refdfn[正射影定理]{thm:Hilbert-projection-theorem}より,
    \[
        𝒛 = 𝒙 + 𝒚,
        \quad 
        \text{where } 
        𝒙 \in \clsr{\symcal{F}}, \  
        𝒚 \in (\clsr{\symcal{F}})^\perp = \symcal{F}^\perp
    \]
    と一意に書ける(途中,\cref{thm:Hilbert-subset-orthogonal-complement-of-closure}を使った.).
    そこで$𝒛 = 𝒗 \in (\symcal{F}^\perp)^\perp$とすれば,
    $𝒚 = \symbf{0}$でなければならないので,
    \[  𝒗 = 𝒙  \in \clsr{\symcal{F}}  \]
    である.したがって$(\symcal{F}^\perp)^\perp \subset \clsr{\symcal{F}}$がいえる.
    一方,閉包$\clsr{\symcal{F}}$は$\symcal{F}$を含む閉集合のうち最小のものであるから,
    $\clsr{\symcal{F}} \subset (\symcal{F}^\perp)^\perp$である.
\end{proof}



稠密であることを判定するために有用な命題を挙げる.

\begin{proposition}
    ヒルベルト空間$\symcal{H}$の部分ベクトル空間$\symcal{F}$が\refdfn[稠密]{dfn:dense}である必要十分条件は,
    $\symcal{F}^\perp = \{ \symbf{0} \}$となることである\cite[命題1.24]{arai-1997}.
\end{proposition}

\begin{proof}
    \bluehead{必要}\quad $\symcal{F}$が稠密であるとする.
    つまり$\clsr{\symcal{F}} = \symcal{H}$である.
    このとき,$𝒗 \in \symcal{F}^\perp$を任意にとると,
    すべての$𝒙 \in \symcal{F}$に対して$\iparen{𝒗, 𝒙} = 0$である.
    ここで\cref{thm:lemma-of-inner-product-continuity}を使うと,
    任意の$𝒙_* \in \symcal{H} = \clsr{\symcal{F}}$に対して$\iparen{𝒙_*, 𝒗}$とわかる.
    したがって$𝒗 = \symbf{0}$がいえた.

    \bluehead{十分}\quad $\symcal{F}^\perp$は閉部分ベクトル空間である\refdfn{thm:Hilbert-subspace-orthogonal-compliment}から,
    \refdfn[正射影定理]{thm:Hilbert-projection-theorem}より,
    任意の$𝒛 \in \symcal{H}$は
    \[  
        𝒛 = 𝒙 + 𝒚, 
        \quad \text{where }
        𝒙 \in \symcal{F}^\perp, \  
        𝒚 \in (\symcal{F}^\perp)^\perp
    \]
    とかける.
    仮定より$𝒙 = \symbf{0}$であり,
    また\cref{thm:Hilbert-subspace-double-orthogonal-compliment}より$(\symcal{F}^\perp)^\perp = \clsr{\symcal{F}}$であるので,
    \[  
        𝒛 = \phantom{𝒙 \mathbin{}+}\mathbin{} 𝒚, 
        \quad \text{where }
        \phantom{𝒙 \in \symcal{F}^\perp,\mathpunct{}} \ 
        \mathrlap{𝒚 \in \clsr{\symcal{F}}}
        \phantom{𝒚 \in (\symcal{F}^\perp)^\perp}
    \]
    すなわち$𝒛 \in \clsr{\symcal{F}}$だから,$\symcal{H} \subset \clsr{\symcal{F}}$である.
\end{proof}


\begin{proposition}
    ヒルベルト空間$\symcal{H}$の部分集合$\symcal{F}, \symcal{G}$について,
    $\symcal{F} \subset \symcal{G}$であり,
    $\symcal{F}$が$\symcal{G}$で稠密,
    $\symcal{G}$が$\symcal{H}$で稠密なら,
    $\symcal{F}$は$\symcal{H}$で稠密である.
\end{proposition}



\subsection{射影定理}




\refdfn[距離]{dfn:distance}%
$d(𝒙, 𝒚) = \norm{𝒙 - 𝒚} = \sqrt{\iparen{𝒙 - 𝒚, \  𝒙 - 𝒚}}$である.

\begin{lemma}
    \label{thm:lemma-of-projection-theorem}
    ヒルベルト空間$\symcal{H}$とその\refdfn-[閉]{dfn:closed-set}部分ベクトル空間$\symcal{L}$について,
    ベクトル$𝒗 \in \symcal{H}$と$\symcal{L}$の距離を
    \begin{equation*}
        \symcal{d}(𝒗, \symcal{L}) 
            \coloneq \inf_{𝒘 \in \symcal{L}} d(𝒗, 𝒘)
            = \inf_{𝒘 \in \symcal{L}} \norm{𝒗 - 𝒘}
    \end{equation*}
    で定める.
    このとき,
    $\symcal{d}(𝒗, \symcal{L}) = d(𝒗, 𝒘)$となるような$𝒘 \in \symcal{L}$がただひとつ存在する.
\end{lemma}

\begin{proof}
    $\symcal{d}(𝒗, \symcal{L})$の定義より,
    $d(𝒗 - 𝒘_n) \xrightarrow{n \to \infty} \symcal{d}(𝒗, \symcal{L})$となるような$\symcal{L}$の点列$\sequ{𝒘_n}[n \in \symbb{N}]$がとれる.
    すると,
    \begin{equation*}
        \begin{split}
            d^2 (𝒘_m, 𝒘_n)
                &= \norm{(𝒘_m - 𝒗) + (𝒗 - 𝒘_n)}^2  \\
                &= 2 \ab( \norm{𝒘_m - 𝒗}^2 + \norm{𝒗 + 𝒘_n}^2 ) 
                    - \norm{(𝒘_m - 𝒗) - (𝒗 - 𝒘_n)}^2
                    \qquad \text{(\refthm[中線定理]{thm:parallelogram-law})}  \\
                &= 2 \ab( \norm{𝒗 - 𝒘_m}^2 + \norm{𝒗 - 𝒘_n}^2 )
                    - 4 \norm*{ \frac{𝒘_m + 𝒘_n}{2} - 𝒗 }^2  \\
                &\leq 2 \ab( \norm{𝒗 - 𝒘_m}^2 + \norm{𝒗 - 𝒘_n}^2 )
                    - 4 \symcal{d}(𝒗, \symcal{L})  \\
                &\xrightarrow{m, n \to \infty}
                    2 \ab\big( \symcal{d}(𝒗, \symcal{L}) + \symcal{d}(𝒗, \symcal{L}) )
                    - 4 \symcal{d}(𝒗, \symcal{L})
                = 0
        \end{split}
    \end{equation*}
    最後の不等号では,$(𝒘_m + 𝒘_n)/2 \in \symcal{L}$であること,
    および$\symcal{d}(𝒗, \symcal{L})$の最小性を使った.

    これにより,$\sequ{𝒘_n}[n \in \symbb{N}]$は\refdfn[コーシー列]{dfn:Cauchy-sequence}であり,
    ヒルベルト空間の完備性からこれはある$𝒘 \in \symcal{L}$に収束する.

    一意性を示す.$𝒘, 𝒘' \in W$がともに$\symcal{d}(𝒗, \symcal{L}) = d(𝒗, 𝒘^{(\prime)})$を満たすとして,
    $d^2 (𝒘, 𝒘')$を上の方法で計算すると,これは$0$になるので,$𝒘 = 𝒘'$である.
\end{proof}


\begin{theorem}[射影定理]
    \label{thm:Hilbert-projection-theorem}
    ヒルベルト空間$\symcal{H}$に対し,閉部分ベクトル空間$\symcal{L} \subset \symcal{H}$とその\refdfn[直交補空間]{dfn:orthogonal-compliment}$\symcal{L}^\perp$をとる.
    このとき,任意のベクトル$𝒛 \in \symcal{H}$に対し,
    $𝒙 \in \symcal{L}$と$𝒚 \in \symcal{L}^\perp$がそれぞれただひとつ存在し,
    $𝒛 = 𝒙 + 𝒚$とかける.
    すなわち$\symcal{H} = \symcal{L} \oplus \symcal{L}^\perp$と\refdfn[直和分解]{dfn:direct-sum-of-vector-space}できる.
\end{theorem}

\begin{proof}
    $𝒛 \in \symcal{H}$とする.
    \cref{thm:lemma-of-projection-theorem}の方法で$𝒙 \in \symcal{L}$をとる.
    このとき$𝒚 \coloneq 𝒛 - 𝒙 \in \symcal{L}^\perp$を示せばよい.
    任意の$\lambda \in \symbb{R}$および$\symbf{a} \in \symcal{L}$に対し,
    \begin{equation*}
        \begin{split}
            0 \leq \norm{𝒚}^2
                &= \norm{𝒛 - 𝒙}^2  \\
                &\leq \norm{𝒛 - (𝒙 - \lambda \symbf{a})}^2
                    \quad \text{($𝒙 - \lambda \symbf{a} \in \symcal{L}$,
                                $\norm{𝒛 - 𝒙}$の最小性)} \\
                &= \norm{𝒚 + \lambda \symbf{a}}^2  \\
                &= \norm{𝒚}^2 + 2 \lambda \Real \iparen{𝒚, \symbf{a}} + \lambda^2 \norm{\symbf{a}}^2  \\
        \end{split}
    \end{equation*}
    したがって,
    \(  2 \lambda \Real \iparen{𝒚, \symbf{a}} + \lambda^2 \norm{\symbf{a}}^2  \geq  0  \)
    であるが,$\lambda \in \symbb{R}$は任意なので,$\Real \iparen{𝒚, \symbf{a}} = 0$.
    また,$\lambda \to \iu \lambda$として同様に考えると,$\Imaginary \iparen{𝒚, \symbf{a}} = 0$.
    よって$\iparen{𝒚, \symbf{a}} = 0$である.
    $\symbf{a} \in \symcal{L}$は任意だったので,$𝒚 \in \symcal{L}$が示された.
\end{proof}



\subsection{ヒルベルト空間の基底}

一般のベクトル空間には\cref{thm:basis-exist}より基底が存在する.
また,有限次元の内積空間では,\cref{thm:Gram-Schmidt-process}によって\refdfn[正規直交基底]{dfn:orthonormal-basis}を構成できる.

無限次元のヒルベルト空間においては,
正規直交基底の概念を拡張した``完全正規直交系''を考える.

それを考えるためには,
まずベクトルの無限級数を定義しよう.
\cref{dfn:convergence-of-vector-sequence}で与えたベクトルの点列に対する収束より,
次のように定義できる.

\begin{definition}[ベクトルの無限級数]
    ヒルベルト空間$\symcal{H}$に属する可算無限個のベクトルの列$\sequ{\symbf{u}_n}[n \in \symbb{N}]$と
    スカラー列$\sequ{c_n}[n \in \symbb{N}]$をとる.
    \begin{equation*}
        \lim_{N \to \infty} \norm*{𝒗 - \sum_{n=1}^{N} c_n \symbf{u}_n} = 0
    \end{equation*}
    となるような$𝒗 \in \symcal{H}$が存在するとき,
    級数$\sum_{n=1}^{\infty} c_n \symbf{u}_n$は$𝒗$に収束するといい,
    \begin{equation*}
        \sum_{n=1}^{\infty} c_n \symbf{u}_n = 𝒗
    \end{equation*}
    とかく\cite[\S 1.5]{arai-1997}.
\end{definition}


\begin{definition}[正規直交系]
    \refdfn[高々可算個]{dfn:at-most-countable}のベクトルの組$\sequ{\symbf{\varphi}_n}[n]$が
    \begin{equation}
        \iparen{\symbf{\varphi}_m, \symbf{\varphi}_n}
            = \kdelta_{mn}
    \end{equation}
    を満たすとき,これを\word{正規直交系}という.
\end{definition}


正規直交系の無限級数が収束する十分条件は,
その係数の和が絶対収束することである.


\begin{lemma}[\word{ベッセルの不等式}(Bessel's inequality)]
    \index{へつせるのふとうしき@ベッセルの不等式}
    \label{thm:Bessel-inequality}
    ヒルベルト空間$\symcal{H}$の正規直交系$\sequ{\symbf{\varphi}_n}[n \in \symbb{N}]$をとる.
    任意の$\symbf{v} \in \symcal{H}$に対し,
    \begin{equation}
        \norm{\symbf{v}}^2 \geq \sum_{n=1}^{\infty} \abs{\iparen{\symbf{v}, \symbf{\varphi}_n}}^2
    \end{equation}
\end{lemma}

\begin{proof}
    
\end{proof}


\begin{lemma}[\word{リース--フィッシャーの定理}(Riesz--Fischer theorem)]
    \label{thm:Riesz-Fischer-theorem}
    \index{りいすふいつしやーのていり@リース--フィッシャーの定理}
    ヒルベルト空間$\symcal{H}$の正規直交系$\sequ{\symbf{\varphi}_n}[n \in \symbb{N}]$をとる.
    複素数列$\sequ{c_n}$について,$\sum_{n=1}^{\infty} \abs{c_n} < \infty$であれば,
    無限級数
    \begin{equation*}
        \sum_{n=1}^{\infty} c_n \symbf{\varphi}_n
    \end{equation*}
    はあるベクトル$𝒗 \in \symcal{H}$に収束する.
    さらにこのとき,
    \begin{align*}
        \iparen{𝒗, \symbf{\varphi}_n} = c_n,
        \quad
        \norm{𝒗} = \sum_{n=1}^{\infty} \abs{c_n}
    \end{align*}
\end{lemma}

\begin{proof}
    部分和$𝒗_N \coloneq \sum_{n=1}^{N} c_n \symbf{\varphi}_n$について,
    三角不等式より
    \begin{equation*}
        \norm{𝒗_M - 𝒗_N}^2
        = \norm*{ \sum_{n=N+1}^{M} c_n \symbf{\varphi}_n }^2
        \leq \sum_{n=N+1}^{M} \abs{c_n}^2 \norm{\symbf{\varphi}_n}^2
        = \sum_{n=N+1}^{M} \abs{c_n}^2
        \xrightarrow{M, N \to \infty} 0
    \end{equation*}
    (ただし$M > N$)であるから,点列$\sequ{𝒗_N}$はコーシー列をなす.
    したがって$\symcal{H}$の完備性から,
    これは$𝒗_N \to 𝒗 \in \symcal{H}$と収束する.
    一方,\refthm[内積の連続性]{thm:inner-product-continuity}および$\sequ{\symbf{\varphi}_n}$が正規直交系であることから,
    \begin{equation*}
        \norm{𝒗} 
        = \lim_{N \to \infty} \norm{𝒗_N}
        = \lim_{N \to \infty} \norm*{ \sum_{n=1}^{N} c_n \symbf{\varphi}_n }
        = \lim_{N \to \infty} \sum_{n=1}^{N} \abs{c_n}
    \end{equation*}
    がいえる.
\end{proof}


\begin{definition}
    $\symcal{H}$をヒルベルト空間とする.
    $\symcal{H}$に属する\refdfn[可算無限個]{dfn:countably-infinite}のベクトルの組$(\symbf{\varphi}_n)_{n=1}^{\infty}$が,
    任意のベクトル$𝒗 \in \symcal{H}$を
    \begin{equation}
        𝒗 = \sum_{n=1}^{\infty} c_n \symbf{\varphi}_n,
        \quad c_n \in \symbb{C}
    \end{equation}
    とあらわせるとき,
    $\sequ{\symbf{\varphi}_n}[n \in \symbb{N}]$を%
    \word{完全正規直交系}(complete orthonormal system)%
    \index{かんせんせいきちよつこうけい@完全正規直交系}%
    という.
\end{definition}

$\sequ{\symbf{u}_\lambda}[\lambda \in \Lambda]$がベクトル空間$V$の\refdfn-[基底]{dfn:basis}であるとは,
任意の$𝒗 \in V$が\emph{有限個を除きゼロである}スカラーの組$\sequ{c_\lambda}[\lambda \in \Lambda]$($c_\lambda \in \symbb{K}$)を用いて
\[  𝒗 = \sum_{\lambda \in \Lambda} c_\lambda \symbf{u}_\lambda  \]
と有限和でかけることだった.
しかし,完全正規直交系の場合は無限和(高々可算個のベクトルの線形結合)も許される.

なお,$\symcal{H}$が有限次元であるときは,
正規直交基底を指して完全正規直交系という.


\begin{proposition}
    \label{thm:complete-orthonormal-system-iff}
    ヒルベルト空間$\symcal{H}$の正規直交系を$\{ \symbf{\varphi}_n \}$とする.
    以下は同値\cite[\S 3.5 c)]{iwanami-functional}.
    \begin{subequations}
    \begin{enumerate}
        \item \label{thm:complete-orthonormal-system-iff:definition}
            $\{ \symbf{\varphi}_n \}$は完全正規直交系である.
        \item \label{thm:complete-orthonormal-system-iff:expansion}
            任意のベクトル$\symbf{v} \in \symcal{H}$に対し,
            \begin{equation*}
                𝒗 = \sum_n \iparen{𝒗, \symbf{\varphi}_n} \symbf{\varphi}_n
            \end{equation*}
        \item \label{thm:complete-orthonormal-system-iff:norm}
            任意のベクトル$\symbf{v} \in \symcal{H}$に対し,
            \begin{equation}
                \label{eq:Parseval-identity-norm}
                \norm{𝒗}^2 = \sum_n \abs{\iparen{𝒗, \symbf{\varphi}_n}}^2
            \end{equation}
        \item \label{thm:complete-orthonormal-system-iff:inner-product}
            任意のベクトル$\symbf{u}, \symbf{v} \in \symcal{H}$に対し,
            \begin{equation}
                \label{eq:Parseval-identity-inner-product}
                \iparen{\symbf{u}, 𝒗} = \sum_n \conj{\iparen{\symbf{u}, \symbf{\varphi}_n}} \iparen{\symbf{v}, \symbf{\varphi}_n}
            \end{equation}
        \item \label{thm:complete-orthonormal-system-iff:sequence}
            任意の$n$に対し,$\iparen{\symbf{u}, \symbf{\varphi}_n} = 0$なら$\symbf{u} = \symbf{0}$である.
    \end{enumerate}
    \end{subequations}
    なお,\cref{eq:Parseval-identity-norm}または\cref{eq:Parseval-identity-inner-product}を,
    \word{パーセヴァルの等式}(Parseval's identity)\index{はあせうあるのとうしき@パーセヴァルの等式}という.
\end{proposition}


\begin{proof}
    \bluehead{\cref*{thm:complete-orthonormal-system-iff:definition}$\implies$\cref*{thm:complete-orthonormal-system-iff:expansion}}\quad

    \bluehead{\cref*{thm:complete-orthonormal-system-iff:expansion}$\implies$\cref*{thm:complete-orthonormal-system-iff:inner-product}}\quad

    \bluehead{\cref*{thm:complete-orthonormal-system-iff:inner-product}$\implies$\cref*{thm:complete-orthonormal-system-iff:norm}$\implies$\cref*{thm:complete-orthonormal-system-iff:sequence}}\quad
    これは明らかである.

    \bluehead{\cref*{thm:complete-orthonormal-system-iff:sequence}$\implies$\cref*{thm:complete-orthonormal-system-iff:definition}}

\end{proof}


\begin{theorem}
    可分なヒルベルト空間は,$\ell^2$空間と\refdfn-[同型]{dfn:isomorphic}である.
\end{theorem}

\begin{corollary}
    可分なヒルベルト空間は互いに\refdfn-[同型]{dfn:isomorphic}である.
\end{corollary}


\begin{theorem}
    \label{thm:basis-of-Hilbert-space}
    \refdfn[可分]{dfn:separable}なヒルベルト空間は完全正規直交系をもつ.
\end{theorem}



\end{document}