\documentclass[../sotsu.tex]{subfiles}


\begin{document}


\section{関数解析}

この節では関数をベクトル空間の\ruby{元}{げん}として扱う.
ベクトル空間の係数体$𝕂$は$ℝ$もしくは$ℂ$に限る.

\subsection{関数空間}

関数$\psi(x_1, x_2, \dotsc)$のようなものを考えよう.

\begin{definition}
    $𝛺 \subset ℝ$を$N$次元ユークリッド空間$ℝ^N$の\refdfn[開集合]{dfn:open-set-and-closed-set}とする.
    $𝛺$における連続関数全体の集合を$C^0 (𝛺)$でかく.
\end{definition}

\begin{proposition}
    \label{thm:continuous-function-space-is-vector-space}
    $C^0 (𝛺)$は以下のように定義される和とスカラー倍に関して,ベクトル空間をなしている.
    \begin{itemize}
        \item $f, g \in C^0 (𝛺)$,$c \in 𝕂$に対して,
        \begin{enumerate}
            \item 関数の和$f \plushat g$は,
                任意の$𝒙 \in 𝛺$に対して$ ( f \plushat g )(𝒙) = f(𝒙) + g(𝒙) $である関数と定める.
            \item 関数のスカラー倍$c \mathbin{\hat{\scaprod}} f$は,
                任意の$𝒙 \in 𝛺$に対して$ ( c \mathbin{\hat{\scaprod}} g )(𝒙) = c \cdotp f(𝒙) $である関数と定める.
        \end{enumerate}
    \end{itemize}
\end{proposition}

\begin{proof}
    
\end{proof}


\begin{definition}
    ある関数$u \in C(𝛺)$に対し,$u(x) \neq 0$であるような$x \in 𝛺$の集合の\refdfn[閉包]{dfn:closure}
    \begin{equation}
        \supp u  \coloneq  \closure{ \Set{  x \in 𝛺  \given  u(x) \neq 0  } }
    \end{equation}
    を$u$の\word{台}[だい][だい](support)という.
\end{definition}

関数$u \in C(𝛺)$の中で,$\supp u$がコンパクト\refdfn{dfn:compact-set}であるもの全体の集合を$C_0 (𝛺)$とかく.
$C_0 (𝛺)$もベクトル空間になっている.


\begin{definition}
    \label{dfn:class-C^k-function}
    複素関数$f$が\word{$C^k$-級関数}[][Cきゆうかんすう](class $C^k$-function)であるとは,
    $f(x)$が$ℂ$上で$k$回微分可能かつ$k$次導関数が連続であることをいう.
    $f(x)$が無限回微分可能であるとき$C^\infty$-級関数であるという.

    $C^k$-級関数全体の集合を$C^k$とかく.
    特に連続関数全体の集合は$C^0$である.
\end{definition}

\begin{example}
    $e^x$, $\sin x$, $\cos x$は$C^\infty$-級関数である.
\end{example}

\begin{example}
    $C^0 \subsetneq C^1 \subsetneq C^2 \subsetneq \dotsb$であり,
    $\bigcap_{k \in \symbb{N}} C^k = C^\infty$である.
\end{example}



\subsection{関数空間のノルム}

ここでは関数空間上にノルムが定義できる場合について考える.

関数空間においてもっとも一般的なノルムは``最大値ノルム''である.

しかし,量子力学において重要なのは,以下の2乗可積分関数空間である.

\begin{definition}
    \label{dfn:square-integrable-function-space}
    $𝛺 \subset ℝ^N$から$ℂ$への%
    \word{2乗可積分関数空間}(square integrable function space)%
    \index{2しようかせきふんくうかん@2乗可積分関数空間}%
    $L²$を
    \begin{equation*}
        L² (𝛺) \coloneq \Set*{ f \colon 𝛺 \to ℂ  \given  \int_𝛺 \abs{f(x)}^2 \dd{x} < \infty }
    \end{equation*}
    で定義する.
    ただし$\int_𝛺 \dd{x}$はルベーグ積分である.
\end{definition}

\begin{proposition}
    \refdfn-[2乗可積分関数空間]{dfn:square-integrable-function-space}%
    $L² (𝛺)$は,
    以下で定義されるノルムに対して\refdfn[ノルム空間]{dfn:norm}になる.
    \begin{equation}
        \label{eq:L^2-norm}
        \norm{f}_{L²} \coloneq \sqrt{ \int_𝛺 \abs{f(𝒙)}^2 \dd{x} }
    \end{equation}
\end{proposition}

\begin{proof}
    \cref{dfn:norm}の条件を確かめればよい.
    \cref{norm:absolute-homogeneity}は,
    スカラー倍$cf$の定義($(cf)(𝒙) = c \cdotp f(𝒙)$)より,
    \(
        \int_𝛺 \abs{(cf)(𝒙)}² \dd{𝒙}
            = \abs{c}² \int_𝛺 \abs{f(𝒙)}^2 \dd{𝒙}
    \)だからいえる.
    三角不等式\cref{norm:triangle-inequality}は,
    和$f + g$の定義($(f+g)(𝒙) = f(𝒙) + g(𝒙)$)より,
    \begin{equation*}
        \begin{split}
            ∫_𝛺 \abs{(f+g)(𝒙)}² \dd{x}
            &= \int_𝛺 \abs{f(𝒙) + g(𝒙)}² \dd{x}
            \\
            &= \int_𝛺 \abs{f(𝒙)}² 
                + 2 \int_𝛺 \Real \ab( \conj{f}(𝒙) g(𝒙) ) \dd{x}
                + \int_𝛺 \abs{g(𝒙)}² \dd{x}
            \\
            &\leq \int_𝛺 \abs{f(𝒙)}² \dd{x} 
                + 2\sqrt{ \int_𝛺 \abs{f(𝒙)}² \dd{x}   \int_𝛺 \abs{g(𝒙)}² \dd{x} } 
                + \int_𝛺 \abs{g(𝒙)}² \dd{x}
            \\
            &= \ab( \sqrt{  \int_𝛺 \abs{f(𝒙)}² \dd{x}  }  
                  + \sqrt{  \int_𝛺 \abs{g(𝒙)}² \dd{x}  }  )²
        \end{split}
    \end{equation*}
    だからいえる(途中コーシー・シュワルツの不等式より
    \begin{equation*}
        \int \abs{\conj{f}(𝒙) g(𝒙)} \dd{x}
            \leq \abs*{ \int \conj{f}(𝒙) g(𝒙) \dd{x} }
            \leq \sqrt{ \int_𝛺 \abs{f(𝒙)}² \dd{x}   \int_𝛺 \abs{g(𝒙)}² \dd{x} }
    \end{equation*}
    を用いた).
    \cref{norm:positivity}の$\norm{f}_{L²} \geq 0$は定義より明らかである.
    しかし,$\norm{f}_{L²} = 0$ならば$f = 0$は,実はいえない.
    $f$が$\Omega$上で恒等的にゼロでなくても,$\Omega$の\refdfn[ほとんどいたるところ]{dfn:almost-everywhere}ゼロであれば,
    $\int_\Omega f(\symbf{x}) \dd{x} = 0$となるからである.
    そこで,関数としての$0$を,(ルベーグ測度ゼロの点を除いて)ほとんどいたるところゼロである関数と再定義する.
    これにより$\norm{\bigdot}_{L²}$はノルムの条件を満たす.
\end{proof}




\subsection{内積空間としての関数空間}

関数空間に内積が定義できる場合.

\begin{proposition}
    \refdfn-[2乗可積分関数空間]{dfn:square-integrable-function-space}$L² (𝛺)$は,以下で定義される内積$(\bigdot, \bigdot)$に対して\refdfn[内積空間]{dfn:inner-product}になる.
    \begin{align}
        \label{eq:L^2-inner-product}
        \iparen{f, g}_{L²} \coloneq \int_𝛺 \conj{f}(x) g(x) \dd{x}
    \end{align}
    この内積から導かれるノルムは\cref{eq:L^2-norm}である.
\end{proposition}

\begin{proof}
    \cref{eq:L^2-inner-product}が内積の公理を満たすのは明らかであるから,
    $\iparen{f, g}_{L²}$がwell-definedであることを示す.
    コーシー・シュワルツの不等式より
    \begin{equation*}
        \abs{ \iparen{f, g} }^2
            \leq \abs*{ \int \conj{f}(𝒙) g(𝒙) \dd{x} }^2
            \leq \int_𝛺 \abs{f(𝒙)}² \dd{x}   \int_𝛺 \abs{g(𝒙)}² \dd{x} 
            \leq \norm{f}_{L²}^2 \norm{g}_{L²}^2
            < +\infty
    \end{equation*}

\end{proof}


\subsubsection*{同値類による\texorpdfstring{$L²$}{L²}-空間}

「関数としての$0$を,
(ルベーグ測度ゼロの点を除いて)ほとんどいたるところゼロである関数と再定義する」という行為は,
同値類を用いて正当化される.

$L²$上の\refdfn[同値関係]{dfn:equivalence-relation}$\sim$を
\begin{equation*}
    f \sim g  \iff  f(\symbf{x}) = g(\symbf{x}) \quad (\text{a.e. } \symbf{x} \in \Omega)
\end{equation*}
と定義する.$\sim$が同値関係になるのは明らかである.
\refdfn[商集合]{dfn:quotient-set}$L²/{\sim}$上の和とスカラー倍を,
代表元を用いて
\begin{align}
    \eqclass{f} + \eqclass{g} &\coloneq \eqclass{f + g}  &
    c \cdotp \eqclass{f} &\coloneq \eqclass{c \cdotp f}
\end{align}
と定める\footnote{
    $+$と$\cdotp$はwell-definedである(すなわち代表元のとりかたによらない).
    なぜなら,$f \sim f'$($\mu(ℱ) = 0$である集合$ℱ$の点を除き一致),
    $g \sim g'$($\mu(𝒢) = 0$である集合$𝒢$の点を除き一致)に対し,
    $f + g \sim f' + g'$(ルベーグ測度$0$の集合$ℱ \cup 𝒢$を除き一致)だからである.
}と,$L²/{\sim}$はベクトル空間をなす.
また,$L²/{\sim}$上の内積を,代表元の内積
\begin{align}
    \iparen{ \eqclass{f}, \eqclass{g} }_{L²/{\sim}}  \coloneq \int_\Omega \conj{f}(\symbf{x}) g(\symbf{x}) \dd{x}
\end{align}
で定めれば\footnote{
    $\iparen{\bigdot, \bigdot}$はwell-definedである.
    なぜなら,ルベーグ測度$0$の集合$ℱ \cup 𝒢$の点における$\conj{f}(\symbf{x}) g(\symbf{x})$の値は,
    積分$\int_\Omega \conj{f}(\symbf{x}) g(\symbf{x}) \dd{x}$に寄与しないからである.
},$L²/{\sim}$は内積空間になる.

このように定義された内積空間$L²/{\sim}$のことを,
$L²$と書いているのである.




\subsection{局所可積分}

$u$を$𝛺 \subset_{\text{open}} ℝ^n$で定義された関数とする.
$u$が$𝛺$で局所可積分であるとは,
任意の\refdfn[コンパクト集合]{dfn:compact-set}$K \subset 𝛺$に対し
\begin{equation*}
    \int_K \abs{ u(x) } \dd{x} < \infty
\end{equation*}
であることをいう.
連続関数$C^0 (𝛺)$は局所可積分である.





\end{document}