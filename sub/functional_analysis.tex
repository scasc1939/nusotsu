\documentclass[../sotsu.tex]{subfiles}


\begin{document}


\section{関数解析}

この節では関数をベクトル空間の\ruby{元}{げん}として扱う。
ベクトル空間の係数体$\symbb{K}$は$\symbb{R}$もしくは$\symbb{C}$に限る。

\subsection{関数空間}

関数$\psi(x_1, x_2, \dotsc)$のようなものを考えよう。

\begin{definition}
    $\Omega \subset \symbb{R}$を$n$次元ユークリッド空間$\symbb{R}^n$の開集合\refdfn{dfn:open-set-and-closed-set}とする。
    $\Omega$における連続関数全体の集合を$C^0 (\Omega)$でかく。
\end{definition}

\begin{proposition}
    \label{thm:continuous-function-space-is-vector-space}
    $C(\Omega)$は以下のように定義される和とスカラー倍に関して、ベクトル空間をなしている。
    \begin{itemize}
        \item $f, g \in C(\Omega)$,$c \in \symbb{K}$に対して、
        \begin{enumerate}
            \item 関数の和$f \plushat g$は,
                任意の$x \in \symbb{C}$に対して$ ( f \plushat g )(\symbf{x}) = f(x) \doubleplus g(x) $である関数と定める.
            \item 関数のスカラー倍$c \mathbin{\hat{\scaprod}} f$は,
                任意の$x \in \symbb{C}$に対して$ ( c \mathbin{\hat{\scaprod}} g )(x) = c \scaprodvar f(x) $である関数と定める.
        \end{enumerate}
    \end{itemize}
\end{proposition}

\begin{proof}
    
\end{proof}


\begin{definition}
    ある関数$u \in C(\Omega)$に対し、$u(x) \neq 0$であるような$x \in \Omega$の集合の閉包\refdfn{dfn:closure}
    \begin{equation}
        \supp u  \coloneq  \cl{ \Set{  x \in \Omega  \given  u(x) \neq 0  } }
    \end{equation}
    を$u$の\word{台}[だい][だい](support)という。
\end{definition}

関数$u \in C(\Omega)$の中で、$\supp u$がコンパクト\refdfn{dfn:compact-set}であるもの全体の集合を$C_0 (\Omega)$とかく。
$C_0 (\Omega)$もベクトル空間になっている。


\begin{definition}
    \label{dfn:class-C^k-function}
    複素関数$f$が\word{$C^k$-級関数}[][Cきゆうかんすう](class $C^k$-function)であるとは,
    $f(x)$が$\symbb{C}$上で$k$回微分可能かつ$k$次導関数が連続であることをいう.
    $f(x)$が無限回微分可能であるとき$C^\infty$-級関数であるという.

    $C^k$-級関数全体の集合を$C^k$とかく.
    特に連続関数全体の集合は$C^0$である.
\end{definition}

\begin{example}
    $e^x$,$\sin x$,$\cos x$は$C^\infty$-級関数である.
\end{example}

\begin{example}
    $C^0 \subsetneq C^1 \subsetneq C^2 \subsetneq \dotsb$であり,
    $\bigcap_{k \in \symbb{N}} C^k = C^\infty$である.
\end{example}



\subsection{関数空間のノルム}

ここでは関数空間上にノルムが定義できる場合について考える。

関数空間においてもっとも一般的なノルムは``最大値ノルム''である。

\begin{definition}
    \label{dfn:square-integrable-function-space}
    $[a, b] \subset \symbb{R}$から$\symbb{C}$への\word{2乗可積分関数空間}(square integrable function space) $L^2$を
    \begin{equation*}
        L^2 (a, b) \coloneq \Set*{ f \colon [a, b] \to \symbb{C}  \given  \int_a^b \abs{f(x)} \dd{x} < \infty }
    \end{equation*}
    で定義する.
\end{definition}





\subsection{内積空間としての関数空間}

関数空間に内積が定義できる場合。

\begin{proposition}
    2乗可積分関数空間$L^2 (a, b)$は,以下で定義される内積$(\bigdot, \bigdot)$に対して内積空間\refdfn{dfn:inner-product}になる.
    \begin{align}
        (f, g)_{L^2} \coloneq \int_a^b \conj{f}(x) g(x) \dd{x}
    \end{align}
    当然ノルム\refdfn{dfn:norm,dfn:norm-by-inner-product}も以下のように定義できる.
    \begin{align}
        \norm{f}_{L^2}^2 \coloneq (f, f)_{L^2} = \int_a^b \abs{f(x)}^2 \dd{x}
    \end{align}
\end{proposition}

\begin{definition}
    連続な2乗可積分関数空間を
    \( L^2 C^0 \coloneq L^2 \cap C^0 \)
    で定める.
\end{definition}


$u$を$\Omega \subset_{\text{open}} \symbb{R}^n$で定義された関数とする。
$u$が$\Omega$で局所可積分であるとは、
任意のコンパクト集合$K \subset \Omega$に対し
\begin{equation*}
    \int_K \abs{ u(x) } \dd{x} < \infty
\end{equation*}
であることをいう。
連続関数$C(\Omega)$は局所可積分である。





\end{document}