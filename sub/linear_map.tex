\documentclass[../sotsu.tex]{subfiles}

\begin{document}

\section{線形写像}

\subsection{線形写像}

ふたたびしばらく$\symbb{K}$を一般の体\refdfn{dfn:field}とする.

\begin{definition}[線形写像]
    \label{dfn:linear-map}
    $V, W$を体$\symbb{K}$上のベクトル空間\refdfn{dfn:vector-space}とする
    \footnote{ここでは$V$上の和を$\tplus$,スカラー倍を$\scaprod$であらわし,$W$上の和を$\doubleplus$,スカラー倍を$\scaprodvar$であらわす.
    また,体$\symbb{K}$上の和を$+$であらわす.}.
    写像$f \colon V \to W$が以下を満たすとき,
    \word{線形写像}[せん|けい|しゃ|ぞう][せんけいしやそう](linear map)
    あるいは
    \word{準同型写像}[じゅん|どう|けい|しゃ|ぞう][せんけいしやそう](homomorphism)という\cite[\S 2.1]{saito-lin-2007}.
    \begin{enumerate}
        \item 任意の$\symbf{u}, \symbf{v} \in V$に対し,$f(\symbf{u} \tplus \symbf{v}) = f(\symbf{u}) \doubleplus f(\symbf{v})$である.
        \item 任意の$c \in \symbb{K}$および任意の$\symbf{v} \in V$に対し,$f(c \scaprod \symbf{v}) = c \scaprodvar f(\symbf{v})$である.
    \end{enumerate}
    $V$から$W$への線形写像全体がなす集合を$\mathrm{Hom}_{\symbb{K}}(V, W)$とかく\cite[\S 4.4]{saito-lin-2007}.

    特に,線形写像$f \colon V \to V$を\word{自己準同型写像}[][しこしゆんとうけいしやそう](endomorphism)といい,
    $f$全体がなす集合を$\mathrm{End}_{\symbb{K}}(V)$とかく\cite[\S 2.1, \S 4.4]{saito-lin-2007}.
\end{definition}


\begin{definition}[同型写像]
    \label{dfn:isomorphism}
    $V, W$を体$\symbb{K}$上のベクトル空間とする.
    線形写像$f \colon V \to W$が可逆である,
    つまり$f \circ f^{-1} = \mathrm{id}_W$かつ$f^{-1} \circ f = \mathrm{id}_V$(恒等写像\refdfn{dfn:identity-map})となるような$f^{-1} \colon W \to V$が存在するとき,
    $f$を($\symbb{K}$上の)\word{同型写像}[どう|けい|しゃ|ぞう][とうけいしやそう](isomorphism)という.

    とくに,同型写像$f \colon V \similarrightarrow V$を\word{自己同型写像}[][しことうけいしやそう](automorphism)という\cite[\S 2.1]{saito-lin-2007}.
\end{definition}

同型写像とは要するに,全単射\refdfn{dfn:bijection}である線形写像のことである.

ベクトル空間$U, V$の間に同型写像が存在するとき,$U$と$V$を同一のベクトル空間と見做すことができる.

\begin{definition}[同型]
    \label{dfn:isomorphic}
    $V, W$を体$\symbb{K}$上のベクトル空間とする.
    同型写像$f \colon V \similarrightarrow W$が存在するとき,$V$と$W$は($\symbb{K}$上に)\word{同型}[どう|けい][とうけい](isomorphic)であるといい,$V \cong W$とかく.
\end{definition}

線形写像全体の集合$\mathrm{Hom}_{\symbb{K}}(V, W)$は,適切な和とスカラー倍の下でベクトル空間\refdfn{dfn:vector-space}をなす.

\begin{proposition}
    \label{thm:linear-map-space}
    $\mathrm{Hom}_{\symbb{K}}(V, W)$は,以下で定義される和$\plushat$とスカラー倍$\hat{\scaprod}$のもとで,体$\symbb{K}$上のベクトル空間になる\cite[\S 4.4]{saito-lin-2007}.
    \begin{itemize}
        \item $f, g \in \mathrm{Hom}_{\symbb{K}}(V, W)$,$c \in \symbb{K}$とする.
        \begin{enumerate}
            \item 和$f \plushat g \in \mathrm{Hom}_{\symbb{K}}(V, W)$は,
                任意の$\symbf{v} \in V$に対して$ ( f \plushat g )(\symbf{v}) = f(\symbf{v}) \doubleplus g(\symbf{v}) $である写像と定める.
            \item スカラー倍$c \mathbin{\hat{\scaprod}} f \in \mathrm{Hom}_{\symbb{K}}(V, W)$は,
                任意の$\symbf{v} \in V$に対して$ ( c \mathbin{\hat{\scaprod}} f )(\symbf{v}) = c \scaprodvar f(\symbf{v}) $である写像と定める.
        \end{enumerate}
    \end{itemize}
\end{proposition}



\begin{proposition}
    $\symbb{K}$上の$n$-次元ベクトル空間と$\symbb{K}^n$は同型である\cite{saito-lin-2007}.
\end{proposition}

\begin{proof}
    $\symbb{K}$上の$n$-次元ベクトル空間$V$の基底$\{\symbf{u}_1, \dots, \symbf{u}_n\}$をひとつとる.
    写像$f \colon \symbb{K}^n \similarrightarrow V$を
    \begin{equation*}
        f \colon \symbb{K}^n \ni \tp{(c_1 \quad \dotsm \quad c_n)}
            \longmapsto
            c_1 \symbf{u}_1 + \dots + c_n \symbf{u}_n \in V
    \end{equation*}
    と定めると,これは同型写像である.
    実際,$\{\symbf{u}_1, \dots, \symbf{u}_n\}$は$V$を生成する\refdfn{dfn:spanning-set}から$f$は全射であり,
    基底による展開係数は一意であるから$f$は単射である.
    $f$が線形であることも明らかである.
\end{proof}

\begin{corollary}
    $\symbb{K}$上の2つの$n$-次元ベクトル空間$U, V$は同型である.
\end{corollary}

\begin{proof}
    同型$f \colon V \similarrightarrow \symbb{K}$と同型$g \colon \symbb{K} \similarrightarrow U$の合成$g \circ f \colon V \similarrightarrow U$が同型であることから従う.
\end{proof}


\subsection{像と核}

\begin{definition}[像]
    \label{dfn:image-of-linear-map}
    \label{dfn:rank}
    $f \colon V \to W$を線形写像とする.
    $W$の部分集合
    \begin{equation}
        \image f  \coloneq  \Set{  f(\symbf{v}) \in W  \given  \symbf{v} \in V  }
    \end{equation}
    を$f$の\word{像}[ぞう][そう](image)という.

    $\image f$の次元$\rank f \coloneq \dim \image f$を,$f$の\word{階数}[かい|すう][かいすう](rank)という\cite{miyake-lin-2008}.
\end{definition}

\begin{definition}[核]
    \label{dfn:kernel-of-linear-map}
    \label{dfn:nullity}
    $f \colon V \to W$を線形写像とする.
    $V$の部分集合
    \begin{equation}
        \ker f  \coloneq  \Set{  \symbf{v} \in V  \given  f(\symbf{v}) = \symbf{0}_W  }
    \end{equation}
    を$f$の\word{核}[かく][かく](kernel)という.

    $\ker f$の次元$\dimker f \coloneq \dim \ker f$を,$f$の\word{退化次数}[たい|か|じ|すう][たいかしすう](nullity)という\cite{miyake-lin-2008}.
\end{definition}

像$\image f$は一般の写像についても定義できた\refdfn{dfn:image}が,
核$\ker f$は線形写像特有の概念である.



\begin{proposition}
    $f \colon V \to W$を線形写像とする.
    \begin{enumerate}
        \item $f$の像$\image f$は,$W$の部分ベクトル空間である.
        \item $f$の核$\ker f$は,$V$の部分ベクトル空間である.
    \end{enumerate}
\end{proposition}

\begin{proof}
    $\image f$と$\ker f$がそれぞれ,部分ベクトル空間であることの必要十分条件\refthm{thm:vector-subspace-iff}を満たすことを確認すればいい.
\end{proof}


\begin{theorem}
    \label{thm:rank-nullity}
    $V, W$を有限次元のベクトル空間とする.
    線形写像$f \colon V \to W$の像と核について,
    $\image f + \dimker f = \dim V$が成り立つ\cite[\S 5.1]{miyake-lin-2008}.
\end{theorem}

\begin{proof}
    $\ker f$の基底を$\symbf{x}_1, \dots, \symbf{x}_n$とし,
    $\symbf{x}_{n+1}, \dots, \symbf{x}_{n+m}$を
    $f(\symbf{x}_{n+1}) = \symbf{y}_1, \dots, f(\symbf{x}_{n+m}) = \symbf{y}_m$
    ($\symbf{y}_1, \dots, \symbf{y}_m$は$\image f$の基底)となるようにとる.
    このとき$\symbf{x}_1, \  \dotsc, \  \symbf{x}_{n+m}$が$V$の基底になる.

    まず,\textgt{$\symbf{x}_1, \dots, \symbf{x}_{n+m}$は一次独立である}ことを示す.
    \begin{equation*}
        c_1 \symbf{x}_1 + \dots + a_n \symbf{x}_n
            + c_{n+1} \symbf{x}_{n+1} + \dots + c_{n+m} \symbf{x}_{n+m}
                = \symbf{0}
    \end{equation*}
    とおく.これに線形写像$f$を施すと,
    \begin{equation*}
        c_{n+1} \symbf{y}_{1} + \dots + c_{n+m} \symbf{y}_{m} = \symbf{0}
    \end{equation*}
    となり,$\symbf{y}_1, \dots, \symbf{y}_m$は一次独立であるから$c_{n+1} = \dots = c_{n+m} = 0$である.
    したがって
    \begin{equation*}
        c_1 \symbf{x}_1 + \dots + c_n \symbf{x}_n = \symbf{0}
    \end{equation*}
    であるから$c_1 = \dots = c_n = 0$である.
    よって$c_1 = \dots = c_{n+m} = 0$がいえた.

    次に\textgt{$\symbf{x}_1, \dots, \symbf{x}_{n+m}$は$V$を生成する}ことを示す.
    任意の$\symbf{x} \in V$をとる.
    $f(\symbf{x}) \in \image f$であるので,$\image f$の基底を用いて
    \begin{equation*}
        f(\symbf{x}) = c_{n+1} \symbf{y}_1 + \dots + c_{n+m} \symbf{y}_m
    \end{equation*}
    とかける.$\symbf{y}_i = f(\symbf{x}_{n+i})$を用いると
    \begin{equation*}
        f(\symbf{x} - c_{n+1} \symbf{x}_{n+1} - \dots - c_{n+m} \symbf{x}_{n+m}) = \symbf{0}
    \end{equation*}
    であるので,$\symbf{x} - c_{n+1} \symbf{x}_{n+1} - \dots - c_{n+m} \symbf{x}_{n+m} \in \ker f$である.
    よって
    \begin{equation*}
        \symbf{x} - c_{n+1} \symbf{x}_{n+1} - \dots - c_{n+m} \symbf{x}_{n+m} 
            = c_1 \symbf{x}_1 + \dots + c_n \symbf{x}_n
    \end{equation*}
    とかける.この式を変形すれば,$\symbf{x}$が$\symbf{x}_1, \dots, \symbf{x}_{n+m}$の線形結合であらわされた.
\end{proof}





\end{document}