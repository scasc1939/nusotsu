\documentclass[../sotsu.tex]{subfiles}

\begin{document}

\section{線形写像}
\label{sec:linear-map}

\cref{sec:coordinate-space}では,行列による数ベクトルの変換を扱った.
これを一般化したものが線形写像という概念である.

\subsection{線形写像}

ふたたびしばらく$𝕂$を一般の\refdfn[体]{dfn:field}とする.

\begin{definition}[線形写像]
    \label{dfn:linear-map}
    $V, W$を体$𝕂$上の\refdfn-[ベクトル空間]{dfn:vector-space}とする
    \footnote{ここでは$V$上の和を$\tplus$,スカラー倍を$\scaprod$であらわし,$W$上の和を$\doubleplus$,スカラー倍を$\scaprodvar$であらわす.
    また,体$𝕂$上の和を$+$であらわす.}.
    写像$f \colon V \to W$が以下を満たすとき,
    \word{線形写像}[せん|けい|しゃ|ぞう](linear map)%
    \index{せんけい@線形!しやそう@写像}%
    あるいは%
    \word{準同型写像}[じゅん|どう|けい|しゃ|ぞう](homomorphism)%
    \index{しゆんとうけい@準同型!しやそう@写像}%
    という\cite[\S 2.1]{saito-lin-2007}.
    \begin{enumerate}
        \item 任意の$𝒖, 𝒗 \in V$に対し,$f(𝒖 \tplus 𝒗) = f(𝒖) \doubleplus f(𝒗)$である.
        \item 任意の$c \in 𝕂$および任意の$𝒗 \in V$に対し,$f(c \scaprod 𝒗) = c \scaprodvar f(𝒗)$である.
    \end{enumerate}
    $V$から$W$への線形写像全体がなす集合を$\mathrm{Hom}_{𝕂}(V, W)$とかく\cite[\S 4.4]{saito-lin-2007}.

    特に,線形写像$f \colon V \to V$を%
    \word{自己準同型写像}(endomorphism)%
    \index{しこしゆんとうけいしやそう@自己準同型写像}%
    といい,
    $f$全体がなす集合を$\mathrm{End}_{𝕂}(V)$とかく\cite[\S 2.1, \S 4.4]{saito-lin-2007}.
\end{definition}


線形写像全体の集合$\mathrm{Hom}_{𝕂}(V, W)$は,適切な和とスカラー倍の下でベクトル空間\refdfn{dfn:vector-space}をなす.

\begin{proposition}
    \label{thm:linear-map-space}
    $\mathrm{Hom}_{𝕂}(V, W)$は,以下で定義される和$\plushat$とスカラー倍$\hat{\scaprod}$のもとで,体$𝕂$上のベクトル空間になる\cite[\S 4.4]{saito-lin-2007}.
    \begin{itemize}
        \item $f, g \in \mathrm{Hom}_{𝕂}(V, W)$,$c \in 𝕂$に対し,
        \begin{enumerate}
            \item 和$f \plushat g \in \mathrm{Hom}_{𝕂}(V, W)$を,
                任意の$𝒗 \in V$に対して$ ( f \plushat g )(𝒗) = f(𝒗) \doubleplus g(𝒗) $である写像と定める.
            \item スカラー倍$c \mathbin{\hat{\scaprod}} f \in \mathrm{Hom}_{𝕂}(V, W)$を,
                任意の$𝒗 \in V$に対して$ ( c \mathbin{\hat{\scaprod}} f )(𝒗) = c \scaprodvar f(𝒗) $である写像と定める.
        \end{enumerate}
    \end{itemize}
\end{proposition}

ベクトル空間のときと同様,
通常は和の記号として$+$を使い,スカラー倍の記号は省略する.



\begin{proposition}
    \label{thm:linear-map-composition}
    $U, V, W$を$\symbb{K}$上のベクトル空間,
    $f, f' \colon U \to V$,
    $g, g' \colon V \to W$を線形写像,
    $c \in \symbb{K}$をスカラーとする.
    線形写像の\refdfn[合成]{dfn:map-composition}について,
    以下が成り立つ.
    \begin{enumerate}
        \item $(f + f') \circ g = f \circ g + f' \circ g$
        \item $f \circ (g + g') = f \circ g + f \circ g'$
        \item $(c f) \circ g = f \circ (c g) = c (f \circ g)$
    \end{enumerate}
\end{proposition}

\begin{proof}
    \cref{dfn:map-composition}に従って計算すればよい.
\end{proof}


\begin{definition}[零写像]
    \label{dfn:zero-linear-map}
    ベクトル空間$V$から$W$への写像で,
    すべての$\symbf{v} \in V$を零ベクトル$\symbf{0} \in W$にうつす写像を\word{零写像}(zero map)\index{れい@零!しやそう@\indexdash 写像}という.
\end{definition}




\subsection{同型写像}


\begin{definition}[同型写像]
    \label{dfn:isomorphism}
    $V, W$を体$𝕂$上のベクトル空間とする.
    線形写像$f \colon V \to W$が可逆である,
    つまり$f \circ f^{-1} = \mathrm{id}_W$かつ$f^{-1} \circ f = \mathrm{id}_V$(\refdfn-[恒等写像]{dfn:identity-map})となるような$f^{-1} \colon W \to V$が存在するとき,
    $f$を($𝕂$上の)%
    \word{同型写像}[どう|けい|しゃ|ぞう](isomorphism)%
    \index{とうけい@同型!しやそう@\indexdash 写像}
    という.

    とくに,同型写像$f \colon V \similarrightarrow V$を%
    \word{自己同型写像}(automorphism)%
    \index{しことうけいしやそう@自己同型写像}%
    という\cite[\S 2.1]{saito-lin-2007}.
\end{definition}

同型写像とは要するに,\refdfn-[全単射]{dfn:bijection}である線形写像のことである.

ベクトル空間$U, V$の間に同型写像が存在するとき,$U$と$V$を同一のベクトル空間と見做すことができる.

\begin{definition}[同型]
    \label{dfn:isomorphic}
    $V, W$を体$𝕂$上のベクトル空間とする.
    同型写像$f \colon V \similarrightarrow W$が存在するとき,$V$と$W$は($𝕂$上に)\word{同型}[どう|けい][とうけい](isomorphic)であるといい,$V \cong W$とかく.
\end{definition}



\begin{proposition}
    \label{thm:finite-dimensional-space-isomorphic}
    $𝕂$上の$n$-次元ベクトル空間と$𝕂^n$は同型である\cite{saito-lin-2007}.
\end{proposition}

\begin{proof}
    $𝕂$上の$n$-次元ベクトル空間$V$の基底$\{𝒖_1, \dots, 𝒖_n\}$をひとつとる.
    写像$f \colon 𝕂^n \similarrightarrow V$を
    \begin{equation*}
        f \colon 𝕂^n \ni \tp{(c_1 \: \dotsm \: c_n)}
            \longmapsto
            c_1 𝒖_1 + \dots + c_n 𝒖_n \in V
    \end{equation*}
    と定めると,これは同型写像である\footnote{
        この$f$を,基底$\{𝒖_1, \dots, 𝒖_n\}$が定める同型という\cite[\S 2.2]{saito-lin-2007}.
    }.
    実際,$\{𝒖_1, \dots, 𝒖_n\}$は$V$を\refdfn[生成する]{dfn:spanning-set}から$f$は全射であり,
    基底による展開係数は一意であるから$f$は単射である.
    $f$が線形であることも明らかである.
\end{proof}

\begin{corollary}
    $𝕂$上の2つの$n$-次元ベクトル空間$U, V$は同型である.
\end{corollary}

\begin{proof}
    同型$f \colon V \similarrightarrow 𝕂$と同型$g \colon 𝕂 \similarrightarrow U$の合成$g \circ f \colon V \similarrightarrow U$が同型であることから従う.
\end{proof}




\subsection{像と核}

\begin{definition}[像]
    \label{dfn:image-of-linear-map}
    \label{dfn:rank}
    $f \colon V \to W$を線形写像とする.
    $W$の部分集合
    \begin{equation}
        \image f  \coloneq  \Set{  f(𝒗) \in W  \given  𝒗 \in V  }
    \end{equation}
    を$f$の\word{像}[ぞう](image)%
    \index{そう@像}という.

    $\image f$の次元$\rank f \coloneq \dim \image f$を,
    $f$の\word{階数}[かい|すう](rank)%
    \index{かいすう@階数}という\cite{miyake-lin-2008}.
\end{definition}

\begin{definition}[核]
    \label{dfn:kernel-of-linear-map}
    \label{dfn:nullity}
    $f \colon V \to W$を線形写像とする.
    $V$の部分集合
    \begin{equation}
        \ker f  \coloneq  \Set{  𝒗 \in V  \given  f(𝒗) = \symbf{0}_W  }
    \end{equation}
    を$f$の\word{核}[かく](kernel)%
    \index{かく@核}という.

    $\ker f$の次元$\dimker f \coloneq \dim \ker f$を,
    $f$の\word{退化次数}[たい|か|じ|すう](nullity)%
    \index{たいかしすう@退化次数}という\cite{miyake-lin-2008}.
\end{definition}

像$\image f$は一般の写像についても定義できた\refdfn{dfn:image}が,
核$\ker f$は線形写像特有の概念である.



\begin{proposition}
    $f \colon V \to W$を線形写像とする.
    \begin{enumerate}
        \item $f$の\refdfn-[像]{dfn:image-of-linear-map}$\image f$は,$W$の部分ベクトル空間である.
        \item $f$の\refdfn-[核]{dfn:kernel-of-linear-map}$\ker f$は,$V$の部分ベクトル空間である.
    \end{enumerate}
\end{proposition}

\begin{proof}
    $\image f$と$\ker f$がそれぞれ,部分ベクトル空間であることの必要十分条件\refthm{thm:vector-subspace-iff}を満たすことを確認すればいい.
\end{proof}


\begin{theorem}
    \label{thm:rank-nullity}
    $V, W$を有限次元のベクトル空間とする.
    線形写像$f \colon V \to W$の像と核について,
    $\image f + \dimker f = \dim V$が成り立つ\cite[\S 5.1]{miyake-lin-2008}.
\end{theorem}

\begin{proof}
    $\ker f$の基底を$𝒙_1, \dots, 𝒙_n$とし,
    $𝒙_{n+1}, \dots, 𝒙_{n+m}$を
    $f(𝒙_{n+1}) = 𝒚_1, \dots, f(𝒙_{n+m}) = 𝒚_m$
    ($𝒚_1, \dots, 𝒚_m$は$\image f$の基底)となるようにとる.
    このとき$𝒙_1, \  \dotsc, \  𝒙_{n+m}$が$V$の基底になる.

    まず,\textgt{$𝒙_1, \dots, 𝒙_{n+m}$は一次独立である}ことを示す.
    \begin{equation*}
        c_1 𝒙_1 + \dots + c_n 𝒙_n
            + c_{n+1} 𝒙_{n+1} + \dots + c_{n+m} 𝒙_{n+m}
                = \symbf{0}
    \end{equation*}
    とおく.これに線形写像$f$を施すと,
    \begin{equation*}
        c_{n+1} 𝒚_{1} + \dots + c_{n+m} 𝒚_{m} = \symbf{0}
    \end{equation*}
    となり,$𝒚_1, \dots, 𝒚_m$は一次独立であるから$c_{n+1} = \dots = c_{n+m} = 0$である.
    したがって
    \begin{equation*}
        c_1 𝒙_1 + \dots + c_n 𝒙_n = \symbf{0}
    \end{equation*}
    であるから$c_1 = \dots = c_n = 0$である.
    よって$c_1 = \dots = c_{n+m} = 0$がいえた.

    次に\textgt{$𝒙_1, \dots, 𝒙_{n+m}$は$V$を生成する}ことを示す.
    任意の$𝒙 \in V$をとる.
    $f(𝒙) \in \image f$であるので,$\image f$の基底を用いて
    \begin{equation*}
        f(𝒙) = c_{n+1} 𝒚_1 + \dots + c_{n+m} 𝒚_m
    \end{equation*}
    とかける.$𝒚_i = f(𝒙_{n+i})$を用いると
    \begin{equation*}
        f(𝒙 - c_{n+1} 𝒙_{n+1} - \dots - c_{n+m} 𝒙_{n+m}) = \symbf{0}
    \end{equation*}
    であるので,$𝒙 - c_{n+1} 𝒙_{n+1} - \dots - c_{n+m} 𝒙_{n+m} \in \ker f$である.
    よって
    \begin{equation*}
        𝒙 - c_{n+1} 𝒙_{n+1} - \dots - c_{n+m} 𝒙_{n+m} 
            = c_1 𝒙_1 + \dots + c_n 𝒙_n
    \end{equation*}
    とかける.この式を変形すれば,$𝒙$が$𝒙_1, \dots, 𝒙_{n+m}$の線形結合であらわされた.
\end{proof}


\begin{proposition}
    \label{thm:linear-map-injective}
    線形写像$f$について,以下は同値.
    \begin{enumerate}
        \item $f$は単射.
        \item $f$の\refdfn-[核]{dfn:kernel-of-linear-map}はゼロベクトルのみ,
            つまり$\ker f = \{ \symbf{0} \}$
    \end{enumerate}
\end{proposition}

\begin{proof}
    線形写像を$f \colon V \to W$とする.

    \bluehead{必要} \quad 
    任意の線形写像$f$に対して$f(\symbf{0}_V) = \symbf{0}_W$である.
    $f$は単射であるから,$f$の核は$\symbf{0}_V$のみである.

    \bluehead{十分} \quad 
    $𝒗, 𝒘 \in V$に対し,$f(𝒗) = f(𝒘)$であったする.
    $f$は線形写像なので,$f(𝒗 - 𝒘) = \symbf{0}_W$.
    したがって$𝒗 - 𝒘 \in \ker f$であるが,
    $\ker f = \{ \symbf{0}_V \}$なので$𝒗 - 𝒘 = \symbf{0}_V$.
\end{proof}



\subsection{表現行列}
\label{representation-matrix}

$V$を$\symbb{K}$上の$n$-次元ベクトル空間,
$W$を$\symbb{K}$上の$m$-次元ベクトル空間とする.
任意のベクトル$\symbf{v} \in V$と$\symbf{w} \in W$は,
\begin{equation*}
    \begin{aligned}
        \symbf{v} &= x_1 \symbf{v}_1 + \dots + x_n \symbf{v}_n  \\
        \symbf{w} &= y_1 \symbf{w}_1 + \dots + y_n \symbf{w}_m
    \end{aligned}
\end{equation*}
とかける.
ここで,$\symbf{v}_1, \dots, \symbf{v}_n$は$V$の基底,
$\symbf{w}_1, \dots, \symbf{w}_m$は$W$の基底である.
いま,線形写像$f \colon V \to W$により,
\begin{equation*}
    \symbf{w} = f(\symbf{v})
\end{equation*}
という関係を定める.


ところで,\cref{thm:finite-dimensional-space-isomorphic}より,
$V$と$\symbb{K}^n$,$W$と$\symbb{K}^m$はそれぞれ\refdfn-[同型]{dfn:isomorphic}であるから,
よって,全単射な線形写像$\varphi \colon \symbb{K}^n \to V$,$\varphi' \colon \symbb{K}^m \to W$が存在して,
\begin{alignat*}{3}
    \varphi  &\colon 
        \tp{ (x_1 \  \cdots \  x_n) }
        = x_1 \symbf{e}_1  + \dots + x_n \symbf{e}_n
        &&\longmapsto x_1 \symbf{v}_1 + \dots + x_n \symbf{v}_n,
    \\
    \varphi' &\colon \tp{ (y_1 \  \cdots \  y_m) }
        = y_1 \symbf{e}'_1 + \dots + y_m \symbf{e}'_m
        &&\longmapsto y_1 \symbf{w}_1 + \dots + y_m \symbf{w}_m
\end{alignat*}
したがって,線形写像$f \colon V \to W$を考える代わりに,
線形写像$\hat{A} \colon \symbb{K}^n \to \symbb{K}^m$を考えることができそうである.
ここで,$\hat{A}$は左から$m \times n$-行列$A$をかける写像とできる.
なぜなら,$\symbb{K}^n$の\refdfn[標準基底]{dfn:canonical-basis-of-coordinate-space}$\symbf{e}_1, \dots, \symbf{e}_n$に対する変換
\begin{equation*}
    \hat{A} (\symbf{e}_j) = a_{1j} \symbf{e}'_1 + \dots + a_{mj} \symbf{e}'_m
    \quad 
    (j = 1, \dots, n)
\end{equation*}
($\symbf{e}'_i$は$\symbb{K}^m$の標準基底)を調べて,
行列$A$を
\begin{equation*}
    A \coloneq 
    \begin{pmatrix}
        a_{11}  &  \cdots  &  a_{1n}  \\
        \vdots  &  \ddots  &  \vdots  \\
        a_{m1}  &  \cdots  &  a_{mn}
    \end{pmatrix}
\end{equation*}
と定義すれば,
一般の$\symbf{x} = x_1 \symbf{e}_1 + \dots + x_n \symbf{e}_n$に対し,
\begin{equation*}
    \begin{split}
        \symbf{y} 
            = A \symbf{x}
            &= x_1 A \symbf{e}_1 + \dots + x_n A \symbf{e}_n
            \\
            &= x_1 \hat{A} (\symbf{e}_1) + \dots + x_n \hat{A} (\symbf{e}_n)
            = \hat{A} (\symbf{x})
    \end{split}
\end{equation*}
$\symbf{y} = A \symbf{x}$をあらわにかくと,
\begin{equation*}
    \begin{pmatrix}
        y_1  \\  \vdots  \\  y_m
    \end{pmatrix}
    =
    A 
    \begin{pmatrix}
        x_1  \\  \vdots  \\  x_n
    \end{pmatrix}
    =
    \begin{pmatrix}
        a_{11} & \cdots & a_{1n}  \\
        \vdots & \ddots & \vdots  \\
        a_{m1} & \cdots & a_{mn}
    \end{pmatrix}
    \begin{pmatrix}
        x_1  \\  \vdots  \\  x_n
    \end{pmatrix}
\end{equation*}

形式的には,以下のように定義する.

\begin{definition}
    $V, W$を$\symbb{K}$上の有限次元\refdfn-[ベクトル空間]{dfn:vector-space},
    $\symbf{v}_1, \dots, \symbf{v}_n, \  \symbf{w}_1, \dots, \symbf{w}_m$をそれぞれ$V, W$の基底とする.
    $\varphi  \colon \symbb{K}^n \to V$,
    $\varphi' \colon \symbb{K}^m \to W$を基底が定める同型,
    つまり
    \begin{gather*}
        \varphi (\symbf{e}_j ) = \symbf{v}_j ,
        \quad 
        \varphi'(\symbf{e}'_i) = \symbf{w}_i
        \\
        \text{where } 
        \symbf{e}_j  = \tp{( 0 \  \dotsm \  \overset{j}{\breve{1}} \  \dotsm \overset{n}{\breve{0}} )} \in \symbb{K}^n ,
        \quad
        \symbf{e}'_i = \tp{( 0 \  \dotsm \  \overset{i}{\breve{1}} \  \dotsm \overset{m}{\breve{0}} )} \in \symbb{K}^m
    \end{gather*}
    \refdfn-[線形写像]{dfn:linear-map}$f \colon V \to W$に対し,
    $\varphi'^{-1} \circ f \circ \varphi \colon \symbb{R}^n \to \symbb{R}^m$は$m \times n$-行列$A$を用いて
    \begin{equation}
        \label{eq:representation-matrix-definition}
        \varphi'^{-1} \circ f \circ \varphi (\symbf{v})
            = A \symbf{v}
    \end{equation}
    とかける.
    $A$を\word{表現行列}(representation matrix)\index{ひようけんきようれつ@表現行列},
    あるいは行列表示\index{きようれつひようし@行列表示}という.
\end{definition}

よって,
線形写像$f$の性質を調べるには,
$\symbf{w} = f(\symbf{v})$を直接扱う代わりに,
行列のかけ算$\symbf{y} = A \symbf{x}$について考察すればよい.

なお,表現行列の形は$V$の基底と$W$の基底をどう取るか\footnote{
    つまり,同型写像$\varphi, \varphi'$の取り方.
}に依存する.
したがって,表現行列を扱うときは,
それがどの基底に対するものなのかを常に意識しなければならない.
特に,基底の順序を変えただけでも表現行列が変わるので,
注意が必要である.

\begin{proposition}
    $f, g$を$\symbb{K}$上の有限次元ベクトル空間$V$から$W$への線形写像とし,
    $c \in \symbb{K}$をスカラーとする.
    $V$の基底と$W$の基底を固定し,
    その基底に対する$f$の表現行列を$A$,$g$の表現行列を$B$とする.
    このとき,
    \begin{enumerate}
        \item $f + g$の表現行列は$A + B$である.
        \item $cf$の表現行列は$cA$である.
    \end{enumerate}
\end{proposition}

\begin{proof}
    \cref{eq:representation-matrix-definition}と\cref{thm:linear-map-composition}を使えば容易に示される.
\end{proof}


\begin{theorem}
    $U, V, W$をそれぞれ$\symbb{K}$上のベクトル空間とし,
    $f \colon U \to V$,$g \colon V \to W$を線形写像とする.
    $U, V, W$の基底$B_U, B_V, B_W$を固定し,
    $B_U, B_V$に対する$f$の表現行列を$A$,
    $B_V, B_W$に対する$g$の表現行列を$B$とする.
    このとき,基底$B_U, B_W$に対する$g \circ f$の表現行列は$BA$である.
\end{theorem}

\begin{proof}
    $\varphi   \colon \symbb{K}^n \to U$,
    $\varphi'  \colon \symbb{K}^m \to V$,
    $\varphi'' \colon \symbb{K}^l \to W$をそれぞれ,
    基底$B_U, B_V, B_W$が定める同型とする.
    線形写像$g \circ f$の基底$B_U, B_W$に対する表現行列は,
    \cref{eq:representation-matrix-definition}より
    \begin{equation*}
        \begin{split}
            \varphi''^{-1} \circ (g \circ f) \circ \varphi (\symbf{u})
            &= \varphi''^{-1} \circ (g \circ \varphi' \circ \varphi'^{-1} \circ f) \circ \varphi (\symbf{u})
            \\
            &= (\varphi''^{-1} \circ g \circ \varphi') \circ (\varphi'^{-1} \circ f \circ \varphi) (\symbf{u})
            \\
            &= B (A \symbf{u})
            = (BA)\symbf{u}
        \end{split}
    \end{equation*}
    最後の等号では\cref{thm:composition-of-transformation-of-Euclidean-vector}を使った.
\end{proof}


扱うベクトル空間を有限次元に限ってしまえば,
線形写像はすべて単なる行列に化けてしまうのである.
これが線形代数学が強力たるゆえんである.



\subsection{固有値と固有ベクトル}
\label{sec:eigenvalue-and-eigenvector}







\end{document}