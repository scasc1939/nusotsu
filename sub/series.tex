\documentclass[../sotsu.tex]{subfiles}

\begin{document}


\section{級数}
\label{sec:series}


\subsection{関数の列}

わかりやすくするために,
実数から実数への関数を考える.

\begin{definition}
    \label{dfn:pointwise-convergence}
    関数の列$\sequ{f_n}[n \in \symbb{N}]$が関数$f$に%
    \word{各点収束}[かく|てん|しゅう|そく]%
    \index{かくてんしゆうそく@各点収束}%
    するとは,
    任意の$x \in \symbb{R}$について,
    任意の$\varepsilon > 0$に対し,
    ある$N \in \symbb{N}$が存在して,
    $n > N$なら$\abs{f(x) - f_n (x)} < \varepsilon$となることをいう.
\end{definition}

各点収束の定義は,
「任意の$x \in \symbb{R}$に対し,
$\lim_{n \to \infty} f_n (x) = f(x)$」とかける.

\begin{definition}
    \label{dfn:uniform-convergence}
    関数の列$\sequ{f_n}[n \in \symbb{N}]$が関数$f$に%
    \word{一様収束}[いち|よう|しゅう|そく]%
    \index{いちようしゆうそく@一様収束}%
    するとは,
    任意の$\varepsilon > 0$に対し,
    ある$N \in \symbb{N}$が存在して,
    任意の$x \in \symbb{R}$に対し,
    $n > N$なら$\abs{f(x) - f_n (x)} < \varepsilon$となることをいう.
\end{definition}

2つの定義はよく似ているが,
「任意の$x \in \symbb{R}$」の位置が違う.
$\varepsilon$を固定したとき,
各点収束では$N$が$x$に依存する数$N_{\varepsilon, x}$でよい.
しかし,一様収束では$N$が$x$に依存しない($\varepsilon$のみに依存する)定数$N_\varepsilon$でなくてはならない.




\end{document}