\documentclass[../sotsu.tex]{subfiles}

\begin{document}


\section{双対空間}

\subsection{双対空間}

\begin{definition}[線形汎関数]
    $V$を$𝕂$上のベクトル空間\refdfn{dfn:vector-space}とする.
    線形写像\refdfn{dfn:linear-map}$f \colon V \to 𝕂$を
    \word{線形形式}[][せんけいけいしき](linear form)
    あるいは\word{線形汎関数}[せん|けい|はん|かん|すう](linear functional)%
    \index{せんけい@線形!はんかんすう@\indexdash 汎関数|see{汎関数}}%
    \index{はんかんすう@汎関数!せんけい@線形\indexdash }という.
\end{definition}


たとえば,$V = ℝ^3$(3次元ユークリッド空間)とする.
$ℝ^3$のベクトルをその第1成分にうつす写像
$f \colon 𝒗 \longmapsto v_1$は$V$の線形汎関数である.



\begin{definition}[代数的な双対空間]
    \label{dfn:dual-space}
    $V$における線形汎関数全体の集合
    \begin{equation}
        \dual{V}  \coloneq  \Set{  f \colon V \to 𝕂  \given  \text{$f$は線形}  }
    \end{equation}
    を$V$の(代数的)\word{双対空間}[そう|つい|くう|かん](dual space)\index{そうつい@双対!くうかん@\indexdash 空間}という\cite[\S 4.1]{saito-lin-2007}.
\end{definition}


\begin{proposition}
    ベクトル空間$V$に対し,
    その双対空間$\dual{V}$は,ベクトル空間をなす\cite[\S 4.1]{saito-lin-2007}.
\end{proposition}

\begin{proof}
    $\dual{V}$の元が,\cref{thm:linear-map-space}で定義される和とスカラー倍について,
    \cref{dfn:vector-space}の公理を満たすことを示せばよい.
\end{proof}


\begin{definition}[双対基底]
    \label{dfn:dual-basis}
    $V$を$𝕂$上の有限次元ベクトル空間,$𝒙_1, \dots, 𝒙_n$を$V$の基底とすると,
    各$i = 1, \dots, n$に対し
    \begin{equation}
        \label{eq:dual-basis}
        f_i (𝒙_j) = \kdelta_{i, j} \quad \text{\refdfn{dfn:Kronecker-delta}}
    \end{equation}
    を満たす線形汎関数$f_i \colon V \to 𝕂$がただひとつ存在する.
    このとき,$f_1, \dots, f_n$は$\dual{V}$の基底になる.

    \cref{eq:dual-basis}で定義される$f_1, \dots, f_n$を$𝒙_1, \dots, 𝒙_n$の\word{双対基底}[そう|つい|き|てい](dual basis)\index{そうつい@双対!きてい@\indexdash 基底}という.
\end{definition}

このような定義が可能であることについては証明を要するが,ここでは省略する.

\begin{example}
    $V = ℝ^n$とし,ある$\symbf{a} \in V$をとる.
    線形汎関数$f_{\symbf{a}} \colon V \to ℝ$を
    \[  f_{\symbf{a}} \colon V \ni 𝒗 \longmapsto \tp{\symbf{a}} \cdotp 𝒗 \in ℝ  \]
    で定める.
    $ℝ^n$の標準基底$\symbf{e}_1, \dots, \symbf{e}_n$に対する双対基底は,
    $f_{\symbf{e}_1}, \dots, f_{\symbf{e}_n}$である.
    これは行ベクトル$\tp{\symbf{e}_1}, \dots, \tp{\symbf{e}_n}$とみることができる..
\end{example}



% \subsection{零化空間}


% \begin{definition}[零化空間]
%     $V$をベクトル空間,$\dual{V}$をその双対空間とする.
%     $V$の部分ベクトル空間$W$に対し,
%     \begin{equation}
%         W^\perp  \coloneq  \Set{  f \in \dual{V}  \given  f(W) = \{ 0 \}  }
%     \end{equation}
% を$W$の\word{零化空間}[れい|か|くう|かん][れいかくうかん](annihilator)という\cite[\S 4.2]{saito-lin-2007}.
% \end{definition}

% \begin{proposition}
%     部分ベクトル空間$W \subset V$の零化空間$W^\perp$は,$\dual{V}$の部分ベクトル空間である.
% \end{proposition}

% \begin{proof}
%     写像$\dual{i}$を,線形汎関数$f \colon V \to 𝕂$を$W$に制限する写像と定める:
%     \[  \dual{i} \colon \dual{V} \ni f \longmapsto f \vert_W \in \dual{W}  \]
%     このとき,任意の$f, g \in \dual{V}$および$c \in 𝕂$について
%     $\dual{i}(f + g) = \dual{i}(f) + \dual{i}(g)$かつ$\dual{i}(cf) = c \cdotp \dual{i}(f)$であるので,
%     $\dual{i}$は線形写像である.
%     すると$W^\perp = \ker( \dual{i} \colon \dual{V} \to \dual{W} )$\refdfn{dfn:kernel-of-linear-map}とかけるので,
%     $W^\perp$は$\dual{V}$の部分ベクトル空間である.
% \end{proof}



\subsection{双対写像}

双対空間から双対空間への写像を定義する.

\begin{definition}
    \label{dfn:dual-map}
    $V, W$を$𝕂$上のベクトル空間とし,
    $f \colon V \to W$を線形写像とする.
    $W$の双対空間の元$g \in \dual{W}$を$V$の線形形式$g \circ f \colon V \to 𝕂$にうつす写像
    $\dual{f} \colon \dual{W} \to \dual{V}$を,
    $f$の\word{双対写像}\index{そうつい@双対!しやそう@\indexdash 写像}という\cite[\S 4.3]{saito-lin-2007}.
\end{definition}

双対写像$\dual{f}$は線形写像である\cite[\S 4.3]{saito-lin-2007}.

\begin{proposition}
    \label{thm:dual-map-property}
    $U, V, W$をベクトル空間,$c \in 𝕂$を体,
    線形写像$f, g \colon U \to V$,$h \colon V \to W$とする.
    双対写像について,以下が成り立つ\cite[\S 4.3]{saito-lin-2007}.
    \begin{enumerate}
        \item $V$上の恒等写像の双対は,$\dual{V}$上の恒等写像である.
            すなわち$\dual{(\mathrm{id}_V)} = \mathrm{id}_{\dual{V}}$が成り立つ.
        \item $\dual{(\bigdot)}$は線形的である.
            すなわち$\dual{(f + g)} = \dual{f} + \dual{g}$であり,$\dual{(cf)} = c\dual{f}$である.
        \item 合成写像の双対は,順序を逆にする.
            すなわち$\dual{(f \circ h)} = \dual{h} \circ \dual{f}$である.
    \end{enumerate}
\end{proposition}

\begin{proof}
    任意の$j \in \dual{V}$あるいは$k \in \dual{W}$をとって証明する.
    \begin{enumerate}
        \item $\dual{(\mathrm{id}_V)}(j) = j \circ \mathrm{id}_V = j$である.
        \item $\dual{(f + g)}(j) = j \circ (f + g) = j \circ f + j \circ g = \dual{f}(j) + \dual{g}(j)$である.
            また,$\dual{(cf)}(j) = j \circ (cf) = c \cdotp (j \circ f) = c\dual{f}$である.
        \item \(
            \dual{(f \circ g)}(j) = j \circ (f \circ g) = (j \circ f) \circ g = \dual{f} \circ g 
                = \dual{h}( \dual{f}(j) ) = (\dual{h} \circ \dual{f})(j)
            \)である.
    \end{enumerate}
\end{proof}





\subsection{位相ベクトル空間における線形汎関数}

ベクトル空間に位相を導入した空間------たとえば,内積空間------を扱う上では,
線形汎関数を連続なものに限るのが都合がよい.
そこで,\refdfn[双対空間]{dfn:dual-space}を次のように定義しなおそう.

\begin{definition}[連続的な双対空間]
    \label{dfn:continuous-dual-space}
    $\symbb{K}$上のベクトル空間$V$における\emph{連続な}線形汎関数全体の集合
    \begin{equation}
        \cdual{V}  \coloneq  \Set{  f \colon V \to 𝕂  \given  \text{$f$は連続かつ線形}  }
    \end{equation}
    を,$V$の(連続的)\word{双対空間}(dual space)\index{そうつい@双対!くうかん@\indexdash 空間}という.
\end{definition}



\cref{sec:inner-product-space}で扱った内積は,
実は線形汎関数の一種である.

\begin{proposition}
    $V$を内積空間\refdfn{dfn:inner-product},$𝒖 \in V$をあるベクトルとする.
    任意のベクトル$𝒗 \in V$に対して内積$\iparen{𝒖, 𝒗}$を与える写像
    \begin{equation}
        \iparen{𝒖, \bigdot} \colon V \ni 𝒗 \longmapsto \iparen{𝒖, 𝒗} \in ℝ
    \end{equation}
    は連続な線形汎関数である.
\end{proposition}

\begin{proof}
    内積の線形性より$\iparen{𝒖, \bigdot}$が線形なのは明らかである.
    連続性は\cref{thm:inner-product-continuity}よりいえる.
\end{proof}



\begin{proposition}
    ノルム空間$V$上の線形汎関数$f$について,
    $f$が有界であることと連続であることは同値である\cite[\S 3.4]{iwanami-functional}.
\end{proposition}


内積が線形汎関数であることはほぼ明らかである.
真に驚くべきことは,すべての有界な線形汎関数が内積の形でかけることである.


\begin{theorem}[\word{リースの表現定理}(Riesz representation theorem)]
    \label{thm:Riesz-representation-theorem}
    $V$を内積空間とする.
    任意の有界線形汎関数$F \colon V \to 𝕂$は,
    あるベクトル$\symbf{a} \in V$を用いて
    \begin{equation}
        F(\symbf{v}) \coloneq \iparen{ \symbf{a}, \symbf{v} }
    \end{equation}
    とかける.
\end{theorem}





\end{document}