\documentclass[../sotsu.tex]{subfiles}

\begin{document}


\section{双対空間}

\subsection{双対空間}

\begin{definition}[線形汎関数]
    $V$を$\symbb{K}$上のベクトル空間\refdfn{dfn:vector-space}とする。
    線形写像\refdfn{dfn:linear-map}$f \colon V \to \symbb{K}$を
    \word{線形形式}[][せんけいけいしき](linear form)
    あるいは\word{線形汎関数}[せん|けい|はん|かん|すう][せんけいはんかんすう](linear functional)という。
\end{definition}

\begin{example}
    $V = \symbb{R}^3$(3次元ユークリッド空間)とする。
    $\symbb{R}^3$のベクトルをその第1成分にうつす写像
    $f \colon \symbf{v} \longmapsto v_1 \in \symbb{R}$は$V$の線形汎関数である。
\end{example}


\begin{definition}[双対空間]
    \label{dfn:dual-space}
    $V$上の線形汎関数全体の集合
    \begin{equation}
        \dual{V}  \coloneq  \Set{  f \colon V \to \symbb{K}  \given  \text{$f$は線形汎関数}  }
    \end{equation}
    を$V$の\word{双対空間}[そう|つい|くう|かん][そうついくうかん](dual space)という\cite[\S 4.1]{saito-lin-2007}。
\end{definition}


\begin{proposition}
    ベクトル空間$V$の双対空間$\dual{V}$は、ベクトル空間をなす\cite[\S 4.1]{saito-lin-2007}。
\end{proposition}

\begin{proof}
    $\dual{V}$の元が、\cref{thm:linear-map-space}で定義される和とスカラー倍について、
    \cref{dfn:vector-space}の公理を満たすことを示せばよい。
\end{proof}


\begin{definition}[双対基底]
    \label{dfn:dual-basis}
    $V$を$\symbb{K}$上の有限次元ベクトル空間、$\symbf{x}_1, \dots, \symbf{x}_n$を$V$の基底とすると、
    各$i = 1, \dots, n$に対し
    \begin{equation}
        \label{eq:dual-basis}
        f_i (\symbf{x}_j) = \delta_{i, j} \quad \text{\refdfn{dfn:Kronecker-delta}}
    \end{equation}
    を満たす線形汎関数$f_i \colon V \to \symbb{K}$がただひとつ存在する。
    このとき、$f_1, \dots, f_n$は$\dual{V}$の基底になる。

    \cref{eq:dual-basis}で定義される$f_1, \dots, f_n$を$\symbf{x}_1, \dots, \symbf{x}_n$の\word{双対基底}[そう|つい|き|てい][そうついきてい](dual basis)という。
\end{definition}

このような定義が可能であることについては証明を要するが、ここでは省略する。

\begin{example}
    $V = \symbb{R}^n$とし、ある$\symbf{a} \in V$をとる。
    線形汎関数$f_{\symbf{a}} \colon V \to \symbb{R}$を
    \[  f_{\symbf{a}} \colon V \ni \symbf{v} \longmapsto \tp{\symbf{a}} \cdotp \symbf{v} \in \symbb{R}  \]
    で定める。
    $\symbb{R}^n$の標準基底$\symbf{e}_1, \dots, \symbf{e}_n$に対する双対基底は、
    $f_{\symbf{e}_1}, \dots, f_{\symbf{e}_n}$である。
    これは行ベクトル$\tp{\symbf{e}_1}, \dots, \tp{\symbf{e}_n}$とみることができる。。
\end{example}



\subsection{零化空間}


\begin{definition}[零化空間]
    $V$をベクトル空間、$\dual{V}$をその双対空間とする。
    $V$の部分ベクトル空間$W$に対し、
    \begin{equation}
        W^\perp  \coloneq  \Set{  f \in \dual{V}  \given  f(W) = \{ 0 \}  }
    \end{equation}
を$W$の\word{零化空間}[れい|か|くう|かん][れいかくうかん](annihilator)という\cite[\S 4.2]{saito-lin-2007}。
\end{definition}

\begin{proposition}
    部分ベクトル空間$W \subset V$の零化空間$W^\perp$は、$\dual{V}$の部分ベクトル空間である。
\end{proposition}

\begin{proof}
    写像$\dual{i}$を、線形汎関数$f \colon V \to \symbb{K}$を$W$に制限する写像と定める:
    \[  \dual{i} \colon \dual{V} \ni f \longmapsto f \vert_W \in \dual{W}  \]
    このとき、任意の$f, g \in \dual{V}$および$c \in \symbb{K}$について
    $\dual{i}(f + g) = \dual{i}(f) + \dual{i}(g)$かつ$\dual{i}(cf) = c \cdotp \dual{i}(f)$であるので、
    $\dual{i}$は線形写像である。
    すると$W^\perp = \ker( \dual{i} \colon \dual{V} \to \dual{W} )$\refdfn{dfn:kernel-of-linear-map}とかけるので、
    $W^\perp$は$\dual{V}$の部分ベクトル空間である。
\end{proof}



\subsection{内積空間における線形汎関数}

内積は線形汎関数の一種である。

\begin{proposition}
    $V$を内積空間\refdfn{dfn:inner-product}、$\symbf{u} \in V$をあるベクトルとする。
    任意のベクトル$\symbf{v} \in V$に対して内積$(\symbf{u}, \symbf{v})$を与える写像
    \begin{equation}
        (\symbf{u}, \bigdot) \colon V \ni \symbf{v} \longmapsto (\symbf{u}, \symbf{v}) \in \symbb{R}
    \end{equation}
    は線形汎関数である。
\end{proposition}

\begin{proof}
    内積の線形性より$(\symbf{u}, \bigdot)$が線形なのは明らかである。
\end{proof}





\end{document}