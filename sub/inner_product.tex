\documentclass[../sotsu.tex]{subfiles}

\begin{document}

\section{内積空間とノルム空間}
\label{sec:inner-product-space}

\cref{sec:vector-space}では,
数ベクトルを一般化した概念である\refdfn-[ベクトル空間]{dfn:vector-space}について扱った.
そこではベクトルの和とスカラー倍が存在するのであった.
しかし,ベクトルの「長さ」あるいは2つのベクトルの「角度」という概念は定義していなかった.

これからベクトル空間の上に「\ruby{内積}{ない|せき}」および「ノルム」という演算を導入する.
内積を用いることで,2つのベクトルが「直交」するということを代数的に表現できる.
また,ノルムを用いれば2点間の距離を定義することができる.

これまでは一般の体$𝕂$上のベクトル空間を考えてきたが,
内積を考える上では,体を$ℝ$もしくは$ℂ$に限る必要がある.
そこでこれからは$𝕂$のかわりに$ℂ$と書くことにする.
以降の$ℂ$を$ℝ$に置き換えることも容易である.


\subsection{内積空間}
\label{sec:inner-product-space}

まず,2つのベクトルから実数を得る演算である内積を定義する.

\begin{definition}[内積]
    \label{dfn:inner-product}
    $V$をベクトル空間\refdfn{dfn:vector-space}とする.
    写像$\iparen{{\bigdot}, {\bigdot}} \colon V \times V \to ℝ$は以下を満たすとき,\word{内積}[ない|せき][ないせき](inner product)であるという.
    \begin{itemize}
        \item 任意の$𝒖, 𝒗, 𝒘 \in V, \  c \in ℂ$に対し,
        \begin{enumerate}
            \item \label{innerp:sum} $\iparen{𝒖, 𝒗 \tplus 𝒘} = \iparen{𝒖, 𝒗} + \iparen{𝒖, 𝒘}$である.
            \item \label{innerp:scalar} $\iparen{𝒖, c𝒗} = c\iparen{𝒖, 𝒗}$である.
            \item \label{innerp:conjugate-symmetry} $\iparen{𝒖, 𝒗} = \conj{\iparen{𝒗, 𝒖}}$である.
            \item \label{innerp:positive-definiteness}$\iparen{𝒗, 𝒗} \geq 0$である.
                また,$\iparen{𝒗, 𝒗} = 0$であるのは$𝒗 = \symbf{0}$であるときに限る.
        \end{enumerate}
    \end{itemize}
    内積をもつベクトル空間を\word{内積空間}[][ないせきくうかん](inner product space)という.
\end{definition}

\begin{corollary}
    内積は以下を満たす.
    \begin{itemize}
        \item 任意の$𝒖, 𝒗, 𝒘 \in V, \  a, b \in ℂ$に対し,
        \begin{enumerate}
            \item \label{innerp:linearity} (\word{線形性}(linearity)\index{せんけいせい@線形性})
                $\iparen{ 𝒖, a𝒗 \tplus b𝒘 } = a \iparen{𝒖, 𝒗} + b \iparen{𝒖, 𝒘}$
            \item \label{innerp:antilinearity} (\word{反線形性}(antilinearity)\word{反線形性}(linearity)\index{はんせんけいせい@反線形性})
                $\iparen{ a𝒖 \tplus b𝒗, 𝒘} = \conj{a} \iparen{𝒖, 𝒘} + \conj{b} \iparen{𝒗, 𝒘}$
            \item $(𝒗, \symbf{0}) = (\symbf{0}, 𝒗) = 0$
        \end{enumerate}

        \cref{innerp:linearity,innerp:antilinearity}をあわせて%
        \word{半双線形}[][はんそうせんけい](sesqui-linear)%
        であるという.
    \end{itemize}
\end{corollary}


\begin{example}
    実ユークリッド空間$ℝ^n$において,
    \begin{equation*}
        \iparen{ \symbf{a}, \symbf{b} } ≔ \sum_{i=1}^{n} a_i b_i
    \end{equation*}
    は内積である.また,複素ユークリッド空間$ℂ^n$において,
    \begin{equation*}
        \iparen{ \symbf{a}, \symbf{b} } ≔ \sum_{i=1}^{n} \conj{a_i} b_i
    \end{equation*}
    は内積である.これらを特に\word{標準内積}[][ひようしゆんないせき]という.
\end{example}


\begin{definition}
    \label{dfn:orthogonal}
    \label{dfn:parallel}
    $V$を内積空間とする.
    ベクトル$𝒗, 𝒖 \in V \setminus \{\symbf{0}\}$は,
    $\iparen{𝒗, 𝒖} = 0$であるとき,\word{直交する}(orthogonal)といい,$𝒗 \perp 𝒖$とかく.
    また,$c \in ℂ \setminus \{0\}$を用いて$𝒗 = c𝒖$とかけるとき,
    \word{平行である}(parallel)といい,$𝒗 \parallel 𝒖$とかく.
\end{definition}


\begin{definition}[直交補空間]
    \label{dfn:orthogonal-compliment}
    $V$をベクトル空間,$W \subset V$を部分ベクトル空間とする.
    $W$のベクトルすべてと直交するようなベクトルの集合,つまり
    \begin{equation}
        W^\perp  ≔  \Set{  𝒗 \in V  
                                \given  \text{任意の$𝒖 \in W$に対して$𝒗 \perp 𝒖$}  }
    \end{equation}
    を$W$の\word{直交補空間}(orthogonal compliment)%
    \index{ちよつこうほくうかん@直交補空間}%
    という.
\end{definition}

\begin{corollary}
    $V$をベクトル空間とする.
    $W \subset V$の直交補空間$W^\perp$は$V$の部分ベクトル空間である.
\end{corollary}

\begin{proof}
    \refthm[部分ベクトル空間の公理]{thm:vector-subspace-iff}を確かめればよい.
    すなわち,任意の$\symbf{v} \in W$に対し,
    \begin{enumerate}
        \item $\symbf{0} \in W^\perp$であることは,
            $\iparen{\symbf{v}, \symbf{0}} = 0$から,
        \item $\symbf{x}, \symbf{y} \in W^\perp$ならば$\symbf{x} + \symbf{y} \in W^\perp$であることは,
            $\iparen{\symbf{x} + \symbf{y}, \  \symbf{v}} = \iparen{\symbf{x}, \symbf{v}} + \iparen{\symbf{y}, \symbf{v}}$から,
        \item $\symbf{x} \in W^\perp$,$c \in \symbb{K}$ならば$c\symbf{x} \in W^\perp$であることは,
            $\iparen{c\symbf{x}, \symbf{v}} = \conj{c} \iparen{\symbf{x}, \symbf{v}}$からいえる.
    \end{enumerate}
\end{proof}


\begin{corollary}
    部分ベクトル空間$W \subset V$とその直交補空間$W^\perp \subset W$について,
    \[  W \cap W^\perp = \{ \symbf{0} \}  \]
\end{corollary}

\begin{proof}
    $\symbf{x} \in W \cap W^\perp$をとる.
    $\symbf{x} \in W^\perp$より,
    任意の$\symbf{v} \in W$に対し$\iparen{\symbf{v}, \symbf{x}} = 0$である.
    ここで,$\symbf{v} = \symbf{x} \in W$ととることができるので,
    $\iparen{\symbf{x}, \symbf{x}} = 0$.したがって$\symbf{x} = \symbf{0}$である.
\end{proof}

任意のベクトル$\symbf{v} \in V$は,
ある部分空間$W$に属するベクトル$\symbf{x}$とその直交補空間に属するベクトル$\symbf{y}$とを用いて,
$\symbf{v} = \symbf{x} + \symbf{y}$と一意にあらわせそうである.
有限次元の場合はたしかに$V = W \oplus W^\perp$と\refdfn[直和分解]{dfn:direct-sum-of-vector-space}できる.
しかし,無限次元の場合これは正しくない.

無限次元ヒルベルト空間が直和分解できる条件については,
\cref{thm:projection-theorem}で議論する.





\subsection{ノルム空間}
\label{sec:norm-space}

ベクトルの長さ,あるいは点の\refdfn-[距離]{dfn:distance}に対応するのがノルムという概念である.

\begin{definition}[ノルム]
    \label{dfn:norm}
    $V$を\refdfn-[ベクトル空間]{dfn:vector-space}とする.
    写像$\norm{\bigdot} \colon V \to ℝ$は以下を満たすとき,
    \word{ノルム}(norm)%
    \index{のるむ@ノルム}%
    という.
    \begin{itemize}
        \item 任意の$𝒖, 𝒗 \in V$および$c \in ℂ$に対し,
        \begin{enumerate}
            \item \label{norm:positivity}$\norm{𝒗} \geq 0$である.
                また,$\norm{𝒗} = 0$となるのは$𝒗 = \symbf{0}$のときに限る.
            \item \label{norm:absolute-homogeneity} $\norm{c𝒗} = \abs{c} \norm{𝒗}$である.
            \item \label{norm:triangle-inequality} 
                (\word{三角不等式}(triangle inequality)\index{さんかくふとうしき@三角不等式})
                $\norm{𝒖 \vecplus 𝒗} \leq \norm{𝒖} + \norm{𝒗}$である.
                また,等号が成立するのは$𝒖$と$𝒗$が\refdfn-[一次従属]{dfn:linearly-independent}のときに限る.
        \end{enumerate}
    \end{itemize}
    ノルムが定義されたベクトル空間のことを%
    \word{ノルム空間}(normed space)%
    \index{のるむ@ノルム!くうかん@\indexdash 空間}%
    という.
\end{definition}

\begin{example}
    実ユークリッド空間のベクトル$𝒗 = (v_1, \dots, v_N) \in ℝ^N$において,以下はいずれもノルムである.
    \begin{align}
        \norm{𝒗}_1      &≔        \sum_{i=1}^{N}  \abs{v_i},      &
        \norm{𝒗}_2      &≔ \sqrt{ \sum_{i=1}^{N}  \abs{v_i}^2 },  &
        \norm{𝒗}_\infty &≔ \max_{1 \leq i \leq N} \abs{v_i}.      &
    \end{align}
\end{example}

ノルムは\refdfn[距離]{dfn:distance}\index{きより@距離}とみなすことができる(\cref{thm:norm-is-distance}).
したがって,ノルム空間は距離空間であり,
\cref{sec:metric-space}で導いたことがらが成り立つ.


\subsection{内積とノルムの関係}

\cref{sec:inner-product-space}では内積を,
\cref{sec:norm-space}ではノルムを導入した.
これらはそれぞれ独立に定義される概念であるが,特殊なベクトル空間においてはこの2つを統一することが可能である.
まずは内積からノルムを定義しよう.

\begin{definition}[内積から導かれるノルム]
    \label{dfn:norm-by-inner-product}
    $V$を内積空間とする.
    写像$\norm{\bigdot} \colon V \to ℝ$を以下で定義すると,これはノルムである.
    \begin{equation}
        \label{eq:norm-by-inner-product}
        \norm{𝒗} ≔ \sqrt{\iparen{𝒗, 𝒗}}
    \end{equation}
    したがって,内積空間は\cref{eq:norm-by-inner-product}で定義されたノルムに対してノルム空間になる.
\end{definition}

\begin{proof}
    \cref{eq:norm-by-inner-product}が\cref{dfn:norm}を満たすことを示せばよい.
    \cref{norm:positivity}は内積の定義(\cref{dfn:inner-product})から明らかなので,\cref{norm:triangle-inequality}を示す.
    そのために,まず次の補題を示す.

    \begin{lemma}[\word{コーシー--シュワルツの不等式}(Cauchy--Schwarz inequality)]
        \index{こおしいしゆわるつのふとうしき@コーシー--シュワルツの不等式}
        \index{しゆわるつのふとうしき@シュワルツの不等式|see{コーシー--シュワルツの不等式}}
        \label{thm:Cauchy-Schwarz-inequality}
        $V$を内積空間,$\norm{\bigdot}$を内積から導かれるノルムとする.
        このとき,任意の$𝒙, 𝒚$について
        \begin{equation}
            \label{eq:Cauchy-Schwarz-inequality}
            \abs{\iparen{𝒙, 𝒚}} \leq \norm{𝒙} \norm{𝒚}
        \end{equation}
        が成り立つ.
    \end{lemma}

    \begin{proof}[\cref{thm:Cauchy-Schwarz-inequality}の証明]
        $𝒚 = 0$のときは,$\abs{\iparen{𝒙, 𝒚}} = \norm{𝒙} \norm{𝒚} = 0$なので成り立つ.
        $𝒚 \neq 0$のときを考える.
        任意の$\lambda, \mu \in ℂ$について内積の正値性から$\norm{\lambda 𝒙 \tplus \mu 𝒚 } \geq \symbf{0}$がいえる.
        そこで$\lambda = -\conj{\iparen{𝒙, 𝒚}}$,$\mu = \norm{𝒙}^2$とおくと,
        \begin{equation*}
            \begin{split}
                0 &\leq \iparen{ \lambda 𝒙 \tplus \mu 𝒚, \  \lambda 𝒙 \tplus \mu 𝒚 }  \\
                  &= \lambda \conj{\lambda} \iparen{ 𝒙, 𝒙 } 
                    + \lambda \conj{\mu} \iparen{𝒙, 𝒚} 
                    + \mu \conj{\lambda} \iparen{𝒚, 𝒙} 
                    + \mu \conj{\mu} \iparen{𝒚, 𝒚}  \\
                  &= \abs{ \iparen{𝒙, 𝒚} }^2 \norm{𝒙}^2
                    - \conj{\iparen{𝒙, 𝒚}} \norm{𝒙}^2 \iparen{𝒙, 𝒚} 
                    - \norm{𝒙}^2 \iparen{𝒙, 𝒚} \iparen{𝒚, 𝒙} 
                    + \norm{𝒙}^4 \iparen{𝒚, 𝒚}  \\
                  &= \norm{𝒙}^2   \Bigl[  -\abs{ \iparen{𝒙, 𝒚} }^2 +  \norm{𝒙}^2 \norm{𝒚}^2 \Bigr]
            \end{split}
        \end{equation*}
        である.$𝒚 \neq 0$なので,内積の正値性より$\norm{𝒙} > 0$.
        そこで両辺を$\norm{𝒙}^2$で割れば\cref{eq:Cauchy-Schwarz-inequality}が示される.
    \end{proof}

    \cref{thm:Cauchy-Schwarz-inequality}を用いると,
    \begin{equation*}
        \begin{split}
            \norm{𝒙 \vecplus 𝒚}^2
                &= \iparen{𝒙 \vecplus 𝒚, \  𝒙 \vecplus 𝒚}  \\
                &= \norm{𝒙}^2 + 2 \Re \iparen{𝒙, 𝒚} + \norm{𝒚}^2  \\
                &\leq \norm{𝒙}^2 + 2 \abs{ \iparen{𝒙, 𝒚} } + \norm{𝒚}^2  \\
                &= \norm{𝒙}^2 + \norm{𝒚}^2
        \end{split}
    \end{equation*}
    と示される.
\end{proof}


一般のノルムについては成立しないが,
内積から導かれるノルムについてのみ成立するのが,
以下の中線定理である.

\begin{theorem}[中線定理]
    \label{thm:parallelogram-law}
    $V$を内積空間,$\norm{\bigdot} = \iparen{\bigdot, \bigdot}$を内積から導かれるノルムとする.
    \begin{equation}
        \norm{𝒗 \vecplus 𝒖}^2 + \norm{𝒗 \vecplus 𝒖}^2
            = 2 \bigl( \norm{𝒗}^2 + \norm{𝒖}^2 \bigr)
    \end{equation}
    が成立する(\word{中線定理}[ちゅう|せん|てい|り][ちゆうせんていり](parallelogram law)).
\end{theorem}

\begin{proof}
    ノルムを内積に直して計算すれば容易に示される.
\end{proof}


次に,内積空間における内積から導かれるノルムについて成立する恒等式を見る.

\begin{theorem}[偏極恒等式]
    \label{thm:polarization-identity}
    $V$を内積空間,$\norm{\bigdot}$を内積から導かれるノルムとする.
    任意のベクトル$𝒙, 𝒚 \in V$に対して,以下の\word{偏極恒等式}(polarization identity)が成立する.
    \begin{equation}
        \label{eq:polarization-identity}
        \iparen{𝒙, 𝒚}
            = \frac{1}{4} \bigl(  
                \norm{𝒙 \vecplus 𝒚}^2
                - \norm{𝒙 \vecminus 𝒚}^2
                - \iu \norm{𝒙 \vecplus \iu 𝒚}^2
                + \iu \norm{𝒙 \vecminus \iu 𝒚}^2
              \bigr)
    \end{equation}
\end{theorem}

\begin{proof}
    ノルムを内積に戻して,半双線形を用いて計算する.
    \begin{alignat*}{2}
        \norm{𝒙 \vecplus 𝒚}^2  
            &= \iparen{𝒙 \vecplus 𝒚, \  𝒙 \tplus 𝒚}
            &&= +\bigl[ \iparen{𝒙, 𝒙} + \iparen{𝒙, 𝒚} + \iparen{𝒚, 𝒙} + \iparen{𝒚, 𝒚} \bigr]
            \\
        -\norm{𝒙 \vecminus 𝒚}^2
            &= -\iparen{𝒙 \vecminus 𝒚, \  𝒙 \vecminus 𝒚}
            &&= -\bigl[ \iparen{𝒙, 𝒙} - \iparen{𝒙, 𝒚} - \iparen{𝒚, 𝒙} + \iparen{𝒚, 𝒚} \bigr]
            \\
        -\iu \norm{𝒙 \vecplus \iu 𝒚}^2
            &= -\iu \iparen{𝒙 \vecplus \iu 𝒚, \  𝒙 \vecplus \iu 𝒚}
            &&= -\iu \bigl[ \iparen{𝒙, 𝒙} + \iu \iparen{𝒙, 𝒚} - \iu \iparen{𝒚, 𝒙} + \iparen{𝒚, 𝒚} \bigr]
            \\
        +\iu \norm{𝒙 \vecminus \iu 𝒚} 
            &= +\iu \iparen{𝒙 \vecminus \iu 𝒚, \  𝒙 \vecminus \iu 𝒚}
            &&= +\iu \bigl[ \iparen{𝒙, 𝒙} - \iu \iparen{𝒙, 𝒚} + \iu \iparen{𝒚, 𝒙} + \iparen{𝒚, 𝒚} \bigr]
    \end{alignat*}
\end{proof}

一般の内積空間においては,\cref{dfn:norm-by-inner-product}によるノルムが存在すること,
そのノルムから内積を再構成できることが分かった.
それでは,一般のノルム空間においても\cref{thm:polarization-identity}によって内積が定義できるのであろうか.
そのことを示すのが次の定理である.

\begin{theorem}
    \word{ジョルダン--フォン・ノイマンの定理}[][しよるたんふおんのいまんのていり](Jordan--von~Neumann theorem):
    $V$をノルム空間とする.
    $V$上のノルムが\refthm-[中線定理]{thm:parallelogram-law}を満たすならば,$V$は内積空間である.
\end{theorem}

\cref{dfn:norm-by-inner-product}により``内積''$\iparen{𝒙, 𝒚}$を定義し,
それが内積の公理を満たすことを示す.



\subsection{内積空間の基底}

一般のベクトル空間において基底\refdfn{dfn:basis}が存在するのであった.
内積空間においては,よい性質をもつ基底をとることができる.

\begin{definition}[正規直交基底]
    $V$を有限次元の内積空間とする.
    $V$の基底$ \{ 𝒗_1, \dots, 𝒗_n \} \subset V$が次の性質を満たすとき,
    \word{正規直交基底}[せい|き|ちょっ|こう|き|てい](orthonormal basis)という.
    \begin{equation}
        \label{eq:orthonormal-basis}
        \iparen{𝒗_i, 𝒗_j} = \delta_{i,j} 
            \quad \text{Kronecker delta\refdfn{dfn:Kronecker-delta}}
    \end{equation}
\end{definition}

無限次元の場合も同様に定義できる.

有限次元の内積空間では,正規直交基底を具体的に構成することができる.

\begin{theorem}
    \label{thm:Gram-Schmidt-process}
    \word{グラム--シュミットの正規直交化}[][くらむしゆみつとのせいきちよつこうか](Gram–Schmidt process):
    $ \{ 𝒗_1, \dots, 𝒗_n \} \subset V$を有限次元の内積空間$V$の基底とする.
    このとき,$V$の正規直交基底$ \{ 𝒖_1, \dots, 𝒖_n \} \subset V$をつくることができる.
\end{theorem}

\begin{proof}
    基底$𝒗_1, \dots, 𝒗_n$から具体的に$𝒖_1, \dots, 𝒖_n$を構成する.

    まず$𝒖_1 ≔ 𝒗_1 / \norm{𝒗_1}$とおく.
    このとき,$\iparen{𝒖_1, 𝒖_1} = 1$が成り立つ.

    次に,$𝒖'_2 ≔ 𝒗_2 - \iparen{𝒖_1, 𝒗_2} 𝒖_1$とし,
    $𝒖_2 ≔ 𝒖'_2 / \norm{𝒖'_2}$と定める.
    このとき$\iparen{𝒖_1, 𝒖_2} = 0$であり,$\iparen{𝒖_2, 𝒖_2} = 1$である.

    これを繰り返し,
    \[  𝒖'_m ≔ 𝒗_m - \sum_{i=1}^{m-1} \iparen{𝒖_i, 𝒗_m} 𝒖_i,
        \qquad 
        𝒖_m ≔ 𝒖'_m  \]
    と定めれば,$𝒖_1, \dots, 𝒖_n$は\cref{eq:orthonormal-basis}を満たす.
\end{proof}

グラム--シュミットの方法が使えるのは基底が有限個,つまり有限次元の場合だけであり,
無限次元の場合は一般に成り立たない.
無限次元ヒルベルト空間の場合は\cite{thm:basis-of-Hilbert-space}で議論する.




\end{document}
