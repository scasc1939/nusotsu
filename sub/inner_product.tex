\documentclass[../sotsu.tex]{subfiles}

\begin{document}

\section{内積空間とノルム空間}
\label{sec:inner-product-and-norm}

\cref{sec:vector-space}では,
数ベクトルを一般化した概念である\refdfn-[ベクトル空間]{dfn:vector-space}について扱った.
ベクトルの和とスカラー倍を定義することで,
数ベクトルに似た様々な性質が演繹的に導かれるのであった.
しかし,数ベクトルがもつ「長さ」,
あるいは2つのベクトルのなす「角度」という概念はまだ定義していない.

\cref{sec:inner-product-and-norm}では,
ベクトル空間の上に「\ruby{内積}{ない|せき}」および「ノルム」という演算を導入する.
2つのベクトルが「直交」することは,内積によって代数的に表現できる.
また,ベクトル空間上にある2点間の「距離」は,
ノルムを用いて定義することができる.

なお,これまでの議論では一般の体$𝕂$上のベクトル空間を考えてきたが,
内積やノルムを考える上では,体を$ℝ$もしくは$ℂ$に限る.
したがって,これからは一般の体$𝕂$のかわりに複素数体$ℂ$を明示して書くことにする.
以降の議論は$ℂ$上で行うが,これを$ℝ$に置き換えて議論することもできる.


\subsection{内積空間}
\label{sec:inner-product-space}

まず,2つのベクトルから実数を得る演算である内積を定義する.

\begin{definition}[内積]
    \label{dfn:inner-product}
    $V$を\refdfn-[ベクトル空間]{dfn:vector-space}とする.
    写像$\iparen{{\bigdot}, {\bigdot}} \colon V \times V \to ℝ$は以下を満たすとき,\word{内積}[ない|せき][ないせき](inner product)であるという.
    \begin{itemize}
        \item 任意の$𝒖, 𝒗, 𝒘 \in V, \  c \in ℂ$に対し,
        \begin{enumerate}
            \item \label{innerp:sum} $\iparen{𝒖, \  𝒗 \vecplus 𝒘} = \iparen{𝒖, 𝒗} + \iparen{𝒖, 𝒘}$である.
            \item \label{innerp:scalar} $\iparen{𝒖, c𝒗} = c\iparen{𝒖, 𝒗}$である.
            \item \label{innerp:conjugate-symmetry} $\iparen{𝒖, 𝒗} = \conj{\iparen{𝒗, 𝒖}}$である.
            \item \label{innerp:positive-definiteness}$\iparen{𝒗, 𝒗} \geq 0$である.
                また,$\iparen{𝒗, 𝒗} = 0$であるのは$𝒗 = 𝟎$であるときに限る.
        \end{enumerate}
    \end{itemize}
    内積をもつベクトル空間を\word{内積空間}[][ないせきくうかん](inner product space)という.
\end{definition}

\begin{corollary}
    内積は以下を満たす.
    \begin{itemize}
        \item 任意の$𝒖, 𝒗, 𝒘 \in V, \  a, b \in ℂ$に対し,
        \begin{enumerate}
            \item \label{innerp:linearity} (\word{線形性}(linearity)\index{せんけいせい@線形性})
                $\iparen{ 𝒖, a𝒗 \vecplus b𝒘 } = a \iparen{𝒖, 𝒗} + b \iparen{𝒖, 𝒘}$
            \item \label{innerp:antilinearity} (\word{反線形性}(antilinearity)\word{反線形性}(linearity)\index{はんせんけいせい@反線形性})
                $\iparen{ a𝒖 \vecplus b𝒗, 𝒘} = \conj{a} \iparen{𝒖, 𝒘} + \conj{b} \iparen{𝒗, 𝒘}$
            \item $(𝒗, 𝟎) = (𝟎, 𝒗) = 0$
        \end{enumerate}

        \cref{innerp:linearity,innerp:antilinearity}をあわせて%
        \word{半双線形}[][はんそうせんけい](sesqui-linear)%
        であるという.
    \end{itemize}
\end{corollary}


実ユークリッド空間$ℝ^n$において,
\begin{equation}
    \label{eq:real-Euclidean-canonical-inner-product}
    \iparen{ \symbf{a}, \symbf{b} }_{\symbb{R}} ≔ \sum_{i=1}^{n} a_i b_i
\end{equation}
は内積である.また,複素ユークリッド空間$ℂ^n$において,
\begin{equation}
    \label{eq:complex-Euclidean-canonical-inner-product}
    \iparen{ \symbf{a}, \symbf{b} }_{\symbb{C}} ≔ \sum_{i=1}^{n} \conj{a_i} b_i
\end{equation}
は内積である.
\cref{eq:real-Euclidean-canonical-inner-product,eq:complex-Euclidean-canonical-inner-product}を特に\word{標準内積}%
\index{ないせき@内積!ひようしゆん@標準\indexdash}%
\index{ひようしゆんないせき@標準内積|see{内積}}%
という.


\begin{definition}
    \label{dfn:orthogonal}
    \label{dfn:parallel}
    $V$を内積空間とする.
    ベクトル$𝒗, 𝒖 \in V \setminus \{𝟎\}$は,
    $\iparen{𝒗, 𝒖} = 0$であるとき,\word{直交する}(orthogonal)といい,$𝒗 \perp 𝒖$とかく.
    また,$c \in ℂ \setminus \{0\}$を用いて$𝒗 = c𝒖$とかけるとき,
    \word{平行である}(parallel)といい,$𝒗 \parallel 𝒖$とかく.
\end{definition}



\subsubsection*{内積の連続性}

\begin{proposition}
    \label{thm:inner-product-continuity}
    内積は連続である.
    つまり,$𝒖_n \to 𝒖_*$,$𝒗_n \to 𝒗_*$であるベクトル列$\sequ{u_n}$,$\sequ{𝒗_n}$に対し,
    \begin{equation*}
        \iparen{𝒖_n, 𝒗_n} \xrightarrow{n \to \infty} \iparen{𝒖_*, 𝒗_*}
    \end{equation*}
\end{proposition}

\begin{proof}
    \refthm[コーシー--シュワルツの不等式]{thm:Cauchy-Schwarz-inequality}より,
    \begin{equation*}
        \begin{split}
            \abs{ \iparen{𝒖_n, 𝒗_n} - \iparen{𝒖_*, 𝒗_*} }
                &= \abs[\big]{ \iparen{𝒖_n, \  𝒗_n - 𝒗_*} + \iparen{𝒖_n - 𝒖_*, \  𝒗_*} }
                \\
                &\leq \abs[\big]{ \norm{𝒖_n} \norm{𝒗_n - 𝒗_*} + \norm{𝒖_n - 𝒖_*} \norm{𝒗_*} }
                \\
                &\xrightarrow{ n \to \infty }
                    \abs[\big]{ \norm{𝒖_*} \cdotp 0 + 0 \cdotp \norm{𝒗_*} }
                = 0
        \end{split}
    \end{equation*}
\end{proof}



\subsubsection*{直交補空間}

\begin{definition}[直交補空間]
    \label{dfn:orthogonal-compliment}
    $V$をベクトル空間,$W \subset V$を部分ベクトル空間とする.
    $W$のベクトルすべてと直交するようなベクトルの集合,つまり
    \begin{equation}
        W^\perp  ≔  \Set{  𝒗 \in V  
                                \given  \text{任意の$𝒖 \in W$に対して$𝒗 \perp 𝒖$}  }
    \end{equation}
    を$W$の\word{直交補空間}(orthogonal compliment)%
    \index{ちよつこうほくうかん@直交補空間}%
    という.
\end{definition}

\begin{proposition}
    $V$をベクトル空間とする.
    $W \subset V$の直交補空間$W^\perp$は$V$の部分ベクトル空間である.
\end{proposition}

\begin{proof}
    \refthm[部分ベクトル空間の条件]{thm:vector-subspace-iff}を確かめればよい.
    すなわち,任意の$𝒗 \in W$に対し,
    \begin{enumerate}
        \item $𝟎 \in W^\perp$であることは,
            $\iparen{𝒗, 𝟎} = 0$から,
        \item $𝒙, 𝒚 \in W^\perp$ならば$𝒙 + 𝒚 \in W^\perp$であることは,
            $\iparen{𝒙 + 𝒚, \  𝒗} = \iparen{𝒙, 𝒗} + \iparen{𝒚, 𝒗}$から,
        \item $𝒙 \in W^\perp$,$c \in \symbb{K}$ならば$c𝒙 \in W^\perp$であることは,
            $\iparen{c𝒙, 𝒗} = \conj{c} \iparen{𝒙, 𝒗}$からいえる.
    \end{enumerate}
\end{proof}


\begin{lemma}
    \label{thm:intersection-of-and-orthogonal-complement}
    部分ベクトル空間$W \subset V$とその直交補空間$W^\perp \subset V$について,
    \[  W \cap W^\perp = \{ 𝟎 \}  \]
\end{lemma}

\begin{proof}
    $𝒙 \in W \cap W^\perp$をとる.
    $𝒙 \in W^\perp$より,
    任意の$𝒗 \in W$に対し$\iparen{𝒗, 𝒙} = 0$である.
    ここで,$𝒗 = 𝒙 \in W$ととることができるので,
    $\iparen{𝒙, 𝒙} = 0$.したがって$𝒙 = 𝟎$である.
\end{proof}

任意のベクトル$𝒗 \in V$は,
ある部分空間$W$に属するベクトル$𝒙$とその直交補空間に属するベクトル$𝒚$とを用いて,
$𝒗 = 𝒙 + 𝒚$と一意にあらわせそうである.
有限次元の場合はたしかに$V = W \oplus W^\perp$と\refdfn[直和分解]{dfn:direct-sum-of-vector-space}できる.
しかし,無限次元の場合これは正しくない.
これについては\cref{sec:Hilbert-space}で議論するが,
先に言葉を定義しておく.

\begin{definition}
    \label{dfn:vector-projection}
    $V$を\refdfn-[内積空間]{dfn:inner-product},
    $W \subset V$を\refdfn-[部分ベクトル空間]{dfn:vector-subspace},
    $W^\perp$をその\refdfn-[直交補空間]{dfn:orthogonal-compliment}とする.
    $𝒙 \in V$が
    \begin{equation}
        \label{eq:vector-projection}
        𝒙 = 𝒚 + \symbf{z}, 
        \quad 
        \text{where }
        \symbf{y} \in W, 
        \ 
        \symbf{z} \in W^\perp
    \end{equation}
    とかけるとき,$\symbf{y}$を$\symbf{x}$の$W$上への正射影\index{せいしやえい@正射影},
    直交射影\index{ちよつこうしやえい@直交射影},
    あるいは単に\word{射影}\index{しやえい@射影}という.
\end{definition}

\begin{lemma}
    \label{thm:proposition-is-unique}
    $\symbf{x}$に射影が存在するとき,
    つまり\cref{eq:vector-projection}のようにかけたとき,
    $\symbf{y} \in W$,$\symbf{z} \in W^\perp$は一意に定まる.
\end{lemma}

\begin{proof}
    $\symbf{x} = \symbf{y} + \symbf{z} = \symbf{y}' + \symbf{z}'$と2通りに分解されたとする.
    すると
    \begin{equation*}
        \symbf{y} - \symbf{y}' = \symbf{z}' - \symbf{z}
    \end{equation*}
    となる.
    右辺は$W$の元,左辺は$W^\perp$の元である.
    よって\cref{thm:intersection-of-and-orthogonal-complement}より,
    $\symbf{y} - \symbf{y}' = \symbf{z}' - \symbf{z} = \symbf{0}$.
    したがって$\symbf{y} = \symbf{y}'$,$\symbf{z} = \symbf{z}'$.
\end{proof}

先に注意したとおり,
内積空間の任意のベクトル$\symbf{x}$に対して射影が存在するとは限らない.
上の証明はあくまで,$\symbf{x}$の射影が存在したと仮定した場合のものである.




\subsection{ノルム空間}
\label{sec:norm-space}

ベクトルの長さ,あるいは点の\refdfn-[距離]{dfn:distance}に対応するのがノルムという概念である.

\begin{definition}[ノルム]
    \label{dfn:norm}
    $V$を\refdfn-[ベクトル空間]{dfn:vector-space}とする.
    写像$\norm{\bigdot} \colon V \to ℝ$は以下を満たすとき,
    \word{ノルム}(norm)%
    \index{のるむ@ノルム}%
    という.
    \begin{itemize}
        \item 任意の$𝒖, 𝒗 \in V$および$c \in ℂ$に対し,
        \begin{enumerate}
            \item \label{norm:positivity}$\norm{𝒗} \geq 0$である.
                また,$\norm{𝒗} = 0$となるのは$𝒗 = 𝟎$のときに限る.
            \item \label{norm:absolute-homogeneity} $\norm{c𝒗} = \abs{c} \norm{𝒗}$である.
            \item \label{norm:triangle-inequality} 
                (\word{三角不等式}(triangle inequality)\index{さんかくふとうしき@三角不等式})
                $\norm{𝒖 \vecplus 𝒗} \leq \norm{𝒖} + \norm{𝒗}$である.
                また,等号が成立するのは$𝒖$と$𝒗$が\refdfn-[一次従属]{dfn:linearly-independent}のときに限る.
        \end{enumerate}
    \end{itemize}
    ノルムが定義されたベクトル空間のことを%
    \word{ノルム空間}(normed space)%
    \index{のるむ@ノルム!くうかん@\indexdash 空間}%
    という.
\end{definition}

\begin{example}
    実ユークリッド空間のベクトル$𝒗 = (v_1, \dots, v_N) \in ℝ^N$において,以下はいずれもノルムである.
    \begin{align}
        \norm{𝒗}_1      &≔        \sum_{i=1}^{N}  \abs{v_i},      &
        \norm{𝒗}_2      &≔ \sqrt{ \sum_{i=1}^{N}  \abs{v_i}^2 },  &
        \norm{𝒗}_\infty &≔ \max_{1 \leq i \leq N} \abs{v_i}.      &
    \end{align}
\end{example}

ノルムは\refdfn[距離]{dfn:distance}\index{きより@距離}とみなすことができる(\cref{thm:norm-is-distance}).
したがって,ノルム空間は距離空間であり,
\cref{sec:metric-space}で導いたことがらが成り立つ.
特に,ノルムを用いると,ベクトルの点列の収束という概念が定義できる.

\begin{definition}
    \label{dfn:convergence-of-vector-sequence}
    ノルム空間$V$のベクトル列$\sequ{𝒗_n}[n \in \symbb{N}]$が$𝒗_* \in V$に収束するとは,
    \begin{equation*}
        \norm{ 𝒗_n \vecminus 𝒗_* } \xrightarrow{n \to \infty} 0
    \end{equation*}
    であることをいう.
\end{definition}


いくつか便利な式を導いておこう.
まず,三角不等式$\norm{𝒖 \vecplus 𝒗} \leq \norm{𝒖} + \norm{𝒗}$から次のような不等式が導かれる.
\begin{lemma}
    \label{thm:triangle-inequality-subtraction}
    ノルム$\norm{\bigdot}$に対し,以下の不等式が成り立つ.
    \begin{equation}
        \label{eq:triangle-inequality-subtraction}
        \abs[\big]{ \norm{𝒖} \vecminus \norm{𝒗} } \leq \norm{𝒖 - 𝒗}
    \end{equation}
\end{lemma}
\begin{proof}
    三角不等式より
    \begin{align*}
        \norm{𝒖} \leq \norm{𝒖 \vecminus 𝒗} + \norm{𝒗},
        \quad 
        \norm{𝒗} \leq \norm{𝒗 \vecminus 𝒖} + \norm{𝒖}
    \end{align*}
    であるから,それぞれ右辺の$\norm{𝒗}, \norm{𝒖}$を左辺へ移せば
    \begin{align*}
        \norm{𝒖} - \norm{𝒗} \leq \norm{𝒖 \vecminus 𝒗},
        \quad 
        \norm{𝒗} - \norm{𝒖} \leq \norm{𝒗 \vecminus 𝒖} = \norm{𝒖 \vecminus 𝒗}
    \end{align*}
    である.この2式をあわせれば,\cref{eq:triangle-inequality-subtraction}が導かれる.
\end{proof}


\begin{lemma}
    \label{thm:norm-sequence-continuity}
    $\sequ{𝒖_n}$がノルム空間のベクトル列で,
    $𝒖_n \to 𝒖_*$と収束するなら,
    $\norm{𝒖_n} \to \norm{𝒖_*}$である\cite[補題1.2]{iwanami-functional}.
\end{lemma}

\begin{proof}
    \cref{eq:triangle-inequality-subtraction}を使うと,
    \begin{equation*}
        \abs[\big]{ \norm{𝒖_n} - \norm{𝒖_*} }
            \leq \norm{𝒖_n \vecminus 𝒖_*}
            \xrightarrow{n \to \infty} 0
    \end{equation*}
    である.
\end{proof}


\subsubsection*{ノルムの連続性}

\begin{proposition}
    \label{thm:norm-continuity}
    ノルムはベクトル空間から実数への写像$\norm{\bigdot} \colon V \to \symbb{R}$として連続である\cite[定理1.1]{iwanami-functional}.
\end{proposition}

\begin{proof}
    $V$の点列$\sequ{𝒖_n}$,$\sequ{𝒗_n}$および$\symbb{C}$の点列(数列)$\sequ{c_n}$をとり,
    $𝒖_n \to 𝒖_*$,$𝒗_n \to 𝒗_*$,$c_n \to c_*$($n \to \infty$)とする.

    \textsf{加法の連続性}\quad
    \begin{equation*}
        \norm{𝒖_n \vecplus 𝒗_n} \xrightarrow{n \to \infty} \norm{𝒖_* \vecplus 𝒗_*}
    \end{equation*}
    を示せばよい.三角不等式から
    \begin{equation*}
        \abs[\big]{ \norm{𝒖_n \vecplus 𝒗_n} - \norm{𝒖_* \vecplus 𝒗_*} }
            \leq \norm{ (𝒖_n \vecplus 𝒗_n) - (𝒖_* \vecplus 𝒗_*) }
            \xrightarrow{n \to \infty} 0
    \end{equation*}

    \textsf{スカラー倍の連続性}\quad
    \begin{equation*}
        \norm{c_n 𝒖_n} \xrightarrow{n \to \infty} \norm{c_* 𝒖_*}
    \end{equation*}
    を示せばよい.これは,三角不等式を使って
    \begin{equation*}
        \begin{split}
            \abs[\big]{ \norm{c_n 𝒖_n} - \norm{c_* 𝒖_*} }
                &\leq \norm{ c_n 𝒖_n \vecminus c_* 𝒖_* }
                \\
                &= \norm[\big]{ (c_n - c_*) 𝒖_n \vecplus c_* (𝒖_n \vecminus 𝒖_*) }
                \\
                &\leq \abs{c_n - c_*} \norm{𝒖_n} \vecplus \abs{c_*} \norm{𝒖_n \vecminus 𝒖_*}
                \xrightarrow{n \to \infty} 0 \cdotp \norm{𝒖_*} + \abs{c_*} \cdotp 0
                = 0
        \end{split}
    \end{equation*}
    と示される\footnote{
        最後の極限では\cref{eq:norm-sequence-continuity}を使った.
    }.
\end{proof}


ノルム空間の部分ベクトル空間は,またノルム空間になる.
より正確に述べると,以下の\namecref{thm:norm-subspace}が成り立つ.

\begin{proposition}
    \label{thm:norm-subspace}
    $\norm{\bigdot}$をもつノルム空間$V$の部分ベクトル空間$W$は,
    $\norm{\bigdot} \colon V \to \symbb{R}$の定義域を$W$に\refdfn[制限]{dfn:restriction}したもの$\norm{\bigdot} \big\vert_W \colon W \to \symbb{R}$をノルムとして,
    ノルム空間になる.
\end{proposition}






\subsection{内積とノルムの関係}

\cref{sec:inner-product-space}では内積を,
\cref{sec:norm-space}ではノルムを導入した.
これらはそれぞれ独立に定義される概念であるが,
実はノルム空間のうち特殊なものが内積空間であることがわかる.
まずは内積からノルムを定義しよう.

\begin{definition}[内積から導かれるノルム]
    \label{dfn:norm-by-inner-product}
    $V$を内積空間とする.
    写像$\norm{\bigdot} \colon V \to ℝ$を以下で定義すると,これはノルムである.
    \begin{equation}
        \label{eq:norm-by-inner-product}
        \norm{𝒗} ≔ \sqrt{\iparen{𝒗, 𝒗}}
    \end{equation}
    したがって,内積空間は\cref{eq:norm-by-inner-product}で定義されたノルムに対してノルム空間になる.
\end{definition}

\begin{proof}
    \cref{eq:norm-by-inner-product}が\cref{dfn:norm}を満たすことを示せばよい.
    \cref{norm:positivity}は内積の定義(\cref{dfn:inner-product})から明らかなので,\cref{norm:triangle-inequality}を示す.
    そのために,まず次の補題を示す.

    \begin{lemma}[\word{コーシー--シュワルツの不等式}(Cauchy--Schwarz inequality)]
        \index{こおしいしゆわるつのふとうしき@コーシー--シュワルツの不等式}
        \index{しゆわるつのふとうしき@シュワルツの不等式|see{コーシー--シュワルツの不等式}}
        \label{thm:Cauchy-Schwarz-inequality}
        $V$を内積空間,$\norm{\bigdot}$を内積から導かれるノルムとする.
        このとき,任意の$𝒙, 𝒚$について
        \begin{equation}
            \label{eq:Cauchy-Schwarz-inequality}
            \abs{\iparen{𝒙, 𝒚}} \leq \norm{𝒙} \norm{𝒚}
        \end{equation}
        が成り立つ\cite{miyake-lin-2008}.
    \end{lemma}

    \begin{proof}[\cref{thm:Cauchy-Schwarz-inequality}の証明]
        $𝒙 = 𝟎$のときは,$\abs{\iparen{𝒙, 𝒚}} = \norm{𝒙} \norm{𝒚} = 0$なので成り立つ.
        $𝒙 \neq 𝟎$のときを考える.
        任意の$𝜆, 𝜇 \in ℂ$について内積の正値性から$\norm{𝜆 𝒙 \vecplus 𝜇 𝒚 } \geq 0$がいえる.
        そこで$𝜆 = -\conj{\iparen{𝒙, 𝒚}}$,$𝜇 = \norm{𝒙}^2$とおくと,
        \begin{equation*}
            \begin{split}
                0 &\leq \iparen{ 𝜆 𝒙 \vecplus 𝜇 𝒚, \  𝜆 𝒙 \vecplus 𝜇 𝒚 }  \\
                  &= 𝜆 \conj{𝜆} \iparen{ 𝒙, 𝒙 } 
                    + 𝜆 \conj{𝜇} \iparen{𝒙, 𝒚} 
                    + 𝜇 \conj{𝜆} \iparen{𝒚, 𝒙} 
                    + 𝜇 \conj{𝜇} \iparen{𝒚, 𝒚}  \\
                  &= \abs{ \iparen{𝒙, 𝒚} }^2 \norm{𝒙}^2
                    - \conj{\iparen{𝒙, 𝒚}} \norm{𝒙}^2 \iparen{𝒙, 𝒚} 
                    - \norm{𝒙}^2 \iparen{𝒙, 𝒚} \iparen{𝒚, 𝒙} 
                    + \norm{𝒙}^4 \iparen{𝒚, 𝒚}  \\
                  &= \norm{𝒙}^2   \Bigl[  -\abs{ \iparen{𝒙, 𝒚} }^2 +  \norm{𝒙}^2 \norm{𝒚}^2 \Bigr]
            \end{split}
        \end{equation*}
        である.$𝒙 \neq 0$なので,内積の正値性より$\norm{𝒙} > 0$.
        そこで両辺を$\norm{𝒙}^2$で割れば\cref{eq:Cauchy-Schwarz-inequality}が示される.
    \end{proof}

    \cref{thm:Cauchy-Schwarz-inequality}を用いると,
    \begin{equation*}
        \begin{split}
            \norm{𝒙 \vecplus 𝒚}^2
                &= \iparen{𝒙 \vecplus 𝒚, \  𝒙 \vecplus 𝒚}  \\
                &= \norm{𝒙}^2 + 2 \Real \iparen{𝒙, 𝒚} + \norm{𝒚}^2  \\
                &\leq \norm{𝒙}^2 + 2 \abs{ \iparen{𝒙, 𝒚} } + \norm{𝒚}^2  \\
                &= ( \norm{𝒙} + \norm{𝒚} )^2
        \end{split}
    \end{equation*}
    と示される.
\end{proof}

\cref{thm:Cauchy-Schwarz-inequality}より,
任意の$𝒙, 𝒚 \in V$に対し,
$-1 \leq \frac{\iparen{𝒙, 𝒚}}{\norm{𝒙} \norm{𝒚}} \leq 1$であるから,
\begin{equation*}
    \cos \theta = \frac{\iparen{𝒙, 𝒚}}{\norm{𝒙} \norm{𝒚}}
\end{equation*}
を満たすような$0 \leq \theta < 2\pi$が存在する.
$\theta$を$𝒙$と$𝒚$のなす\word{角度}(angle)\index{かくと@角度}という.


一般のノルムについては成立しないが,
内積から導かれるノルムについてのみ成立するのが,
以下の中線定理である.

\begin{theorem}[中線定理]
    \label{thm:parallelogram-law}
    $V$を内積空間,$\norm{\bigdot} = \iparen{\bigdot, \bigdot}$を内積から導かれるノルムとする.
    \begin{equation}
        \norm{𝒗 \vecplus 𝒖}^2 + \norm{𝒗 \vecplus 𝒖}^2
            = 2 \bigl( \norm{𝒗}^2 + \norm{𝒖}^2 \bigr)
    \end{equation}
    が成立する(\word{中線定理}[ちゅう|せん|てい|り][ちゆうせんていり](parallelogram law)).
\end{theorem}

\begin{proof}
    ノルムを内積に直して計算すれば容易に示される.
\end{proof}


次に,内積空間における内積から導かれるノルムについて成立する恒等式を見る.

\begin{theorem}[偏極恒等式]
    \label{thm:polarization-identity}
    $V$を内積空間,$\norm{\bigdot}$を内積から導かれるノルムとする.
    任意のベクトル$𝒙, 𝒚 \in V$に対して,以下の\word{偏極恒等式}(polarization identity)が成立する.
    \begin{equation}
        \label{eq:polarization-identity}
        \iparen{𝒙, 𝒚}
            = \frac{1}{4} \bigl(  
                \norm{𝒙 \vecplus 𝒚}^2
                - \norm{𝒙 \vecminus 𝒚}^2
                - \iu \norm{𝒙 \vecplus \iu 𝒚}^2
                + \iu \norm{𝒙 \vecminus \iu 𝒚}^2
              \bigr)
    \end{equation}
\end{theorem}

\begin{proof}
    ノルムを内積に戻して,半双線形を用いて計算すれば示される.
    % \begin{alignat*}{2}
    %     \norm{𝒙 \vecplus 𝒚}^2  
    %         &= \iparen{𝒙 \vecplus 𝒚, \  𝒙 \vecplus 𝒚}
    %         &&= +\bigl[ \iparen{𝒙, 𝒙} + \iparen{𝒙, 𝒚} + \iparen{𝒚, 𝒙} + \iparen{𝒚, 𝒚} \bigr]
    %         \\
    %     -\norm{𝒙 \vecminus 𝒚}^2
    %         &= -\iparen{𝒙 \vecminus 𝒚, \  𝒙 \vecminus 𝒚}
    %         &&= -\bigl[ \iparen{𝒙, 𝒙} - \iparen{𝒙, 𝒚} - \iparen{𝒚, 𝒙} + \iparen{𝒚, 𝒚} \bigr]
    %         \\
    %     -\iu \norm{𝒙 \vecplus \iu 𝒚}^2
    %         &= -\iu \iparen{𝒙 \vecplus \iu 𝒚, \  𝒙 \vecplus \iu 𝒚}
    %         &&= -\iu \bigl[ \iparen{𝒙, 𝒙} + \iu \iparen{𝒙, 𝒚} - \iu \iparen{𝒚, 𝒙} + \iparen{𝒚, 𝒚} \bigr]
    %         \\
    %     +\iu \norm{𝒙 \vecminus \iu 𝒚} 
    %         &= +\iu \iparen{𝒙 \vecminus \iu 𝒚, \  𝒙 \vecminus \iu 𝒚}
    %         &&= +\iu \bigl[ \iparen{𝒙, 𝒙} - \iu \iparen{𝒙, 𝒚} + \iu \iparen{𝒚, 𝒙} + \iparen{𝒚, 𝒚} \bigr]
    % \end{alignat*}
\end{proof}

一般の内積空間においては,\cref{dfn:norm-by-inner-product}によるノルムが存在すること,
そのノルムから内積を再構成できることが分かった.
それでは,一般のノルム空間においても\cref{thm:polarization-identity}によって内積が定義できるのであろうか.
そのことを示すのが次の定理である.

\begin{theorem}
    \word{ジョルダン--フォン・ノイマンの定理}[][しよるたんふおんのいまんのていり](Jordan--von~Neumann theorem):
    $V$をノルム空間とする.
    $V$上のノルムが\refthm-[中線定理]{thm:parallelogram-law}を満たすならば,$V$は内積空間である.
\end{theorem}

証明は,
\cref{dfn:norm-by-inner-product}により``内積''$\iparen{𝒙, 𝒚}$を定義し,
それが内積の公理を満たすことを示す.



\subsection{内積空間の基底}

一般のベクトル空間において\refdfn[基底]{dfn:basis}が存在するのであった.
内積空間においては,よい性質をもつ基底をとることができる.

\begin{definition}[正規直交基底]
    $V$を有限次元の内積空間とする.
    $V$の基底$ \{ 𝒗_1, \dots, 𝒗_n \} \subset V$が次の性質を満たすとき,
    \word{正規直交基底}[せい|き|ちょっ|こう|き|てい](orthonormal basis)という.
    \begin{equation}
        \label{eq:orthonormal-basis}
        \iparen{𝒗_i, 𝒗_j} = \delta_{i,j} 
            \quad \text{Kronecker delta\refdfn{dfn:Kronecker-delta}}
    \end{equation}
\end{definition}

無限次元の場合も同様に定義できる.

有限次元の内積空間では,正規直交基底を具体的に構成することができる.

\begin{theorem}
    \label{thm:Gram-Schmidt-process}
    \word{グラム--シュミットの正規直交化}[][くらむしゆみつとのせいきちよつこうか](Gram–Schmidt process):
    $ \{ 𝒗_1, \dots, 𝒗_n \} \subset V$を有限次元の内積空間$V$の基底とする.
    このとき,$V$の正規直交基底$ \{ 𝒖_1, \dots, 𝒖_n \} \subset V$をつくることができる.
\end{theorem}

\begin{proof}
    基底$𝒗_1, \dots, 𝒗_n$から具体的に$𝒖_1, \dots, 𝒖_n$を構成する.

    まず$𝒖_1 ≔ 𝒗_1 / \norm{𝒗_1}$とおく.
    このとき,$\iparen{𝒖_1, 𝒖_1} = 1$が成り立つ.

    次に,$𝒖'_2 ≔ 𝒗_2 - \iparen{𝒖_1, 𝒗_2} 𝒖_1$とし,
    $𝒖_2 ≔ 𝒖'_2 / \norm{𝒖'_2}$と定める.
    このとき$\iparen{𝒖_1, 𝒖_2} = 0$であり,$\iparen{𝒖_2, 𝒖_2} = 1$である.

    これを繰り返し,
    \[  𝒖'_m ≔ 𝒗_m - \sum_{i=1}^{m-1} \iparen{𝒖_i, 𝒗_m} 𝒖_i,
        \qquad 
        𝒖_m ≔ 𝒖'_m  \]
    と定めれば,$𝒖_1, \dots, 𝒖_n$は\cref{eq:orthonormal-basis}を満たす.
\end{proof}

グラム--シュミットの方法が使えるのは基底が有限個,つまり有限次元の場合だけであり,
無限次元の場合は一般に成り立たない\footnote{
    基底が\refdfn-[可算無限個]{dfn:countably-infinite}なら帰納的に示されるという考え方もあるだろうが,
    非可算無限個の場合はやはり成り立たない.
}.
無限次元ヒルベルト空間の場合は\cref{thm:basis-of-Hilbert-space}で議論する.



\subsection{内積空間の直和}

\cref{dfn:abstract-direct-sum}ではベクトル空間の直和を定義した.
内積空間の直和も定義できる.

\begin{definition}
    \label{dfn:abstract-direct-sum-of-inner-product-space}
    $\sequ{V_𝜆}[𝜆 \in 𝛬]$を内積空間の\refdfn-[族]{dfn:family-of-sets}とする.
    直積$\prod_{𝜆 \in 𝛬} V_𝜆$の部分ベクトル空間を
    \begin{equation}
        \tag{\ref*{eq:abstract-direct-sum}$'$}
        \bigoplus_{𝜆 \in 𝛬} V_𝜆
            ≔ \Set*{ \sequ{𝒗_𝜆}[𝜆 \in 𝛬] \in \prod_{𝜆 \in 𝛬} V_𝜆  
                        \given  
                        \begin{array}{c}
                            \text{有限個の$𝜆 \in 𝛬$を除いて$𝒗_𝜆 = 𝟎$}
                            \\
                            \text{かつ$\sum_{𝜆 \in 𝛬} \iparen{\symbf{x}_𝜆, \symbf{y}_𝜆} < \infty$}
                        \end{array}
                    }
    \end{equation}
    で定める.
    $\bigoplus_{𝜆 \in 𝛬} V_𝜆$の元$\sequ{𝒗_𝜆}[𝜆 \in 𝛬]$について,
    和とスカラー倍を,各成分の和とスカラー倍
    \begin{align*}
        \tag{\ref*{eq:operation-on-abstract-direct-sum}}
        \sequ{\symbf{x}_𝜆}[𝜆 \in 𝛬] + \sequ{\symbf{y}_𝜆}[𝜆 \in 𝛬]
            \coloneq \sequ{\symbf{x}_𝜆 + \symbf{y}_𝜆}[𝜆 \in 𝛬],
        \quad
        c \scaprod \sequ{\symbf{x}_𝜆}[𝜆 \in 𝛬]
            \coloneq \sequ{c \scaprod \symbf{x}_𝜆}[𝜆 \in 𝛬]
    \end{align*}
    で定めると,$\bigoplus_{𝜆 \in 𝛬} V_𝜆$はベクトル空間をなす.
    さらに$\sequ{\symbf{x}_𝜆}[𝜆 \in 𝛬], \sequ{\symbf{y}_𝜆}[𝜆 \in 𝛬] \in \bigoplus_{𝜆 \in 𝛬} V_𝜆$の内積を
    \begin{equation}
        \label{eq:inner-product-on-abstract-direct-sum}
        \iparen{\sequ{\symbf{x}_𝜆}[𝜆 \in 𝛬], \sequ{\symbf{y}_𝜆}[𝜆 \in 𝛬]}
            \coloneq \sum_{𝜆 \in 𝛬} \iparen{\symbf{x}_𝜆, \symbf{y}_𝜆}
    \end{equation}
    で定めれば,$\bigoplus_{𝜆 \in 𝛬} V_𝜆$は内積空間になる.
    これを,(抽象的な)\word{直和}\index{ちよくわ@直和!ちゆうしようてきな@抽象的な}という.
\end{definition}

直和の直観的理解については\cref{dfn:abstract-direct-sum}で議論した.
ここでは直和の内積について考えてみよう.
$\symbf{x}, \symbf{x}' \in V$,$\symbf{y}, \symbf{y}' \in W$として,
$\symbf{v} = \symbf{x} + \symbf{y}$,$\symbf{v}' = \symbf{x}' + \symbf{y}' \in V \oplus W$という``和''を考える.
形式的に内積を計算すると,
\begin{equation*}
    \iparen{ \symbf{v}, \symbf{v}' }
        = \iparen{ \symbf{x}, \symbf{x}' }
        + \iparen{ \symbf{x}, \symbf{y}' }
        + \iparen{ \symbf{y}, \symbf{x}' }
        + \iparen{ \symbf{y}, \symbf{y}' }
\end{equation*}
である.
異なるベクトル空間に属するベクトルは\refdfn[直交]{dfn:orthogonal}する\emph{だろう}から,
$\iparen{\symbf{x}, \symbf{y}'} = \iparen{\symbf{y}, \symbf{x}'} = 0$である.
したがって
\begin{equation*}
    \iparen{ \symbf{v}, \symbf{v}' }
        = \iparen{ \symbf{x}, \symbf{x}' }
        + \iparen{ \symbf{y}, \symbf{y}' }
\end{equation*}
という\cref{eq:inner-product-on-abstract-direct-sum}が得られる.

内積空間の直和として有名なものがフォック空間である.
これは,\refdfn[ヒルベルト空間]{dfn:Hilbert-space}の$n$-階\refdfn[テンソル積]{dfn:tensor-product}$\symcal{H}^{\otimes n}$の無限直和
\begin{equation*}
    F_{+,-} = \bigoplus_{n = 0}^{\infty} S_{+,-} \symcal{H}^{\otimes n}
\end{equation*}
で定義される.
$S_{+,-}$はボソン・フェルミオンに対応する対称化・反対称化の演算子$S_+, S_-$であるが,
ここでは重要でない.
$\symcal{H}^{\otimes n}$は,$n$個の粒子がある状態$\ket{1, \dots, n}$が属する空間である.
本来$n$-粒子状態$\ket{1, \dots, n}$と$m$-粒子状態$\ket{1', \dots, m'}$($m \neq n$)は,
住んでいる空間が違うので,足すことはできない.
これを直和によって足すことで,
異なる粒子数をもつ状態間の遷移を扱うことができる.



\end{document}
