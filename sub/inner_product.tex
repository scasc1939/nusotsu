\documentclass[../sotsu.tex]{subfiles}

\begin{document}

\section{内積空間とノルム空間}

これからベクトル空間の上に「\ruby{内積}{ない|せき}」および「ノルム」という演算を導入する.
内積を用いることで,2つのベクトルが「直交」するということを代数的に表現できる.
また,ノルムを用いれば2点間の距離を定義することができる.

これまでは一般の体$\symbb{K}$上のベクトル空間を考えてきたが,
内積を考える上では,体を$\symbb{R}$もしくは$\symbb{C}$に限る必要がある.
そこでこれからは$\symbb{K}$のかわりに$\symbb{C}$と書くことにする.
以降の$\symbb{C}$を$\symbb{R}$に置き換えることも容易である.


\subsection{内積空間}
\label{sec:inner-product-space}

まず、2つのベクトルから実数を得る演算である内積を定義する。

\begin{definition}[内積]
    \label{dfn:inner-product}
    $V$をベクトル空間\refdfn{dfn:vector-space}とする.
    写像$({\bigdot}, {\bigdot}) \colon V \times V \to \symbb{R}$は以下を満たすとき,\word{内積}[ない|せき][ないせき](inner product)であるという.
    \begin{itemize}
        \item 任意の$\symbf{u}, \symbf{v}, \symbf{w} \in V, \  c \in \symbb{C}$に対し,
        \begin{enumerate}
            \item \label{innerp:sum} $(\symbf{u}, \symbf{v} \tplus \symbf{w}) = (\symbf{u}, \symbf{v}) + (\symbf{u}, \symbf{w})$である.
            \item \label{innerp:scalar} $(\symbf{u}, c\symbf{v}) = c(\symbf{u}, \symbf{v})$である.
            \item \label{innerp:conjugate-symmetry} $(\symbf{u}, \symbf{v}) = \conj{(\symbf{v}, \symbf{u})}$である.
            \item \label{innerp:positive-definiteness}$(\symbf{v}, \symbf{v}) \geq 0$である.
                また,$(\symbf{v}, \symbf{v}) = 0$であるのは$\symbf{v} = \symbf{0}$であるときに限る.
        \end{enumerate}
    \end{itemize}
    内積をもつベクトル空間を\word{内積空間}[][ないせきくうかん](inner product space)という.
\end{definition}

\begin{corollary}
    内積は以下を満たす.
    \begin{itemize}
        \item 任意の$\symbf{u}, \symbf{v}, \symbf{w} \in V, \  a, b \in \symbb{C}$に対し,
        \begin{enumerate}
            \item \label{innerp:linearity} (\word{線形性}(linearity)) $(\symbf{u}, a\symbf{v} \tplus b\symbf{w}) = a (\symbf{u}, \symbf{v}) + b (\symbf{u}, \symbf{w})$
            \item \label{innerp:antilinearity} (\word{反線形性}(antilinearity)) $(a\symbf{u} \tplus b\symbf{v}, \symbf{w}) = \conj{a} (\symbf{u}, \symbf{w}) + \conj{b} (\symbf{v}, \symbf{w})$
            \item $(\symbf{v}, \symbf{0}) = (\symbf{0}, \symbf{v}) = 0$
        \end{enumerate}

        \cref{innerp:linearity,innerp:antilinearity}をあわせて\word{半双線形}[][はんそうせんけい](sesqui-linear)であるという.
    \end{itemize}
\end{corollary}


\begin{example}
    実ユークリッド空間$\symbb{R}^n$において,
    \begin{equation*}
        (\symbf{a}, \symbf{b}) \coloneq \sum_{i=1}^{n} a_i b_i
    \end{equation*}
    は内積である.また,複素ユークリッド空間$\symbb{C}^n$において,
    \begin{equation*}
        (\symbf{a}, \symbf{b}) \coloneq \sum_{i=1}^{n} \conj{a_i} b_i
    \end{equation*}
    は内積である.これらを特に\word{標準内積}[][ひようしゆんないせき]という.
\end{example}


\begin{definition}
    \label{dfn:orthogonal}
    \label{dfn:parallel}
    $V$を内積空間とする.
    ベクトル$\symbf{v}, \symbf{u} \in V \setminus \{\symbf{0}\}$は,
    $(\symbf{v}, \symbf{u}) = 0$であるとき,\word{直交する}(orthogonal)といい,$\symbf{v} \perp \symbf{u}$とかく.
    また,$c \in \symbb{C} \setminus \{0\}$を用いて$\symbf{v} = c\symbf{u}$とかけるとき,
    \word{平行である}(parallel)といい,$\symbf{v} \parallel \symbf{u}$とかく.
\end{definition}


\begin{definition}[直交補空間]
    $V$をベクトル空間、$W \subset V$を部分ベクトル空間とする。
    $W$のベクトルすべてと直交するようなベクトルの集合、つまり
    \begin{equation}
        W^\perp  \coloneq  \Set{  \symbf{v} \in V  
                                \given  \text{任意の$\symbf{u} \in W$に対して$\symbf{v} \perp \symbf{u}$}  }
    \end{equation}
    を$W$の\word{直交補空間}[][ちよつこうほくうかん](orthogonal compliment)という。
\end{definition}

\begin{corollary}
    $V$をベクトル空間とする。
    $W \subset V$の直交補空間$W^\perp$は$V$の部分ベクトル空間である。
\end{corollary}





\subsection{ノルム空間}
\label{sec:norm-space}

\subsubsection{ノルムの一般論}

\begin{definition}[ノルム]
    \label{dfn:norm}
    $V$をベクトル空間とする.
    写像$\norm{\bigdot} \colon V \to \symbb{R}$は以下を満たすとき,\word{ノルム}(norm)という.
    \begin{itemize}
        \item 任意の$\symbf{u}, \symbf{v} \in V$および$c \in \symbb{C}$に対し,
        \begin{enumerate}
            \item \label{norm:positivity}$\norm{\symbf{v}} \geq 0$である.
                また,$\norm{\symbf{v}} = 0$となるのは$\symbf{v} = \symbf{0}$のときに限る.
            \item \label{norm:absolute-homogeneity} $\norm{c\symbf{v}} = \abs{c} \norm{\symbf{v}}$である.
            \item \label{norm:triangle-inequality} (\word{三角不等式}(triangle inequality))
                $\norm{\symbf{u} \tplus \symbf{v}} \leq \norm{\symbf{u}} + \norm{\symbf{v}}$である.
                また,等号が成立するのは$\symbf{u}$と$\symbf{v}$が一次従属\refdfn{dfn:linearly-independent}のときに限る.
        \end{enumerate}
    \end{itemize}
    ノルムが定義されたベクトル空間のことを\word{ノルム空間}(normed space)という.
\end{definition}

\begin{example}
    実ユークリッド空間のベクトル$\symbf{v} = (v_1, \dots, v_n) \in \symbb{R}^n$において,以下はいずれもノルムである.
    \begin{align}
        \norm{\symbf{v}}_1      &\coloneq        \sum_{i=1}^{n}  \abs{v_i},      &
        \norm{\symbf{v}}_2      &\coloneq \sqrt{ \sum_{i=1}^{n}  \abs{v_i}^2 },  &
        \norm{\symbf{v}}_\infty &\coloneq \max_{1 \leq i \leq n} \abs{v_i}.      &
    \end{align}
\end{example}

ノルムは距離とみなすことができる(\cref{thm:norm-is-distance}).


\subsection{内積とノルムの関係}

\cref{sec:inner-product-space}では内積を,
\cref{sec:norm-space}ではノルムを導入した.
これらはそれぞれ独立に定義される概念であるが,特殊なベクトル空間においてはこの2つを統一することが可能である.
まずは内積からノルムを定義しよう.

\begin{definition}[内積から導かれるノルム]
    \label{dfn:norm-by-inner-product}
    $V$を内積空間とする.
    写像$\norm{\bigdot} \colon V \to \symbb{R}$を以下で定義すると,これはノルムである.
    \begin{equation}
        \label{eq:norm-by-inner-product}
        \norm{\symbf{v}} \coloneq \sqrt{(\symbf{v}, \symbf{v})}
    \end{equation}
    したがって,内積空間は\cref{eq:norm-by-inner-product}で定義されたノルムに対してノルム空間になる.
\end{definition}

\begin{proof}
    \cref{eq:norm-by-inner-product}が\cref{dfn:norm}を満たすことを示せばよい.
    \cref{norm:positivity}は内積の定義(\cref{dfn:inner-product})から明らかなので,\cref{norm:triangle-inequality}を示す.
    そのために,まず次の補題を示す.

    \begin{lemma}[コーシー--シュワルツの不等式]
        \label{thm:Cauchy-Schwarz-inequality}
        $V$を内積空間,$\norm{\bigdot}$を内積から導かれるノルムとする.
        このとき,任意の$\symbf{x}, \symbf{y}$について
        \begin{equation}
            \label{eq:Cauchy-Schwarz-inequality}
            \abs{(\symbf{x}, \symbf{y})} \leq \norm{\symbf{x}} \norm{\symbf{y}}
        \end{equation}
        が成り立つ(\word{コーシー--シュワルツの不等式}(Cauchy--Schwarz inequality)).
    \end{lemma}

    \begin{proof}[\cref{thm:Cauchy-Schwarz-inequality}の証明]
        $\symbf{y} = 0$のときは,$\abs{(\symbf{x}, \symbf{y})} = \norm{\symbf{x}} \norm{\symbf{y}} = 0$なので成り立つ.
        $\symbf{y} \neq 0$のときを考える.
        任意の$\lambda, \mu \in \symbb{C}$について内積の正値性から$\norm{\lambda \symbf{x} \tplus \mu \symbf{y} } \geq \symbf{0}$がいえる.
        そこで$\lambda = -\conj{(\symbf{x}, \symbf{y})}$,$\mu = \norm{\symbf{x}}^2$とおくと,
        \begin{equation*}
            \begin{split}
                0 &\leq ( \lambda \symbf{x} \tplus \mu \symbf{y}, \  \lambda \symbf{x} \tplus \mu \symbf{y} )  \\
                  &= \lambda \conj{\lambda} ( \symbf{x}, \symbf{x} ) 
                    + \lambda \conj{\mu} (\symbf{x}, \symbf{y}) 
                    + \mu \conj{\lambda} (\symbf{y}, \symbf{x}) 
                    + \mu \conj{\mu} (\symbf{y}, \symbf{y})  \\
                  &= \abs{ (\symbf{x}, \symbf{y}) }^2 \norm{\symbf{x}}^2
                    - \conj{(\symbf{x}, \symbf{y})} \norm{\symbf{x}}^2 (\symbf{x}, \symbf{y}) 
                    - \norm{\symbf{x}}^2 (\symbf{x}, \symbf{y}) (\symbf{y}, \symbf{x}) 
                    + \norm{\symbf{x}}^4 (\symbf{y}, \symbf{y})  \\
                  &= \norm{\symbf{x}}^2   \Bigl[  -\abs{ (\symbf{x}, \symbf{y}) }^2 +  \norm{\symbf{x}}^2 \norm{\symbf{y}}^2 \Bigr]
            \end{split}
        \end{equation*}
        である.$\symbf{y} \neq 0$なので,内積の正値性より$\norm{\symbf{x}} > 0$.
        そこで両辺を$\norm{\symbf{x}}^2$で割れば\cref{eq:Cauchy-Schwarz-inequality}が示される.
    \end{proof}

    \cref{thm:Cauchy-Schwarz-inequality}を用いると,
    \begin{equation*}
        \begin{split}
            \norm{\symbf{x} \vecplus \symbf{y}}^2
                &= (\symbf{x} \vecplus \symbf{y}, \  \symbf{x} \vecplus \symbf{y})  \\
                &= \norm{\symbf{x}}^2 + 2 \Re (\symbf{x}, \symbf{y}) + \norm{\symbf{y}}^2  \\
                &\leq \norm{\symbf{x}}^2 + 2 \abs{ (\symbf{x}, \symbf{y}) } + \norm{\symbf{y}}^2  \\
                &= \norm{\symbf{x}}^2 + \norm{\symbf{y}}^2
        \end{split}
    \end{equation*}
    と示される.
\end{proof}


\begin{theorem}[中線定理]
    \label{thm:parallelogram-law}
    $V$を内積空間、$\norm{\bigdot} = (\bigdot, \bigdot)$を内積から導かれるノルムとする。
    \begin{equation}
        \norm{\symbf{v} \vecplus \symbf{u}}^2 + \norm{\symbf{v} \vecplus \symbf{u}}^2
            = 2 \bigl( \norm{\symbf{v}}^2 + \norm{\symbf{u}}^2 \bigr)
    \end{equation}
    が成立する(\word{中線定理}[ちゅう|せん|てい|り][ちゆうせんていり](parallelogram law))。
\end{theorem}

\begin{proof}
    ノルムを内積に直して計算すれば容易に示される。
\end{proof}


一般の内積空間には,\cref{dfn:norm-by-inner-product}によるノルムが存在することが分かった.
それでは逆に,一般のノルム空間に内積は存在するのだろうか.
それを考えるために,まず内積空間で成立する恒等式を見る.


\begin{theorem}[偏極恒等式]
    \label{thm:polarization-identity}
    $V$を内積空間,$\norm{\bigdot}$を内積から導かれるノルムとする.
    任意のベクトル$\symbf{x}, \symbf{y} \in V$に対して,以下の\word{偏極恒等式}(polarization identity)が成立する.
    \begin{equation}
        \label{eq:polarization-identity}
        (\symbf{x}, \symbf{y})
            = \frac{1}{4} \bigl(  
                \norm{\symbf{x} \vecplus \symbf{y}}^2
                - \norm{\symbf{x} \vecminus \symbf{y}}^2
                - \iu \norm{\symbf{x} \vecplus \iu \symbf{y}}^2
                + \iu \norm{\symbf{x} \vecminus \iu \symbf{y}}^2
              \bigr)
    \end{equation}
\end{theorem}

\begin{proof}
    ノルムを内積に戻して,半双線形を用いて計算する.
    \begin{alignat*}{2}
        \norm{\symbf{x} \vecplus \symbf{y}}^2  
            &= (\symbf{x} \vecplus \symbf{y}, \  \symbf{x} \tplus \symbf{y})
            &&= +\bigl[ (\symbf{x}, \symbf{x}) + (\symbf{x}, \symbf{y}) + (\symbf{y}, \symbf{x}) + (\symbf{y}, \symbf{y}) \bigr]
            \\
        -\norm{\symbf{x} \vecminus \symbf{y}}^2
            &= -(\symbf{x} \vecminus \symbf{y}, \  \symbf{x} \vecminus \symbf{y})
            &&= -\bigl[ (\symbf{x}, \symbf{x}) - (\symbf{x}, \symbf{y}) - (\symbf{y}, \symbf{x}) + (\symbf{y}, \symbf{y}) \bigr]
            \\
        -\iu \norm{\symbf{x} \vecplus \iu \symbf{y}}^2
            &= -\iu (\symbf{x} \vecplus \iu \symbf{y}, \  \symbf{x} \vecplus \iu \symbf{y})
            &&= -\iu \bigl[ (\symbf{x}, \symbf{x}) + \iu (\symbf{x}, \symbf{y}) - \iu (\symbf{y}, \symbf{x}) + (\symbf{y}, \symbf{y}) \bigr]
            \\
        +\iu \norm{\symbf{x} \vecminus \iu \symbf{y}} 
            &= +\iu (\symbf{x} \vecminus \iu \symbf{y}, \  \symbf{x} \vecminus \iu \symbf{y})
            &&= +\iu \bigl[ (\symbf{x}, \symbf{x}) - \iu (\symbf{x}, \symbf{y}) + \iu (\symbf{y}, \symbf{x}) + (\symbf{y}, \symbf{y}) \bigr]
    \end{alignat*}
\end{proof}


内積空間においてはノルムから内積を構成できることが分かった.
しかし,一般のノルム空間で内積が構成できるとは限らない.
そのことを示すのが次の定理である.


\begin{theorem}[ジョルダン--フォン・ノイマンの定理]
    $V$をノルム空間とする.
    $V$上のノルムが中線定理\refthm{thm:parallelogram-law}を満たすならば,$V$は内積空間である
    (\word{ジョルダン--フォン・ノイマンの定理}(Jordan--von~Neumann theorem)).
\end{theorem}


\subsection{内積空間の基底}

一般のベクトル空間において基底\refdfn{dfn:basis}が存在するのであった.
内積空間においては,よい性質をもつ基底をとることができる.

\begin{definition}[正規直交基底]
    $V$を有限次元の内積空間とする.
    $V$の基底$ \{ \symbf{v}_1, \dots, \symbf{v}_n \} \subset V$が次の性質を満たすとき,
    \word{正規直交基底}[せい|き|ちょっ|こう|き|てい](orthonormal basis)という.
    \begin{equation*}
        (\symbf{v}_i, \symbf{v}_j) = \delta_{i,j} 
            \quad \text{Kronecker delta\refdfn{dfn:Kronecker-delta}}
    \end{equation*}
\end{definition}

有限次元の内積空間では,正規直交基底を具体的に構成することができる.

\begin{theorem}[グラム--シュミットの正規直交化]
    $ \{ \symbf{v}_1, \dots, \symbf{v}_n \} \subset V$を有限次元の内積空間$V$の基底とする.
    このとき,$V$の正規直交基底$ \{ \symbf{u}_1, \dots, \symbf{u}_n \} \subset V$をつくることができる
    (\word{グラム--シュミットの正規直交化}(Gram–Schmidt process)).
\end{theorem}


\end{document}
