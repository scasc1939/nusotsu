\documentclass[../sotsu.tex]{subfiles}

\begin{document}

\section{演算子}

\subsection{演算子とは}

\begin{definition}[演算子]
    $\symcal{H}, \symcal{H}'$を$\symbb{C}$上のヒルベルト空間\refdfn{dfn:Hilbert-space}とする.
    部分ベクトル空間\refdfn{dfn:vector-subspace}$\symcal{A}$から$\symcal{H}'$への写像\refdfn{dfn:map}
    \begin{equation*}
        \hat{A} \colon \symcal{A} \to \symcal{H}'
    \end{equation*}
    のことを,$\symcal{H}$から$\symcal{H}'$への\word{演算子}[えん|ざん|し][えんさんし](operator)または\word{作用素}[さ|よう|そ][さようそ]という.
\end{definition}

\begin{definition}[演算子の定義域]
    $\symcal{A}$のことを$\hat{A}$の\word{定義域}[てい|ぎ|いき][ていきいき](domain)といい,$D(\hat{A})$などとかく.
\end{definition}

$f$を$\symcal{H}$から$\symcal{H}'$への写像としたとき,任意の$x \in \symcal{H}$に対して$f(x)$が定義されていなければいけなかった.
言い換えれば,定義域が$\symcal{H}$全体である.
それに対して,$\symcal{H}$から$\symcal{H}'$への演算子$\hat{A}$は,
任意の$x \in \symcal{H}$に対して$\hat{A}(x)$が定義されている必要はなく,
部分ベクトル空間$D(\hat{A})$の元に対して定義されていれば十分である.


\begin{definition}[演算子の一致]
    $\hat{A}, \hat{B} \colon \symcal{H} \to \symcal{H}'$を演算子とする.
    $D(\hat{A}) = D(\hat{B})$かつ任意の$x \in D(A)$に対し$\hat{A}(x) = \hat{B}(x)$であるとき,
    演算子$\hat{A}$と$\hat{B}$は一致するという.
\end{definition}

2つの写像$f, g$において,定義域(もしくは終域)が一致しない場合,異なる写像とみなすのであった.
同様に,2つの演算子$\hat{A}, \hat{B}$の定義域$D(\hat{A}), D(\hat{B})$が一致しない場合,違う演算子とみなす.


\subsection{線形演算子}

ベクトル空間上の写像として,線形写像\refdfn{dfn:linear-map}というものを考えることができた.
同じように,線形な演算子を考える.
\begin{definition}[線形演算子]
    $\symbb{C}$上のヒルベルト空間$\symcal{H}$から$\symcal{H}'$への演算子$\hat{A}$は,次の性質を満たすとき,
    \word{線形演算子}[せん|けい|えん|ざん|し][せんけいえんさんし](linear operator)という.
    \begin{itemize}
        \item 任意の$\symbf{x}, \symbf{y} \in \symcal{H}$,任意の$a, b \in \symbb{C}$に対し,
        \item \begin{enumerate}
            \item (線形性)$\hat{A}( a \symbf{x} + b \symbf{y} ) = a \hat{A} (\symbf{x}) + b \hat{A} (\symbf{y})$である.
        \end{enumerate}
    \end{itemize}
\end{definition}

量子力学において,演算子が線形であることは本質的である.



\subsection{エルミート演算子}








\end{document}