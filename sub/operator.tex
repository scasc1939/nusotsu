\documentclass[../sotsu.tex]{subfiles}

\begin{document}

\section{演算子}
\label{sec:operator}

\subsection{演算子とは}
\label{sec:operator-intro}

\cref{sec:map}で扱ったように、
集合$X$の元を集合$Y$の元にうつす規則のことを\refdfn-[写像]{dfn:map}というのであった。

\begin{definition}[演算子]
    $\symcal{H}, \symcal{H}'$を$\symbb{C}$上の\refdfn-[ヒルベルト空間]{dfn:Hilbert-space}とする.
    \refdfn-[部分ベクトル空間]{dfn:vector-subspace}$\symcal{A}$から$\symcal{H}'$への写像\refdfn{dfn:map}
    \begin{equation*}
        \hat{A} \colon \symcal{A} \to \symcal{H}'
    \end{equation*}
    のことを,$\symcal{H}$から$\symcal{H}'$への\word{演算子}[えん|ざん|し][えんさんし](operator)または\word{作用素}[さ|よう|そ][さようそ]という.
\end{definition}

\begin{definition}[演算子の定義域]
    $\symcal{A}$のことを$\hat{A}$の\word{定義域}[てい|ぎ|いき][ていきいき](domain)といい,$\operatorname{Dom}(\hat{A})$,$D(\hat{A})$などとかく.
\end{definition}

$f$を$\symcal{H}$から$\symcal{H}'$への写像としたとき,任意の$x \in \symcal{H}$に対して$f(x)$が定義されていなければいけなかった.
言い換えれば,定義域が$\symcal{H}$全体である.
それに対して,$\symcal{H}$から$\symcal{H}'$への演算子$\hat{A}$は,
任意の$x \in \symcal{H}$に対して$\hat{A}(x)$が定義されている必要はなく,
部分ベクトル空間$\domain(\hat{A})$の元に対して定義されていれば十分である.


\begin{definition}[演算子の一致]
    $\hat{A}, \hat{B} \colon \symcal{H} \to \symcal{H}'$を演算子とする.
    $\domain(\hat{A}) = \domain(\hat{B})$かつ任意の$x \in \domain(A)$に対し$\hat{A}(x) = \hat{B}(x)$であるとき,
    演算子$\hat{A}$と$\hat{B}$は一致するという.
\end{definition}

2つの写像$f, g$において,定義域(もしくは終域)が一致しない場合,異なる写像とみなすのであった.
同様に,2つの演算子$\hat{A}, \hat{B}$の定義域$\domain(\hat{A}), \domain(\hat{B})$が一致しない場合,違う演算子とみなす.

\par

集合$X$から$X$自身への写像においては,すべての元を自身へとうつす\refdfn[恒等写像]{dfn:identity-map}が存在した.
それと同様に,任意の$\symbf{x} \in \symcal{H}$をそれ自身にうつす演算子
$\idop \colon \symcal{H} \to \symcal{H}$を\word{恒等演算子}[][こうとうえんさんし](identity operator)という:
$\idop(\symbf{x}) = \symbf{x}$.

\refdfn[逆写像]{dfn:inverse-map}に対応する概念も定義できる.
$\hat{A} \colon \symcal{H} \to \symcal{H}$を$\domain(\hat{A})$で定義された演算子とする.
任意の$\symbf{x} \in \domain(\hat{A})$に対し,
$\hat{A}^{-1} ( \hat{A} (\symbf{x}) ) = \hat{A} ( \hat{A}^{-1} (\symbf{x}) ) $
をみたす演算子$\hat{A}^{-1} \colon \symcal{H} \to \symcal{H}$が存在するとき,
これを$\hat{A}$の\word{逆演算子}[][きやくえんさんし](inverse operator)という.

写像$f \colon X \to Y$の定義域を$A \subset X$に\refdfn[制限]{dfn:restriction}した写像
$f \vert_A \colon A \to Y$を考えることができた.
同様に,演算子$\hat{A} \colon \symcal{H} \to \symcal{H}'$の定義域を
$\domain(\hat{B}) \subset \domain(\hat{A})$に制限した演算子$\hat{B} \colon \symcal{H} \to \symcal{H}'$を考えることができる.
写像のときと同様に,$\hat{B}$を$\hat{A}$の\word{制限}[][せいけん](restriction)という.


\subsection{線形演算子}
\label{sec:linear-operator}

ベクトル空間上の写像として,\refdfn-[線形写像]{dfn:linear-map}というものを考えることができた.
同じように,線形な演算子を考える.
\begin{definition}[線形演算子]
    $\symbb{C}$上のヒルベルト空間$\symcal{H}$から$\symcal{H}'$への演算子$\hat{A}$は,次の性質を満たすとき,
    \word{線形演算子}[せん|けい|えん|ざん|し][せんけいえんさんし](linear operator)という.
    \begin{itemize}
        \item 任意の$\symbf{x}, \symbf{y} \in \symcal{H}$,任意の$a, b \in \symbb{C}$に対し,
        \begin{enumerate}
            \item (線形性)$\hat{A}( a \symbf{x} + b \symbf{y} ) = a \hat{A} (\symbf{x}) + b \hat{A} (\symbf{y})$である.
        \end{enumerate}
    \end{itemize}
\end{definition}

量子力学において,演算子が線形であることは本質的である.

\begin{example}
    関数$f$に導関数$f'$を対応させる$C[a, b]$上の演算子$\hat{T}$を考える。
    $\hat{T}$を$C[a, b]$全域で定義することはできないので、
    定義域$\domain(T) = C^1 [a, b]$に限ると、
    \begin{equation*}
        \hat{T}(f) = f',  \qquad  
        \domain(T) = C^1 [a, b]
    \end{equation*}
    とでき、$\hat{T}$は演算子になっている\cite[\S 2.1]{kuroda-qphys}。
    また、定義域をさらに小さくして
    \begin{equation*}
        \hat{T}(f) = f',  \qquad  
        \domain(\hat{T}_0) = \Set{ f \in C^1 [a, b]  \given  f(a) = f(b) = 0 }
    \end{equation*}
    ととってもよい。
    $\hat{T}$と$\hat{T}_0$は別の演算子である。
\end{example}

線形写像における零写像と同じように,\word{零演算子}[][れいえんざんし](zero operator)を定義できる.
すなわち,任意の$\symbf{x} \in \symcal{H}$をゼロベクトル$\symbf{0} \in \symcal{H}'$にうつす演算子
$\zeroop \colon \symcal{H} \to \symcal{H}'$を零演算子という:
$\zeroop(\symbf{x}) = \symbf{0}$.

\begin{definition}
    \label{dfn:operator-sum-scalar}
    $\hat{A}, \hat{B} \colon \symcal{H} \to \symcal{H}'$を$\domain(\hat{A})$で定義された演算子,
    $c \in \symbb{K}$をスカラーとする.
    演算子の和・スカラー倍を,以下のように定義する.
    \begin{enumerate}
        \item 演算子の和$\hat{A} + \hat{B}$を,任意の$\symbf{x} \in \domain(\hat{A})$に対し,
            $( \hat{A} + \hat{B} )(\symbf{x}) = \hat{A}(\symbf{x}) + \hat{B}(\symbf{x})$であるような演算子と定める.
        \item 演算子のスカラー倍$c\hat{A}$を,任意の$\symbf{x} \in \domain(\hat{A})$に対し,
            $( c\hat{A} )(\symbf{x}) = c \cdotp \hat{A}(\symbf{x})$であるような演算子と定める.
    \end{enumerate}
    また,$\hat{E} \colon \symcal{H} \to \symcal{H}'$,$\hat{F} \colon \symcal{H}' \to \symcal{H}''$を演算子とすると,
    演算子の積は次のように定義する.
    \begin{enumerate}[resume]
        \item 演算子の積$\hat{E} \hat{F}$を,任意の$\symbf{x} \in \domain(\hat{E})$に対し,
            $ \hat{E} \hat{F} (\symbf{x}) = \hat{E} ( \hat{F} (\symbf{x}) )$であるような演算子と定める.
    \end{enumerate}
    この定義より,$\domain(\hat{F}) \supset \range(\hat{F})$でなければ演算子の積$\hat{E} \hat{F}$が定義できないことがわかる.
\end{definition}

微分演算子$\dv{}{x} \colon L^2 \to L^2$は,
ヒルベルト空間$L^2$\refdfn{dfn:square-integrable-function-space}




\subsection{エルミート演算子}
\label{sec:Hermitian-operator}










\end{document}