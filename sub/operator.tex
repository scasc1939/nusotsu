\documentclass[../sotsu.tex]{subfiles}

\begin{document}

\section{演算子}
\label{sec:operator}

\subsection{演算子とは}
\label{sec:operator-intro}

\cref{sec:map}で扱ったように,
集合$X$の元を集合$Y$の元にうつす規則のことを\refdfn-[写像]{dfn:map}というのであった.
$f \colon X \to Y$が写像であるとき,
任意の$x \in X$に対して$f(x) \in Y$が存在しなければならない.
この条件を少し弱めてみよう.

\begin{definition}[演算子]
    $\symcal{H}, \symcal{K}$を$ℂ$上の\refdfn-[ヒルベルト空間]{dfn:Hilbert-space}とする.
    \refdfn-[部分ベクトル空間]{dfn:vector-subspace}$\symcal{A}$から$\symcal{H}'$への写像\refdfn{dfn:map}
    \begin{equation*}
        \hat{A} \colon \symcal{A} \to \symcal{K}
    \end{equation*}
    のことを,$\symcal{H}$から$\symcal{K}$への\word{演算子}[えん|ざん|し][えんさんし](operator)または\word{作用素}[さ|よう|そ][さようそ]という.

    特に$\symcal{H} = \symcal{K}$のとき,
    $\hat{T}$を$\symcal{H}$上の演算子という.
\end{definition}

\begin{definition}[演算子の定義域]
    $\symcal{A}$のことを$\hat{A}$の\word{定義域}[てい|ぎ|いき][ていきいき](domain)といい,$\operatorname{Dom}(\hat{A})$,$D(\hat{A})$などとかく.
\end{definition}

$\symcal{H}$から$\symcal{K}$への写像$f$の定義域は$\symcal{H}$全体である.
それに対して,$\symcal{H}$から$\symcal{K}$への演算子$\hat{A}$は,
任意の$x \in \symcal{H}$に対して$\hat{A}(x)$が定義されている必要はなく,
部分ベクトル空間$\domain(\hat{A})$の元に対して定義されていれば十分である.


\begin{definition}[演算子の一致]
    $\hat{A}, \hat{B} \colon \symcal{H} \to \symcal{K}$を演算子とする.
    $\domain(\hat{A}) = \domain(\hat{B})$かつ任意の$x \in \domain(A)$に対し$\hat{A}(x) = \hat{B}(x)$であるとき,
    演算子$\hat{A}$と$\hat{B}$は一致するという.
\end{definition}

2つの写像$f, g$において,定義域(もしくは終域)が一致しない場合,異なる写像とみなすのであった.
同様に,2つの演算子$\hat{A}, \hat{B}$の定義域$\domain(\hat{A}), \domain(\hat{B})$が一致しない場合,違う演算子とみなす.

\par

集合$X$から$X$自身への写像においては,すべての元を自身へとうつす\refdfn[恒等写像]{dfn:identity-map}が存在した.
それと同様に,任意の$𝒙 \in \symcal{H}$をそれ自身にうつす演算子
$\idop \colon \symcal{H} \to \symcal{H}$を\word{恒等演算子}[][こうとうえんさんし](identity operator)という:
$\idop(𝒙) = 𝒙$.

\refdfn[逆写像]{dfn:inverse-map}に対応する概念も定義できる.
$\hat{A} \colon \symcal{H} \to \symcal{H}$を$\domain(\hat{A})$で定義された演算子とする.
任意の$𝒙 \in \domain(\hat{A})$に対し,
$\hat{A}^{-1} ( \hat{A} (𝒙) ) = \hat{A} ( \hat{A}^{-1} (𝒙) ) $
をみたす演算子$\hat{A}^{-1} \colon \symcal{H} \to \symcal{H}$が存在するとき,
これを$\hat{A}$の\word{逆演算子}[][きやくえんさんし](inverse operator)という.

写像$f \colon X \to Y$の定義域を$A \subset X$に\refdfn[制限]{dfn:restriction}した写像
$f \vert_A \colon A \to Y$を考えることができた.
同様に,演算子$\hat{A} \colon \symcal{H} \to \symcal{K}$の定義域を
$\domain(\hat{B}) \subset \domain(\hat{A})$に制限した演算子$\hat{B} \colon \symcal{H} \to \symcal{K}$を考えることができる.
写像のときと同様に,$\hat{B}$を$\hat{A}$の\word{制限}[][せいけん](restriction)という.


\subsection{線形演算子}
\label{sec:linear-operator}

\cref{sec:linear-map}では,
ベクトル空間上の\refdfn-[線形写像]{dfn:linear-map}を扱った.
同じように,ベクトル空間上の線形な演算子を考えることができる.
\begin{definition}[線形演算子]
    $ℂ$上のヒルベルト空間$\symcal{H}$から$\symcal{H}'$への演算子$\hat{A}$は,次の性質を満たすとき,
    \word{線形演算子}[せん|けい|えん|ざん|し][せんけいえんさんし](linear operator)という.
    \begin{itemize}
        \item 任意の$𝒙, 𝒚 \in \symcal{H}$,任意の$a, b \in ℂ$に対し,
        \begin{enumerate}
            \item (線形性)$\hat{A}( a 𝒙 + b 𝒚 ) = a \hat{A} (𝒙) + b \hat{A} (𝒚)$である.
        \end{enumerate}
    \end{itemize}
\end{definition}

量子力学において,演算子が線形であることは本質的である.

\begin{example}
    関数$f$に導関数$f'$を対応させる$C[a, b]$上の演算子$\hat{D}$を考える.
    $\hat{D}$を$C[a, b]$全域で定義することはできないので,
    定義域$\domain(T) = C^1 [a, b]$に限ると,
    \begin{equation*}
        \hat{D}(f) = f',  \qquad  
        \domain(D) = C^1 [a, b]
    \end{equation*}
    とでき,$\hat{D}$は$C[a, b]$から$C[a, b]$への演算子になっている\cite[\S 2.1]{kuroda-qphys-2007}.
\end{example}

線形写像における\refdfn-[零写像]{dfn:zero-linear-map}と同じように,
\word{零演算子}(zero operator)%
\index{れい@零!えんさんし@\indexdash 演算子}を定義できる.
すなわち,任意の$𝒙 \in \symcal{H}$をゼロベクトル$\symbf{0} \in \symcal{K}$にうつす演算子
$\zeroop \colon \symcal{H} \to \symcal{K}$を零演算子という:
$\zeroop(𝒙) = \symbf{0}$.

線形演算子全体もまたベクトル空間をなす.

\begin{definition}
    \label{dfn:operator-sum-scalar}
    $\hat{A}, \hat{B} \colon \symcal{H} \to \symcal{H}'$を$\domain(\hat{A})$で定義された演算子,
    $c \in 𝕂$をスカラーとする.
    演算子の和・スカラー倍を,以下のように定義する.
    \begin{enumerate}
        \item 演算子の和$\hat{A} + \hat{B}$を,任意の$𝒙 \in \domain(\hat{A})$に対し,
            $( \hat{A} + \hat{B} )(𝒙) = \hat{A}(𝒙) + \hat{B}(𝒙)$であるような演算子と定める.
        \item 演算子のスカラー倍$c\hat{A}$を,任意の$𝒙 \in \domain(\hat{A})$に対し,
            $( c\hat{A} )(𝒙) = c \cdotp \hat{A}(𝒙)$であるような演算子と定める.
    \end{enumerate}
    また,$\hat{E} \colon \symcal{H} \to \symcal{H}'$,$\hat{F} \colon \symcal{H}' \to \symcal{H}''$を演算子とすると,
    演算子の積は次のように定義する.
    \begin{enumerate}[resume]
        \item 演算子の積$\hat{E} \hat{F}$を,
            定義域を
            \begin{equation*}
                \domain(\hat{E} \hat{F}) \coloneq 
                    \Set{  𝒙 \in \domain(\hat{E}) \given  \hat{E} (𝒙) \in \domain(\hat{F}) }
            \end{equation*}
            とし,任意の$𝒙 \in \domain(\hat{E} \hat{F})$に対し,
            $ \hat{E} \hat{F} (𝒙) = \hat{E} ( \hat{F} (𝒙) )$であるような演算子と定める.
    \end{enumerate}
\end{definition}

よく$\hat{T} (𝒙) = \hat{T} 𝒙$と書く.




\subsection{エルミート演算子}
\label{sec:Hermitian-operator}



\subsection{ユニタリ演算子}
\label{sec:unitary-operator}

\begin{definition}
    \label{dfn:unitary-operator}
    ヒルベルト空間$\symcal{H}$から$\symcal{K}$への演算子$\hat{U}$が以下を満たすとき,
    \word{ユニタリ演算子}(unitary operator)%
    \index{ゆにたり@ユニタリ!えんさんし@\indexdash 演算子}%
    という.
    \begin{enumerate}
        \item $\hat{U}$の定義域は$\symcal{H}$全体である.
            すなわち$\domain(\hat{U}) = \symcal{H}$.
        \item $\hat{U}$は全射である.
            すなわち$\range(\hat{U}) = \symcal{K}$.
        \item $\hat{U}$は内積を保つ.
            すなわち,任意の$𝒙, 𝒚 \in \symcal{H}$に対し,
            \[ \iparen{\hat{U} 𝒙, \hat{U} 𝒚}_{\symcal{K}} = \iparen{𝒙, 𝒚}_{\symcal{H}} \]
            特に,ユニタリ演算子は\word{等長}(isometric)\index{とうちよう@等長}である:
            \[ \norm{\hat{U} 𝒙}_{\symcal{K}} = \norm{𝒙}_{\symcal{H}} \]
    \end{enumerate}
\end{definition}

ユニタリ演算子は\refdfn-[単射]{dfn:injection}である.
なぜなら,$\hat{U} 𝒙 = \hat{U} 𝒚$とすると,線形性より$\hat{U} (𝒙 - 𝒚) = \symbf{0}$.
等長性より$\norm{𝒙 - 𝒚}_{\symcal{H}} = \norm{\symbf{0}}_{\symcal{K}} = 0$であるから,$𝒙 = 𝒚$がいえる.

\begin{definition}[ヒルベルト空間の同型]
    \label{dfn:Hilbert-space-isomorphic}
    ヒルベルト空間$\symcal{H}$から$\symcal{K}$へのユニタリ演算子が存在するとき,
    $\symcal{H}$と$\symcal{K}$は\word{同型}(isomorphic)\index{とうけい@同型}であるという.
\end{definition}

2つのベクトル空間のあいだに,線形構造を変えない全単射が存在するとき,
これらは\refdfn[同型]{dfn:isomorphic}であるといった.
ヒルベルト空間においては,
これに加えて内積(ノルム)構造を変えないことも要求する.



\subsection{演算子ノルム}

\begin{definition}[有界な演算子]
    \label{dfn:bounded-operator}
    ヒルベルト空間$\symcal{H}$から$\symcal{K}$への線形演算子$\hat{T}$が\word{有界}[ゆう|かい](bounded)\index{ゆうかい@有界}であるとは,
    ある定数$C > 0$が存在して,
    任意の$\symbf{x} \in \domain(\hat{T})$に対し,
    \begin{equation*}
        \hat{T} \symbf{x} \leq C \symbf{x}
    \end{equation*}
    となることをいう.
    有界でない演算子は,\word{非有界}(unbounded)\index{ひゆうかい@非有界}であるという.
\end{definition}

\begin{definition}[演算子ノルム]
    \label{dfn:operator-norm}
    有界な線形演算子$\hat{T}$に対し,
    \begin{equation}
        \opnorm{\hat{T}} \coloneq \sup_{\symbf{x} \in \domain(\hat{T})} \frac{\norm{\hat{T} \symbf{x}}}{\norm{\symbf{x}}}
    \end{equation}
    を演算子$\hat{T}$の\word{ノルム}(norm)\index{のるむ@ノルム}という.
\end{definition}

例えば,\refdfn-[ユニタリ演算子]{dfn:unitary-operator}$\hat{U}$のノルムは,
等長性より$\opnorm{\hat{U}} = 1$である.

演算子の有界性を議論するうえで有用な\namecref{thm:unbounded-operator-iff}を述べる.

\begin{proposition}
    \label{thm:unbounded-operator-iff}
    線形演算子$\hat{T}$が非有界である必要十分条件は,
    点列$\sequ{\symbf{x}_n} \subset \domain(\hat{T})$が存在して,
    $n \to \infty$で$\hat{T} \symbf{x}_n \to \infty$となることである.
\end{proposition}

このことはほぼ自明であるから,証明は省略する.


\begin{proposition}
    定義域を$C_0^1 (\symbb{R})$とする$L^2 (\symbb{R}^N)$上の微分演算子$\hat{D}$は,
    $L^2 (\symbb{R}^N)$の位相の下で(つまりノルム$\norm{\bigdot}_{L^2 (\symbb{R}^N)}$を用いて)有界でない.
\end{proposition}


\begin{lemma}
    \label{thm:norm-and-operator-norm}
    有界な線形演算子$\hat{T}$について,
    任意の$\symbf{x} \in \hat{T}$に対し,
    \begin{equation*}
        \norm{\hat{T} \symbf{x}} \leq \opnorm{T} \norm{\symbf{x}}
    \end{equation*}
\end{lemma}

\begin{proof}
    \refdfn-[演算子ノルムの定義]{dfn:operator-norm}より明らかである.
\end{proof}



\begin{proposition}
    有界な線形演算子$\hat{S}, \hat{T}$について,
    \begin{equation*}
        \opnorm{\hat{S} \hat{T}} \leq \opnorm{\hat{S}} \opnorm{\hat{T}}
    \end{equation*}
\end{proposition}

\begin{proof}
    \cref{thm:norm-and-operator-norm}より,
    任意の$\symbf{x} \in \domain(\hat{S} \hat{T})$に対し,
    \begin{equation*}
        \norm{(\hat{S} \hat{T}) \symbf{x}}
        = \norm{\hat{S} (\hat{T} \symbf{x})}
        \leq \opnorm{\hat{S}} \norm{\hat{T} \symbf{x}}
        \leq \opnorm{\hat{S}} \opnorm{\hat{T}} \norm{\symbf{x}}
    \end{equation*}
    であることから従う.
\end{proof}




\end{document}