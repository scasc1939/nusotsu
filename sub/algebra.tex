\documentclass[../sotsu.tex]{subfiles}

\begin{document}

\section{抽象代数学の基礎}

\subsection{群}

\begin{definition}[群]
    \label{dfn:group}
    $G$を集合,$\dotprod \colon G \times G \to G$を二項演算とする.
    $G$の元が,演算$\dotprod$に対して以下を満たすとき,
    $(G, \dotprod)$は\ltword{群}[ぐん](group)\index{くん@群}であるという.
    \begin{enumerate}
        \item 任意の$a, b, c \in G$に対し,$(a \dotprod b) \dotprod c = a \dotprod (b \dotprod c)$である.
        \item ある$e \in G$が存在して,任意の$a \in G$に対し,$a \dotprod e = e \dotprod a = a$である.
        \item 任意の$a \in G$に対し,ある$a^{-1} \in G$が存在して,$a \dotprod a^{-1} = a^{-1} \dotprod a = e$である.
    \end{enumerate}
    さらに以下を満たすとき,$(G, \dotprod)$を%
    \ltword{アーベル群}(Abelian group)%
    \index{ああへるくん@アーベル群}%
    であるという.
    \begin{enumerate}[resume]
        \item 任意の$a, b \in G$に対し,$a \dotprod b = b \dotprod a$である.
    \end{enumerate}
    群$(G, \dotprod)$のことを単に$G$とかくこともある.
\end{definition}


自然数全体の集合$\symbb{N}$は,通常の和$+$に対して群をなす.

実数全体の集合$ℝ$は,通常の積$\times$に対して群をなさない.
なぜなら,$0 \in ℝ$に対して,$0 \times x = 1$となるような$x \in ℝ$が存在しないため.


\subsection{環}

\begin{definition}[環]
    \label{dfn:ring}
    $R$を集合,$+$と$\dotprod$を$R$上の二項演算とする.
    以下がみたされるとき,$(R, +, \dotprod)$は%
    \ltword{環}[かん](ring)%
    \index{かん@環}%
    であるという.
    \begin{itemize}
        \item 和について,$(R, +)$はアーベル群である.すなわち
        \begin{enumerate}
            \item \label{ring:sum-associative} 任意の$a, b, c \in R$に対し,$(a + b) + c = a + (b + c)$である.
            \item \label{ring:sum-zero} ある$0 \in R$が存在して,任意の$a \in R$に対し,$a + 0 = 0 + a = a$である.
            \item \label{ring:sum-opposite} 任意の$a \in R$に対し,ある$-a \in R$が存在して,$a + (-a) = (-a) + a = 0$である.
            \item \label{ring:sum-commutative} 任意の$a, b \in R$に対し,$a + b = b + a$である.
        \end{enumerate}
        \item 積について,
        \begin{enumerate}[resume]
            \item \label{ring:prod-associative} 任意の$a, b, c \in R$に対し,$(a \dotprod b) \dotprod c = a \dotprod (b \dotprod c)$である.
            \item \label{ring:prod-one} ある$1 \in R$が存在して,任意の$a \in R$に対し,$a \dotprod 1 = 1 \dotprod a = a$である.
        \end{enumerate}
        \item 和と積の関係について,
        \begin{enumerate}[resume]
            \item \label{ring:distributive} 任意の$a, b, c \in R$に対し,$(a + b) \dotprod c = a \dotprod c + b \dotprod c$であり,$a \dotprod (b + c) = a \dotprod b + a \dotprod c$である.
        \end{enumerate}
    \end{itemize}
    さらに積について以下がみたされるとき,$(R, +, \dotprod)$は%
    \ltword{可換環}(commutative ring)%
    \index{かん@環!かかん@可換\indexdash}%
    \index{かかん@可換!かん@\indexdash 環|see{環(かん)}}%
    であるという.
    \begin{enumerate}[start=9]
        \item 任意の$a, b \in R$に対し,$a \dotprod b = b \dotprod a$である.
    \end{enumerate}
    二項演算$+$のことを%
    \ltword{和}[わ](sum)%
    \index{わ@和}%
    あるいは%
    \ltword{加|法}[か|ほう](addition)%
    \index{かほう@加法}%
    ,
    $\dotprod$のことを%
    \ltword{積}[せき](product)%
    \index{せき@積}%
    あるいは%
    \ltword{乗|法}[じょう|ほう](multiplication)%
    \index{しようほう@乗法}%
    という.
\end{definition}

\begin{example}
    $n$-次正方行列全体の集合は,行列の和$+$と積$\dotprod$に関して環であるが,$n = 1$の場合をのぞき可換環でない.
\end{example}



\subsection{体}

これから実数・複素数のもつ性質を抽象化した「\ltjruby{体}{たい}」を定義する.
量子力学において扱う体は基本的に複素数体$\symbb{C}$であるので,わざわざ抽象化を行う動機が見えづらいかもしれない.
しかし,例えば量子情報の分野においては体として有限体を扱うことがあるので,ここで一般の体についての性質を議論しておく.

\begin{definition}[体]
    \label{dfn:field}
    $𝕂$を集合,$+$と$\dotprod$を$𝕂$上の二項演算とする.
    以下がみたされるとき,$(𝕂, +, \dotprod)$は\ltword{体}[たい](field)\index{たい@体}であるという.
    \begin{itemize}
        \item 和について,$(𝕂, +)$はアーベル群である.
            すなわち,
        \begin{enumerate}
            \item \label{field:sum-associative} 任意の$a, b, c \in 𝕂$に対し,$(a + b) + c = a + (b + c)$である.
            \item \label{field:sum-zero} ある$0 \in 𝕂$が存在して,任意の$a \in 𝕂$に対し,$a + 0 = 0 + a = a$である.
            \item \label{field:sum-opposite} 任意の$a \in 𝕂$に対し,ある$-a \in 𝕂$が存在して,$a + (-a) = (-a) + a = 0$である.
            \item \label{field:sum-commutative} 任意の$a, b \in 𝕂$に対し,$a + b = b + a$である.
        \end{enumerate}
        \item 積について,$(𝕂 \setminus \{ 0 \}, \  \dotprod)$はアーベル群である.
            すなわち,
        \begin{enumerate}[resume]
            \item \label{field:prod-associative} 任意の$a, b, c \in 𝕂$に対し,$(a \dotprod b) \dotprod c = a \dotprod (b \dotprod c)$である.
            \item \label{field:prod-one} ある$1 \in 𝕂$が存在して,任意の$a \in 𝕂$に対し,$a \dotprod 1 = 1 \dotprod a = a$である.
            \item \label{field:prod-reciprocal} 任意の$a \in 𝕂 \setminus \{0\}$に対し,ある$a^{-1} \in 𝕂$が存在して,$a \dotprod a^{-1} = a^{-1} \dotprod a = 1$である.
            \item \label{field:prod-commutative} 任意の$a, b \in 𝕂$に対し,$a \dotprod b = b \dotprod a$である.
        \end{enumerate}
        \item 和と積の関係について,
        \begin{enumerate}[resume]
            \item \label{field:distributive} 任意の$a, b, c \in 𝕂$に対し,$(a + b) \dotprod c = (a \dotprod c) + (b \dotprod c)$であり,$a \dotprod (b + c) = (a \dotprod b) + (a \dotprod c)$である.
        \end{enumerate}    
    \end{itemize}
\end{definition}

\begin{example}
    有理数全体の集合$\symbb{Q}$,実数全体の集合$ℝ$,複素数全体の集合$ℂ$は,
    それぞれ通常の和と積に関して体になる.
\end{example}

体をひとことでいえば,「四則演算(和・差・積・商)ができる集合」である.
実数や複素数は体の典型的なものである.
本稿で「体$𝕂$」と書いてある場合,実数$ℝ$もしくは複素数$ℂ$と読み替えて差支えない.



\subsection{実数}
\label{sec:real-number-field}



\begin{lemma}[相加相乗平均の関係]
    \label{thm:arithmetic-mean-vs-geometric-mean}
    2つの正の実数$a, b > 0$に対し,
    \begin{equation}
        \label{eq:arithmetic-mean-vs-geometric-mean}
        2 \sqrt{ a b } \leq a + b
    \end{equation}
    が成り立つ\footnote{
        $\displaystyle \frac{a + b}{2}$を\ltjruby{相|加}{そう|か}平均(または算術平均),
        $\sqrt{ab}$を\ltjruby{相|乗}{そう|じょう}平均(または幾何平均)という.
    }.
    等号($=$)が成立するのは$a = b$のとき,
    またその時に限る.
\end{lemma}


\begin{proof}
    両辺は正であるので,両辺を2乗して右辺から左辺を引けば,
    \begin{equation*}
        (a + b)^2 - (2 \sqrt{ a b })^2
            = a^2 - 2 a b + b^2
            = (a - b)^2
            \geq 0
    \end{equation*}
    等号が成り立つのは$a - b = 0$すなわち$a = b$のとき.
\end{proof}


コーシー・シュワルツの不等式と呼ばれるものにはいくつかのバリエーションがある.
下に挙げる\cref{thm:real-Cauchy-Schwarz-inequality}のほかに,
\cref{thm:Euclidean-Cauchy-Schwarz}(\refdfn-[内積]{dfn:inner-product})などがある.

\begin{lemma}
    \label{thm:real-Cauchy-Schwarz-inequality}
    $a_1, \dots, a_N, b_1, \dots, b_N$を実数とする.
    このとき,以下の\ltword{コーシー・シュワルツの不等式}(Cauchy--Schwarz inequality)%
    \index{こおしいしゆわるつのふとうしき@コーシー--シュワルツの不等式}%
    が成立する.
    \begin{equation}
        \label{eq:real-Cauchy-Schwarz-inequality}
        \ab( \sum_{i = 1}^{N} a_i b_i )^2
        \leq
        \ab( \sum_{i = 1}^{N} a_i^2 )
        \ab( \sum_{i = 1}^{N} b_i^2 )
    \end{equation}
\end{lemma}


\begin{proof}
    いろいろな証明方法がある(たとえばユークリッド空間$\symbb{R}^N$に対して\cref{thm:Euclidean-Cauchy-Schwarz}を用いる方法)が,
    ここでは泥臭い方法で示す.
    \begin{enumerate}
        \item \bluehead{$N = 2$のとき}\quad
            示すべきは
            \begin{equation}
                \tag{\ref*{eq:real-Cauchy-Schwarz-inequality}$'$}
                \label{eq:eq:real-2-Cauchy-Schwarz-inequality}
                \ab( a_1 b_1 + a_2 b_2 )^2
                \leq
                \ab( a_1^2 + a_2^2 )
                \ab( b_1^2 + b_2^2 )
            \end{equation}
            そこで,右辺から左辺を引くと,
            \begin{equation*}
                \begin{split}
                    \MoveEqLeft
                    \ab( a_1^2 + a_2^2 ) \ab( b_1^2 + b_2^2 )
                        - \ab( a_1 b_1 + a_2 b_2 )^2
                    \\
                    &= a_1^2 b_2^2 + a_2^2 b_1^2 - 2 a_1 a_2 b_1 b_2
                    \\
                    &= (a_1 b_2 - a_2 b_1)^2
                    \geq 0
                \end{split}
            \end{equation*}
        \item \bluehead{一般の$N$について}\quad
            ある$N$に対し\cref{eq:real-Cauchy-Schwarz-inequality}が成り立つと仮定する.
            このとき,
            \begin{equation*}
                \begin{split}
                    \ab( \sum_{i = 1}^{N+1} a_i b_i )^2
                    &= \ab( \sum_{i = 1}^{N} a_i b_i + a_{N+1} b_{N+1} )^2
                    \\
                    &= \ab( \sum_{i = 1}^{N} a_i b_i )^2 
                        + 2 a_{N+1} b_{N+1} \sum_{i = 1}^{N} a_i b_i
                        + a_{N+1}^2 b_{N+1}^2
                    \\
                    &\leq 
                        \ab( \sum_{i = 1}^{N} a_i^2 )
                        \ab( \sum_{i = 1}^{N} b_i^2 )
                        + 2 a_{N+1} b_{N+1} 
                            \sqrt{ \sum_{i = 1}^{N} a_i^2 }
                            \sqrt{ \sum_{i = 1}^{N} b_i^2 }
                        + a_{N+1}^2 b_{N+1}^2
                    \\
                    &\leq 
                        \ab( \sum_{i = 1}^{N} a_i^2 )
                        \ab( \sum_{i = 1}^{N} b_i^2 )
                        + \ab( b_{N+1}^2 \sum_{i = 1}^{N} a_1^2 )
                        + \ab( a_{N+1}^2 \sum_{i = 1}^{N} b_1^2 )
                        + a_{N+1}^2 b_{N+1}^2
                    \\
                    &=
                        \ab( \sum_{i = 1}^{N} a_i^2 + a_{N+1}^2 )
                        \ab( \sum_{i = 1}^{N} b_i^2 + b_{N+1}^2 )
                    \\
                    &= 
                    \ab( \sum_{i = 1}^{N+1} a_i^2 )
                    \ab( \sum_{i = 1}^{N+1} b_i^2 )
                \end{split}
            \end{equation*}
            1つ目の$\leq$で仮定を使った.
            2つ目の$\leq$は\refdfn[相加相乗平均の関係]{thm:arithmetic-mean-vs-geometric-mean}である.
            したがって,$N + 1$に対しても\cref{eq:real-Cauchy-Schwarz-inequality}が成り立つ.
            $N = 2$のときに成り立つことはすでに見たので,
            すべての自然数$N$($\geq 2$)に対して\cref{eq:real-Cauchy-Schwarz-inequality}が成り立つ.
    \end{enumerate}
\end{proof}






\subsection{複素数}
\label{sec:complex-number-field}

複素数は物理学を扱ううえで必須のツールである.
この節では,まず複素数はどのようにして定義されるのかを見たのちに,
複素数に関する用語を定義し,その性質を調べる.


複素数は次のように構成される\footnote{
    導出方法を理解する必要はないが,
    複素数がユークリッド平面($\symbb{R}^2$)と同じだということを知っていると,
    複素平面を理解しやすい.
}.
集合$\symbb{C}$を
\begin{equation}
    \symbb{C} \coloneq \Set{  (a, b)  \given  a, b \in \symbb{R}  }
\end{equation}
で定める.
もちろん$\symbb{C} = \symbb{R}^2$である.
$\symbb{C}$の元である順序対$(a, b) \in \symbb{R}$に対し,和と積を次のように定義する.
\begin{align}
    \label{eq:definition-of-sum-of-complex-number}
    (a, b) \underset{\symbb{C}}{+} (c, d) &\coloneq (a+c, \  b+d)  \\
    \label{eq:definition-of-product-of-complex-number}
    (a, b) \underset{\symbb{C}}{\times} (c, d) &\coloneq (ac-bd, \  ad+bc)
\end{align}
すると,$(\symbb{C}, \underset{\symbb{C}}{+}, \underset{\symbb{C}}{\times})$は\refdfn-[体]{dfn:field}になるので,
これを複素数体と呼ぶ.
$\symbb{C}$の元を\ltword{複|素|数}[ふく|そ|すう]\index{ふくそすう@複素数}と呼び,
記号$\iu$を用いて$(a, b) \eqcolon a + b\iu$とかく.

\begin{proof}
    $\symbb{C}$の元を$x = (a, b)$,$y = (c, d)$,$z = (e, f)$とする.
    積の記号$\underset{\symbb{C}}{\times}$は一部省略する.

    \bluehead{和について}\quad
    和が可換かつ結合的であることは明らか.
    零元は$0_{\symbb{C}} \coloneq (0, 0) = 0 + 0\iu$であり,
    $x$に対する和の逆元は$-x = (-a, -b)$である.

    \bluehead{積について}\quad
    積が可換であることは明らか.
    単位元は$1_{\symbb{C}} \coloneq (1, 0) = 1 + 0\iu$であり,
    $x \neq 0_{\symbb{C}}$に対する積の逆元は$(a / \sqrt{a^2 + b^2}, \  -b / \sqrt{a^2 + b^2} )$である.
    積が結合的であることは
    \begin{equation*}
        \begin{split}
            (xy)z &= (ac-bd, \  bc+ad) \underset{\symbb{C}}{\times} (e, f)   \\
                %   &= \cdots  \\
                  &= (a, b) \underset{\symbb{C}}{\times} (ce-df, \  de+cf)   \\
                  &= x(yz)
        \end{split}
    \end{equation*}

    \bluehead{分配法則}\quad
    \begin{equation*}
        \begin{split}
            (x \underset{\symbb{C}}{+} y) \underset{\symbb{C}}{\times} z
            &= (a+c, \  b+d) \underset{\symbb{C}}{\times} (e, f)   \\ 
            &= (ae+ce-bf-df, \  af+cf+be+de)   \\
            &= (ae-bf, \  af+be) \underset{\symbb{C}}{+} (ce-df, \  cf+de)   \\
            &= (x \underset{\symbb{C}}{\times} z) \underset{\symbb{C}}{+}
               (y \underset{\symbb{C}}{\times} z)
        \end{split}
    \end{equation*}
\end{proof}

複素数の積を形式的に展開すると,
$(ac + \iu^2 bd) + \iu (bc + ad)$となる.
これと積の定義(\cref{eq:definition-of-product-of-complex-number})を見比べると,
$\iu^2 \equiv -1$と解釈できる.
これが虚数単位$\iu$である.

実数$a \in \symbb{R}$は,複素数$a = (a, 0)$とみなすことで,
複素数に組み込むことができる($\symbb{R} \subset \symbb{C}$).
部分集合$\symbb{C} \setminus \symbb{R}$の元を\ltword{虚|数}[きょ|すう]\index{きよすう@虚数}という%
\footnote{
    虚数とは$a + \iu b$($a, b \in \symbb{R}$,$b \neq 0$)とあらわせる数のことである.
    したがって,$3\iu$のみならず$1 + 3\iu$も虚数である.
    $\iu b$($b \in \symbb{R}$)とあらわされる数(たとえば$3\iu, -2\iu$)を指す用語は\word{純虚数}である.
}.

複素数$z = a + b\iu$($a, b$は実数)に対し,
$\Real z = a$を$z$の\ltword{実|部}[じつ|ぶ]\index{しつふ@実部},
$\Imaginary z = b$を$z$の\ltword{虚|部}[きょ|ぶ]\index{きよふ@虚部}という.

複素数の\ltword{絶|対|値}[ぜっ|たい|ち](absolute value)\index{せつたいち@絶対値}を
\begin{equation}
    \label{eq:complex-number-abs}
    \abs{z} 
        = \abs{a + b\iu}
        \coloneq \sqrt{a^2 + b^2}
\end{equation}
で定める.$0 \leq \abs{z} < \infty$であり,
$\abs{z} = 0$となるのは$a = b = 0$,つまり$z = 0$のときに限る.



\subsubsection*{複素共役}

複素数$a + \iu b$に対する複素数$a - \iu b$のことを,
\ltword{共|役|複|素|数}[きょう|やく|ふく|そ|すう]%
\index{きようやく@共役!ふくそすう@複素数}%
あるいは\ltword{複|素|共|役}[ふく|そ|きょう|やく](complex conjugate)%
\index{複素共役@ふくそきようやく}%
という%
\footnote{\label{footnote:joyo-kanji}
    \ltjruby{共|役}{きょう|えき}は誤読.
    もともとは\ltjruby{軛}{くびき}(\ltjruby{牛|車}{ぎっ|しゃ}などを引く牛馬に取り付ける横向きの棒)という字を用いて\ltjruby{共|軛}{きょう|やく}とかいていたが,
    1946年に定められた当用漢字,1981年の常用漢字への書き換えに伴い\ltjruby{共|役}{きょう|やく}と書かれるようになった.
    同様の理由で,
    \ltjruby{交|叉}{こう|さ}は交\underline{差},
    \ltjruby{函|数}{かん|すう}(函は「箱」の意)は\underline{関}数と書き換えられた.
    \cite{ito-lebesgue-1963}(1963年)のように古い書籍は「函数」表記を用いている.
    
    常用漢字表に含まれる漢字は意外に少ない.
    例えば\underline{楕}円は表外字である.
    \underline{汎}関数,\underline{勾}配は2010年の改訂でくわえられた.
    常用漢字への書き換えの例については,
    Wikipedia日本語版の「\href{https://ja.wikipedia.org/wiki/同音の漢字による書きかえ}{同音の漢字による書きかえ}」に詳しい.
}.
複素数$z$の複素共役を,$\bar{z}$や$z^*$と書く.

複素共役を用いると,
$z$の絶対値は
\begin{equation}
    \abs{z}^2 = z \cdotp \conj{z}
\end{equation}
とかける.
実際の計算でもよく使う,非常に有用な関係式である.




\subsubsection*{複素数についての性質}

\begin{proposition}
    $\frac{1}{\iu} \equiv \iu^{-1} = -\iu$である.
\end{proposition}

\begin{proof}
    $\iu^{-1}$とは$\iu$の積に関する逆元であるから,
    $\iu^{-1} \cdotp \iu = 1$でなければならない.
    ここで,$(-\iu) \cdotp \iu = 1$であること,
    および逆元の一意性(\refdfn-[体]{dfn:field}の公理)から,$\iu^{-1} = -\iu$である.
\end{proof}

\begin{proposition}
    複素数$z_1, z_2$の絶対値について,
    以下が成り立つ.
    \begin{enumerate}
        \item $\abs{z_1 z_2} = \abs{z_1} \abs{z_2}$
        \item $\abs{z_1 + z_2} \leq \abs{z_1} + \abs{z_2}$
            (\ltword{三角不等式}\index{さんかくふとうしき@三角不等式})
    \end{enumerate}
\end{proposition}

\begin{proof}
    $z_1 = a_1 + \iu b_1$,$z_2 = a_2 + \iu b_2$とおく.
    \begin{enumerate}
        \item 次のような計算で示される.
            \begin{equation*}
                \begin{split}
                    \abs{z_1 z_2}^2 
                        &= \abs{ a_1 a_2 + \iu a_1 b_2 + \iu a_2 b_1 - a_2 b_2 }^2   \\
                        &= (a_1 a_2 - b_1 b_2)^2 + (a_1 b_2 + a_2 b_1)^2   \\
                        &= (a_1^2 + b_1^2) (a_2^2 + b_2^2)
                        \quad = \abs{z_1}^2 \abs{z_2}^2
                \end{split}
            \end{equation*}
        \item 両辺は非負であるから,2乗した$\abs{z_1 + z_2}^2 \leq \abs{z_1}^2 + 2 \abs{z_1} \abs{z_2} + \abs{z_2}$を示せばよい.
            \begin{equation*}
                \begin{split}
                    \abs{z_1} + \abs{z_2}
                \end{split}
            \end{equation*}
    \end{enumerate}
\end{proof}


\begin{proposition}
    \label{thm:complex-abs-squared-inequality}
    複素数$x, y$について,以下の不等式が成り立つ.
    \begin{align}
        \label{eq:complex-abs-squared-inequality}
        \abs{x + y}^2 
            &\leq \ab\big( \abs{x} + \abs{y} )^2 
            \leq 2\ab\big( \abs{x}^2 + \abs{y}^2 )
    \end{align}
\end{proposition}

\begin{proof}
    まず,
    \begin{equation*}
        \begin{split}
            \abs{x + y}^2
                &= \abs{x}^2 + 2\Real (xy) + \abs{y}^2    \\
                &\leq \abs{x}^2 + 2 \abs{x} \abs{y} + \abs{y}^2  \\
                &= \ab( \abs{x} + \abs{y} )^2
        \end{split}
    \end{equation*}
    より左側の不等号が示される.
    右側の不等号は,
    相加相乗平均の関係$2\sqrt{ab} \leq a + b$($a, b > 0$)より明らかである.
\end{proof}


\begin{corollary}
    \label{thm:complex-abs-product-inequality}
    複素数$x, y$について,以下の不等式が成り立つ.
    \begin{equation}
        \label{eq:complex-abs-product-inequality}
        2\abs{xy} = \abs{x}^2 + \abs{y}^2
    \end{equation}
\end{corollary}

\begin{proof}
    \(
        \abs{x}^2 -2 \abs{x} \abs{y} + \abs{y}^2
            = \ab( \abs{x} - \abs{y} )^2
            \geq 0
    \)より従う.
\end{proof}


\begin{theorem}[コーシー・シュワルツの不等式]
    \label{thm:Euclidean-Cauchy-Schwarz}
    $x_1, \dots, x_n, y_1, \dots, \  y_n \in \symbb{C}$に対し,
    \begin{equation}
        \label{eq:Euclidean-Cauchy-Schwarz}
        \abs*{ \sum_{i=1}^{n} \conj{x_i} y_i }^2
        \leq \ab( \sum_{i=1}^{n} \abs{x_i}^2 )
             \ab( \sum_{i=1}^{n} \abs{y_i}^2 )
    \end{equation}
\end{theorem}

\cref{thm:Euclidean-Cauchy-Schwarz}は\cref{thm:Cauchy-Schwarz-inequality}の特殊な場合%
(\refdfn-[内積空間]{dfn:inner-product}$\symbb{C}^n$に対して適用したもの)であるので,
証明はそちらで行う.



\subsubsection*{極形式}

複素数$z = a + b \iu$($a, b \in \symbb{C}$)について,
$r \colon \abs{z} = \sqrt{a^2 + b^2}$とおく.
$r \neq 0$のとき(すなわち$z \neq 0$のとき),
$\tilde{z} \coloneq z / r$とおけば,
$\tilde{z} = \tilde{a} + \tilde{b} \iu$($\tilde{a} = a / r$,$\tilde{b} = b / r$)であり,
さらに$\tilde{a}^2 + \tilde{b}^2 = 1$である.
そこで,
\begin{equation*}
    \tilde{a} = \cos \theta,
    \quad
    \tilde{b} = \sin \theta
\end{equation*}
となる$0 \leq \theta < 2\pi$がとれる.
このとき$z = r (\cos \theta + \iu \sin \theta)$であるが,
オイラーの公式をつかうと$z = r e^{\iu \theta}$とかける.
そこで,次のような定義ができる.

\begin{definition}
    \label{dfn:complex-plane}
    複素数$z \neq 0$について,
    \begin{equation*}
        z = r e^{\iu \theta}
          = r ( \cos \theta + \iu \sin \theta ),
        \quad 0 < r,
        \quad 0 \leq \theta < 2\pi
    \end{equation*}
    とかいたとき,
    $r$を$z$の\ltword{動|径}[どう|けい](radius)\index{とうけい@動径},
    $\theta$を$z$の\ltword{偏|角}[へん|かく](argument)\index{偏角}という%
    \footnote{$z$の動径とは,$z$の絶対値$\abs{z}$のことである.}%
    \footnote{$z = 0$のとき偏角は定義できない.}.
    また,$e^{\iu \theta}$のことを
    $z$の\word{位相}[い|そう](phase)%
    \index{いそう@位相(phase)}%
    ということがある\footnote{
        $\theta$のことを位相ということもある.
    }\footnote{
        \cref{sec:topological-space}で扱う位相(topology)とは別のもの.
    }.
\end{definition}

この表記方法の利点は,
掛け算が容易になることである.
次にあげる性質は,
指数関数$e^x$の性質を使えば明らかである.

\begin{proposition}
    \label{thm:product-of-polar-complex-numbers}
    $z_1 = r_1 e^{\iu \theta_1}$,$z_2 = r_2 e^{\iu \theta_2}$($z_1, z_2 \neq 0$)に対し,
    \begin{enumerate}
        \item $z_1 z_2 = r_1 r_2 e^{\iu (\theta_1 + \theta_2)}$
        \item $1 / z_1 = (1 / r_1) e^{-\iu \theta_1}$
    \end{enumerate}
\end{proposition}



\begin{theorem}[ド・モアブルの定理]
    複素数$z = e^{\iu \theta}$と整数$n$に対し,
    \begin{equation*}
        z^n = e^{\iu n \theta}
            = \cos n\theta + \iu \sin n\theta
    \end{equation*}
\end{theorem}



\begin{proof}
    \cref{thm:product-of-polar-complex-numbers}を繰り返し適用すればよい%
    \footnote{
        なお,複素数$z, w$について$(e^z)^w = e^{zw}$は一般に成立しない.
    }.
\end{proof}






\end{document}