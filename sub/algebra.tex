\documentclass[../sotsu.tex]{subfiles}

\begin{document}

\section{抽象代数学の基礎}

\subsection{群}

\begin{definition}[群]
    \label{dfn:group}
    $G$を集合,$\dotprod \colon G \times G \to G$を二項演算とする.
    $G$の元が,演算$\dotprod$に対して以下を満たすとき,
    $(G, \dotprod)$は\word{群}[ぐん][くん](group)であるという.
    \begin{enumerate}
        \item 任意の$a, b, c \in G$に対し,$(a \dotprod b) \dotprod c = a \dotprod (b \dotprod c)$である.
        \item ある$e \in G$が存在して,任意の$a \in G$に対し,$a \dotprod e = e \dotprod a = a$である.
        \item 任意の$a \in G$に対し,ある$a^{-1} \in G$が存在して,$a \dotprod a^{-1} = a^{-1} \dotprod a = e$である.
    \end{enumerate}
    さらに以下を満たすとき,$(G, \dotprod)$を\word{アーベル群}[][ああへるくん](Abelian group)であるという
    群$(G, \dotprod)$のことを単に$G$とかくこともある.
\end{definition}

\begin{example}
    \begin{equation*}
        \mathrm{SO}(n, ℝ) \coloneq \Set{ A \colon \text{実成分$n$-次正方行列} \given \det A = 1 }
    \end{equation*}
    とする.これは行列としての積に対して群をなす.
\end{example}

\begin{example}
    自然数全体の集合$\symbb{N}$は,通常の和$+$に対して群をなす.
\end{example}

\begin{example}
    実数全体の集合$ℝ$は,通常の積$\times$に対して群をなさない.
    なぜなら,$0 \in ℝ$に対して,$0 \times x = 1$となるような$x \in ℝ$が存在しないため.
\end{example}


\subsection{環}

\begin{definition}[環]
    \label{dfn:ring}
    $R$を集合,$+$と$\dotprod$を$R$上の二項演算とする.
    以下がみたされるとき,$(R, +, \dotprod)$は\word{環}[かん][かん](ring)であるという.
    \begin{itemize}
        \item 和について,$(R, +)$はアーベル群である.すなわち
        \begin{enumerate}
            \item \label{ring:sum-associative} 任意の$a, b, c \in R$に対し,$(a + b) + c = a + (b + c)$である.
            \item \label{ring:sum-zero} ある$0 \in R$が存在して,任意の$a \in R$に対し,$a + 0 = 0 + a = a$である.
            \item \label{ring:sum-opposite} 任意の$a \in R$に対し,ある$-a \in R$が存在して,$a + (-a) = (-a) + a = 0$である.
            \item \label{ring:sum-commutative} 任意の$a, b \in R$に対し,$a + b = b + a$である.
        \end{enumerate}
        \item 積について,
        \begin{enumerate}[resume]
            \item \label{ring:prod-associative} 任意の$a, b, c \in R$に対し,$(a \dotprod b) \dotprod c = a \dotprod (b \dotprod c)$である.
            \item \label{ring:prod-one} ある$1 \in R$が存在して,任意の$a \in R$に対し,$a \dotprod 1 = 1 \dotprod a = a$である.
        \end{enumerate}
        \item 和と積の関係について,
        \begin{enumerate}[resume]
            \item \label{ring:distributive} 任意の$a, b, c \in R$に対し,$(a + b) \dotprod c = a \dotprod c + b \dotprod c$であり,$a \dotprod (b + c) = a \dotprod b + a \dotprod c$である.
        \end{enumerate}
    \end{itemize}
    さらに積について以下がみたされるとき,$(R, +, \dotprod)$は\word{可換環}[][かかんかん](commutative ring)であるという.
    \begin{enumerate}[start=9]
        \item 任意の$a, b \in R$に対し,$a \dotprod b = b \dotprod a$である.
    \end{enumerate}
    二項演算$+$のことを\word{和}あるいは\word{加法},
    $\dotprod$のことを\word{積}あるいは\word{乗法}という.
\end{definition}

\begin{example}
    $n$-次正方行列全体の集合は,行列の和$+$と積$\dotprod$に関して環であるが,$n = 1$の場合をのぞき可換環でない.
\end{example}



\subsection{体}

これから実数・複素数のもつ性質を抽象化した「\ruby{体}{たい}」を定義する.
量子力学において扱う体は基本的に複素数体であるので,わざわざ抽象化を行う動機が見えづらいかもしれない.
しかし,例えば量子情報の分野においては体として有限体を扱うことがあるので,ここで一般の体についての性質を議論しておく.

\begin{definition}[体]
    \label{dfn:field}
    $𝕂$を集合,$+$と$\dotprod$を$𝕂$上の二項演算とする.
    以下がみたされるとき,$(𝕂, +, \dotprod)$は\word{体}[たい][たい](field)であるという.
    \begin{itemize}
        \item 和について,$(𝕂, +)$はアーベル群である.
        \begin{enumerate}
            \item \label{field:sum-associative} 任意の$a, b, c \in 𝕂$に対し,$(a + b) + c = a + (b + c)$である.
            \item \label{field:sum-zero} ある$0 \in 𝕂$が存在して,任意の$a \in 𝕂$に対し,$a + 0 = 0 + a = a$である.
            \item \label{field:sum-opposite} 任意の$a \in 𝕂$に対し,ある$-a \in 𝕂$が存在して,$a + (-a) = (-a) + a = 0$である.
            \item \label{field:sum-commutative} 任意の$a, b \in 𝕂$に対し,$a + b = b + a$である.
        \end{enumerate}
        \item 積について,$(𝕂 \setminus \{ 0 \}, \  \dotprod)$はアーベル群である.
        \begin{enumerate}[resume]
            \item \label{field:prod-associative} 任意の$a, b, c \in 𝕂$に対し,$(a \dotprod b) \dotprod c = a \dotprod (b \dotprod c)$である.
            \item \label{field:prod-one} ある$1 \in 𝕂$が存在して,任意の$a \in 𝕂$に対し,$a \dotprod 1 = 1 \dotprod a = a$である.
            \item \label{field:prod-reciprocal} 任意の$a \in 𝕂 \setminus \{0\}$に対し,ある$a^{-1} \in 𝕂$が存在して,$a \dotprod a^{-1} = a^{-1} \dotprod a = 1$である.
            \item \label{field:prod-commutative} 任意の$a, b \in 𝕂$に対し,$a \dotprod b = b \dotprod a$である.
        \end{enumerate}
        \item 和と積の関係について,
        \begin{enumerate}[resume]
            \item \label{field:distributive} 任意の$a, b, c \in 𝕂$に対し,$(a + b) \dotprod c = (a \dotprod c) + (b \dotprod c)$であり,$a \dotprod (b + c) = (a \dotprod b) + (a \dotprod c)$である.
        \end{enumerate}    
    \end{itemize}
\end{definition}

\begin{example}
    有理数全体の集合$\symbb{Q}$,実数全体の集合$ℝ$,複素数全体の集合$ℂ$は,
    それぞれ通常の和と積に関して体になる.
\end{example}

体をひとことでいえば,「四則演算(和・差・積・商)ができる集合」である.
実数や複素数は体の典型的なものである.
本稿で「体$𝕂$」と書いてある場合,実数$ℝ$もしくは複素数$ℂ$と読み替えて差支えない.



\subsection{複素数}

集合$\symbb{C}$を
\begin{equation}
    \symbb{C} \coloneq \Set{  (a, b)  \given  a, b \in \symbb{R}  }
\end{equation}
で定める.
もちろん$\symbb{C} = \symbb{R}^2$である.
$\symbb{C}$の元である順序対$(a, b) \in \symbb{R}$に対し,和と積を次のように定義する.
\begin{align}
    (a, b) \underset{\symbb{C}}{+} (c, d) &\coloneq (a+c, \  b+d)  \\
    (a, b) \underset{\symbb{C}}{\times} (c, d) &\coloneq (ac-bd, \  ad+bc)
\end{align}
すると,$(\symbb{C}, \underset{\symbb{C}}{+}, \underset{\symbb{C}}{\times})$は\refdfn-[体]{dfn:field}になるので,
これを\word{複素数体}と呼ぶ.
$\symbb{C}$の元を\word{複素数}と呼び,
$(a, b) \eqcolon a + b\iu$とかく.
また,実数のときと同様に積の記号はよく省略する.

\begin{proof}
    $\symbb{C}$の元を$x = (a, b)$,$y = (c, d)$,$z = (e, f)$とする.

    \textsf{和について} 和が可換かつ結合的であることは明らか.
    零元は$0_{\symbb{C}} = (0, 0) = 0 + 0\iu$であり,
    $x$に対する和の逆元は$-x = (-a, -b)$である.

    \textsf{積について} 積が可換であることは明らか.
    単位元は$1_{\symbb{C}} = (1, 0) = 1 + 0\iu$であり,
    $x \neq 0_{\symbb{C}}$に対する積の逆元は$(a / \sqrt{a^2 + b^2}, \  -b / \sqrt{a^2 + b^2} )$である.
    積が結合的であることは
    \begin{equation*}
        \begin{split}
            (xy)z &= (ac-bd, \  bc+ad) \underset{\symbb{C}}{\times} (e, f)   \\
                  &= \cdots  \\
                  &= (a, b) \underset{\symbb{C}}{\times} (ce-df, \  de+cf)   \\
                  &= x(yz)
        \end{split}
    \end{equation*}

    \textsf{分配法則} 
    \begin{equation*}
        \begin{split}
            (x \underset{\symbb{C}}{+} y) \underset{\symbb{C}}{\times} z
            &= (a+c, \  b+d) \underset{\symbb{C}}{\times} (e, f)   \\ 
            &= (ae+ce-bf-df, \  af+cf+be+de)   \\
            &= (ae-bf, \  af+be) \underset{\symbb{C}}{+} (ce-df, \  cf+de)   \\
            &= (x \underset{\symbb{C}}{\times} z) \underset{\symbb{C}}{+}
               (y \underset{\symbb{C}}{\times} z)
        \end{split}
    \end{equation*}
\end{proof}

実数$a \in \symbb{R}$は,複素数$a = (a, 0)$とみなすことで,
複素数に組み込むことができる($\symbb{R} \subset \symbb{C}$).

複素数の\word{絶対値}[ぜっ|たい|ち](absolute value)\index{せつたいち@絶対値}を
\begin{equation}
    \label{eq:complex-number-abs}
    \abs{z} 
        = \abs{a + b\iu}
        \coloneq \sqrt{a^2 + b^2}
\end{equation}
で定める.$0 \leq \abs{z} < \infty$であり,
$\abs{z} = 0$となるのは$a = b = 0$,つまり$z = 0$のときに限る.


\subsubsection*{複素数についての性質}

\begin{proposition}
    複素数$z_1, z_2$の絶対値について,
    以下が成り立つ.
    \begin{enumerate}
        \item $\abs{z_1 z_2} = \abs{z_1} \abs{z_2}$
        \item $\abs{z_1 + z_2} \leq \abs{z_1} + \abs{z_2}$
    \end{enumerate}
\end{proposition}

\begin{proof}
    \begin{enumerate}
        \item $z_1 = a_1 + b_1 \iu$,$z_2 = a_2 + b_2 \iu$とすると,
            \begin{equation*}
                \begin{split}
                    \abs{z_1 z_2}^2 
                        &= \abs{ a_1 a_2 + a_1 b_2 \iu + a_2 b_1 \iu - a_2 b_2 }^2   \\
                        &= (a_1 a_2 - b_1 b_2)^2 + (a_1 b_2 + a_2 b_1)^2   \\
                        &= (a_1^2 + b_1^2) (a_2^2 + b_2^2)
                        \quad = \abs{z_1}^2 \abs{z_2}^2
                \end{split}
            \end{equation*}
        \item 
    \end{enumerate}
\end{proof}


\begin{proposition}
    複素数$x, y$について,以下の不等式が成り立つ.
    \begin{align}
        \label{eq:complex-abs-squared}
        \abs{x + y}^2 
            &\leq \ab( \abs{x} + \abs{y} )^2 
            \leq 2\ab( \abs{x}^2 + \abs{y}^2 )
    \end{align}
\end{proposition}

\begin{proof}
    まず,
    \begin{equation*}
        \begin{split}
            \abs{x + y}^2
                &= \abs{x}^2 + 2\Re (xy) + \abs{y}^2    \\
                &\leq \abs{x}^2 + 2 \abs{x} \abs{y} + \abs{y}^2  \\
                &= \ab( \abs{x} + \abs{y} )^2
        \end{split}
    \end{equation*}
    より左側の不等号が示される.
    右側の不等号は,
    相加相乗平均の関係$2\sqrt{ab} \leq a + b$($a, b > 0$)より明らかである.
\end{proof}




\end{document}