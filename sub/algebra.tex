\documentclass[../sotsu.tex]{subfiles}

\begin{document}

\section{抽象代数学の基礎}

\subsection{群}

\begin{definition}[群]
    \label{dfn:group}
    $G$を集合,$\dotprod \colon G \times G \to G$を二項演算とする.
    $G$の元が,演算$\dotprod$に対して以下を満たすとき,
    $(G, \dotprod)$は\word{群}[ぐん][くん](group)であるという.
    \begin{enumerate}
        \item 任意の$a, b, c \in G$に対し,$(a \dotprod b) \dotprod c = a \dotprod (b \dotprod c)$である.
        \item ある$e \in G$が存在して,任意の$a \in G$に対し,$a \dotprod e = e \dotprod a = a$である.
        \item 任意の$a \in G$に対し,ある$a^{-1} \in G$が存在して,$a \dotprod a^{-1} = a^{-1} \dotprod a = e$である.
    \end{enumerate}
    さらに以下を満たすとき,$(G, \dotprod)$を\word{アーベル群}[][ああへるくん](Abelian group)であるという
    群$(G, \dotprod)$のことを単に$G$とかくこともある.
\end{definition}

\begin{example}
    \begin{equation*}
        \mathrm{SO}(n, ℝ) \coloneq \Set{ A \colon \text{実成分$n$-次正方行列} \given \det A = 1 }
    \end{equation*}
    とする.これは行列としての積に対して群をなす.
\end{example}

\begin{example}
    自然数全体の集合$\symbb{N}$は,通常の和$+$に対して群をなす.
\end{example}

\begin{example}
    実数全体の集合$ℝ$は,通常の積$\times$に対して群をなさない.
    なぜなら,$0 \in ℝ$に対して,$0 \times x = 1$となるような$x \in ℝ$が存在しないため.
\end{example}


\subsection{環}

\begin{definition}[環]
    \label{dfn:ring}
    $R$を集合,$+$と$\dotprod$を$R$上の二項演算とする.
    以下がみたされるとき,$(R, +, \dotprod)$は\word{環}[かん][かん](ring)であるという.
    \begin{itemize}
        \item 和について,$(R, +)$はアーベル群である.すなわち
        \begin{enumerate}
            \item \label{ring:sum-associative} 任意の$a, b, c \in R$に対し,$(a + b) + c = a + (b + c)$である.
            \item \label{ring:sum-zero} ある$0 \in R$が存在して,任意の$a \in R$に対し,$a + 0 = 0 + a = a$である.
            \item \label{ring:sum-opposite} 任意の$a \in R$に対し,ある$-a \in R$が存在して,$a + (-a) = (-a) + a = 0$である.
            \item \label{ring:sum-commutative} 任意の$a, b \in R$に対し,$a + b = b + a$である.
        \end{enumerate}
        \item 積について,
        \begin{enumerate}[resume]
            \item \label{ring:prod-associative} 任意の$a, b, c \in R$に対し,$(a \dotprod b) \dotprod c = a \dotprod (b \dotprod c)$である.
            \item \label{ring:prod-one} ある$1 \in R$が存在して,任意の$a \in R$に対し,$a \dotprod 1 = 1 \dotprod a = a$である.
        \end{enumerate}
        \item 和と積の関係について,
        \begin{enumerate}[resume]
            \item \label{ring:distributive} 任意の$a, b, c \in R$に対し,$(a + b) \dotprod c = a \dotprod c + b \dotprod c$であり,$a \dotprod (b + c) = a \dotprod b + a \dotprod c$である.
        \end{enumerate}
    \end{itemize}
    さらに積について以下がみたされるとき,$(R, +, \dotprod)$は\word{可換環}[][かかんかん](commutative ring)であるという.
    \begin{enumerate}[start=9]
        \item 任意の$a, b \in R$に対し,$a \dotprod b = b \dotprod a$である.
    \end{enumerate}
    二項演算$+$のことを\word{和}あるいは\word{加法},
    $\dotprod$のことを\word{積}あるいは\word{乗法}という.
\end{definition}

\begin{example}
    $n$-次正方行列全体の集合は,行列の和$+$と積$\dotprod$に関して環であるが,$n = 1$の場合をのぞき可換環でない.
\end{example}



\subsection{体}

これから実数・複素数のもつ性質を抽象化した「\ruby{体}{たい}」を定義する.
量子力学において扱う体は基本的に複素数体であるので,わざわざ抽象化を行う動機が見えづらいかもしれない.
しかし,例えば量子情報の分野においては体として有限体を扱うことがあるので,ここで一般の体についての性質を議論しておく.

\begin{definition}[体]
    \label{dfn:field}
    $𝕂$を集合,$+$と$\dotprod$を$𝕂$上の二項演算とする.
    以下がみたされるとき,$(𝕂, +, \dotprod)$は\word{体}[たい][たい](field)であるという.
    \begin{itemize}
        \item 和について,$(𝕂, +)$はアーベル群である.
        \begin{enumerate}
            \item \label{field:sum-associative} 任意の$a, b, c \in 𝕂$に対し,$(a + b) + c = a + (b + c)$である.
            \item \label{field:sum-zero} ある$0 \in 𝕂$が存在して,任意の$a \in 𝕂$に対し,$a + 0 = 0 + a = a$である.
            \item \label{field:sum-opposite} 任意の$a \in 𝕂$に対し,ある$-a \in 𝕂$が存在して,$a + (-a) = (-a) + a = 0$である.
            \item \label{field:sum-commutative} 任意の$a, b \in 𝕂$に対し,$a + b = b + a$である.
        \end{enumerate}
        \item 積について,$(𝕂 \setminus \{ 0 \}, \  \dotprod)$はアーベル群である.
        \begin{enumerate}[resume]
            \item \label{field:prod-associative} 任意の$a, b, c \in 𝕂$に対し,$(a \dotprod b) \dotprod c = a \dotprod (b \dotprod c)$である.
            \item \label{field:prod-one} ある$1 \in 𝕂$が存在して,任意の$a \in 𝕂$に対し,$a \dotprod 1 = 1 \dotprod a = a$である.
            \item \label{field:prod-reciprocal} 任意の$a \in 𝕂 \setminus \{0\}$に対し,ある$a^{-1} \in 𝕂$が存在して,$a \dotprod a^{-1} = a^{-1} \dotprod a = 1$である.
            \item \label{field:prod-commutative} 任意の$a, b \in 𝕂$に対し,$a \dotprod b = b \dotprod a$である.
        \end{enumerate}
        \item 和と積の関係について,
        \begin{enumerate}[resume]
            \item \label{field:distributive} 任意の$a, b, c \in 𝕂$に対し,$(a + b) \dotprod c = (a \dotprod c) + (b \dotprod c)$であり,$a \dotprod (b + c) = (a \dotprod b) + (a \dotprod c)$である.
        \end{enumerate}    
    \end{itemize}
\end{definition}

\begin{example}
    有理数全体の集合$\symbb{Q}$,実数全体の集合$ℝ$,複素数全体の集合$ℂ$は,
    それぞれ通常の和と積に関して体になる.
\end{example}

体をひとことでいえば,「四則演算(和・差・積・商)ができる集合」である.
実数や複素数は体の典型的なものである.
本稿で「体$𝕂$」と書いてある場合,実数$ℝ$もしくは複素数$ℂ$と読み替えて差支えない.


\end{document}