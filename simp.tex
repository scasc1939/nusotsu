\documentclass[
    % draft,
]{sotsu}

\usepackage{ifdraft}

\usepackage{sotsu}
\usepackage{sotsu-symb}
\usepackage{sotsu-thm}

\crefname{equation}{式}{式}

\usepackage[
    style=sotsu,
    maxnames=99,
    minnames=99,
    ]{biblatex}
\addbibresource{sotsu.bib}


\ModifyHeading{subsubsection}{
    label_format={\thesubsubsection},
    lines=2,
    font={\large\sffamily\gtfamily\bfseries},
    format={%
        \jlreqHeadingLabel{{\rmfamily\mcfamily\bfseries \thesubsubsection} \quad}%
        #2%
    },
}



\definecolor{fire}{RGB}{255, 90, 0}
\definecolor{water}{RGB}{0, 120, 255}

\newcommand{\fire}[1]{\textcolor{fire}{#1}}
\newcommand{\water}[1]{\textcolor{water}{#1}}


\title{なんか}
\date{\today}


\ExplSyntaxOn
\makeatletter
\renewcommand{\maketitle}{
    \begin{titlepage}
        \null\vfil
        \centering
        {\LARGE \@title}
        \par
        {\large \@date}
        \vfil\null
        \par
        \ifdraft{}{
        \begin{minipage}{0.7\linewidth}
            \begin{minipage}{0.2\linewidth}
                \centering
                \includegraphics[width=2cm]{misc/scasc_nusotsu.pdf}
            \end{minipage}
            \hfil
            \begin{minipage}{0.7\linewidth}
                PDFファイルと\LaTeX ソースは,
                \url{https://github.com/scasc1939/nusotsu}で配布しています.
            \end{minipage}
            \vskip 1cm
            \begin{minipage}{0.2\linewidth}
                \centering
                \includegraphics[width=2cm]{misc/cc-by-sa.pdf}
            \end{minipage}
            \hfil
            \begin{minipage}{0.7\linewidth}
                This~work~is~licensed~under~CC~BY-SA~4.0.
                \\
                \url{https://creativecommons.org/licenses/by-sa/4.0/}
            \end{minipage}
        \end{minipage}
        }
    \end{titlepage}
}
\makeatother
\ExplSyntaxOff


\begin{document}

\maketitle


\section{イントロダクション}

17世紀から18世紀にかけての物理学者,ニュートンが大成した古典力学(ニュートン力学)によって,
物体の運動は完全に記述できるようになった.
この世界は原子という物体によって構成されているのだから\footnote{
    実は,原子の存在が証明されたのは1905年.
    アインシュタインによるブラウン運動の解明による.
},
現時点での宇宙の全原子の座標と運動量(および相互作用)を完全に決定できる存在,
いわゆる\emph{ラプラスの悪魔}は,
この世界の過去も未来もすべて知ることができる.
のちに形成された解析力学のことばを使えば,
各原子のハミルトニアンと初期状態が,その原子の時間発展を決定する%
%(また正準方程式は時間反転に対して対称である),
といえる.

もちろん実際には,
宇宙の全原子はおろか,
小さな箱に閉じ込められた気体を構成する全原子の座標と運動量を特定することすら,
人間には不可能である.
このような多数------おおよそ$10^{23}$個のオーダー-----の原子がつくる\emph{多体系}は,
各粒子を支配しているはずのニュートン力学からは想像もつかないようなふるまいを示す.
ニュートン方程式は可逆であるにもかかわらず,
多体系は本質的に不可逆的である.
実際,熱湯と水を混ぜればぬるま湯になるが,
逆にぬるま湯が熱湯と水に分離されるところを見た人はいないだろう.
このような「\emph{マクロ}な世界」を扱うのが\emph{統計力学}である.
ニュートン力学には存在しない「温度」や「エントロピー」といった概念が,
統計力学において定義されることは,
よく知られている.

マクロな理論である統計力学の形成から少し遅れた20世紀初頭,
\emph{ミクロ}な世界の理論である\emph{量子力学}が形成された.
量子力学によればエネルギーは離散的な値しかとることができないし,
物理量は確率的にしか予言できず,
さらには状態を重ね合わせることができる.
また,そこで扱われる量子力学的粒子も,
古典粒子に存在しない「スピン」や「位相」といった値を持つ.
このような点で,量子力学は古典力学と根本的に異なる理論体系である.
しかしながら,面白いことに,視野を広げる(ミクロな世界から遠ざかる)ことで,
量子力学は古典力学に漸近するのである.
水素原子のスペクトル問題は,
量子力学によって見事に解き明かされた.


1905年にはアインシュタインが特殊相対論を提唱し,
絶対時間の存在を暗黙裡に仮定していたニュートン力学はここに崩壊した.
量子力学を相対論的に書き直す試みがなされるまで,
それほど時間はかからなかった.
1928年,ディラックは,量子力学に相対論的効果を導入したディラック方程式を提唱した.
この方程式からは,
すべての粒子に対して,
質量が同じで,電荷の符号が逆である粒子(\emph{反粒子})が存在することが示唆される.
反粒子が実在することは,
1932年のアンダーソンによる陽電子の発見によって裏付けられた.


量子力学を使って統計力学を捉えなおすこともなされた.
すなわち,多体系を構成する粒子が量子力学的であるとするのである.
一切の相互作用なくしておこる相転移,
ボース・アインシュタイン凝縮はまさに量子統計力学の醍醐味である.


1911年,カメルリング・オネス\footnote{
    Kamerlingh Onnesのカタカナ表記は,
    圧力$p$・$P$問題,
    状態密度$D$・$N$問題,
    ボーズ・ボース問題,
    Liouville問題と並ぶ統計力学の難問である.
}は極低温に冷やした水銀の電気抵抗がゼロになることを発見した.
のちに超伝導と呼ばれるこの現象の理論的解明は,
1957年にバーディーン,クーパー,シュリーファーによってなされた.
3人の頭文字をとってBCS理論と呼ばれるこの理論は,
量子力学を多体系に適用したものである.


このように,量子力学を応用することによって多くの新奇的な物理現象が明らかになったわけであるが,
しかしその中で量子力学の数学的位置づけは等閑視されてきた.
これを読んでいる人でヒルベルト空間を知らない人はいないだろうが,
ではヒルベルト空間の定義を述べよ,といわれると動揺する人が多いのではないだろうか.
むろん基礎から量子力学を理解するよりも,
実際に量子力学を使って物理現象を理解することが重要であるという考え方も大いにあり得よう.
量子力学の厳密な数学的定式化はフォン・ノイマン(1932年)によってなされた.







\section{ヒルベルト空間の定義}

量子状態はヒルベルト空間の元であるベクトルによって記述される.
そこで,この節ではまず,一般のヒルベルト空間について定義する.
スピン系のような有限次元のヒルベルト空間のベクトルとは,
単なる数ベクトルであることを示す.


\subsection{ベクトル空間}

量子力学は線形代数である,
量子力学的状態はベクトルであるというのはよく言われることである.
このことを理解するために,
まずはベクトルとは何かをきちんと定義することにしよう.


\subsubsection{ベクトル空間の定義}

まずは\emph{ベクトル空間}を定義する.
$\symbb{C}$を複素数全体の集合とする.
集合$V$があって,
$V$の\ltjruby{元}{げん}に対して\emph{和}$\vecplus \colon V \times V \to V$,
\emph{スカラー倍}$\scaprod \colon \symbb{C} \times V \to V$が定義されているとする.
$V$の元を$\psi, \varphi, \xi$,
複素数$a, b$としたとき,
以下の8条件
\begin{enumerate}
    \item \label{vector:sum-associative}
        \bluehead{和は結合的である}------すなわち,
        \fire{任意の$\psi, \varphi, \xi \in V$}に対し,
        $(\psi \vecplus \varphi) \vecplus \xi = \psi \vecplus (\varphi \vecplus \xi)$.
    \item \label{vector:sum-commutative}
        \bluehead{和は交換する}------すなわち,
        \fire{任意の$\psi, \varphi, \xi \in V$}に対し,
        $\psi \vecplus \varphi = \varphi \vecplus \psi$.
    \item \label{vector:sum-zero}
        \bluehead{和に対する零元の存在}------すなわち,
        \water{ある$\symbf{0} \in V$}が存在して,
        \fire{任意の$\psi \in V$}に対し,$\psi \vecplus \symbf{0} = \psi$.
    \item \label{vector:sum-opposite}
        \bluehead{和に対する逆元の存在}------すなわち,
        \fire{任意の$\psi \in V$}に対し,
        \water{ある$\varphi \in V$}が存在して,
        $\psi \vecplus \varphi = \symbf{0}$.
    \item \label{vector:scalar-sum}
        \bluehead{スカラー倍は複素数の和に対して分配的である}------すなわち,
        \fire{任意の$\psi \in V$}と\fire{任意の複素数$a, b$}に対し,
        $(a + b) \scaprod \psi = (a \scaprod \psi) + (b \scaprod \psi)$.
    \item \label{vector:scalar-distributive}
        \bluehead{スカラー倍は$V$上の和に対して分配的である}------すなわち,
        \fire{任意の$\psi, \varphi \in V$}と\fire{任意の複素数$a$}に対し,
        $a \scaprod (\psi \vecplus \varphi) = (a \scaprod \psi) \vecplus (a \scaprod \varphi)$.
    \item \label{vector:scalar-prod}
        \bluehead{複素数の積とスカラー倍の結合}------すなわち,
        \fire{任意の$\psi \in V$}と\fire{任意の複素数$a, b$}に対し,
        $(ab) \scaprod \psi = a \scaprod (b \scaprod \psi)$.
    \item \label{vector:scalar-identity}
        \bluehead{$1$はスカラー倍の単位元}------すなわち,
        複素数$1$について,\fire{任意の$\psi \in V$}に対し,
        $1 \scaprod \psi = \psi$.
\end{enumerate}
を満たすのならば,
$V$を\ltword{ベクトル空間}(vector space)と呼び,
$V$の元(上で挙げた$\psi, \varphi, \xi, \dotsc$)を\ltword{ベクトル}という.

ここで,上の条件に出てくる「\fire{任意の}---」と「\water{ある}---」の順序は重要である.
\cref{vector:sum-zero}では$\symbf{0}$はただひとつであり,
その$\symbf{0}$について$\psi \vecplus \symbf{0} = \psi$が満たされ,
さらに$\varphi \vecplus \symbf{0} = \varphi$が成り立たねばならない.
その一方で,\cref{vector:sum-opposite}における$\varphi$は,
$\psi$のとりかたによって変わりうる.
あるベクトル$\psi$に対して$\psi \vecplus \varphi = \symbf{0}$が満たされても,
別のベクトル$\xi$に対して$\xi \vecplus \varphi = \symbf{0}$が成り立つとは限らない.



\subsubsection{ベクトル空間の例}

\bluehead{ユークリッド空間}\quad
ベクトル空間の典型例は\word{ユークリッド空間}(Euclidean space)である.
ある自然数$N$($\geq 1$)に対し,
集合$\symbb{C}^N$を
\begin{equation*}
    \symbb{C}^N 
    \coloneq
    \Set*{
    \symbf{x} = 
    \begin{pmatrix}
        x_1  \\  \vdots  \\  x_N
    \end{pmatrix}
    \given 
    \text{$x_1, \dots, x_N$は複素数}
    }
\end{equation*}
で定める.
和$\vecplus$とスカラー倍$\scaprod$は,
各成分の和と積,つまり
\begin{align*}
    \symbf{x} \vecplus \symbf{y}
        &\coloneq 
        \begin{pmatrix}
            x_1 + y_1  \\
            \vdots     \\
            x_N + y_N
        \end{pmatrix},
    \quad
    c \scaprod \symbf{x}
        \coloneq 
        \begin{pmatrix}
            c x_1  \\
            \vdots \\
            c x_N
        \end{pmatrix}
\end{align*}
で定める.
このとき,$\symbb{C}^N$がベクトル空間の定義を満たすことを確認するのは容易である.
まず,
\cref{vector:sum-associative,vector:sum-commutative}は,
複素数の性質から明らかである.
\cref{vector:sum-zero}における$\symbf{0}$は,
\begin{equation*}
    \symbf{0} \coloneq 
    \begin{pmatrix}
        0  \\  \vdots  \\  0
    \end{pmatrix}
\end{equation*}
で与えられる.
また,\cref{vector:sum-opposite}でいう,$\symbf{x}$に対して$\symbf{x} \vecplus \symbf{y} = \symbf{0}$を満たすベクトル
$\symbf{y}$は,
\begin{equation*}
    \symbf{y} \coloneq 
    -\symbf{x} =
    \begin{pmatrix}
        -x_1  \\  \vdots  \\  -x_N
    \end{pmatrix}
\end{equation*}
である.
\cref{vector:scalar-sum,vector:scalar-distributive}は複素数の分配法則に帰着されるし,
\cref{vector:scalar-prod}は複素数の結合法則そのものである.
\cref{vector:scalar-identity}は複素数$1$が乗法の単位元であることから直ちに従う.

以上のことから$\symbb{C}^N$がベクトル空間であることが示されたが,
実は\pageref{vector:sum-associative}ページで与えたベクトル空間の定義は,
この$\symbb{C}^N$が満たす性質を抽出して羅列したものである.
物理学の教科書で「具体例から基本法則を導く」ことが多いのとは対照的に,
数学ではこのような「一般から具体」という論理展開が好まれる\footnote{
    上のように,具体例を与えずに初めに定義を書くことは,
    物理では「\ltjruby{天|下}{あま|くだ}り的」とされて忌避される傾向にある.
}.
しかし,論理の形式としては一般から具体であっても,
たいていは抽象的な定義のもととなった典型例が存在する.
今回の場合,抽象的に定義された「ベクトル空間」の典型例がユークリッド空間である.


\bluehead{関数空間}\quad
もう少し特殊なベクトル空間として,
関数のなす空間を考えよう.
実数係数の\ltword{多項式空間}$\symbb{R}[x]$を
\begin{equation*}
    \symbb{R}[x] \coloneq 
    \Set*{ \sum_{i=1}^{n} c_i x^i
        \given
        \begin{array}{c}
            \text{$c_1, \dots, c_n$は実数}  \\
            \text{$n = 0, 1, 2, \dots$}
        \end{array}
        }
\end{equation*}
で定義する.
$\symbb{R}[x]$がベクトル空間になっていることもすぐにわかる.
例えば多項式の和はまた多項式であるし,
多項式のスカラー倍(つまり,全体に数$c$をかけたもの)も多項式である\footnote{
    本来は「多項式の和$\vecplus$とスカラー倍$\scaprod$を定義しないといけないが,
    煩雑さを避けるため割愛する.
}.
\pageref{vector:sum-associative}ページの8条件を満たすことは,
これらの演算が\cref{vector:sum-zero}の$\symbf{0}$が多項式としての$0$であることに注意すれば,
ほとんど自明である.
$\symbb{R}[x]$に属するベクトルの例は,
$2x$, $3x^3 - \sqrt{2} x + 1$などである.

同様にして,連続関数全体$C^0$や,
$n$回微分可能な関数全体$C^n$もベクトル空間であることが示される.
このような事実を踏まえて,
量子力学的状態$\psi(x)$がベクトルであるといっているのである.



\subsection{基底と次元}

\subsubsection{有限個の基底}

ベクトル空間の和とスカラー倍について,
進んだ考察をしてみよう.

考えるベクトル空間を$V$とする.
ベクトル空間の公理(より正確には,和とスカラー倍の定義)から,
$V$のベクトル$\symbf{u}_1, \dots, \symbf{u}_n$と複素数$c_1, \dots, c_n$に対して,
\begin{equation}
    \label{eq:linear-combination}
    \symbf{v} \coloneq c_1 \symbf{u}_1 + \dots + c_n \symbf{u}_n
\end{equation}
で定義される$\symbf{v}$はまた$V$のベクトルになる.
そこで\cref{eq:linear-combination}のことを,
ベクトル$\symbf{u}_1, \dots, \symbf{u}_n$の\ltword{線形結合}(linear combination)という.
線形結合はベクトル空間論の基礎をなす重要概念であり,
ある意味ベクトルの本質ともいえる.
ベクトル空間における概念の多くが,
この線形結合を用いて定義される.
以下でその例を見てみよう.

ベクトルの組$\symbf{u}_1, \dots, \symbf{u}_n \in V$が,
複素数$c_1, \dots, c_n$に対して
\begin{equation}
    \label{eq:linearly-independent}
    \symbf{0} = c_1 \symbf{u}_1 + \dots + c_n \symbf{u}_n
\end{equation}
を満たすとする.
\cref{eq:linearly-independent}のことを$\symbf{u}_1, \dots, \symbf{u}_n$の\ltword{一次関係}(linear relation)という.
もし$c_1 = \dots = c_n = 0$ならば,
\cref{eq:linearly-independent}は当然成り立つ.
逆に,\cref{eq:linearly-independent}が成り立つのが$c_1 = \dots = c_n = 0$の場合に限るとき,
$\symbf{u}_1, \dots, \symbf{u}_n$は\ltword{一次独立}(linearly independent)であるという.

また,$V$のベクトル$\symbf{u}_1, \dots, \symbf{u}_n$の線形結合の全体が$V$に一致する,
つまり
\begin{equation*}
    V = \Set{
            c_1 \symbf{u}_1 + \dots + c_n \symbf{u}_n
            \given
            \text{$c_1, \dots, c_n$は複素数}
            }
\end{equation*}
であるとき,$\symbf{u}_1, \dots, \symbf{u}_n$は$V$を\ltword{生成する}という.
あるいは$V$のすべてのベクトルが\cref{eq:linear-combination}の形で書けるといってもよい.

ここまでの準備によって,
基底という概念を定義できる.
$\symbf{u}_1, \dots, \symbf{u}_n$を$V$のベクトルとする.
以下の2条件
\begin{enumerate}
    \item $\symbf{u}_1, \dots, \symbf{u}_n$は一次独立である.
    \item $\symbf{u}_1, \dots, \symbf{u}_n$は$V$を生成する.
\end{enumerate}
が満たされるとき,
$\symbf{u}_1, \dots, \symbf{u}_n$は$V$の\ltword{基|底}[き|てい](basis)であるという.

例えば,$V$をユークリッド空間$\symbb{C}^N$としよう.
$V$の基底は,例えば
\begin{equation}
    \label{eq:Euclidean-basis-1}
    \symbf{u}_1 \coloneq
    \begin{pmatrix}
        1  \\  0  \\  0  \\  \vdots  \\  0
    \end{pmatrix}
    , \quad
    \symbf{u}_2 \coloneq
    \begin{pmatrix}
        0  \\  1  \\  0  \\  \vdots  \\  0
    \end{pmatrix}
    , \quad \dotsc, \quad
    \symbf{u}_N \coloneq
    \begin{pmatrix}
        0  \\  0  \\  \vdots  \\  0  \\  1
    \end{pmatrix}
\end{equation}
が考えられる.
$\symbf{u}_1, \dots, \symbf{u}_N$が$V$を生成することは,
$\symbb{C}^N$の任意のベクトル$\symbf{v}$が
\begin{equation*}
    \symbf{v}
    \equiv
    \begin{pmatrix}
        c_1  \\  c_2  \\  \vdots  \\  c_N
    \end{pmatrix}
    =
    c_1 \symbf{u}_1 + c_2 \symbf{u}_2 + \dots + c_N \symbf{u}_N
\end{equation*}
(ただし$c_1, c_2, \dots, c_N$は任意の複素数)とあらわされることからわかる.
また,一次独立性は,
\begin{equation*}
    \symbf{0} 
        \equiv
        \begin{pmatrix}
            0  \\  0  \\  \vdots  \\  0
        \end{pmatrix}
        = c_1 \symbf{u}_1 + c_2 \symbf{u}_2 + \dots + c_N \symbf{u}_N
        \equiv
        \begin{pmatrix}
            c_1  \\  c_2  \\  \vdots  \\  c_N
        \end{pmatrix}
\end{equation*}
より$c_1 = c_2 = \dots = c_N = 0$と確認される.

ところが,\cref{eq:Euclidean-basis-1}は$\symbb{C}^N$の\emph{唯一の}基底ではない.
これを確かめるには,
実際に別の基底を構成すればよい.
たとえば,
\begin{equation}
    \tag{\ref*{eq:Euclidean-basis-1}$'$}
    \label{eq:Euclidean-basis-2}
    \symbf{u}'_1 \coloneq
    \begin{pmatrix}
        1  \\  -1  \\  0  \\  \vdots  \\  0  \\  0
    \end{pmatrix}
    , \quad
    \symbf{u}'_2 \coloneq
    \begin{pmatrix}
        0  \\  1  \\  -1  \\  0  \\  \vdots  \\  0
    \end{pmatrix}
    , \quad \dotsc, \quad
    \symbf{u}'_{N-1} \coloneq
    \begin{pmatrix}
        0  \\  0  \\  \vdots  \\  0  \\  1  \\  -1
    \end{pmatrix}
    , \quad
    \symbf{u}'_N \coloneq
    \begin{pmatrix}
        0  \\  0  \\  \vdots  \\  0  \\  0  \\  1
    \end{pmatrix}
\end{equation}
で$\symbf{u}'_1, \dots, \symbf{u}'_N$を定めても,
これは基底である.
一次独立であることは,
$c_1 \symbf{u}'_1 + \dots + c_N \symbf{u}'_N = \symbf{0}$とおくことで,
\begin{equation*}
    c_1 = 0, 
    \quad 
    c_2 - c_1 = 0, 
    \quad
    \dotsb,
    \quad 
    c_{N-1} - c_N = 0,
    \quad
    c_N = 0
\end{equation*}
という連立不等式が得られ,
ここから$c_1 = c_2 = \dots = c_N = 0$が従うことよりわかる.
また$V$を生成することは,
任意のベクトル$\symbf{v} = (a_1 \; a_2 \; \cdots \; a_N) \in V$が
\begin{equation*}
    \symbf{v} \equiv 
    \begin{pmatrix}
        a_1  \\  a_2  \\  \vdots  \\  a_N
    \end{pmatrix}
    =
      a_1 \symbf{u}'_1
    + (a_2 + a_1) \symbf{u}'_2
    + \dotsb 
    + (a_{N-1} + a_{N-2}) \symbf{u}'_{N-1}
    + (a_{N-1} + a_N) \symbf{u}'_{N-1}
\end{equation*}
と表せることから従う.


\subsubsection{無限個の基底}

今までの議論では,
基底が有限個の場合についてのみ考えていた.
ここでは基底が無限個ある場合を考える.
そのためには,
線形結合の定義を拡張する必要がある.

以降,表記を簡単にするために,
可算無限(自然数$\symbb{N}$と同じくらいの数)の場合についてだけ考える.
物理学で扱うヒルベルト空間は,
無限次元であっても可分である(すなわち稠密な高々可算集合をもつ)ので,
この仮定を行っても特に支障はない.

考えるベクトル空間を$V$とする.
$V$のベクトル$\symbf{u}_1, \symbf{u}_2, \dots$に対して,
\begin{equation}
    \label{eq:infinite-linear-combination}
    \symbf{v} \coloneq
    \sum_{i = 1}^{\infty} c_i \symbf{u}_i,
    \quad
    \text{ただし$c_1, c_2, \dots$は有限個を除きゼロである複素数}
\end{equation}
で定義される$V$のベクトル$\symbf{v}$を,
$\symbf{u}_1, \symbf{u}_2, \dots$の\ltword{線形結合}という.

ここで,無限個の複素数$c_1, c_2, \dots$が有限個を除きゼロである,つまり\cref{infinite-linear-combination}が実は
\begin{equation*}
    \symbf{v} = c_{i_1} \symbf{u}_{i_1} + \dots + c_{i_n} \symbf{u}_{i_n}
\end{equation*}
という有限の和でかけるという条件は必要不可欠である.
実際,和が有限でない場合,
すべての$i$に対して$\symbf{u}_i \in V$でも,
$\sum_{i=1}^{\infty} c_i \symbf{u}_i \notin V$となることがあり得る.
例えば関数$1, x, x^2, \dots, x^n, \dots$はすべて多項式関数であるが,
それらの線形結合
\begin{equation*}
    1 + x + \frac{1}{2!} x^2 + \dots + \frac{1}{n!} x^n + \dotsb = e^x
\end{equation*}
は多項式関数でない.

ベクトル空間$V$に属するベクトル$\symbf{u}_1, \symbf{u}_2, \dots$が\ltword{一次独立}であるとは,
\fire{有限個を除きゼロである}複素数の組$c_1, c_2, \dots$を用いて
\begin{equation*}
    \symbf{0} = \sum_{i = 1}^{\infty} c_i \symbf{u}_i
\end{equation*}
と書けたなら,$c_1 = c_2 = \dots = 0$が成り立つことをいう.
また,
$\symbf{u}_1, \symbf{u}_2, \dots$が$V$を\ltword{生成する}とは,
すべての$V$のベクトル$\symbf{v}$がそれらの線形結合で書ける,
つまり\fire{有限個を除きゼロである}複素数の組$c_1, c_2, \dots$を用いて
\begin{equation*}
    \symbf{v} = \sum_{i = 1}^{\infty} c_i \symbf{u}_i
\end{equation*}
が成り立つことをいう.

以上のように再定義された概念を用いて,
無限個のベクトルからなる基底を定義できる.
つまり,$V$のベクトル$\symbf{u}_1, \symbf{u}_2, \dots$が$V$の\ltword{基底}であるとは,
これらが一次独立であり,
かつ$V$を生成することをいう.





\subsubsection{ベクトル空間の次元}

ベクトル空間が与えられても,
その基底の取り方には任意性がある.
たとえば\cref{eq:Euclidean-basis-1}と\cref{eq:Euclidean-basis-2}は,
ともにユークリッド空間$\symbb{C}^N$の基底をなす.
しかし,基底を構成するベクトルの数は,
どのような基底をとっても常に一定である.
実際,\cref{eq:Euclidean-basis-1},\cref{eq:Euclidean-basis-2}は,
ともに$N$個のベクトルからなる.
これを証明しよう.

あるベクトル空間$V$に対して,
基底$\sequ{\symbf{u}_i} \coloneq \symbf{u}_1, \dots, \symbf{u}_M$と,
基底$\sequ{\symbf{u}'_j} \coloneq \symbf{u}'_1, \dots, \symbf{u}'_N$がとれたとする.
まず$\sequ{\symbf{u}_i}$は$V$の基底であるから,
任意の$\symbf{u}'_j$は$\sequ{\symbf{u}_i}$の線形結合で書ける:
\begin{equation*}
    \symbf{u}'_j = \sum_{j = 1}^{M} c_{ij} \symbf{u}_i
    \quad ( 1 \leq i \leq N )
\end{equation*}
ここでもし$M < N$なら,
$\sequ{\symbf{u}'_j}$は一次従属になる.
なぜなら,複素数$a_1, \dots, a_N$を用いた$\sequ{\symbf{u}'_j}$の一次関係は
\begin{equation*}
    \begin{split}
        \symbf{0} &= \sum_{j = 1}^{N} a_j \symbf{u}'_j
            = \sum_{i = 1}^{M} \sum_{j = 1}^{N} a_j c_{ij} \symbf{u}_i
            \\
            &= 
            \begin{pmatrix}
                \symbf{u}_1  &  \cdots  &  \symbf{u}_M
            \end{pmatrix}
            \begin{pmatrix}
                c_{11}  &  \cdots  &  c_{1N}  \\
                \vdots  &  \ddots  &  \vdots  \\
                c_{M1}  &  \cdots  &  c_{MN}
            \end{pmatrix}
            \begin{pmatrix}
                a_1  \\  \vdots  \\  a_N
            \end{pmatrix}
    \end{split}
\end{equation*}
と書けるが,
行列の方程式
\begin{equation*}
    \begin{pmatrix}
        c_{11}  &  \cdots  &  c_{1N}  \\
        \vdots  &  \ddots  &  \vdots  \\
        c_{M1}  &  \cdots  &  c_{MN}
    \end{pmatrix}
    \symbf{a}
    =
    \symbf{0}
\end{equation*}
は$M < N$($\text{行数} < \text{列数}$)のとき必ず非自明な解$\symbf{a} \neq \symbf{0}$を持つからである.
仮定より$\sequ{\symbf{u}'_j}$は一次独立であるから,
$M \geq N$である.
同様の議論は$\sequ{\symbf{u}_i}$と$\sequ{\symbf{u}'_j}$を入れ替えても成立するので,
$N \geq M$である.
よって$M = N$であるといえる.

ベクトル空間$V$について,
その基底を構成するベクトルの数を,
$V$の\ltword{次|元}[じ|げん](dimension)という.
\cref{eq:Euclidean-basis-1}よりユークリッド空間$\symbb{C}^N$の次元は$N$である.





\subsection{内積空間}

次に,ベクトル空間において定義される\ltjruby{内|積}{ない|せき}という概念を導入する.
内積を定義することは,ベクトル空間に幾何学的な性質を入れることに対応する.


\subsubsection{内積}

$\iparen{\bigdot, \bigdot} \colon V \times V \to \symbb{C}$は,
2つのベクトルを複素数に対応させる写像であって,
\begin{enumerate}
    \item \label{innerp:linear} 
        \bluehead{右線形性}------すなわち,
        \fire{任意のベクトル$\psi, \varphi, \xi$}および\fire{任意の複素数$a, b$}に対し,
        $\iparen{\psi, \  a \varphi \vecplus b \xi} = a \iparen{\psi, \varphi} + b \iparen{\psi, \xi}$.
    \item \label{innerp:conjugate-symmetry} 
        \bluehead{共役対称性}------すなわち,
        \fire{任意のベクトル$\psi, \varphi$}に対し,
        $\iparen{\psi, \varphi} = \conj{\iparen{\varphi, \psi}}$.
        ここで,$\conj{\bigdot}$は$\bigdot$の共役複素数を表す.
    \item \label{innerp:positive-definiteness}
        \bluehead{正定値性}------すなわち,
        \fire{任意のベクトル$\psi$}に対し,
        $\iparen{\psi, \psi} \geq 0$であり,
        しかも$\iparen{\psi, \psi} = 0$ならば$\psi = \symbf{0}$.
\end{enumerate}
を満たすものを,$V$上の\ltword{内|積}[ない|せき](inner product)と呼ぶ.

ここで,\cref{innerp:linear,innerp:conjugate-symmetry}を合わせると,
左側のベクトルについては以下の\emph{反線形性}
\begin{equation}
    \label{eq:anti-linear}
    \iparen{a \psi \vecplus b \varphi, \  \xi}
        = \conj{a} \iparen{\psi, \xi} + \conj{b} \iparen{\varphi, \xi}
\end{equation}
が成り立つことに注意しなければならない.
なお,数学ではふつう\cref{innerp:linear}のかわりに\emph{左線形性}%
------つまり\(
    \iparen{a \psi \vecplus b \varphi, \  \xi}
    = a \iparen{\psi, \xi} + b \iparen{\varphi, \xi}
\)------%
を課すので,
かわりに右側に反線形性が要請される.
右線形性と左線形性のどちらを採用するかは単に定義の問題であり,
両者は本質的に同じものである.
しかし,量子力学の理論で用いるブラ・ケット記法は右線形性を前提としたものであるから,
そちらに統一したほうが物理を扱ううえでは便利である.


内積の具体例として,
やはりユークリッド空間$V = \symbb{C}^N$について考えよう.
\begin{equation}
    \label{eq:Euclidean-inner-product}
    \iparen{\symbf{x}, \symbf{y}}
        \coloneq \sum_{i = 1}^{N} \conj{x_i} \cdotp y_i
\end{equation}
で与えられる.
% ここで,$\conj{x}$は複素数$x$の共役複素数を表す.
\cref{eq:Euclidean-inner-product}が内積の公理を満たすことは,
容易に確認できる.
すなわち,\cref*{innerp:linear}は
\begin{equation*}
    \begin{split}
        \iparen{\symbf{x}, \  a \symbf{y} + b \symbf{z}}
            &= \sum_{i = 1}^{N} \conj{x_i} \cdotp ( a y_i + b z_i )
            \\
            &= a \sum_{i = 1}^{N} \conj{x_i} \cdotp y_i
            + b \sum_{i = 1}^{N} \conj{x_i} \cdotp z_i
            = a \iparen{\symbf{x}, \symbf{y}} + b \iparen{\symbf{y}, \symbf{z}}
    \end{split}
\end{equation*}
と示されるし,\cref*{innerp:conjugate-symmetry}は,
\begin{equation*}
    \iparen{\symbf{x}, \symbf{y}}
        = \sum_{i = 1}^{N} \conj{x_i} \cdotp y_i
        = \sum_{i = 1}^{N} \conj*{ \conj{y_i} \cdotp x_i }
        = \conj{\ab(  \sum_{i = 1}^{N} \conj{y_i} \cdotp x_i  )}
        = \conj{\ab\big( \iparen{\symbf{y}, \symbf{x}} )}
\end{equation*}
と計算できる.
\cref*{innerp:positive-definiteness}を示すには,
\begin{equation*}
    \iparen{\symbf{x}, \symbf{x}}
        = \sum_{i = 1}^{N} \conj{x_i} \cdotp y_i
        = \sum_{i = 1}^{N} \abs{x_i}^2
        \geq 0
\end{equation*}
とすればよい.
最後の不等号は,各$i$に対して$\abs{x_i} \geq 0$であることによる.
もし$\iparen{\symbf{x}, \symbf{x}} = 0$なら,
すべての$i$に対して$\abs{x_i} = 0$,
つまり$x_i = 0$であるから,$\symbf{x} = \symbf{0}$である.

この内積が\cref{eq:anti-linear}の反線形性を満たすことも,
直接的に確かめられる.
\begin{equation*}
    \iparen{a \symbf{x} + b \symbf{y}, \  \symbf{z}}
    = \sum_{i = 1}^{N} \conj*{a x_i + b y_i} \cdotp z_i
    = \conj{a} \sum_{i = 1}^{N} \conj{x_i} \cdotp z_i
    + \conj{b} \sum_{i = 1}^{N} \conj{y_i} \cdotp z_i
    = \conj{a} \iparen{\symbf{x}, \symbf{z}} + \conj{b} \iparen{\symbf{y}, \symbf{z}}
\end{equation*}



\subsubsection{ノルム}

$V$を内積空間とすると,
ベクトル$\psi$に対して,
内積の公理\cref{innerp:positive-definiteness}より,
\begin{equation}
    \label{eq:norm}
    \norm{\psi} \coloneq \sqrt{\vphantom{\psi}\smash{\iparen{\psi, \psi}}}
\end{equation}
は正の実数またはゼロとなる.
\cref{eq:norm}を(内積空間$V$に付随する)$\psi$の\ltword{ノルム}(norm)という.

たとえば$V = \symbb{R}^N$であれば,
ベクトル$\symbf{x} \in \symbb{R}^N$のノルム
\begin{equation*}
    \norm{\symbf{x}} = \sqrt{\sum_{i = 1}^{N} \abs{x_i}^2}
\end{equation*}
が$\symbf{x}$の``長さ''に対応することは,
$N$次元空間における三平方の定理を考えればわかる.

$V$を内積空間,$\norm{\bigdot}$を$V$に付随するノルムとする.
ノルムについて,以下の\ltword{三角不等式}
\begin{equation}
    \label{eq:triangle}
    \norm{\psi + \varphi}
    \leq \norm{\psi} + \norm{\varphi}
\end{equation}
が成り立つ.
幾何的に解釈すると,
ベクトル$\vec{a}$とベクトル$\vec{b}$を合成したベクトル$\vec{a} + \vec{b}$
(つまり,$\vec{a}$の始点と$\vec{b}$の終点をむすんだベクトル)の長さは,
$\vec{a}$の長さと$\vec{b}$の長さを足したものよりも短い(または同じ)であるということである.


これにより,
ノルムは単にベクトルの長さというだけでなく,
ベクトル空間上の距離と考えることができる.

\bluehead{\cref{eq:triangle}は次のように示される.}
まず,内積空間では,
任意のベクトル$\psi, \varphi$に対して,
\begin{equation}
    \label{eq:Cauchy-Schwarz}
    \abs{\iparen{\psi, \varphi}} \leq \norm{\psi} \norm{\varphi}
\end{equation}
が成り立つ.
これを使うと,
\begin{equation*}
    \begin{split}
        \norm{\psi + \varphi}^2
        &= \iparen{\psi + \varphi, \  \psi + \varphi}
        \\
        &= \iparen{\psi, \psi}
            + \iparen{\psi, \varphi}
            + \iparen{\varphi, \psi}
            + \iparen{\varphi, \varphi}
        \\
        &= \norm{\psi}^2
            + 2 \Real \iparen{\psi, \varphi}
            + \norm{\varphi}^2
        \\
        &\leq \norm{\psi}^2 + 2 \abs{ \iparen{\psi, \varphi} } + \norm{\varphi}^2
        \\
        &\leq \norm{\psi}^2 + 2 \norm{\psi} \norm{\varphi} + \norm{\varphi}^2
        = \ab(\norm{\psi} + \norm{\varphi})^2
    \end{split}
\end{equation*}
とできる.
ここで,ひとつめの不等号は単なる複素数の絶対値の定義であり,
ふたつめの不等号が\cref{eq:triangle}である.

\bluehead{\cref{eq:Cauchy-Schwarz}の証明}は次のようにできる.




\subsubsection{正規直交基底}




\subsection{空間の完備性}

完備?

ベクトル空間$V$の点列とは,
自然数$n$に対して$\psi_n \in V$であるような$\psi_1, \psi_2, \dots$のセットをいう.
この列のことを$\sequ{\psi_n}[n]$と書くことにする.

$V$を内積空間とする.
$V$の点列$\sequ{\psi_n}[n]$が
\begin{equation*}
    \lim_{m, n \to \infty} \norm{\psi_m - \psi_n} = 0
\end{equation*}
を満たすとき,
$\sequ{\psi_n}[n]$は\ltword{コーシー列}(Cauchy sequence)であるという.

$V$で収束する点列はコーシー列である.
なぜなら,点列$\sequ{\psi_n}[n]$が$\psi_n \to \psi_*$($n \to \infty$)に収束するとすれば,
$\norm{\psi_n - \psi_*} \to 0$($n \to \infty$)だから,
\begin{equation*}
    \norm{\psi_m - \psi_n}
    = \norm{(\psi_m - \psi_*) + (\psi_* - \psi_n)}
    \leq \norm{\psi_m - \psi_*} + \norm{\psi_* - \psi_n}
    \xrightarrow{n, m \to \infty} 0
\end{equation*}
となり,コーシー列の条件を満たす.

一方,$V$のコーシー列が$V$で収束するとは限らない.





\subsection{ヒルベルト空間}

完備な内積空間を\ltword{ヒルベルト空間}(Hilbert space)という.




\section{もう少し準備}

\subsection{線形写像}

2つの集合$X, Y$が与えられたとき,
$X$の各元をそれぞれ$Y$のある元に対応させる規則のことを,
$X$から$Y$への\ltword{写|像}[しゃ|ぞう]という.
ある写像を$f$と書くことにすると,
元$x \in X$が元$y \in Y$に対応することを,
$f(x) = y$と書く.

$f$を集合$X$から$Y$への写像とする.
$X$の任意の元$x, y$について,
$x \neq y$なら$f(x) \neq f(y)$が必ず成り立つとき,
$f$を$X$から$Y$への\ltword{単|射}[たん|しゃ](injection)という.
また,$Y$の任意の元$y$について,
$y = f(x)$を満たす$X$の元$x$が存在するとき,
$f$を$X$から$Y$への\ltword{全|射}[ぜん|しゃ](surjection)という.
$f$が単射であり,かつ全射であるとき,
$f$を\ltword{全|単|射}[ぜん|たん|しゃ](bijection)という.

$X$から$Y$への写像$f$が全単射であるとは,
$X$の元と$Y$の元が$f$によってそれぞれ1対1に結びついていることを意味する.
$f$は\fire{$X$の元\emph{を}}\water{$Y$の元\emph{に}}対応させる写像だが,
同じ対応について,
逆に\water{$Y$の元\emph{を}}\fire{$X$の元\emph{に}}対応させる写像(つまり$Y$から$X$への写像)$f^{-1}$を考えることができる.
$f^{-1}$を,$f$の\ltword{逆|写|像}[ぎゃく|しゃ|ぞう]という.

写像そのものも非常に重要な概念であるが,
量子力学(あるいはもっと一般の線形代数)でとりわけ重要な役割を果たすのは,
写像のうち特別な性質(\emph{線形性})を満たすものである.
2つのベクトル空間$V, W$が与えられたとき,
$V$から$W$への写像$f$があったとする.
すべての$\psi \in V$とすべての$\varphi \in W$に対して,
\begin{equation}
    \label{eq:linear-map}
    f(a \psi + b \varphi)
        = a f(\psi) + b f(\varphi)
\end{equation}
を満たすとき,$f$を($V$から$W$への)\ltword{線形写像}(linear map)という.
あるいは,$f$は\ltword{線形}(linear)であるともいう.

線形写像の例としてなじみ深いのが行列である.
複素成分の$M \times N$行列$\hat{A}$は,
ユークリッド空間$\symbb{C}^M$から$\symbb{C}^N$への線形写像になっている.
$\hat{A}$が\cref{eq:linear-map}を満たすことは,
行列計算によって容易に示されるが,
ここでは省略する.

微分演算子$\hat{D} \coloneq \dv{}{x}$も線形写像である.
$\hat{D}$は,
ベクトル空間$C^1$の元(1回微分可能な連続関数)を$C^0$の元(連続関数)に対応させる写像である.
$\hat{D}$が線形であることは,
関数$f, g \in C^1$および複素数$a, b$に対して,
\begin{equation*}
    \hat{D}(af + bg)(x) = \dv{}{x} \ab(a f(x) + b g(x))
        = a \dv{f(x)}{x} + b \dv{g(x)}{x}
        = \ab( a \hat{D}f + b \hat{D}g )(x)
\end{equation*}
であることから従う($\hat{D}(f)$を$\hat{D}f$と略記している)\footnote{
    ここの議論を精緻に行うには,
    連続関数全体が関数としての和とスカラー倍に対してベクトル空間をなすことを示さないといけないが,
    本稿では$\hat{D}$を扱わないので例示だけにとどめる.
}.
$\hat{D}$が線形であることから,
1次元の時間に依存しないシュレーディンガー方程式
\begin{equation*}
    -\frac{\hbar^2}{2m} \dv[2]{\psi(x)}{x} + V(x) \psi(x) = E \psi(x)
\end{equation*}
は線型方程式である.




\subsection{ベクトル空間の同型}

$V, W$をそれぞれベクトル空間とする.
$V$から$W$への線形写像であって,全単射である$f$が存在するとき,
$V$と$W$は\ltword{同|型}[どう|けい](isomorphic)であるという.
互いに同型なベクトル空間は,
(本質的に)同じベクトル空間であるとみなすことができる.





\section{ブラ・ケット記法}








\end{document}