\documentclass{sotsu}

\usepackage{sotsu}
\usepackage{sotsu-symb}
\usepackage{sotsu-thm}

\usepackage[
    style=sotsu,
    maxnames=99,
    minnames=99,
    ]{biblatex}
\addbibresource{sotsu.bib}


\definecolor{fire}{RGB}{255, 90, 0}
\definecolor{water}{RGB}{0, 120, 255}

\newcommand{\fire}[1]{\textcolor{fire}{#1}}
\newcommand{\water}[1]{\textcolor{water}{#1}}


\title{なんか}
\date{\today}


\begin{document}

\maketitle


\section{イントロダクション}

17世紀から18世紀にかけての物理学者,ニュートンが大成した古典力学(ニュートン力学)によって,
物体の運動は完全に記述できるようになった.
この世界は原子という物体によって構成されているのだから\footnote{
    実は,原子の存在が証明されたのは1905年.
    アインシュタインによるブラウン運動の解明による.
},
現時点での宇宙の全原子の座標と運動量(および相互作用)を完全に決定できる存在,
いわゆる\emph{ラプラスの悪魔}は,
この世界の過去も未来もすべて知ることができる.
のちに形成された解析力学のことばを使えば,
各原子のハミルトニアンと初期状態が,その原子の時間発展を決定する%
%(また正準方程式は時間反転に対して対称である),
といえる.

もちろん実際には,
宇宙の全原子はおろか,
小さな箱に閉じ込められた気体を構成する全原子の座標と運動量を特定することすら,
人間には不可能である.
このような多数------おおよそ$10^{23}$個のオーダー-----の原子がつくる\emph{多体系}は,
各粒子を支配しているはずのニュートン力学からは想像もつかないようなふるまいを示す.
ニュートン方程式は可逆であるにもかかわらず,
多体系は本質的に不可逆的である.
実際,熱湯と水を混ぜればぬるま湯になるが,
逆にぬるま湯が熱湯と水に分離されるところを見た人はいないだろう.
このような「\emph{マクロ}な世界」を扱うのが\emph{統計力学}である.
ニュートン力学には存在しない「温度」や「エントロピー」といった概念が,
統計力学において定義されることは,
よく知られている.

マクロな理論である統計力学の形成から少し遅れた20世紀初頭,
\emph{ミクロ}な世界の理論である\emph{量子力学}が形成された.
量子力学によればエネルギーは離散的な値しかとることができないし,
物理量は確率的にしか予言できず,
さらには状態を重ね合わせることができる.
また,そこで扱われる量子力学的粒子も,
古典粒子に存在しない「スピン」や「位相」といった値を持つ.
このような点で,量子力学は古典力学と根本的に異なる理論体系である.
しかしながら,面白いことに,視野を広げる(ミクロな世界から遠ざかる)ことで,
量子力学は古典力学に漸近するのである.
水素原子のスペクトル問題は,
量子力学によって見事に解き明かされた.


1905年にはアインシュタインが特殊相対論を提唱し,
絶対時間の存在を暗黙裡に仮定していたニュートン力学はここに崩壊した.
量子力学を相対論的に書き直す試みがなされるまで,
それほど時間はかからなかった.
1928年,ディラックは,量子力学に相対論的効果を導入したディラック方程式を提唱した.
この方程式からは,
すべての粒子に対して,
質量が同じで,電荷の符号が逆である粒子(\emph{反粒子})が存在することが示唆される.
反粒子が実在することは,
1932年のアンダーソンによる陽電子の発見によって裏付けられた.


量子力学を使って統計力学を捉えなおすこともなされた.
すなわち,多体系を構成する粒子が量子力学的であるとするのである.
一切の相互作用なくしておこる相転移,
ボース・アインシュタイン凝縮はまさに量子統計力学の醍醐味である.


1911年,カメルリング・オネス\footnote{
    Kamerlingh Onnesのカタカナ表記は,
    圧力$p$・$P$問題,
    状態密度$D$・$N$問題,
    ボーズ・ボース問題,
    Liouville問題と並ぶ統計力学の難問である.
}は極低温に冷やした水銀の電気抵抗がゼロになることを発見した.
のちに超伝導と呼ばれるこの現象の理論的解明は,
1957年にバーディーン,クーパー,シュリーファーによってなされた.
3人の頭文字をとってBCS理論と呼ばれるこの理論は,
量子力学を多体系に適用したものである.


このように,量子力学を応用することによって多くの新奇的な物理現象が明らかになったわけであるが,
しかしその中で量子力学の数学的位置づけは等閑視されてきた.
これを読んでいる人でヒルベルト空間を知らない人はいないだろうが,
ではヒルベルト空間の定義を述べよ,といわれると動揺する人が多いのではないだろうか.
むろん基礎から量子力学を理解するよりも,
実際に量子力学を使って物理現象を理解することが重要であるという考え方も大いにあり得よう.
量子力学の厳密な数学的定式化はフォン・ノイマン(1932年)によってなされた.




たった2つのスピンであっても,
非常に奇妙なふるまいを示す.



\section{数学的準備}

\subsection{ヒルベルト空間}

量子状態はヒルベルト空間の元であるベクトルによって記述される.
そこで,この節ではまず,一般のヒルベルト空間について定義する.
スピン系のような有限次元のヒルベルト空間のベクトルとは,
単なる数ベクトルであることを示す.

まずは\emph{ベクトル空間}を定義する.
$\symbb{C}$を複素数全体の集合とする.
集合$V$があって,
$V$の\ltjruby{元}{げん}に対して\emph{和}$\vecplus \colon V \times V \to V$,
\emph{スカラー倍}$\scaprod \colon \symbb{C} \times V \to V$が定義されているとする.
$V$の元を$\psi, \varphi, \xi$,
複素数$a, b$としたとき,
以下の8条件
\begin{enumerate}
    \item \label{vector:sum-associative}
        \bluehead{和は結合的である}------すなわち,
        \fire{任意の$\psi, \varphi, \xi \in V$}に対し,
        $(\psi \vecplus \varphi) \vecplus \xi = \psi \vecplus (\varphi \vecplus \xi)$.
    \item \label{vector:sum-commutative}
        \bluehead{和は交換する}------すなわち,
        \fire{任意の$\psi, \varphi, \xi \in V$}に対し,
        $\psi \vecplus \varphi = \varphi \vecplus \psi$.
    \item \label{vector:sum-zero}
        \bluehead{和に対する零元の存在}------すなわち,
        \water{ある$\symbf{0} \in V$}が存在して,
        \fire{任意の$\psi \in V$}に対し,$\psi + \symbf{0} = \psi$.
    \item \label{vector:sum-opposite}
        \bluehead{和に対する逆元の存在}------すなわち,
        \fire{任意の$\psi \in V$}に対し,
        \water{ある$\varphi \in V$}が存在して,
        $\psi + \varphi = \symbf{0}$.
    \item \label{vector:scalar-sum}
        \bluehead{スカラー倍は複素数の和に対して分配的である}------すなわち,
        \fire{任意の$\psi \in V$}と\fire{任意の複素数$a, b$}に対し,
        $(a + b) \scaprod \psi = (a \scaprod \psi) + (b \scaprod \psi)$.
    \item \label{vector:scalar-distributive}
        \bluehead{スカラー倍は$V$上の和に対して分配的である}------すなわち,
        \fire{任意の$\psi, \varphi \in V$}と\fire{任意の複素数$a$}に対し,
        $a \scaprod (\psi + \varphi) = (a \scaprod \psi) + (a \scaprod \varphi)$.
    \item \label{vector:scalar-prod}
        \bluehead{複素数の積とスカラー倍の結合}------すなわち,
        \fire{任意の$\psi \in V$}と\fire{任意の複素数$a, b$}に対し,
        $(ab) \scaprod \psi = a \scaprod (b \scaprod \psi)$.
    \item \label{vector:scalar-identity}
        \bluehead{$1$はスカラー倍の単位元}------すなわち,
        複素数$1$について,\fire{任意の$\psi \in V$}に対し,
        $1 \scaprod \psi = \psi$.
\end{enumerate}
を満たすのならば,
$V$を\ltword{ベクトル空間}(vector space)と呼び,
$V$の元(上で挙げた$\psi, \varphi, \xi, \dotsc$)を\ltword{ベクトル}という.

ここで,上の条件に出てくる「\fire{任意の}---」と「\water{ある}---」の順序は重要である.
\cref{vector:sum-zero}では$\symbf{0}$はただひとつであり,
その$\symbf{0}$について$\psi \vecplus \symbf{0} = \psi$が満たされ,
さらに$\varphi \vecplus \symbf{0} = \varphi$が成り立たねばならない.
その一方で,\cref{vector:sum-opposite}における$\varphi$は,
$\psi$のとりかたによって変わりうる.
あるベクトル$\psi$に対して$\psi \vecplus \varphi = \symbf{0}$が満たされても,
別のベクトル$\xi$に対して$\xi \vecplus \varphi = \symbf{0}$が成り立つとは限らない.



\clearpage



ベクトル空間の典型例は\word{ユークリッド空間}(Euclidean space)である.
ある自然数$N$($\geq 1$)に対し,
集合$\symbb{C}^N$を
\begin{equation*}
    \symbb{C}^N 
    \coloneq
    \Set*{
    \symbf{x} = 
    \begin{pmatrix}
        x_1  \\  \vdots  \\  x_N
    \end{pmatrix}
    \given 
    \text{$x_1, \dots, x_N$は複素数}
    }
\end{equation*}
で定める.
和$\vecplus$とスカラー倍$\scaprod$は,
各成分の和と積,つまり
\begin{align*}
    \symbf{x} \vecplus \symbf{y}
        &\coloneq 
        \begin{pmatrix}
            x_1 + y_1  \\
            \vdots     \\
            x_N + y_N
        \end{pmatrix},
    \quad
    c \scaprod \symbf{x}
        \coloneq 
        \begin{pmatrix}
            c x_1  \\
            \vdots \\
            c x_N
        \end{pmatrix}
\end{align*}
で定める.
このとき,$\symbb{C}^N$がベクトル空間の定義を満たすことを確認するのは容易である.
まず,
\cref{vector:sum-associative,vector:sum-commutative}は,
複素数の性質から明らかである.
\cref{vector:sum-zero}における$\symbf{0}$は,
\begin{equation*}
    \symbf{0} \coloneq 
    \begin{pmatrix}
        0  \\  \vdots  \\  0
    \end{pmatrix}
\end{equation*}
で与えられる.
また,\cref{vector:sum-opposite}でいう,$\symbf{x}$に対して$\symbf{x} \vecplus \symbf{y} = \symbf{0}$を満たすベクトル
$\symbf{y}$は,
\begin{equation*}
    \symbf{y} \coloneq 
    -\symbf{x} =
    \begin{pmatrix}
        -x_1  \\  \vdots  \\  -x_N
    \end{pmatrix}
\end{equation*}
である.
\cref{vector:scalar-sum,vector:scalar-distributive}は複素数の分配法則に帰着されるし,
\cref{vector:scalar-prod}は複素数の結合法則そのものである.
\cref{vector:scalar-identity}は複素数$1$が乗法の単位元であることから直ちに従う.


\clearpage



次に,ベクトル空間において定義される\ltjruby{内|積}{ない|せき}という概念を導入する.
内積を定義することは,ベクトル空間に幾何学的な性質を入れることに対応する.

$\iparen{\bigdot, \bigdot} \colon V \times V \to \symbb{C}$は,
2つのベクトルを複素数に対応させる写像であって,
\begin{enumerate}
    \item \label{innerp:linear} 
        \bluehead{右線形性}------すなわち,
        \fire{任意のベクトル$\psi, \varphi, \xi$}および\fire{任意の複素数$a, b$}に対し,
        $\iparen{\psi, \  a \varphi \vecplus b \xi} = a \iparen{\psi, \varphi} + b \iparen{\psi, \xi}$.
    \item \label{innerp:conjugate-symmetry} 
        \bluehead{共役対称性}------すなわち,
        \fire{任意のベクトル$\psi, \varphi$}に対し,
        $\iparen{\psi, \varphi} = \conj{\iparen{\varphi, \psi}}$.
        ここで,$\conj{\bigdot}$は$\bigdot$の共役複素数を表す.
    \item \label{innerp:positive-definiteness}
        \bluehead{正定値性}------すなわち,
        \fire{任意のベクトル$\psi$}に対し,
        $\iparen{\psi, \psi} \geq 0$であり,
        しかも$\iparen{\psi, \psi} = 0$ならば$\psi = \symbf{0}$.
\end{enumerate}
を満たすものを,$V$上の\ltword{内|積}[ない|せき](inner product)と呼ぶ.

\cref{innerp:linear,innerp:conjugate-symmetry}を合わせると,
左側のベクトルについては以下の\emph{反線形性}
\begin{equation*}
    \iparen{a \psi \vecplus b \varphi, \  \xi}
        = \conj{a} \iparen{\psi, \xi} + \conj{b} \iparen{\varphi, \xi}
\end{equation*}
が成り立つことに注意しなければならない.
なお,数学ではふつう\cref{innerp:linear}のかわりに\emph{左線形性}%
------つまり\(
    \iparen{a \psi \vecplus b \varphi, \  \xi}
    = a \iparen{\psi, \xi} + b \iparen{\varphi, \xi}
\)------%
を課すので,
かわりに右側に反線形性が要請される.
右線形性と左線形性のどちらを採用するかは単に定義の問題であり,
両者は本質的に同じものである.
しかし,量子力学の理論で用いるブラ・ケット記法は右線形性を前提としたものであるから,
そちらに統一したほうが物理を扱ううえでは便利である.




\end{document}