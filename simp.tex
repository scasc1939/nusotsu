\documentclass{sotsu}

\usepackage{sotsu}
\usepackage{sotsu-symb}
\usepackage{sotsu-thm}

\usepackage[
    style=sotsu,
    maxnames=99,
    minnames=99,
    ]{biblatex}
\addbibresource{sotsu.bib}


\title{なんか}
\date{\today}


\begin{document}

\maketitle


\section{イントロダクション}

17世紀から18世紀にかけての物理学者・ニュートンが大成した古典力学によって,
物体の運動は完全に記述できるようになった.
この世界は原子という物体によって構成されているのだから\footnote{
    実は,原子の存在が証明されたのは1905年.
    アインシュタインによるブラウン運動の解明による.
},
ここに世界のすべてが理解できるようになったのである.
宇宙の原子のある時点での座標と運動量を完全に特定することができれば,
過去も未来もすべて確定する.
これが有名な\emph{ラプラスの悪魔}である.
現実的には不可能であるが,
理論上はそのようなことが可能である.

ところで,この世で目にする物体(固体でもよいし,気体でも,あるいは人体でも!)は,
ふつう粒子がたくさん,
具体的には$10^{23}$個ほど集まってできている.
こうした多体系は,
各粒子を支配しているはずのニュートン力学からは想像もつかないようなふるまいを示す.
これを扱うのが統計力学である.
ニュートン力学とは「質点」の理論,
一方で統計力学は「マクロ」な理論といわれる.
マクロな世界では,
質点が持たない様々な概念,
例えば「温度」「エントロピー」といったものが定義される.
さらに,ニュートン方程式は可逆であるにもかかわらず,
多体系は本質的に不可逆的である.

一方,これと同時期にこのようなマクロな物理学とは別の壁が,
やはりニュートン力学の前に現れた.
もともとはプランクが黒体放射を解明するためのアイデアであった量子力学は,
水素原子のスペクトル問題などを扱うミクロな物理学として,
20世紀の物理学者たちによって整備された.
量子力学においては,
物理量は確率的にしか予言できない,
状態を重ね合わせることができるなど,
古典力学とはまったく相容れないことを前提に議論をする.
また,そこで扱われる量子力学的粒子も,
古典粒子に存在しない「スピン」「位相」などといった値を持つ.

たった2つのスピンであっても,
非常に奇妙なふるまいを示す.



\end{document}