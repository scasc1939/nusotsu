% \documentclass[draft]{sotsu}
\documentclass{sotsu}

\newif\ifdraft
\drafttrue
\draftfalse % これをコメントアウト(先頭に % )すると draft モードになる

\usepackage{sotsu}

\usepackage[
    style=sotsu,
    maxnames=99,
    minnames=99,
    ]{biblatex}
\addbibresource{sotsu.bib}

\usepackage{subfiles}


\graphicspath{{./fig/}}


% 索引
\ifdraft
    \DeclareDocumentCommand{\index}{O{} m}{}
\else
    \makeindex
\fi




\begin{document}


% 目次
\tableofcontents


\chapter{集合と空間}
\label{sec:set-and-space}

数学において最も基本的な概念が集合である.
すべての数学は,究極的には集合の言葉で記述される.
そこまでいかなくても,集合は数学を使うのに必要不可欠なツールであるし,
集合のことばに慣れておくと,数学のみならず物理学でも非常に便利である.
この章では「集合とは何か」といった公理には立ち入らず,
素朴に集合を定義したうえで,
集合にかかわる諸概念を導入し,性質を見ることにする.

集合に対して`距離'を導入したものが「空間」である.
距離を導入することで,「収束」や「開集合」「閉集合」といった重要な概念を定義できる.
この意味で,空間とは線形代数や微分積分学など数学の諸分野の基礎となる,極めて重要な概念である.
この章では抽象的な距離について定義するが,
抽象的な定義を用いた証明よりも,
具体的な空間(ユークリッド空間や関数空間)について成り立つ性質を理解するほうが(物理においては)有益である.



\subfile{sub/set_theory.tex}

\subfile{sub/topology.tex}

\subfile{sub/algebra.tex}


\chapter{ベクトル空間の線形代数}
\label{sec:linear-algebra}

量子力学においては状態を「ベクトル」としてあらわすことは,
量子力学を学んだばかりの学生でも知っている.
しかし,状態がふつうの数ベクトルであらわされるわけではない.
それではここでいう「ベクトル」とは何のことなのだろうか.

数ベクトルのもつ性質を抽出し,抽象的な「ベクトル」を定義する.
さらにベクトルの内積,行列をも抽象化し,「線形写像」というものを定義する.
線形写像は量子状態から観測可能な物理量を取り出す線形演算子の基礎となる考え方であり,
それゆえ線形写像を理解せずに量子力学を理解することはできない.
線形代数学とはこのベクトルと線形写像を扱う数学の分野であるが,
量子力学とは線形代数であるという人もいるくらいである.


\subfile{sub/matrix.tex}

\subfile{sub/vector_space.tex}

\subfile{sub/inner_product.tex}

\subfile{sub/linear_map.tex}

\subfile{sub/dual_space.tex}


\chapter{解析学と関数解析}

量子力学の根底をなす「ヒルベルト空間」は,
\cref{sec:linear-algebra}で扱ったベクトル空間の一種である.
しかし,ヒルベルト空間を構成する要素は単なる数ベクトルではなく関数,しかも無限次元である.
それゆえ,有限次元では自明であるかのように思える定理のいくつかが破綻する.
このような無限次元ベクトル空間を扱う数学の分野が関数解析である.

\subfile{sub/measure.tex}

\subfile{sub/functional_analysis.tex}

\subfile{sub/hilbert.tex}


\subfile{sub/operator.tex}



\chapter{量子力学へ}

\subfile{sub/bra_ket.tex}


\ModifyHeading{section}{
    pagebreak=nariyuki,
}


% もとの bibintoc だと目次からのハイパーリンクがうまく作動しない

\defbibheading{newbibintoc}[\bibname]{%
    \chapter*{#1}% 
    \phantomsection % ハイパーリンク用
    \addcontentsline{toc}{chapter}{\bibname}% 目次に追加
    \markboth{#1}{#1}% ヘッダーの更新
}

\printbibheading[
    heading=newbibintoc,
    title={参考文献}
]

\printbibliography[
    keyword=set,
    heading=subbibliography,
    title={集合論}
]
\printbibliography[
    keyword=linear, 
    heading=subbibliography, 
    title={線形代数}
]
\printbibliography[
    keyword=functional, 
    heading=subbibliography, 
    title={関数解析}
]
\printbibliography[
    notkeyword=set,
    notkeyword=lin,
    notkeyword=functional,
    heading=subbibliography,
    title={その他}
]





\ifdraft
\else
    \printindex
\fi


\end{document}
