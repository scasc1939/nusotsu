% \documentclass[draft]{sotsu}
\documentclass{sotsu}

\newif\ifdraft
\drafttrue
\draftfalse % これをコメントアウト(先頭に % )すると draft モードになる

\usepackage{sotsu}


\usepackage{subfiles}

\graphicspath{{./fig/}}



\addbibresource{sotsu.bib}


% \newcommand{\dotprod}{\mathbin{\mdsmwhtcircle}}
\newcommand{\dotprod}{\mathbin{\ast}}

\makeatletter

% \newcommand{\scaprod}{\mathbin{\diamondcdot}}
\newcommand{\scaprod}{\mathbin{\star}}
\newcommand{\scaprodvar}{\mathbin{\varstar}}

\newcommand{\vecplus}{\tplus}
\newcommand{\vecminus}{\tminus}

\makeatother


\ifdraft
    \DeclareDocumentCommand{\index}{O{} m}{}
\else
    \makeindex
\fi




\begin{document}


\tableofcontents


\subfile{sub/set_theory.tex}

\subfile{sub/vector_space.tex}

\subfile{sub/inner_product.tex}



\section{ヒルベルト空間}

\subsection{距離空間}

\begin{definition}[距離]
    \label{dfn:distance}
    $X$を集合とする.写像$d \colon X \times X \to \symbb{R}$が以下を満たすとき,
    \word{距離}(distance)あるいは\word{計量}(metric)という.
    \begin{itemize}
        \item 任意の$x, y, z \in X$に対し,
        \begin{enumerate}
            \item $d(x, y) \geq 0$である.ただし,$d(x, y) = 0$となるのは$x = y$のときに限る.
            \item $d(x, y) = d(y, x)$である.
            \item $d(x, z) \leq d(x, y) + d(y, z)$である.
        \end{enumerate}
    \end{itemize}
    $(X, d)$を\word{距離空間}[][きよりくうかん](metric space)という.
\end{definition}


\begin{example}
    $X = \symbb{R}^n$(実ユークリッド空間)とする.
    $\symbf{x}, \symbf{y} \in \symbb{R}^n$に対して,
    $d_2 (\symbf{x}, \symbf{y}) = \sqrt{ \sum_{i = 1}^{n} \abs{x_i - y_i}^2 }$と定めると,これは距離である.

    特に,実数全体の集合$\symbb{R}$は,距離$d(x, y) \coloneq \abs{x - y}$に対して距離空間になる.
\end{example}

\begin{proposition}
    \label{thm:norm-is-distance}
    $V$をノルム空間\refdfn{dfn:norm},$\norm{\bigdot}$を$V$上のノルムとする.
    $\symbf{x}, \symbf{y} \in V$に対し,$\norm{\symbf{x} \tminus \symbf{y}}$は,$V$上の距離である.
    すなわち,ノルム空間は距離空間である.
\end{proposition}

\begin{proof}
    ノルムの定義より従う.
\end{proof}




\subsection{収束列とコーシー列}

ヒルベルト空間を扱ううえで避けて通れないのが,「\ruby{完備}{かん|び}」という概念である.
完備性を定義するための準備として,ある値に収束する数列について議論する.

これから$X$の点列といった場合,
$\symbb{N}$で順序付けられた$X$の加算部分集合,つまり
$(x_1, x_2, \dots, x_i, \dotsc)$であり,$i \in \symbb{N}$,$x_i \in X$であるものをいうことにする.

\begin{definition}[収束列]
    \label{dfn:convergent-sequence}
    $(X, d)$を距離空間とする.
    $X$の点列$(x_1, x_2, \dotsc)$が$x \in X$に\word{収束する}(converge)とは,
    任意の$\varepsilon > 0$に対し,ある$N \in \symbb{N}$が存在して,
    任意の$n > N$に対し,$d(x_n, x) < \varepsilon$となることをいう.
    
    このときの$x$のことを\word{極限}[きょく|げん](limit)という.
    \ruby{収束先}{しゅう|そく|さき}ということもある.

    また,$X$の点列$(x_1, x_2, \dotsc)$が\word{収束列}(convergent sequence)であるとは,
    点列がある$x \in X$に収束することをいう.
\end{definition}

記号$\lim$を用いると,点列$(x_i)$が$x$に収束することを
\[  \lim_{n \to \infty} d(x_n, x) = 0  \]
とかける.

収束列の極限はただ一つに定まる.
実際,$X$の点列$(x_i)$の極限が$x$と$x'$の2つあったとすると,
\begin{equation*}
    \begin{split}
        d(x, x') &\leq d(x, x_n) + d(x_n, x)  \qquad \text{(三角不等式)}  \\
            &= d(x_n, x) + d(x_n, x)  
            \xrightarrow{n \to \infty} 0
    \end{split}
\end{equation*}
であるので,$x = x'$である\refdfn{dfn:distance}.


\begin{definition}[コーシー列]
    \label{dfn:Cauchy-sequence}
    $(X, d)$を距離空間とする.
    $X$の点列$(x_1, x_2, \dotsc)$が\word{コーシー列}(Cauchy sequence)であるとは,
    任意の$\varepsilon > 0$に対し,ある$N \in \symbb{N}$が存在して,
    任意の$n, m > N$に対し,$ d(x_n, x_m) < \varepsilon $となることをいう.
\end{definition}

記号$\lim$を用いると,コーシー列の定義は
$\lim_{n, m \to \infty} d(x_n, x_m) = 0$とかける.

収束列の定義とコーシー列の定義はよく似ているが,前者は収束先$x \in X$の存在を要請しているのに対し,後者はそうでない.
収束列とコーシー列には,次のような関係がある.
\begin{theorem}
    \label{thm:convergent-is-Cauchy}
    距離空間の収束列は常にコーシー列である.
\end{theorem}

\begin{proof}
    $(x_i)$を収束列,その極限を$x \in X$とする.
    定義より,任意の$\delta > 0$に対し,ある$N > \symbb{N}$が存在して,
    任意の$n > N$に対し,$d(x_n, x) < \delta$である.
    すると,任意の$m, n > N$に対し,
    \begin{equation*}
        \begin{split}
            d(x_m, x_n) &\leq d(x_m, x) + d(x, x_n)  \qquad \text{(三角不等式)}  \\
                &= d(x_m, x) + d(x_n, x)  \\
                &< 2\delta
        \end{split}
    \end{equation*}
    である.
    $\delta \coloneq \varepsilon/2$とおけば,コーシー列の条件\refdfn{dfn:Cauchy-sequence}が成立する.
\end{proof}

距離空間において,すべての収束列はコーシー列であるが,その逆は必ずしも成立しない.
コーシー列が収束列でない例をいくつか挙げる.

\begin{example}
    $\symbb{Q}$の点列$(1, 1.4, 1.41, 1.414, \dotsc)$は明らかにコーシー列である.
    しかし,この点列の収束先は$\sqrt{2} \notin \symbb{Q}$であり,$\symbb{Q}$の収束列でない.
\end{example}

\begin{example}
    開区間$(0, 1) \subset \symbb{R}$の点列$(0.1, 0.01, 0.001, 0.0001, \dotsc)$は明らかにコーシー列である.
    しかし,この点列の収束先は$0 \notin (0, 1)$であり,$(0, 1)$の収束列でない.
\end{example}


\subsection{距離空間の完備性}

\begin{definition}
    距離空間$(X, d)$において,任意のコーシー列が収束列であるとき,\word{完備}[かん|び](complete)であるという.
    このとき,$(X, d)$のことを\word{完備距離空間}(complete metric space)という.
\end{definition}

\begin{proposition}[実数の完備性]
    実数全体の集合$\symbb{R}$は完備である.
\end{proposition}

\begin{proof}
    実数を,有理数列のうちコーシー列であるものの同値類\refdfn{dfn:equivalence-class}として公理的に構成することでわかる.
\end{proof}

\begin{definition}[閉包]
    \label{dfn:closure-by-sequence}
    $(X, d)$を距離空間,$\symcal{D} \subset X$を部分集合とする.
    $\symcal{D}$に属するすべての収束列の極限からなる集合を
    $\symcal{D}$の\word{閉包}[へい|ほう](closure)といい,$\cl{\symcal{D}}$とかく.
    つまり
    \begin{equation}
        \cl{\symcal{D}} \coloneq \Set{ x \in X  \given  \exists (x_i) \subset \symcal{D} \text{ s.t. } \lim_{n \to \infty} d(x_i, x) = 0 }
    \end{equation}
\end{definition}

$x \in \symcal{D}$であることと$x \in \cl{\symcal{D}}$であることは,定義上は全く関係ない.

\begin{corollary}
    任意の$\symcal{D} \subset X$の閉包$\cl{\symcal{D}}$について,
    $\symcal{D} \subset \cl{\symcal{D}}$である.
\end{corollary}

\begin{proof}
    $x \in \symcal{D}$とする.
    $\symcal{D}$の点列$(x, x, x, \dotsc)$は明らかに$x$に収束するので$x \in \cl{\symcal{D}}$である.
\end{proof}

次に稠密を定義する.

\begin{definition}[稠密]
    $(X, d)$を距離空間とする.部分集合$A \subset X$の閉包$\cl{A}$が$X$に一致するとき,
    $A$は$X$の\word{稠密}[ちゅう|みつ][ちゆうみつ](dense)な部分集合であるという.
\end{definition}

\begin{corollary}
    $A$が$X$の稠密な部分集合である必要十分条件は,
    $x \in X$の任意の近傍が$A$と共通部分を持つことである.
\end{corollary}

\begin{proof}
    
\end{proof}



\subsection{ヒルベルト空間}

ベクトル空間における完備性について考えてみよう.

$V$をノルム空間\refdfn{dfn:norm}とする.
$\symbf{x}, \symbf{y} \in V$の距離\refdfn{dfn:distance}は,
$d(\symbf{x}, \symbf{y}) \coloneq \norm{\symbf{x} - \symbf{y}}$で定義できる(\cref{thm:norm-is-distance}).
すると,$V$の収束列\refdfn{dfn:convergent-sequence}とコーシー列\refdfn{dfn:Cauchy-sequence}は,以下のように定義できる.
\begin{itemize}
    \item $V$の点列$(\symbf{v}_i)$が収束列であるとは,$(\symbf{v}_i)$がある$\symbf{v} \in V$に収束する,
        つまり$\lim_{n \to \infty} x_n = x \in V$となることをいう.
    \item $V$の点列$(\symbf{v}_i)$がコーシー列であるとは,
        $\lim_{n, m \to \infty} = 0$であることをいう.
\end{itemize}
これらを用いて,ノルム空間の完備性を定義することができる.

\begin{definition}
    ノルム空間$V$が完備であるとは,$V$のコーシー列が収束列であることをいう.
\end{definition}

\begin{definition}[バナッハ空間]
    \label{dfn:Banach-space}
    完備なノルム空間を\word{バナッハ空間}(Banach space)
\end{definition}

内積空間\refdfn{dfn:inner-product}では,内積から導かれるノルムが存在するのであった.
このノルムを用いれば,完備な内積空間というものを定義することができる.

\begin{definition}[ヒルベルト空間]
    \label{dfn:Hilbert-space}
    完備な内積空間を\word{ヒルベルト空間}(Hilbert space)という.
\end{definition}

\begin{proposition}
    実ユークリッド空間$\symbb{R}^n$および複素ユークリッド空間$\symbb{C}^n$はヒルベルト空間である.
\end{proposition}



\section{関数空間}

関数とは,$\symbb{R} \text{ or } \symbb{C}$から$\symbb{R} \text{ or } \symbb{C}$への写像をいうのであった.

これからの議論では$\symbb{C}$から$\symbb{C}$への関数を扱うが,どちらの$\symbb{C}$を$\symbb{R}$もしくは(良い)部分集合$X \subset \symbb{C}$に置き換えてもよい.

\subsection{ベクトル空間としての関数空間}

\begin{proposition}
    \label{thm:function-space-is-vector-space}
    $\symbb{C}$から$\symbb{C}$への関数全体の集合は,以下で定義される和$\plushat$とスカラー倍$\hat{\scaprod}$のもとで,体$\symbb{K}$上のベクトル空間になる.
    \begin{itemize}
        \item $f, g$を,$c \in \symbb{K}$とする.
        \begin{enumerate}
            \item 関数の和$f \plushat g$は,
                任意の$x \in \symbb{C}$に対して$ ( f \plushat g )(\symbf{x}) = f(x) \doubleplus g(x) $である関数と定める.
            \item 関数のスカラー倍$c \mathbin{\hat{\scaprod}} f$は,
                任意の$x \in \symbb{C}$に対して$ ( c \mathbin{\hat{\scaprod}} g )(x) = c \scaprodvar f(x) $である関数と定める.
        \end{enumerate}
    \end{itemize}
\end{proposition}

\begin{proof}
    \refdfn{dfn:vector-space}の条件を満たすことを示せばよい.
\end{proof}



\subsection{内積空間としての関数空間}

\begin{definition}
    \label{dfn:square-integrable-function-space}
    $[a, b] \subset \symbb{R}$から$\symbb{C}$への\word{2乗可積分関数空間}(square integrable function space)$L^2$を
    \begin{equation*}
        L^2 (a, b) \coloneq \Set*{ f \colon [a, b] \to \symbb{C}  \given  \int_a^b \abs{f(x)} \dd{x} < \infty }
    \end{equation*}
    で定義する.
\end{definition}

\begin{proposition}
    2乗可積分関数空間$L^2 (a, b)$は,以下で定義される内積$(\bigdot, \bigdot)$に対して内積空間\refdfn{dfn:inner-product}になる.
    \begin{align}
        (f, g)_{L^2} \coloneq \int_a^b \conj{f}(x) g(x) \dd{x}
    \end{align}
    当然ノルム\refdfn{dfn:norm,dfn:norm-by-inner-product}も以下のように定義できる.
    \begin{align}
        \norm{f}_{L^2}^2 \coloneq (f, f)_{L^2} = \int_a^b \abs{f(x)}^2 \dd{x}
    \end{align}
\end{proposition}


\subfile{sub/linear_map.tex}


\subfile{sub/dual_space.tex}


\subfile{sub/operator.tex}



\printbibheading

\printbibliography[
    keyword=set,
    heading=subbibliography,
    title={集合論}
]
\printbibliography[
    keyword=lin, 
    heading=subbibliography, 
    title={線形代数}
]




\ifdraft
\else
    \printindex
\fi


\end{document}