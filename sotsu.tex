% \documentclass[draft]{sotsu}
\documentclass{sotsu}

\newif\ifdraft
\drafttrue
\draftfalse % これをコメントアウト(先頭に % )すると draft モードになる

\usepackage{sotsu}

\usepackage{subfiles}


\graphicspath{{./fig/}}



\addbibresource{sotsu.bib}


% 索引
\ifdraft
    \DeclareDocumentCommand{\index}{O{} m}{}
\else
    \makeindex
\fi




\begin{document}


% 目次
\tableofcontents


\chapter{数学の基礎}


\subfile{sub/set_theory.tex}

\subfile{sub/algebra.tex}

\subfile{sub/vector_space.tex}

\subfile{sub/inner_product.tex}

\subfile{sub/topology.tex}


\section{ヒルベルト空間}

\subsection{ヒルベルト空間}

ベクトル空間における完備性について考えてみよう.

$V$をノルム空間\refdfn{dfn:norm}とする.
$\symbf{x}, \symbf{y} \in V$の距離\refdfn{dfn:distance}は,
$d(\symbf{x}, \symbf{y}) \coloneq \norm{\symbf{x} - \symbf{y}}$で定義できる(\cref{thm:norm-is-distance}).
すると,$V$の収束列\refdfn{dfn:convergent-sequence}とコーシー列\refdfn{dfn:Cauchy-sequence}は,以下のように定義できる.
\begin{itemize}
    \item $V$の点列$\sequ{\symbf{v}_i}$が収束列であるとは,$\sequ{\symbf{v}_i}$がある$\symbf{v} \in V$に収束する,
        つまり$\lim_{n \to \infty} x_n = x \in V$となることをいう.
    \item $V$の点列$\sequ{\symbf{v}_i}$がコーシー列であるとは,
        $\lim_{m, n \to \infty} \norm{\symbf{v}_m - \symbf{v}_n} = 0$であることをいう.
\end{itemize}
これらを用いて,ノルム空間の完備性を定義することができる.

\begin{definition}
    ノルム空間$V$が完備であるとは,$V$のコーシー列が収束列であることをいう.
\end{definition}

\begin{definition}[バナッハ空間]
    \label{dfn:Banach-space}
    完備なノルム空間を\word{バナッハ空間}(Banach space)という。
\end{definition}

内積空間\refdfn{dfn:inner-product}では,内積から導かれるノルムが存在するのであった.
このノルムを用いれば,完備な内積空間というものを定義することができる.

\begin{definition}[ヒルベルト空間]
    \label{dfn:Hilbert-space}
    完備な内積空間を\word{ヒルベルト空間}(Hilbert space)という.
\end{definition}

\begin{proposition}
    実ユークリッド空間$\symbb{R}^n$および複素ユークリッド空間$\symbb{C}^n$はヒルベルト空間である.
\end{proposition}


\subfile{sub/linear_map.tex}


\subfile{sub/dual_space.tex}


\subfile{sub/functional_analysis.tex}


\subfile{sub/operator.tex}



\chapter{量子力学へ}

\subfile{sub/bra_ket.tex}


\ModifyHeading{section}{
    pagebreak=nariyuki,
}



\printbibheading

\printbibliography[
    keyword=set,
    heading=subbibliography,
    title={集合論}
]
\printbibliography[
    keyword=lin, 
    heading=subbibliography, 
    title={線形代数}
]
\printbibliography[
    keyword=functional, 
    heading=subbibliography, 
    title={関数解析}
]
\printbibliography[
    notkeyword=set,
    notkeyword=lin,
    notkeyword=functional,
    heading=subbibliography,
    title={その他}
]




\ifdraft
\else
    \printindex
\fi


\end{document}