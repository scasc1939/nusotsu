% \documentclass[draft]{sotsu}
\documentclass{sotsu}

\newif\ifdraft
\drafttrue
\draftfalse % これをコメントアウト(先頭に % )すると draft モードになる

\usepackage{sotsu}


\usepackage{subfiles}

\graphicspath{{./fig/}}



\addbibresource{sotsu.bib}


% \newcommand{\dotprod}{\mathbin{\mdsmwhtcircle}}
\newcommand{\dotprod}{\mathbin{\ast}}

\makeatletter

% \newcommand{\scaprod}{\mathbin{\diamondcdot}}
\newcommand{\scaprod}{\mathbin{\star}}
\newcommand{\scaprodvar}{\mathbin{\varstar}}

\newcommand{\vecplus}{\tplus}
\newcommand{\vecminus}{\tminus}

\makeatother


\ifdraft
    \DeclareDocumentCommand{\index}{O{} m}{}
\else
    \makeindex
\fi




\begin{document}


\tableofcontents


\subfile{sub/set_theory.tex}

\subfile{sub/vector_space.tex}

\subfile{sub/inner_product.tex}

\subfile{sub/topology.tex}


\section{ヒルベルト空間}

\subsection{ヒルベルト空間}

ベクトル空間における完備性について考えてみよう.

$V$をノルム空間\refdfn{dfn:norm}とする.
$\symbf{x}, \symbf{y} \in V$の距離\refdfn{dfn:distance}は,
$d(\symbf{x}, \symbf{y}) \coloneq \norm{\symbf{x} - \symbf{y}}$で定義できる(\cref{thm:norm-is-distance}).
すると,$V$の収束列\refdfn{dfn:convergent-sequence}とコーシー列\refdfn{dfn:Cauchy-sequence}は,以下のように定義できる.
\begin{itemize}
    \item $V$の点列$(\symbf{v}_i)$が収束列であるとは,$(\symbf{v}_i)$がある$\symbf{v} \in V$に収束する,
        つまり$\lim_{n \to \infty} x_n = x \in V$となることをいう.
    \item $V$の点列$(\symbf{v}_i)$がコーシー列であるとは,
        $\lim_{n, m \to \infty} = 0$であることをいう.
\end{itemize}
これらを用いて,ノルム空間の完備性を定義することができる.

\begin{definition}
    ノルム空間$V$が完備であるとは,$V$のコーシー列が収束列であることをいう.
\end{definition}

\begin{definition}[バナッハ空間]
    \label{dfn:Banach-space}
    完備なノルム空間を\word{バナッハ空間}(Banach space)
\end{definition}

内積空間\refdfn{dfn:inner-product}では,内積から導かれるノルムが存在するのであった.
このノルムを用いれば,完備な内積空間というものを定義することができる.

\begin{definition}[ヒルベルト空間]
    \label{dfn:Hilbert-space}
    完備な内積空間を\word{ヒルベルト空間}(Hilbert space)という.
\end{definition}

\begin{proposition}
    実ユークリッド空間$\symbb{R}^n$および複素ユークリッド空間$\symbb{C}^n$はヒルベルト空間である.
\end{proposition}



\section{関数空間}

関数とは,$\symbb{R} \text{ or } \symbb{C}$から$\symbb{R} \text{ or } \symbb{C}$への写像をいうのであった.

これからの議論では$\symbb{C}$から$\symbb{C}$への関数を扱うが,どちらの$\symbb{C}$を$\symbb{R}$もしくは(良い)部分集合$X \subset \symbb{C}$に置き換えてもよい.

\subsection{ベクトル空間としての関数空間}

\begin{proposition}
    \label{thm:function-space-is-vector-space}
    $\symbb{C}$から$\symbb{C}$への関数全体の集合は,以下で定義される和$\plushat$とスカラー倍$\hat{\scaprod}$のもとで,体$\symbb{K}$上のベクトル空間になる.
    \begin{itemize}
        \item $f, g$を,$c \in \symbb{K}$とする.
        \begin{enumerate}
            \item 関数の和$f \plushat g$は,
                任意の$x \in \symbb{C}$に対して$ ( f \plushat g )(\symbf{x}) = f(x) \doubleplus g(x) $である関数と定める.
            \item 関数のスカラー倍$c \mathbin{\hat{\scaprod}} f$は,
                任意の$x \in \symbb{C}$に対して$ ( c \mathbin{\hat{\scaprod}} g )(x) = c \scaprodvar f(x) $である関数と定める.
        \end{enumerate}
    \end{itemize}
\end{proposition}

\begin{proof}
    \refdfn{dfn:vector-space}の条件を満たすことを示せばよい.
\end{proof}



\subsection{内積空間としての関数空間}

\begin{definition}
    \label{dfn:class-C^k-function}
    複素関数$f$が\word{$C^k$-級関数}[][Cきゆうかんすう](class $C^k$-function)であるとは、
    $f(x)$が$\symbb{C}$上で$k$回微分可能かつ$k$次導関数が連続であることをいう。
    $f(x)$が無限回微分可能であるとき$C^\infty$-級関数であるという。

    $C^k$-級関数全体の集合を$C^k$とかく。
    特に連続関数全体の集合は$C^0$である。
\end{definition}

\begin{example}
    $e^x$、$\sin x$、$\cos x$は$C^\infty$-級関数である。
\end{example}

\begin{example}
    $C^0 \subsetneq C^1 \subsetneq C^2 \subsetneq \dotsb$であり、
    $\bigcap_{k \in \symbb{N}} C^k = C^\infty$である。
\end{example}


\begin{definition}
    \label{dfn:square-integrable-function-space}
    $[a, b] \subset \symbb{R}$から$\symbb{C}$への\word{2乗可積分関数空間}(square integrable function space) $L^2$を
    \begin{equation*}
        L^2 (a, b) \coloneq \Set*{ f \colon [a, b] \to \symbb{C}  \given  \int_a^b \abs{f(x)} \dd{x} < \infty }
    \end{equation*}
    で定義する.
\end{definition}

\begin{proposition}
    2乗可積分関数空間$L^2 (a, b)$は,以下で定義される内積$(\bigdot, \bigdot)$に対して内積空間\refdfn{dfn:inner-product}になる.
    \begin{align}
        (f, g)_{L^2} \coloneq \int_a^b \conj{f}(x) g(x) \dd{x}
    \end{align}
    当然ノルム\refdfn{dfn:norm,dfn:norm-by-inner-product}も以下のように定義できる.
    \begin{align}
        \norm{f}_{L^2}^2 \coloneq (f, f)_{L^2} = \int_a^b \abs{f(x)}^2 \dd{x}
    \end{align}
\end{proposition}

\begin{definition}
    連続な2乗可積分関数空間を
    \( L^2 C \coloneq L^2 \cap C^0 \)
    で定める。
\end{definition}






\subfile{sub/linear_map.tex}


\subfile{sub/dual_space.tex}


\subfile{sub/operator.tex}



\printbibheading

\printbibliography[
    keyword=set,
    heading=subbibliography,
    title={集合論}
]
\printbibliography[
    keyword=lin, 
    heading=subbibliography, 
    title={線形代数}
]




\ifdraft
\else
    \printindex
\fi


\end{document}