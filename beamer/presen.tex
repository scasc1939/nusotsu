\documentclass[
    10pt,
    % handout
    ]{sotsu-beamer}

\usepackage{../sotsu-symb}

\usepackage{multicol}

\setmathfont{STIX Two Math}
\setmathfont{XCharterMath}[range=\symscr]

\definecolor{red}{RGB}{255,120,30}
\definecolor{blue}{RGB}{30,120,255}

\newcommand{\upspin}{\mathcolor{red}{\ket{\uparrow}}}
\newcommand{\dwspin}{\mathcolor{blue}{\ket{\downarrow}}}


\begin{document}


\title{デモ}
\date{\number\year 年\number\month 月\number\day 日}


\begin{frame}

    \maketitle

\end{frame}


\section{なぜ線形代数?}


\begin{frame}{量子力学の基本方程式}

    \structure{量子力学の基本方程式}といえば?

    \pause

    \begin{block}{シュレーディンガー方程式}
        \begin{equation}
            \label{eq:time-undependent-Schroedinger-equation}
            \ab[ \frac{\hbar^2}{2m} \ab( \pdv[2]{}{x} + \pdv[2]{}{y} + \pdv[2]{}{z} ) + V(\symbf{r})] \psi(\symbf{r}) = E \psi(\symbf{r})
        \end{equation}
    \end{block}

    \pause
    
    \eqref{eq:time-undependent-Schroedinger-equation}の解を$\psi(\symbf{r}), \varphi(\symbf{r})$とすると,
    任意の複素数$a, b$に対して,
    \[  \xi(\symbf{r}) \coloneq a \psi(\symbf{r}) + b \varphi(\symbf{r}) 
        \quad \text{($a, b$は複素数)}  \]
    は\eqref{eq:time-undependent-Schroedinger-equation}の解になっている.

    \pause

    $\implies$\alert{\eqref{eq:time-undependent-Schroedinger-equation}は線型方程式}.

\end{frame}



\begin{frame}
    \label{frame:psi-is-vector}
    \frametitle{数学的には・・・}

    \begin{center}
        方程式\eqref{eq:time-undependent-Schroedinger-equation}の解$\psi(\symbf{r})$全体は,
        \structure{ベクトル空間}をなす.
    \end{center}

\end{frame}


\section{ベクトル空間って?}


\begin{frame}
    \frametitle{ユークリッド空間の例}

    3次元ユークリッド空間$\symbb{R}^3$の縦ベクトル
    \begin{equation*}
        \symbf{x} = 
        \begin{pmatrix}
            x_1 \\ x_2  \\ x_3
        \end{pmatrix}
        \quad
        \text{($x_1, x_2, x_3$は複素数)}
    \end{equation*}
    とする.

    このとき,
    \begin{equation*}
        a \symbf{x} + b \symbf{y}
        = \begin{pmatrix}
            a x_1 + b y_1  \\
            a x_2 + b y_2  \\
            a x_3 + b y_3
        \end{pmatrix}
    \end{equation*}
    もベクトル.

    \pause

    $\symbf{x}$はある意味\alert{シュレーディンガー方程式の解$\symbf{\psi}$に似ている}.

\end{frame}



\begin{frame}{ベクトル空間の定義}

    ふつうの縦ベクトルについて考えれば当たり前のこと:
    \begin{block}{ベクトル空間の公理}
        $\symbf{\psi}, \symbf{\varphi}, \symbf{\xi}$を\structure{ベクトル},
        $a, b, c$を\structure{スカラー}(複素数)とする.
        \begin{enumerate}
            \begin{multicols}{2}
                \item $(\symbf{\psi} + \symbf{\varphi}) + \symbf{\xi} = \symbf{\psi} + (\symbf{\varphi} + \symbf{\xi})$
                \item $\symbf{\psi} + \symbf{0} = \symbf{0} + \symbf{\psi} = \symbf{\psi}$
                \item $\symbf{\psi} + (-\symbf{\psi}) = \symbf{0}$
                \item $\symbf{\psi} + \symbf{\varphi} = \symbf{\varphi} + \symbf{\psi}$
                \item $c (\symbf{\psi} + \symbf{\varphi}) = c \symbf{\psi} + c \symbf{\varphi}$
                \item $(a + b) \symbf{\psi} = a \symbf{\psi} + b \symbf{\psi}$
                \item $(ab) \symbf{\psi} = a(b\symbf{\psi})$
                \item $1 \symbf{\psi} = \symbf{\psi}$
            \end{multicols}
        \end{enumerate}
    \end{block}

    \pause

    \alert{逆に,1--8を満たすすものは,ぜんぶ「\structure{ベクトル}」ということにする}.
    ベクトル全体がつくる集合を\structure{ベクトル空間}という.
    
    $\implies$\ref{frame:psi-is-vector}ページで考えた$\symbf{\psi}$もベクトル.

\end{frame}


\begin{frame}
    \frametitle{ベクトルの例}

    \begin{itemize}
        \item $N$次の縦ベクトル:$(a_1 \,\, a_2 \,\, \cdots \,\, a_N)$
        \item $N$次多項式:$p(x) = a_N x^N + a_{N-1} x^{N-1} + \dots + a_1 x + a_0$
        \item 線形方程式の解:$\varphi(x_1, \dots, x_N)$
    \end{itemize}

    

\end{frame}



\begin{frame}
    \frametitle{ユークリッド内積}

    縦ベクトルがもっている重要な性質:\structure{内積}の復習.

    \begin{block}{縦ベクトルの内積}
        $\symbf{x}, \symbf{y}$:$N$次元ベクトル
        \begin{equation}
            \label{eq:Euclid-inner-product}
            \iparen{\symbf{x}, \symbf{y}}
                \coloneq \sum_{i=1}^{N} x_i^* y_i
        \end{equation}
        を\structure{内積}という.
    \end{block}

    ベクトルの\structure{長さ}は,
    \begin{equation}
        \label{eq:Euclid-norm}
        \norm{\symbf{x}} \coloneq \sqrt{\iparen{\symbf{x}, \symbf{x}}}
    \end{equation}

    \begin{alertblock}{注意}
        \eqref{eq:Euclid-norm}の右辺の$\sqrt{\quad}$の中身$\geq 0$になるように,
        \eqref{eq:Euclid-inner-product}の$x_i$を複素共役にしてある.
    \end{alertblock}

\end{frame}

\begin{frame}
    \frametitle{内積の定義}

    \eqref{eq:Euclid-inner-product}は次の性質を満たす.

    \begin{block}{内積の公理}
        ベクトル$\symbf{\psi}, \symbf{\varphi}, \symbf{\xi}$と
        スカラー(複素数)$a, b$について
        \begin{enumerate}
            \item $\iparen{\symbf{\psi}, \  a\symbf{\varphi} + b\symbf{\xi}} = a\iparen{\symbf{\psi}, \symbf{\xi}} + b\iparen{\symbf{\psi}, \symbf{\varphi}}$(\structure{線形性})
            \item $\iparen{\symbf{\psi}, \symbf{\varphi}} = \conj{\iparen{\symbf{\varphi}, \symbf{\psi}}}$
            \item $\iparen{\symbf{\psi}, \symbf{\psi}} \geq 0$であり,
                $\iparen{\symbf{\psi}, \symbf{\psi}} = 0 \iff \symbf{\psi} = \symbf{0}$(\structure{正定値性})
        \end{enumerate}
    \end{block}

    上の条件を満たすものを改めて「\structure{内積}」と呼ぶことにする.

    \begin{alertblock}{注意}
        1と2をあわせると,
        $\iparen{a\symbf{\psi} + b\symbf{\varphi}, \  \symbf{\xi}} = \conj{a}\iparen{\symbf{\psi}, \symbf{\xi}} + \conj{b}\iparen{\symbf{\varphi}, \symbf{\xi}}$(\structure{反線形性})である.
    \end{alertblock}


\end{frame}

\begin{frame}
    \frametitle{関数の内積}

    では,関数$\psi(\symbf{r}), \varphi(\symbf{r})$の内積は?

    \pause

    \begin{equation}
        \iparen{\psi, \varphi}
            \coloneq \iiint \conj{\psi}(\symbf{r}) \varphi(\symbf{r}) \dd{x} \dd{y} \dd{z}
    \end{equation}
    とすれば,前のページの性質を満たす.

\end{frame}


\begin{frame}
    \frametitle{基底}

    すべての縦ベクトル$\symbf{x} = (x_1 \,\, x_2 \,\, x_3)$は,
    \begin{equation}
        \label{eq:Euclidean-basis}
        \symbf{e}_1 = \begin{pmatrix} 1 \\ 0 \\ 0 \end{pmatrix}, \quad 
        \symbf{e}_2 = \begin{pmatrix} 0 \\ 1 \\ 0 \end{pmatrix}, \quad 
        \symbf{e}_3 = \begin{pmatrix} 0 \\ 0 \\ 1 \end{pmatrix}
    \end{equation}
    を使って,$\symbf{x} = x_1 \symbf{e}_1 + x_2 \symbf{e}_2 + x_3 \symbf{e}_3$と書ける.

    さらに,
    \begin{math}
        x_1 \symbf{e}_1 + x_2 \symbf{e}_2 + x_3 \symbf{e}_3 = \symbf{0}
        \iff
        x_1 = x_2 = x_3 = 0
    \end{math}(\structure{一次独立}).

    \pause

    同じように,\alert{すべてのベクトル}$\symbf{\psi}$は,
    \structure{一次独立}なベクトル$\symbf{\chi}_1, \symbf{\chi}_2, \dotsc$を使って
    \begin{equation}
        \label{eq:basis}
        \symbf{\chi} = \sum_i c_i \symbf{\chi}_i
        \quad \text{($c_i$は複素数)}
    \end{equation}
    と書ける.
    \pause
    $\symbf{\chi}_i$:\alert{基底}という.

\end{frame}


\begin{frame}
    \frametitle{ベクトル空間の次元}

    \eqref{eq:basis}を満たす一次独立なベクトルの組$\symbf{\chi}_i$は,
    いくつも存在している.

    例えば,\eqref{eq:Euclidean-basis}のかわりに
    \begin{equation*}
        \symbf{e}'_1 = (1 \,\, 1 \,\, 0), \quad 
        \symbf{e}'_2 = (1 \,\, {-1} \,\, 0), \quad
        \symbf{e}'_3 = (0 \,\, 0 \,\, \sqrt{2})
    \end{equation*}
    としてもよい.

    しかし,$\symbf{\chi}_1, \symbf{\chi}_2, \dotsc$の数は同じ(上の例だと,必ず3つ).

    ベクトル空間がもつ基底の数を\alert{次元}という.


\end{frame}


\subsection{同型}


\begin{frame}
    \frametitle{同型とは}

    \begin{itemize}
        \item ベクトルの\structure{かず}が同じ
        \item \structure{線形構造}(和とスカラー倍)が同じ
    \end{itemize}
    である2つのベクトル空間は,同じものと見なしてもよい.
    
\end{frame}


\begin{frame}[allowframebreaks]
    \frametitle{同型の例}

    2次多項式$a_2 x^2 + a_1 x + a_0$のベクトル空間$\symbb{R}[x]_2$は,
    3次元縦ベクトルの空間$\symbb{R}^3$と同型.

    対応
    \begin{align*}
        p(x) = a_2 x^2 + a_1 x + a_0
        &\leftrightarrow \text{縦ベクトル} \, (a_2 \,\, a_1 \,\, a_0) = \symbf{a}
        \\
        q(x) = b_2 x^2 + b_1 x + b_0
        &\leftrightarrow \text{縦ベクトル} \, (b_2 \,\, b_1 \,\, b_0) = \symbf{b}
    \end{align*}
    とすれば,
    \begin{align*}
        p + q \leftrightarrow \symbf{a} + \symbf{b},
        \quad
        c \cdotp p \leftrightarrow c \symbf{a} \, \text{($c$:複素数)}
    \end{align*}
    線形演算がまったく同じ.


    \framebreak

    実は,基底の対応
    \begin{equation*}
        x^2 \leftrightarrow \begin{pmatrix} 1 \\ 0 \\ 0 \end{pmatrix},
        \quad 
        x \leftrightarrow \begin{pmatrix} 0 \\ 1 \\ 0 \end{pmatrix},
        \quad 
        1 \leftrightarrow \begin{pmatrix} 0 \\ 0 \\ 1 \end{pmatrix},
    \end{equation*}
    を定めるだけで,対応$a_2 x^2 + a_1 x + a_0 \leftrightarrow (a_2 \,\, a_1 \,\, a_0)$が決まる.

\end{frame}


\begin{frame}
    \frametitle{$N$次元ベクトル空間どうしは同型}

    \begin{block}{定理}
        2つの$N$次元ベクトル空間は同型である.
    \end{block}

    2つのベクトル空間の基底$\{\symbf{\chi}_i\}, \{\symbf{\chi}'_i\}$をそれぞれ
    \begin{align*}
        \symbf{\chi}_1 &\leftrightarrow \symbf{\chi}'_1, 
        \quad
        \cdots, 
        \quad
        \symbf{\chi}_N \leftrightarrow \symbf{\chi}'_N
    \end{align*}
    のように対応させれば簡単に証明できる.

\end{frame}


\subsection{表現行列}


\begin{frame}
    \frametitle{表現行列}

    多項式関数$f(x) = a_2 x^2 + a_1 x + a_0$を微分する:$\hat{D} \coloneq \dv{}{x}$.

    \pause

    もちろん,
    \begin{equation*}
        (\hat{D} f) (x) = \dv{f}{x} = 2 a_2 x + a_1
    \end{equation*}

    \pause

    ところで,対応$f \leftrightarrow (a_2 \,\, a_1 \,\, a_0)$を考えると,
    これは
    \begin{equation*}
        (\hat{D} f)
        \leftrightarrow
        \begin{pmatrix}
            0  \\  2 a_2  \\  a_1
        \end{pmatrix}
        =
        \underbrace{
        \begin{pmatrix}
            0  &  0  &  0  \\
            2  &  0  &  0  \\
            0  &  1  &  0
        \end{pmatrix}
        }_{\text{\normalsize$\symsf{D}$}}
        \begin{pmatrix}
            a_2  \\  a_1  \\  a_0
        \end{pmatrix}
        \leftrightarrow
        f
    \end{equation*}
    と行列の計算としてかける.

    \pause

    $\symsf{D}$を\alert{表現行列}という.

\end{frame}



\section{スピン系}

\begin{frame}
    \frametitle{スピン}

    1粒子のスピンは,
    $\upspin$:アップスピン,
    $\dwspin$:ダウンスピンの2次元.

    \pause

    よって,$\symbb{R}^2$と同型,
    具体的には
    \begin{equation*}
        \upspin \leftrightarrow (1 \,\, 0),
        \quad 
        \dwspin \leftrightarrow (0 \,\, 1)
    \end{equation*}
    と対応させられる.

    \pause

    スピンの重ね合わせ:
    \[  a \upspin + b \dwspin \leftrightarrow 
    \begin{pmatrix}
        a  \\  b
    \end{pmatrix}  \]
    と簡潔に表現できる.

\end{frame}


\begin{frame}
    \frametitle{スピンに作用する演算子}

    スピンに作用する演算子:
    \begin{itemize}
        \item $\hat{S}_+$……$\hat{S}_+ \upspin = \symbf{0}$,$\hat{S}_+ \dwspin = \upspin$
        \item $\hat{S}_-$……$\hat{S}_- \upspin = \dwspin$,$\hat{S}_+ \dwspin = \symbf{0}$
        \item $\hat{S}_z$……$\hat{S}_z \upspin = \frac{1}{2} \upspin$,$\hat{S}_z \dwspin = \frac{1}{2} \dwspin$
    \end{itemize}

    \pause

    行列を使えば,
    \begin{alignat*}{3}
        % S_+
        \hat{S}_+ &\colon 
        &
        \begin{pmatrix}
            0 & 1 \\
            0 & 0
        \end{pmatrix}
        \mathcolor{red}{ \begin{pmatrix} 1 \\ 0 \end{pmatrix} }
        &= 
        \mathcolor{black}{ \begin{pmatrix} 0 \\ 0 \end{pmatrix} }
        , \quad
        &
        \begin{pmatrix}
            0 & 1 \\
            0 & 0
        \end{pmatrix}
        \mathcolor{blue}{ \begin{pmatrix} 0 \\ 1 \end{pmatrix} }
        &= 
        \mathcolor{red}{ \begin{pmatrix} 1 \\ 0 \end{pmatrix} }
        \\
        %
        % S_-
        % 
        \hat{S}_- &\colon 
        &
        \begin{pmatrix}
            0 & 0 \\
            1 & 0
        \end{pmatrix}
        \mathcolor{red}{ \begin{pmatrix} 1 \\ 0 \end{pmatrix} }
        &= 
        \mathcolor{blue}{ \begin{pmatrix} 0 \\ 1 \end{pmatrix} }
        , \quad
        &
        \begin{pmatrix}
            0 & 0 \\
            1 & 0
        \end{pmatrix}
        \mathcolor{blue}{ \begin{pmatrix} 0 \\ 1 \end{pmatrix} }
        &= 
        \mathcolor{black}{ \begin{pmatrix} 0 \\ 0 \end{pmatrix} }
        \\
        %
        % S_z
        %
        \hat{S}_z &\colon 
        &
        \begin{pmatrix}
            \frac{1}{2} & 0 \\
            0 & \frac{1}{2}
        \end{pmatrix}
        \mathcolor{red}{ \begin{pmatrix} 1 \\ 0 \end{pmatrix} }
        &= 
        \frac{1}{2}
        \mathcolor{red}{ \begin{pmatrix} 1 \\ 0 \end{pmatrix} }
        , \quad
        &
        \begin{pmatrix}
            \frac{1}{2} & 0 \\
            0 & \frac{1}{2}
        \end{pmatrix}
        \mathcolor{blue}{ \begin{pmatrix} 0 \\ 1 \end{pmatrix} }
        &= 
        \frac{1}{2}
        \mathcolor{blue}{ \begin{pmatrix} 0 \\ 1 \end{pmatrix} }
    \end{alignat*}
    

\end{frame}



\begin{frame}
    \frametitle{多粒子系の扱い方:テンソル積}

    $\ket{\psi} = (a \,\, b)$, $\ket{\varphi} = (c \,\, d)$とする.

    2状態をあわせた$\ket{\psi}\ket{\varphi}$はどう表現する?

    \pause

    答え:\alert{テンソル積}
    \begin{equation}
        \ket{\psi} \otimes \ket{\varphi}
            \coloneq \begin{pmatrix}
                a \ket{\varphi}  \\  b \ket{\varphi}
            \end{pmatrix}
            = \begin{pmatrix}
                ac \\ ad \\ bc \\ bd
            \end{pmatrix}
    \end{equation}

    \pause

    3つ以上のテンソル積も,
    $\ket{\psi} \otimes \bigl( \ket{\varphi} \otimes \ket{\xi} \bigr)$のようにして定義できる.

\end{frame}


\begin{frame}
    \frametitle{行列のテンソル積}

    行列のテンソル積も,
    \begin{equation*}
        \begin{pmatrix}
            a & b \\ c & d
        \end{pmatrix}
        \otimes 
        \underbrace{
        \begin{pmatrix}
            \alpha & \beta \\ \gamma & \delta
        \end{pmatrix}
        }_{\text{\normalsize$\symsf{B}$}}
        = 
        \begin{pmatrix}
            a \symsf{B} & b \symsf{B} \\
            c \symsf{B} & d \symsf{B}
        \end{pmatrix}
        =
        \begin{pmatrix}
            a\alpha & a\beta & b\alpha & b\beta \\
            a\gamma & a\delta & b\gamma & b\delta \\
            c\alpha & c\beta & d\alpha & d\beta \\
            c\gamma & c\delta & d\gamma & d\delta
        \end{pmatrix}
    \end{equation*}

    さらに,2スピン系$\ket{\psi} \otimes \ket{\varphi}$に作用する演算子は,
    それぞれのスピンに作用する演算子$\symcal{O}_1, \symcal{O}_2$を使うと
    \begin{equation*}
        \ab(\hat{\symcal{O}}_1 \otimes \hat{\symcal{O}}_2)
        \ab(\ket{\psi} \otimes \ket{\varphi})
        = \hat{\symcal{O}}_1 \ket{\psi}
          \otimes
          \hat{\symcal{O}}_2 \ket{\varphi}
    \end{equation*}

\end{frame}




\end{document}