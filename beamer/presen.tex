\documentclass[
    10pt,
    % handout
    ]{sotsu-beamer}

\usepackage{../sotsu-symb}

\usepackage{multicol}

\setmathfont{STIX Two Math}
\setmathfont{XCharterMath}[range={\symcal,\symscr}]

\definecolor{fire}{RGB}{255,120,30}
\definecolor{water}{RGB}{30,120,255}

\newcommand{\bitone}{\mathcolor{fire}{\uparrow}}
\newcommand{\bittwo}{\mathcolor{water}{\downarrow}}


\begin{document}


\title{手作りエンタングル}
\date{2025年3月5日(\number\year 年\number\month 月\number\day 日更新)}


\begin{frame}

    \maketitle

\end{frame}


\begin{frame}[allowframebreaks]{目次}

    \tableofcontents

\end{frame}


\section{なぜ線形代数?}


\begin{frame}{量子力学の基本方程式}

    \structure{量子力学の基本方程式}といえば?

    \pause

    \begin{block}{(時間に依存しない)シュレーディンガー方程式}
        \begin{equation}
            \label{eq:time-undependent-Schroedinger-equation}
            \ab[ \frac{\hbar^2}{2m} \ab( \pdv[2]{}{x} + \pdv[2]{}{y} + \pdv[2]{}{z} ) + V(\symbf{r})] \psi(\symbf{r}) = E \psi(\symbf{r})
        \end{equation}
    \end{block}

    \pause
    
    \eqref{eq:time-undependent-Schroedinger-equation}の解を$\psi(\symbf{r}), \xi(\symbf{r})$とすると,
    任意の複素数$a, b$に対して,
    \[  \eta(\symbf{r}) \coloneq a \psi(\symbf{r}) + b \xi(\symbf{r}) 
        \quad \text{($a, b$は複素数)}  \]
    は\eqref{eq:time-undependent-Schroedinger-equation}の解になっている.

    \pause

    $\implies$\alert{\eqref{eq:time-undependent-Schroedinger-equation}は線型方程式}.

\end{frame}



\begin{frame}
    \label{frame:psi-is-vector}
    \frametitle{数学的には・・・}

    \begin{center}
        方程式\eqref{eq:time-undependent-Schroedinger-equation}の解$\psi(\symbf{r})$全体は,
        \structure{ベクトル空間}をなす.
    \end{center}

\end{frame}


\section{ヒルベルト空間って?}

\begin{frame}[label={frame:definition-of-Hilbert-space}]
    \frametitle{ヒルベルト空間}

    \only<1-2>{
        状態はヒルベルト空間のベクトルとしてあらわされる.

        ではヒルベルト空間とは何か?
    }

    \pause

    \begin{block}{ヒルベルト空間の定義}
        \structure{内積}をもつ\structure{ベクトル空間}であって,
        内積から導かれる\structure{ノルム}に関して\structure{完備}なものを\structure{ヒルベルト空間}という.
    \end{block}

\end{frame}


\subsection{ベクトル}

\begin{frame}
    \frametitle{ベクトルの性質}

    そもそもベクトルとは?

    ベクトルの性質として
    \begin{itemize}
        \item ベクトル同士の和\( \symbf{x} + \symbf{y} \)
        \item ベクトルの複素数倍(スカラー倍)\( c \symbf{x} \)
    \end{itemize}
    の2つ(\structure{線形演算})が定義できる.

    →2つの計算を合わせた\alert{線形結合$a \symbf{x} + b \symbf{y}$}が基本.

\end{frame}

\begin{frame}
    \frametitle{ユークリッド空間の例}

    3次元ユークリッド空間$\symbb{R}^3$の縦ベクトル
    \begin{equation*}
        \symbf{x} = 
        \begin{pmatrix}
            x_1 \\ x_2  \\ x_3
        \end{pmatrix}
        \quad
        \text{($x_1, x_2, x_3$は複素数)}
    \end{equation*}
    とする.

    このとき,
    \begin{equation*}
        a \symbf{x} + b \symbf{y}
        = \begin{pmatrix}
            a x_1 + b y_1  \\
            a x_2 + b y_2  \\
            a x_3 + b y_3
        \end{pmatrix}
    \end{equation*}
    もベクトル.

    \pause

    $\symbf{x}$はある意味\alert{シュレーディンガー方程式の解$\symbf{\psi}$に似ている}.

\end{frame}



\begin{frame}
    [label={frame:definition-of-vector-space}]
    \frametitle{ベクトル空間の定義}
    
    ふつうの縦ベクトルについて考えれば当たり前のこと:
    \begin{block}{ベクトル空間の公理}
        集合$V$の元$\symbf{\psi}, \symbf{\varphi}, \symbf{\xi}$を\structure{ベクトル},
        $a, b, c$を\structure{スカラー}(複素数)とする.
        \begin{enumerate}
            \begin{multicols}{2}
                \item $(\symbf{\psi} + \symbf{\varphi}) + \symbf{\xi} = \symbf{\psi} + (\symbf{\varphi} + \symbf{\xi})$
                \item $\symbf{\psi} + \symbf{0} = \symbf{0} + \symbf{\psi} = \symbf{\psi}$
                \item $\symbf{\psi} + (-\symbf{\psi}) = \symbf{0}$
                \item $\symbf{\psi} + \symbf{\varphi} = \symbf{\varphi} + \symbf{\psi}$
                \item $c (\symbf{\psi} + \symbf{\varphi}) = c \symbf{\psi} + c \symbf{\varphi}$
                \item $(a + b) \symbf{\psi} = a \symbf{\psi} + b \symbf{\psi}$
                \item $(ab) \symbf{\psi} = a(b\symbf{\psi})$
                \item $1 \symbf{\psi} = \symbf{\psi}$
            \end{multicols}
        \end{enumerate}
    \end{block}

    \pause

    \alert{逆に,1--8を満たすすものは,ぜんぶ「\structure{ベクトル}」ということにする}.
    
    このとき,集合$V$を\structure{ベクトル空間}という.
    
    $\implies$\ref{frame:psi-is-vector}ページで考えた$\psi$もベクトル.

\end{frame}


\begin{frame}
    \frametitle{ベクトルの例}

    \begin{itemize}
        \item $N$次の縦ベクトル $\cvector{a_1 ; a_2 ; \cdots ; a_N}$
        \item $N$次多項式 $p(x) = a_N x^N + a_{N-1} x^{N-1} + \dots + a_1 x + a_0$
        \item 線形方程式の解 $\varphi(x_1, \dots, x_N)$
            ←\alert{シュレーディンガー方程式の解!}
    \end{itemize}
    これらはすべてベクトル空間の公理を満たすベクトル.
\end{frame}


\begin{frame}
    \frametitle{ベクトル空間の基底と次元}

    このように定義されたベクトル空間には必ず\structure{基底}と呼ばれるベクトルの組が存在する.
    
    ベクトル空間$V$の基底を構成するベクトルの数を\structure{次元}という.

\end{frame}


\subsection{内積}

\begin{frame}
    \frametitle{なぜ内積が必要か?}

    ベクトル空間の定義(\ref{frame:definition-of-vector-space}ページ)だけでは,
    \begin{itemize}
        \item ベクトル$\symbf{\psi}$の\structure{長さ}
        \item ベクトル$\symbf{\psi}$と$\symbf{\xi}$が\structure{直交}すること
    \end{itemize}
    (\structure{幾何的な性質})を定義できない.

    \pause

    つまり,\alert{正規直交基底(\structure{完全正規直交系})が定義できない!}

    →ベクトル空間に幾何的な性質を導入しよう.

    % \begin{block}{(参考)位相的性質}
    %     このセクションでの議論は,
    %     ベクトル空間に位相的(topological)な性質を入れることに対応する.
    % \end{block}

\end{frame}

\begin{frame}
    \frametitle{ユークリッド内積}

    縦ベクトルがもっている重要な性質:\structure{内積}の復習.

    \pause 

    \begin{block}{縦ベクトルの内積}
        $\symbf{x}, \symbf{y}$:$N$次元ベクトル
        \begin{equation}
            \label{eq:Euclid-inner-product}
            \iparen{\symbf{x}, \symbf{y}}
                \coloneq \sum_{i=1}^{N} x_i^* y_i
        \end{equation}
        を\structure{内積}という.
    \end{block}

    ベクトルの長さ(\structure{ノルム})は,
    \begin{equation}
        \label{eq:Euclid-norm}
        \norm{\symbf{x}} \coloneq \sqrt{\iparen{\symbf{x}, \symbf{x}}}
            = \sum_{i = 1}^{N} \abs{x_i}^2
    \end{equation}

    \begin{alertblock}{注意}
        \eqref{eq:Euclid-norm}の右辺の$\sqrt{\quad}$の中身$\geq 0$になるように,
        \eqref{eq:Euclid-inner-product}の$x_i$を複素共役にしてある.
    \end{alertblock}

\end{frame}


\begin{frame}[label={frame:definition-of-inner-product}]
    \frametitle{内積の定義}

    \only<1>{
        \eqref{eq:Euclid-inner-product}は次の性質を満たす.
    }

    \begin{block}{内積の公理}
        ベクトル$\symbf{\psi}, \symbf{\varphi}, \symbf{\xi}$と
        スカラー(複素数)$a, b$について
        \begin{enumerate}
            \item $\iparen{\symbf{\psi}, \  a\symbf{\varphi} + b\symbf{\xi}} = a\iparen{\symbf{\psi}, \symbf{\xi}} + b\iparen{\symbf{\psi}, \symbf{\varphi}}$(\structure{線形性})
            \item $\iparen{\symbf{\psi}, \symbf{\varphi}} = \conj{\iparen{\symbf{\varphi}, \symbf{\psi}}}$
            \item $\iparen{\symbf{\psi}, \symbf{\psi}} \geq 0$であり,
                $\iparen{\symbf{\psi}, \symbf{\psi}} = 0 \iff \symbf{\psi} = \symbf{0}$(\structure{正定値性})
        \end{enumerate}
    \end{block}

    \pause

    上の条件を満たす計算$\iparen{\bigdot, \bigdot}$を改めて「\structure{内積}」と呼ぶことにする.

    内積をもつベクトル空間を,\structure{内積空間}と呼ぶ.

    \pause

    \begin{alertblock}{注意}
        1と2をあわせると,
        $\iparen{a\symbf{\psi} + b\symbf{\varphi}, \  \symbf{\xi}} = \conj{a}\iparen{\symbf{\psi}, \symbf{\xi}} + \conj{b}\iparen{\symbf{\varphi}, \symbf{\xi}}$(\structure{反線形性})である.
    \end{alertblock}


\end{frame}


\begin{frame}
    \frametitle{正規直交基底}

    内積空間はベクトル空間なので,
    基底をもつ.

    特に,内積空間では\structure{正規直交基底}という良い基底をとれる(ことがある):

    基底$\sequ{\symbf{\varphi}_i}$が正規直交基底である条件:
    \begin{equation*}
        \iparen{ \symbf{\varphi}_i, \symbf{\varphi}_j }
        = \begin{cases}
            1  &  i = j  \\
            0  &  i \neq j  
        \end{cases}
    \end{equation*}

\end{frame}


\begin{frame}
    \frametitle{関数の内積}

    シュレーディンガー方程式の解はベクトル$\psi(\symbf{r})$(関数).

    では,関数$\psi(\symbf{r})$と$\xi(\symbf{r})$の内積は?

    \pause

    \begin{equation}
        \iparen{\psi, \xi}
            \coloneq \iiint \conj{\psi}(\symbf{r}) \, \xi(\symbf{r}) \dd{\symbf{r}}
    \end{equation}
    とすれば,前のページの性質を満たす.

\end{frame}


\subsection{完備性}


\begin{frame}
    \frametitle{内積空間の完備性}

    「収束しそうなベクトルの列」の極限が存在する内積空間を,
    \structure{完備}であるという.

    \alert{有限次元の内積空間はすべて完備である}から,
    あまり気にしなくてよい.

    参考までに,
    ベクトルの列が収束することの定義は以下のとおり.
    \begin{block}{(参考)ベクトルの列の収束}
        ベクトルの列$\symbf{\psi}_1, \symbf{\psi}_2, \dotsc$が$\symbf{\psi}_*$に収束するとは,
        \begin{equation*}
            \lim_{n \to \infty} \norm{ \symbf{\psi}_* - \symbf{\psi}_n } = 0
        \end{equation*}
        となることをいう.
    \end{block}

\end{frame}


\begin{frame}
    \frametitle{(参考)完備性の定義}

    内積空間$V$に属するベクトルの列$\symbf{\psi}_1, \symbf{\psi}_2, \dotsc$が\structure{コーシー列}であるとは,
    \begin{equation*}
        \lim_{m, n \to \infty} \norm{\symbf{\psi}_m - \symbf{\psi}_n} = 0
    \end{equation*}

    コーシー列は,番号が大きくなるにつれて1点に収束するように見える.
    その1点が$V$に含まれているかが問題である.

    $V$のすべてのコーシー列が収束する,つまり
    \begin{equation*}
        \lim_{n \to \infty} \norm{ \symbf{\psi}_* - \symbf{\psi}_n } = 0
    \end{equation*}
    となる$V$のベクトル$\symbf{\psi}_*$が存在するとき,
    $V$は\structure{完備}であるという.

\end{frame}


\subsection{ヒルベルト空間}

\againframe<3>{frame:definition-of-Hilbert-space}




\section{ヒルベルト空間から\texorpdfstring{$\symbb{C}^N$}{ℂ\textasciicircum 𝑁}へ}

\subsection{同型}


\begin{frame}
    \frametitle{同型(どうけい)とは:ベクトル空間の同型}

    \ref{frame:definition-of-vector-space}ページのベクトル空間の定義に出てくるのは3つだけ:
    \begin{itemize}
        \item ベクトル($\symbf{\psi}, \symbf{\xi}, \dotsc$)
        \item ベクトルの和($+$)
        \item スカラー倍($\cdotp$)
    \end{itemize}

    \pause

    よって,2つのベクトル空間の
    \begin{itemize}
        \item それぞれのベクトルが\structure{1対1に対応}
        \item \structure{線形構造}(和とスカラー倍)が同じ
    \end{itemize}
    なら,2つは(ある意味)同じ空間と見なせる.

    このとき,2つは(ベクトル空間として)\structure{同型}{\small (どうけい)}であるという.
    
\end{frame}


\begin{frame}
    \frametitle{同型とは:線形構造}

    \begin{block}{(参考)線形構造が同じ?}
        ベクトル空間$V$(上)と$W$(下)の1対1対応について,
        \begin{align*}
            \begin{array}{ccccl}
                \symbf{\psi} &+ &\symbf{\xi} &= &\symbf{\eta},
                \\
                \updownarrow &  &\updownarrow &  &\Updownarrow 
                \\
                \symbf{x} &+ &\symbf{y} &= &\symbf{z}
            \end{array}
            \qquad 
            \begin{array}{cccl}
                c &\symbf{\psi} &= &\symbf{\psi}'
                \\
                \rotatebox[origin=c]{90}{$=$} &\updownarrow &  &\Updownarrow
                \\
                c &\symbf{x} &= &\symbf{x}'
            \end{array}
        \end{align*}
        対応$\leftrightarrow$が成立すれば,対応$\Leftrightarrow$が必ず成立する.
    \end{block}

\end{frame}



\begin{frame}
    \frametitle{同型とは:内積空間の同型}

    2つの内積空間が,
    ベクトル空間として同型であり,さらに
    \begin{itemize}
        \item \structure{内積構造}が同じ
    \end{itemize}
    なら,内積空間として同型であるといえる.

\end{frame}



\begin{frame}
    [label={frame:finite-dimensional-vector-space-isomorphic}]
    \frametitle{$N$次元ベクトル空間どうしは同型}

    \begin{block}{定理}
        2つの$N$次元内積空間(or ベクトル空間)は同型である.
    \end{block}

    空間$V$の(正規直交)基底$\{\symbf{\varphi}_i\}$,$W$の(正規直交)基底$\{\symbf{\varphi}'_i\}$をそれぞれ
    \begin{align*}
        \symbf{\varphi}_1 &\leftrightarrow \symbf{\varphi}'_1, 
        \quad
        \cdots, 
        \quad
        \symbf{\varphi}_N \leftrightarrow \symbf{\varphi}'_N
    \end{align*}
    のように対応させれば簡単に証明できる.

    \pause

    \begin{alertblock}{同型の意味}
        結局,\structure{$N$次元ヒルベルト空間は実質1つだけ!}
    \end{alertblock}

    $\implies$一番簡単な$\symbb{C}^N$を考えればよい.

\end{frame}



\subsection{表現行列}

\begin{frame}
    \frametitle{表現行列の導入}

    多項式関数$f(x) = a_2 x^2 + a_1 x + a_0$を微分する線形演算子$\hat{D} \coloneq \dv{}{x}$.

    もちろん,
    \begin{equation*}
        (\hat{D} f) (x) = \dv{f}{x} = 2 a_2 x + a_1
    \end{equation*}

    \pause

    ところで,対応$f \leftrightarrow (a_2 \,\, a_1 \,\, a_0)$を考えると,
    これは次のように書ける
    \begin{equation*}
        (\hat{D} f)
        \leftrightarrow
        \begin{pmatrix}
            0  \\  2 a_2  \\  a_1
        \end{pmatrix}
        =
        \underbrace{
        \begin{pmatrix}
            0  &  0  &  0  \\
            2  &  0  &  0  \\
            0  &  1  &  0
        \end{pmatrix}
        }_{\text{\normalsize$\symsf{D}$}}
        \begin{pmatrix}
            a_2  \\  a_1  \\  a_0
        \end{pmatrix}
    \end{equation*}

    \pause

    $\symsf{D}$を\structure{表現行列}という.

\end{frame}


\begin{frame}
    \frametitle{表現行列の形式的な定義}

    形式的には次のとおり.

    $N$次元ベクトル空間$V$から$V$自身への線形写像(\structure{線形演算子})$f$について,
    \begin{enumerate}
        \item ベクトル空間$V$は,$\symbb{C}^N$と同型.
        \item $V$の基底$\symbf{\varphi}_i$は,$\symbb{C}^N$の基底$\symbf{e}_i = \cvector{0 ; \cdots ; 1 ; \cdots ; 0}$と対応.
        \item $V$上の線形写像$f$は,$\symbb{C}^N$上の線形写像$A$に対応.
        \item $\symbb{C}^N$上の線形写像$A$は,行列$\symsf{A}$に対応.
    \end{enumerate}
    $\symsf{A}$を$f$の\structure{表現行列}という.

    \pause

    \begin{alertblock}{表現行列の力}
        線形写像$f$を考える代わりに,
        行列$\symsf{A}$を考えればよい!
    \end{alertblock}

\end{frame}




\section{量子力学の公理}


\begin{frame}
    \frametitle{有限次元の状態}

    ここでは量子系の\alert{有限次元である状態}について扱う.

    \bigskip

    有限次元($N$次元)ヒルベルト空間$\symscr{H}$は$\symbb{C}^N$と同型だった(\ref{frame:finite-dimensional-vector-space-isomorphic}ページ).
    
    よって,$\symscr{H} = \symbb{C}^N$上のベクトルと線形写像を使って議論すればよい.

    つまり,\alert{数ベクトルと行列}を使って議論すればよい.

\end{frame}


\subsection{量子系の状態}

\begin{frame}
    \frametitle{量子系の状態とは?}

    量子力学的な系の状態は,
    次のような公理によって定義される.

    \begin{block}{公理:量子系の状態}
        量子系の状態は,
        系に対応するヒルベルト空間$\symscr{H}$のベクトル$\ket{\psi}$であらわされる.
        ただし$\ket{\psi}$は単位ベクトル:
        \begin{equation*}
            \braket{\psi}{\psi} = 1
        \end{equation*}
        である.
        また,位相のみが異なる2つのベクトル
        \begin{equation*}
            \ket{\psi'}, \  e^{\iu \theta} \ket{\psi}
            \quad \text{($\theta$は実数)}
        \end{equation*}
        は同じ状態とみなす.
    \end{block}

\end{frame}

\subsection{測定とは?}

\begin{frame}
    \frametitle{オブザーバブルとエルミート行列}

    観測可能な物理量のことを\structure{オブザーバブル}という.

    \begin{block}{公理:オブザーバブルとエルミート行列}
        オブザーバブルには,対応する$\symscr{H}$上の自己共役演算子(\structure{エルミート行列})$\symsf{A}$が存在する.
    \end{block}

    $\symsf{A}$の固有ベクトル$\ket{\varphi_i}$は$\symscr{H}$の正規直交基底になっている{\small ($\because$エルミート行列の性質)}.

    よって,$\symscr{H}$のすべての状態$\ket{\psi}$は,
    \begin{equation}
        \label{eq:expansion-of-eigenvector}
        \ket{\psi} = \sum_i c_i \ket{\varphi_i}
    \end{equation}
    と展開できる.

\end{frame}


\begin{frame}
    \frametitle{オブザーバブルの測定}

    以下,\alert{$\symsf{A}$の固有ベクトルに縮退がない場合}を考える.

    \begin{block}{公理:オブザーバブルの測定}

        状態$\ket{\psi}$にある系に対してオブザーバブル$A$を観測すると,
        行列$\symsf{A}$の固有値のいずれか1つが得られる.
        
        固有値$a$が得られる確率は
        \begin{equation}
            \label{eq:probability-of-eigenvalue}
            \braket{\psi}{\hat{P}_a \psi}
        \end{equation}
        である.
        ただし$\hat{P}_a$は\structure{射影演算子}であり,
        縮退がない場合は
        \begin{equation*}
            \hat{P}_a \coloneq \ketbra{\varphi_a}{\varphi_a}
        \end{equation*}
        (ただし$\ket{\varphi_a}$は$\symsf{A}$の固有値$a$に属する固有ベクトル).
    
    \end{block}

\end{frame}


\begin{frame}
    \frametitle{オブザーバブルの測定(つづき)}

    固有値$a$が得られる確率\eqref{eq:probability-of-eigenvalue}に,
    \eqref{eq:expansion-of-eigenvector}を代入すれば,
    \begin{equation*}
        \braket{\psi}{\hat{P}_a \psi}
            = \abs{c_a}^2
    \end{equation*}
    という,なじみ深い式が得られる.
    
\end{frame}


\subsection{基底測定}


\begin{frame}
    \frametitle{オブザーバブルとエルミート行列の関係}

    オブザーバブル$A$の固有状態系$\ket{\varphi_i}$は,
    ヒルベルト空間$\symscr{H}$の正規直交基底をなす.
    \begin{equation*}
        \text{オブザーバブル$A$}
        \implies
        \text{エルミート行列$\symsf{A}$}
        \implies 
        \text{正規直交基底$\ket{\varphi_i}$}
    \end{equation*}

    \pause

    逆に,
    $\symscr{H}$の正規直交基底$\ket{\varphi_i}$に対して,
    それらを固有状態系にもつオブザーバブル$A$は存在するか?

    つまり,\alert{任意のエルミート行列$\symsf{A}$に,
    対応するオブザーバブル$A$が存在するか?}

\end{frame}


\begin{frame}
    \frametitle{基底測定}

    量子力学の公理では,
    任意のエルミート行列$\symsf{A}$に対して,
    オブザーバブル$A$が存在するとは限らない.

    しかし,ここでは次のように仮定する.

    \begin{block}{公理ではない仮定:基底測定}
        $\symcal{H}$の任意の正規直交基底$\ket{\varphi_i}$を固有状態系にもつオブザーバブル$A$が存在する.
    \end{block}

    $\symscr{H}$の正規直交基底$\ket{\varphi_i}$に対応するオブザーバブル$A$の測定を,
    (その基底に対する)\structure{基底測定}ということにする.

\end{frame}


\subsection{状態の時間発展}

\begin{frame}
    \frametitle{状態の時間発展}

    \begin{block}{公理:状態の時間発展}
        量子系の状態の時間発展は,ユニタリ行列$\symsf{U}$で表される:
        \begin{equation*}
            \ket{\psi} \to \symsf{U} \ket{\psi}
        \end{equation*}
    \end{block}

    \structure{時間発展}には,実験者による操作も含まれる.

    つまり,実験者が行える操作は(測定を除き)すべてユニタリ的である.

    \pause

    ここでは逆に,すべてのユニタリ的な操作が可能であると\alert{仮定}する.

    \begin{block}{公理ではない仮定:ユニタリ操作}
        実験者は量子系に対して,
        任意のユニタリ行列$\symsf{U}$で表される操作を行うことができる.
    \end{block}

\end{frame}


\subsection{波束の収縮}


\begin{frame}
    \frametitle{波束の収縮}

    状態$\ket{\psi}$にある量子系のオブザーバブル$A$を測定しよう.

    $\symsf{A}$の固有ベクトル系を$\ket{\varphi_i}$とする.

    \begin{block}{公理:波束の収縮}
        オブザーバブル$A$の測定により,
        固有状態$\ket{\varphi_a}$に対応する値を得たとする.
    
        この測定によって,系の状態は
        \begin{equation*}
            \ket{\psi} = \sum_i c_i \ket{\varphi_i}
                \longrightarrow \ket{\varphi_a}
        \end{equation*}
        と変化する.
        (固有ベクトルに縮退はないとした)
    \end{block}

\end{frame}


\section{多体系の量子力学}

\subsection{スピン系}

\begin{frame}
    \frametitle{スピン}

    スピン$1/2$をもつ粒子のスピン状態:
    \begin{equation*}
        \left.
            \begin{array}{c}
                \text{$\ket{\bitone}$:アップスピン}  \\
                \text{$\ket{\bittwo}$:ダウンスピン}
            \end{array}
        \right\}
        \text{ヒルベルト空間$\symscr{H}$の正規直交基底}
    \end{equation*}

    \pause

    $\symscr{H}$は2次元ヒルベルト空間なので,
    \begin{equation*}
        \ket{\bitone} \leftrightarrow \cvector{ 1 ; 0 },
        \quad 
        \ket{\bittwo} \leftrightarrow \cvector{ 0 ; 1 }
    \end{equation*}
    という対応によって,$\symbb{C}^2$と同型である(\ref{frame:finite-dimensional-vector-space-isomorphic}ページ).

    \pause

    スピンの重ね合わせ:
    \[  a \ket{\bitone} + b \ket{\bittwo} \leftrightarrow 
    \begin{pmatrix}
        a  \\  b
    \end{pmatrix}  \]
    と簡潔に表現できる.

\end{frame}


% \begin{frame}
%     \frametitle{スピンに作用する演算子}

%     スピンに作用する演算子:
%     \begin{itemize}
%         \item $\hat{S}_+$……$\hat{S}_+ \ket{\bitone} = \symbf{0}$,$\hat{S}_+ \ket{\bittwo} = \ket{\bitone}$
%         \item $\hat{S}_-$……$\hat{S}_- \ket{\bitone} = \ket{\bittwo}$,$\hat{S}_+ \ket{\bittwo} = \symbf{0}$
%         \item $\hat{S}_z$……$\hat{S}_z \ket{\bitone} = \frac{1}{2} \ket{\bitone}$,$\hat{S}_z \ket{\bittwo} = \frac{1}{2} \ket{\bittwo}$
%     \end{itemize}

%     \pause

%     行列を使えば,
%     \begin{alignat*}{3}
%         % S_+
%         \hat{S}_+ &\colon 
%         &
%         \begin{pmatrix}
%             0 & 1 \\
%             0 & 0
%         \end{pmatrix}
%         \mathcolor{fire}{ \begin{pmatrix} 1 \\ 0 \end{pmatrix} }
%         &= 
%         \mathcolor{black}{ \begin{pmatrix} 0 \\ 0 \end{pmatrix} }
%         , \quad
%         &
%         \begin{pmatrix}
%             0 & 1 \\
%             0 & 0
%         \end{pmatrix}
%         \mathcolor{water}{ \begin{pmatrix} 0 \\ 1 \end{pmatrix} }
%         &= 
%         \mathcolor{fire}{ \begin{pmatrix} 1 \\ 0 \end{pmatrix} }
%         \\
%         %
%         % S_-
%         % 
%         \hat{S}_- &\colon 
%         &
%         \begin{pmatrix}
%             0 & 0 \\
%             1 & 0
%         \end{pmatrix}
%         \mathcolor{fire}{ \begin{pmatrix} 1 \\ 0 \end{pmatrix} }
%         &= 
%         \mathcolor{water}{ \begin{pmatrix} 0 \\ 1 \end{pmatrix} }
%         , \quad
%         &
%         \begin{pmatrix}
%             0 & 0 \\
%             1 & 0
%         \end{pmatrix}
%         \mathcolor{water}{ \begin{pmatrix} 0 \\ 1 \end{pmatrix} }
%         &= 
%         \mathcolor{black}{ \begin{pmatrix} 0 \\ 0 \end{pmatrix} }
%         \\
%         %
%         % S_z
%         %
%         \hat{S}_z &\colon 
%         &
%         \begin{pmatrix}
%             \frac{1}{2} & 0 \\
%             0 & \frac{1}{2}
%         \end{pmatrix}
%         \mathcolor{fire}{ \begin{pmatrix} 1 \\ 0 \end{pmatrix} }
%         &= 
%         \frac{1}{2}
%         \mathcolor{fire}{ \begin{pmatrix} 1 \\ 0 \end{pmatrix} }
%         , \quad
%         &
%         \begin{pmatrix}
%             \frac{1}{2} & 0 \\
%             0 & \frac{1}{2}
%         \end{pmatrix}
%         \mathcolor{water}{ \begin{pmatrix} 0 \\ 1 \end{pmatrix} }
%         &= 
%         \frac{1}{2}
%         \mathcolor{water}{ \begin{pmatrix} 0 \\ 1 \end{pmatrix} }
%     \end{alignat*}
    

% \end{frame}



\subsection{テンソル積}

\begin{frame}
    \frametitle{多粒子系の扱い方:テンソル積}

    スピン$1/2$を持つ2粒子の状態$\ket{\psi} = \cvector{a ; b}$,$\ket{\xi} = \cvector{c ; d}$とする.

    2状態をあわせた$\ket{\psi}\ket{\xi}$はどう表現する?

    \pause

    答え:\alert{テンソル積}
    \begin{equation}
        \ket{\psi} \otimes \ket{\xi}
            \coloneq \begin{pmatrix}
                a \ket{\xi}  \\  b \ket{\xi}
            \end{pmatrix}
            = \begin{pmatrix}
                ac \\ ad \\ bc \\ bd
            \end{pmatrix}
    \end{equation}

    \pause

    3つ以上のテンソル積も,
    $\ket{\psi} \otimes \bigl( \ket{\xi} \otimes \ket{\xi} \bigr)$のようにして定義できる.

\end{frame}


\begin{frame}
    \frametitle{テンソル積の空間}

    テンソル積空間$\symbb{C}^2 \otimes \symbb{C}^2$の基底は,
    たとえば
    \begin{equation*}
        \begin{pmatrix}
            1  \\  0
        \end{pmatrix}
        \otimes 
        \begin{pmatrix}
            1  \\  0
        \end{pmatrix}
        , 
        \quad 
        \begin{pmatrix}
            1  \\  0
        \end{pmatrix}
        \otimes 
        \begin{pmatrix}
            0  \\  1
        \end{pmatrix}
        , 
        \quad 
        \begin{pmatrix}
            0  \\  1
        \end{pmatrix}
        \otimes 
        \begin{pmatrix}
            1  \\  0
        \end{pmatrix}
        , 
        \quad 
        \begin{pmatrix}
            0  \\  1
        \end{pmatrix}
        \otimes 
        \begin{pmatrix}
            0  \\  1
        \end{pmatrix}        
    \end{equation*}
    の4つ.
    あるいは,これらを計算して,
    \begin{equation*}
        \cvector{1 ; 0 ; 0 ; 0},
        \quad 
        \cvector{0 ; 1 ; 0 ; 0},
        \quad 
        \cvector{0 ; 0 ; 1 ; 0},
        \quad 
        \cvector{0 ; 0 ; 0 ; 1}
    \end{equation*}
    つまり,
    \begin{equation*}
        \symbb{C}^2 \otimes \symbb{C}^2 = \symbb{C}^4
    \end{equation*}

\end{frame}


\begin{frame}
    \frametitle{行列のテンソル積}

    行列のテンソル積も,
    \begin{equation*}
        \begin{pmatrix}
            a & b \\ c & d
        \end{pmatrix}
        \otimes 
        \underbrace{
        \begin{pmatrix}
            \alpha & \beta \\ \gamma & \delta
        \end{pmatrix}
        }_{\text{\normalsize$\symsf{B}$}}
        = 
        \begin{pmatrix}
            a \symsf{B} & b \symsf{B} \\
            c \symsf{B} & d \symsf{B}
        \end{pmatrix}
        =
        \begin{pmatrix}
            a\alpha & a\beta & b\alpha & b\beta \\
            a\gamma & a\delta & b\gamma & b\delta \\
            c\alpha & c\beta & d\alpha & d\beta \\
            c\gamma & c\delta & d\gamma & d\delta
        \end{pmatrix}
    \end{equation*}

    さらに,2スピン系$\ket{\psi} \otimes \ket{\varphi}$に作用する演算子は,
    それぞれのスピンに作用する演算子(行列)$\hat{\symscr{O}}_1, \hat{\symscr{O}}_2$を使うと
    \begin{alertblock}{演算子のテンソル積の作用}
        \begin{equation*}
            \ab(\hat{\symscr{O}}_1 \otimes \hat{\symscr{O}}_2)
            \ab(\ket{\psi} \otimes \ket{\varphi})
            = \hat{\symscr{O}}_1 \ket{\psi}
              \otimes
              \hat{\symscr{O}}_2 \ket{\varphi}
        \end{equation*}    
    \end{alertblock}

\end{frame}


\begin{frame}
    \frametitle{テンソル積で表せないテンソル積}

    $\symbb{C}^2 \otimes \symbb{C}^2 = \symbb{C}^4$のベクトル
    \begin{equation}
        \label{eq:tensor-product-not-separable}
        \cvector{0 ; 1 ; 1 ; 0}
        = \ab\big\{ \cvector{1 ; 0} \otimes \cvector{0 ; 1} \}
        + \ab\big\{ \cvector{0 ; 1} \otimes \cvector{1 ; 0} \}
    \end{equation}
    は,
    $\symbb{C}^2$の2つのベクトルを使って
    \begin{equation*}
        \cvector{0 ; 1 ; 1 ; 0}
        = \cvector{a ; b} \otimes \cvector{c ; d}
        = \cvector{ac ; ad ; bc ; bd}
    \end{equation*}
    という形で書くことができない.

\end{frame}


\begin{frame}
    \frametitle{エンタングル状態}
    
    \eqref{eq:tensor-product-not-separable}をスピン状態で書き換えれば
    \begin{equation}
        \label{eq:entangled-state}
        \frac{1}{\sqrt{2}} \ab\Big( 
            \ket{\bitone}<A> \otimes \ket{\bittwo}<B>
            + \ket{\bittwo}<A> \otimes \ket{\bitone}<B>
        )
    \end{equation}
    この状態は,$\ket{\psi} \otimes \ket{\xi}$と書くことができない.

    そのような状態を,
    \structure{エンタングル状態}という.
    
\end{frame}


\begin{frame}
    \frametitle{エンタングル状態の測定}

    2つのスピン$\symup{A}, \symup{B}$のエンタングル状態
    \begin{equation*}
        \tag{\ref*{eq:entangled-state}}
        \frac{1}{\sqrt{2}} \ab\Big( 
            \ket{\bitone}<A> \otimes \ket{\bittwo}<B>
            + \ket{\bittwo}<A> \otimes \ket{\bitone}<B>
        )
    \end{equation*}
    に対し,
    粒子$\symup{A}$のスピンの向きを測定する.

    \pause

    たとえば$\symup{A}$のスピン上向き($\bitone$)を観測したら,
    状態は
    \begin{equation*}
            \ket{\bitone}<A> \otimes \ket{\bittwo}<B>
    \end{equation*}
    に収束する.

    つまり,
    \alert{$\symup{A}$のスピンの向きを測定すると,
    $\symup{B}$のスピンの向きも確定する.}

\end{frame}


\section{量子テレポーテーション}


\begin{frame}
    \frametitle{量子テレポーテーション}

    Aliceが遠方にいるBobに,
    自身が持っている状態
    \begin{equation*}
        \ket{\xi}<X> = a \ket{\bitone} + b \ket{\bittwo}
    \end{equation*}
    を転送したいとする.

    \alert{$\ket{\xi}<X>$は未知.}
    つまり,Aliceは$a, b$の値を知らない.

    \pause

    Aliceは$a, b$を知らないまま,
    Bobに$\ket{\xi}<X>$を転送することができる.

    もちろん状態$\ket{\xi}<X>$にある粒子がBobの居場所へテレポートするわけではなく,
    $\ket{\xi}$をBobの手元で再現できるということ.

    {\small
        とはいえ,状態が同じ2つの粒子は区別できないので,
        粒子がテレポートしたのと同じである.
    }

\end{frame}


\begin{frame}
    \frametitle{ベル基底}

    $\symbb{C}^2 \otimes \symbb{C}^2$の正規直交基底として,
    \structure{ベル基底}
        \begin{align*}
            \Ket{\Phi^+}<A,B> 
                &\coloneq \frac{1}{\sqrt{2}}
                    \ab\Big( \ket{\bitone}<A> \otimes \ket{\bitone}<B> + \ket{\bittwo}<A> \otimes \ket{\bittwo}<B> )
            \\
            \Ket{\Phi^-}<A,B> 
                &\coloneq \frac{1}{\sqrt{2}}
                    \ab\Big( \ket{\bitone}<A> \otimes \ket{\bitone}<B> - \ket{\bittwo}<A> \otimes \ket{\bittwo}<B> )
            \\
            \Ket{\Psi^+}<A,B> 
                &\coloneq \frac{1}{\sqrt{2}}
                    \ab\Big( \ket{\bitone}<A> \otimes \ket{\bittwo}<B> + \ket{\bittwo}<A> \otimes \ket{\bitone}<B> )
            \\
            \Ket{\Psi^-}<A,B> 
                &\coloneq \frac{1}{\sqrt{2}}
                    \ab\Big( \ket{\bitone}<A> \otimes \ket{\bittwo}<B> - \ket{\bittwo}<A> \otimes \ket{\bitone}<B> )
        \end{align*}
        をとれる.

\end{frame}


\begin{frame}
    \frametitle{量子テレポーテーションの手順①-1}

    AliceとBobが,
    エンタングル状態
    \begin{equation*}
        \Ket{\Phi^+}<A,B> 
        \coloneq \frac{1}{\sqrt{2}}
        \ab\Big( \ket{\bitone}<A> \otimes \ket{\bitone}<B> + \ket{\bittwo}<A> \otimes \ket{\bittwo}<B> )
    \end{equation*}
    を所持しているとする.

    これを使って,Aliceが持っている未知状態
    \begin{equation*}
        \ket{\xi}<X> = a \ket{\bitone} + b \ket{\bittwo}
    \end{equation*}
    をBobに転送する.

\end{frame}


\begin{frame}
    \frametitle{量子テレポーテーションの手順①-2}

    $\symup{X}, \symup{A}, \symup{B}$を合わせた系の状態は
    \begin{equation}
        \label{eq:system-XAB}
        \begin{split}
            \ket{\xi}<X> \otimes \Ket{\Phi^+}<A,B>
            &= \frac{1}{2} \Ket{\Phi^+}<X,A> \otimes 
                \ab\Big( a \ket{\bitone}<B> + b \ket{\bittwo}<B> )
                \\
            &+ \frac{1}{2} \Ket{\Phi^-}<X,A> \otimes 
                \ab\Big( a \ket{\bitone}<B> - b \ket{\bittwo}<B> )
                \\
            &+ \frac{1}{2} \Ket{\Psi^+}<X,A> \otimes 
                \ab\Big( b \ket{\bitone}<B> + a \ket{\bittwo}<B> )
            \\
            &+ \frac{1}{2} \Ket{\Psi^-}<X,A> \otimes 
                \ab\Big(-b \ket{\bitone}<B> + a \ket{\bittwo}<B> )
        \end{split}
    \end{equation}
    と書ける.

\end{frame}


\begin{frame}
    \frametitle{量子テレポーテーションの手順②}

    Aliceは系$\symup{X}\symup{A}$に対し,
    ベル基底による測定を行う.

    すると,
    状態$\Ket{\Phi^+}, \Ket{\Phi^-}, \Ket{\Psi^+}, \Ket{\Psi^-}$のどれかをそれぞれ$1/4$の確率で観測する.

    \eqref{eq:system-XAB}より,
    Aliceが観測した系$\symup{X}\symup{A}$の状態によって,
    Bobのもつ系$\symup{B}$の状態が確定する.
    \begin{center}
        \begin{tabular}{c|c}
            Alice ($\symup{X}\symup{A}$)  &  Bob ($\symup{B}$)  \\
            \hline
            $\Ket{\Phi^+}$ & $a \ket{\bitone} + b \ket{\bittwo}$  \\
            $\Ket{\Phi^-}$ & $a \ket{\bitone} - b \ket{\bittwo}$  \\
            $\Ket{\Psi^+}$ & $b \ket{\bitone} + a \ket{\bittwo}$  \\
            $\Ket{\Psi^-}$ & $-b \ket{\bitone} + a \ket{\bittwo}$  \\
        \end{tabular}
    \end{center}
    このとき,一瞬で(光速を超えて)Bobの状態が収束する.

\end{frame}


\begin{frame}
    \frametitle{量子テレポーテーションの手順③-1}

    Aliceは何らかの手段(\structure{古典通信})によって,
    自分がどの状態を観測したかBobに伝える.

    Bobは系$\symup{B}$に対して,
    次のような操作を行う.
    \begin{enumerate}
        \item Aliceが$\Ket{\Phi^+}$を観測した場合,何もしない.
        \item Aliceが$\Ket{\Phi^-}$を観測した場合,$\ket{\bittwo}$の符号を反転する.
        \item Aliceが$\Ket{\Psi^+}$を観測した場合,$\ket{\bitone}$と$\ket{\bittwo}$の係数を入れ替える.
        \item Aliceが$\Ket{\Psi^-}$を観測した場合,$\ket{\bitone}$と$\ket{\bittwo}$の係数を入れ替え,さらに$\ket{\bitone}$と$\ket{\bittwo}$の係数を入れ替える.
    \end{enumerate}
    この操作により,系$\symup{B}$の状態が$\ket{\xi} = a \ket{\bitone} + b \ket{\bittwo}$になる.

\end{frame}


\begin{frame}
    \frametitle{量子テレポーテーションの手順③-2}

    具体的には,Bobは
    \begin{enumerate}
        \item Aliceが$\Ket{\Phi^+}$を観測した場合,何もしない.
        \item Aliceが$\Ket{\Phi^-}$を観測した場合,系$\symup{B}$にユニタリ変換$\sigma^z$を施す.
        \item Aliceが$\Ket{\Psi^+}$を観測した場合,系$\symup{B}$にユニタリ変換$\sigma^x$を施す.
        \item Aliceが$\Ket{\Psi^-}$を観測した場合,系$\symup{B}$にユニタリ変換$\sigma^z \sigma^x = \iu \sigma^y$を施す.
    \end{enumerate}
    ただし,$\sigma^i$はパウリ行列として表される:
    \begin{equation}
        \label{eq:Pauli-matrix}
        \sigma^x = 
        \begin{pmatrix}
            0  &  1  \\
            1  &  0
        \end{pmatrix},
        \quad 
        \sigma^y = 
        \begin{pmatrix}
            0  &  -\iu  \\
            \iu  &  0  
        \end{pmatrix},
        \quad 
        \sigma^z = 
        \begin{pmatrix}
            1  &  0  \\
            0  &  -1
        \end{pmatrix}
    \end{equation}
    
\end{frame}


\begin{frame}
    \frametitle{量子テレポーテーションに関する注意}

    Aliceが持っていた状態$\ket{\xi}<X>$は,
    手順の途中(Aliceによる系$\symup{X} \symup{A}$の測定)で破壊されている.

    この手順により,
    \alert{超光速でAliceからBobに情報が伝わるわけではない}.

    Bobが状態$\ket{\xi}$を得ることができるのは,
    Aliceから観測結果を\alert{光速以下の速度で}伝えられた後である.

\end{frame}


\begin{frame}[standout,plain,noframenumbering]

    \begin{center}
        \normalsize
        \includegraphics[width=0.2\linewidth]{../misc/github_url.pdf}
        \hspace{0.1\linewidth}
        \includegraphics[width=0.3\linewidth]{../misc/cc-by-sa.pdf}

        This file is licensed under CC-BY-SA 4.0.

        Want you get PDF or \TeX{} file?
        Visit \url{https://github.com/scasc1939/nusotsu}.
    \end{center}

\end{frame}



\end{document}