\documentclass{standalone}
\usepackage{tikz}
\usetikzlibrary{arrows.meta}
\usetikzlibrary{math}


\usepackage{xcolor}


\usepackage[bold-style=ISO]{unicode-math}


\directlua{mylib = require("rational_countable")}
\newcommand{\iscoprime}[2]{ \directlua{ tex.sprint(mylib.coprime( #1 , #2 )) } }


\begin{document}

\large

\begin{tikzpicture}
    % グラフの軸
    \node at (0, 0) [above left] {$O$};
    \draw [very thick, -Stealth] (-1,  0) -- ( 6,  0) node[above] {$p$};
    \draw [very thick, -Stealth] ( 0,  1) -- ( 0, -6) node[left]  {$q$};

    % x-軸めもり
    \foreach \x in {1, 2, ..., 5} {
        \draw [semithick] 
               (\x, 0.1) node[above] {$\x$}
            -- (\x, -0.1);
        \draw [dashed] (\x,  0) -- (\x, -6);
    }
    % y-軸めもり
    \foreach \y in {1, 2, ..., 5} {
        \draw [semithick]
               (-0.1, -\y) node[left] {$\y$}
            -- (0.1, -\y);
        \draw [dashed] (0, -\y) -- (6, -\y);
    }

    % \foreach \x in {1, 2, ..., 5} {
    %     \foreach \y in {-1, -2, ..., -5} {
    %         \fill (\x, \y) circle (1mm);
    %     }
    % }

    % 点の描画
    \tikzmath{
        int \a, \b;
        for \a in {1, 2, ..., 5} {
            for \b in {1, 2, ..., 5} {
                if \iscoprime{\a}{\b} < 1 then {
                    { \draw (\a, -\b) circle (1mm); };
                } else {
                    { \fill (\a, -\b) circle (1mm); };
                };
            };
        };
    }

    % 矢印の描画
    \foreach \n in {1, 2, ..., 4} {
        \draw [-{Stealth[length=5mm, open]}] 
            (\n + 2,  1) node[above, xshift=3mm] {(\n)}
             -- (-1,  -\n - 2);
    }
\end{tikzpicture}

\end{document}